\ProvidesFile{EDITION.tex}[2017/06/13]

\begin{edition}{Kritischer Text}
\markboth{Die \emph{Anathomie}}{Kritischer Text}

\beginnumbering
\pstart
%
[14v\hoch{o}a] Ou nom\wdx{nom}{m. `mot servant à désigner
les êtres, les choses qui appartiennent à une même catégorie
logique'}{\textbf{au nom de} \emph{`en vertu de'}}
de Dieu misericord\wdx{misericort}{adj. `qui a de la
miséricorde; miséricordieux'}{misericord \emph{m.sg.}}.
Cy co\emph{m}mence\wdx{*comencier}{v.intr. `entrer dans
son commencement'}{commence \emph{3.p.sg. ind.prés.}}
le premier\wdx{premier}{adj. `qui vient avant les autres,
dans un ordre; premier'}{} traictier\wdx{*traitié}{m. `ouvrage
didactique, où est exposé d'une manière systématique un
sujet ou un ensemble de sujets concernant une matière;
traité'}{traictier} de ceste oevre\wdx{*uevre}{f. 2\hoch{o}
`texte scientifique, technique ou littéraire'}{oevre} qui
parle\wdx{parler}{\textbf{\emph{parler de}} v.tr.indir.
`s'entretenir de; parler de'}{} de l'anathomie\wdx{*anatomie}{f.
1\hoch{o} `structure et composition du corps humain et animal,
et, en parlant dans un sens abstrait, science de cette
structure'}{anathomie} et\wdx{et}{conj. de coordination `et'}{}
contient\wdx{contenir}{v.tr. `comprendre en soi, dans sa capacité,
son étendue, sa substance; contenir'}{contient \emph{3.p.sg. ind.prés.}}
deux doctrines\wdx{doctrine}{f. `ensemble de notions qu'on affirme
être vraies et par lesquelles on veut fourner une interprétation
des faits, orienter ou diriger l'action'}{}: la premiere\wdx{premier}{adj.
`qui vient avant les autres, dans un ordre; premier'}{}
doctrine\wdx{doctrine}{f. `ensemble de notions qu'on affirme
être vraies et par lesquelles on veut fourner une interprétation des
faits, orienter ou diriger l'action'}{}
parle\wdx{parler}{\textbf{\emph{parler de}} v.tr.indir.
`s'entretenir de; parler de'}{} de l'anathomie\wdx{*anatomie}{f. 1\hoch{o}
`structure et composition du corps humain et animal, et, en parlant dans
un sens abstrait, science de cette structure'}{anathomie}
des membres\wdx{membre}{m. terme d'anat. `chacune des parties du
corps humain ou animal remplissant une fonction déterminée'}{}
\emph{com}muns\wdx{*comun}{adj. `qui appartient à plusieurs
personnes ou choses'}{commun}, universelz\wdx{universel}{adj. `qui
s'applique à la totalité des objets (personnes, choses) que l'on
considère'}{} et\wdx{et}{conj. de coordination `et'}{}
simples\wdx{simple}{adj. terme d'anat. `qui n'est pas composé de
plusieurs parties à distinguer (dit d'un membre du corps considéré
comme un tout)'}{}. 
La seconde sera
des me\emph{m}bres\wdx{membre}{m.
terme d'anat. `chacune des parties du
corps humain ou animal remplissant une fonction déterminée'}{}
propres\wdx{propre}{adj. 2\hoch{o} `qui
est particulier à (qn, qch.)'}{},
particuliers\wdx{*particuler}{adj. `qui
appartient en propre (à qn, à qch., ou à une catégorie
de personnes, de choses); particulier'}{particulier}
[14v\hoch{o}b]
et\wdx{et}{conj. de coordination `et'}{}
compost\wdx{compost}{adj.
terme d'anat.
`qui est formé
de plusieurs membres simples (dit d'un
membre du corps)'}{compost
\emph{m.pl.}}. La premiere\wdx{premier}{adj.
`qui vient avant les
autres, dans un ordre; premier'}{}
doctrine\wdx{doctrine}{f.
`ensemble de
notions qu'on affirme être vraies et par
lesquelles on veut fourner une interprétation
des faits, orienter ou diriger l'action'}{} co\emph{n}tient\wdx{contenir}{v.tr. `comprendre en soi,
dans sa capacité, son étendue, sa substance;
contenir'}{contient \emph{3.p.sg. ind.prés.}}
.v.
chappitres\wdx{chapitre}{m. `chacune des
parties qui se suivent dans un livre et en articulent la lecture;
chapitre'}{chappitre}:
le premier\wdx{premier}{adj.
`qui vient avant les
autres, dans un ordre; premier'}{}, c'est ung
chappitre\wdx{chapitre}{m. `chacune des
parties qui se suivent dans un livre et en articulent la lecture;
chapitre'}{chappitre}
universel\wdx{universel}{adj. `qui
s'applique à la totalité des objets (personnes, choses) que l'on
considère'}{}
qui parle\wdx{parler}{\textbf{\emph{parler de}} v.tr.indir.
`s'entretenir de;
parler de'}{} de
l'anathomie\wdx{*anatomie}{f. 1\hoch{o} `structure et composition du corps
humain et animal, et, en parlant dans un sens abstrait, science
de cette structure'}{anathomie}
et\wdx{et}{conj. de coordination `et'}{}
de
la nature\wdx{nature}{f. 1\hoch{o} `ensemble des
caractères, des propriétés qui définissent un être,
une chose concrète ou abstraite'}{} des me\emph{m}bres\wdx{membre}{m.
terme d'anat. `chacune des parties du
corps humain ou animal remplissant une fonction déterminée'}{}
du corps\wdx{cors}{m. `ce qui fait l'existence
matérielle d'un homme ou d'un animal'}{corps}.
\pend
%
% \memorybreak
%
\pstartueber
[14v\hoch{o}a]
POUR CE QUE, selon Galien\adx{Galien}{}{},
lumiere\wdx{lumiere}{f. au fig. `homme de grande
intelligence, de grande valeur'}{} des
medicins\wdx{*medecin}{m. 1\hoch{o} `personne
habilitée
à exercer la médecine'}{medicin}, ou
.xvij.\hoch{e}
[14v\hoch{o}b] livre\wdx{livre}{m. `assemblage d'un assez grand nombre
de feuilles, portant des signes destinés à être lus'}{} qui
se intitule\wdx{intituler}{v.pron. `avoir
pour titre'}{}
\flq De
utilité\wdx{utilité}{f.
`caractère
de ce qui est utile; utilité'}{}
des parties\frq , ou
penultime\wdx{*penultieme}{adj.
`avant-dernier'}{penultime}
chappitre\wdx{chapitre}{m. `chacune des
parties qui se suivent dans un livre et en articulent la lecture;
chapitre'}{chappitre},
y sont quatre utilités\wdx{utilité}{f. `caractère
de ce qui est utile; utilité'}{} de la
science\wdx{scïence}{f.
`connaissance
exacte et approfondie; science'}{science} de
anathomie\wdx{*anatomie}{f. 1\hoch{o} `structure et composition du corps
humain et animal, et, en parlant dans un sens abstrait, science
de cette structure'}{anathomie}:
l'une, qui est la tres grande\wdx{grant}{adj.
3\hoch{o} dans
l'ordre qualitatif, non quantifiable `qui est d'un
degré supérieur à la moyenne en ce
qui concerne la
qualité, l'intensité, l'importance'}{grande
\emph{f.sg.}}, pour
amiracion\wdx{amiracion}{f. `sentiment de joie et d'épanouissement
devant ce qu'on juge supérieurement beau ou grand; admiration'}{} de la
puissance\wdx{*poissance}{f. `état de celui qui
peut beaucoup, qui a une grande action sur les
personnes, les choses, aussi la domination qui en
résulte; puissance'}{puissance} de Dieu;
la seconde, po\emph{ur} cognoistre\wdx{*conoistre}{v.tr.
1\hoch{o}
`avoir une idée de (qch.); connaître'}{cognoistre
\emph{inf.}}
les parties\wdx{partie}{f.
`élément d'un tout;
partie'}{} des paciens\wdx{pacïent}{m. `malade qui
est l'objet d'un examen médical, d'un
traitement médical; patient'}{pacient};
la tierce, pour
pronostiquer\wdx{pronostiquer}{v.tr. `émettre un
pronostic au sujet de l'évolution d'une maladie, de sa
gravité; pronostiquer'}{} des
[15r\hoch{o}a] disposicions\wdx{disposicïon}{f.
1\hoch{o} `action de mettre dans un certain ordre, le
résultat de cette action'}{disposicion} du
corps\wdx{cors}{m. `ce qui fait l'existence
matérielle d'un homme ou d'un animal'}{corps} qui
doivent\wdx{devoir}{v.tr. + inf. `être dans
l'obligation de (faire qch.); devoir'}{}
avenir\wdx{avenir}{v.tr.indir.
`venir ou être sur le point d'être; arriver'}{}; la
quarte si est pour \text{curer}\fnb{Erstes \emph{r} über der Zeile
nachgetragen.}/\wdx{curer}{v.tr.
`soumettre à un traitement médical'}{} les
maladies\wdx{maladie}{f. `altération organique ou
fonctionnelle considérée dans son évolution, et comme
une entité définissable; maladie'}{}.
Et pour ce, c'est chose\wdx{chose}{f. `toute réalité
concrète ou abstraite qu'on désigne d'une
manière déterminé'}{}
necessaire\wdx{necessaire}{adj. `dont l'existence, la
présence est requise pour répondre au besoin (de qn,
de qch.)'}{} et\wdx{et}{conj. de coordination `et'}{}
prouffitable\wdx{*porfitable}{adj. `qui est
avantageux, utile; profitable'}{prouffitable} a\wdx{a}{prép.
marquant des rapports de direction, de position `à'}{}
ung
ch\emph{acu}m
medicin\wdx{*medecin}{m. 1\hoch{o}
`personne habilitée à
exercer la médecine'}{medicin} de
savoir\wdx{savoir}{v.tr. `avoir présent à l'esprit
(un objet de pensée qu'on identifie et qu'on tient
pour réel); savoir'}{}
la anathomie\wdx{*anatomie}{f. 1\hoch{o} `structure et composition du corps
humain et animal, et, en parlant dans un sens abstrait, science
de cette structure'}{anathomie}.
Et c'est ce q\emph{ue} disoit
Galien\adx{Galien}{}{} ou livre\wdx{livre}{m. `assemblage d'un
assez grand nombre de feuilles, portant des signes destinés à
être lus'}{}
qui se intitule\wdx{intituler}{v.pron. `avoir
pour titre'}{}
\flq Liber
scienciarum
sive interioru\emph{m}
medico\emph{rum}\frq , ou il\wdx{il}{troisième personne, masculin ou féminin, singulier ou
pluriel, du pron. pers., aussi à valeur neutre}{}
dit ainsi\wdx{ainsi}{adv. `de cette façon'}{}: les
jeunes clers\wdx{*clerc}{m. `celui qui
est lettré, savant'}{clers
\emph{pl.}}, et\wdx{et}{conj. de coordination `et'}{}
les anciens\wdx{anciens}{m.pl. `ceux qui ont vécu dans des temps
fort éloignés de nous'}{} aussi,
estudient\wdx{estudiier}{v.tr.indir. `apporter une
attention soutenue à qch.; s'appliquer'}{estudient
\emph{3.p.pl. ind.prés.}} a
cognoistre\wdx{*conoistre}{v.tr. 1\hoch{o}
`avoir une idée de (qch.); connaître'}{cognoistre
\emph{inf.}} les parties\wdx{partie}{f.
`élément d'un tout;
partie'}{}
et\wdx{et}{conj. de coordination `et'}{}
les passions\wdx{passïon}{f. `souffrance physique'}{} d'icelles,
\text{car selon la difference
d'icelles}\lemma{car{\dots}
d'icelles}\fnb{Unvollständig übersetzt (cp.
GuiChaul\textsc{jl} 19,21s. \emph{quia curam oportet
diversificare secundum differentias
ipsarum}).}/\wdx{car}{conj. qui
unit à une proposition une
proposition suivante qui donne
la raison de ce qu'affirme la
première}{}
\wdx{*diference}{f.
`caractère ou ensemble de caractères
qui distingue une chose d'une autre, un être
d'un autre'}{difference}.
Et\wdx{et}{conj. de coordination `et'}{}
ja soit ce que\wdx{ja soit ce
que}{loc.conj. `bien que assurément'}{} \text{les
parties}\fnb{Ms. \emph{laes
parties}.}/\wdx{partie}{f.
`élément d'un tout;
partie'}{}
qui appare\emph{n}t\wdx{*aparoir}{v.intr. `se montrer aux yeux; se
manifester'}{apparent \emph{3.p.pl. ind.prés.}} ou
sens\wdx{sens}{m. 1\hoch{o} `faculté d'éprouver
les impressions que font les objets matériels, i.e.
goût, odorat, ouïe, toucher, vue'}{} soient
cogneues\wdx{*conoistre}{v.tr. 2\hoch{o} `saisir
(qch.) par la pensée; reconnaître'}{cogneues
\emph{p.p. f.pl.}}
appertemant\wdx{*apertement}{adv. `d'une manière
évidente'}{appertemant},
toutesvoies\wdx{*totes voies}{loc.adv. `en
considérant toutes les raisons, toutes les
circonstances qui pourraient s'y opposer, et
malgré elles; toutefois'}{toutesvoies}, celles qui
sont en\wdx{en}{prép. marquant en général la position
à l'intérieur de limites spatiales, temporelles ou
notionelles `en'}{} p\emph{ar}font\wdx{parfont}{m. `ce
qui est profond; profond'}{}
occultes\wdx{*ocult}{adj. `qui se cache, garde le
secret ou l'incognito; occulte'}{occult}, elles
\text{ont}\fnb{\emph{ont} l. \emph{sont}\,?}/
mestier de
ho\emph{m}me\wdx{*ome}{m. `être appartenant à l'espèce
animale la plus évoluée de la terre; être
humain'}{homme} qui soit excercités\wdx{*exerciter}{v.tr. `avoir une
activité réglée pour acquérir la pratique;
s'exercer'}{excercité \emph{p.p.}}
en\wdx{en}{prép. marquant en général la position
à l'intérieur de limites spatiales, temporelles ou
notionelles `en'}{} l'anathomie\wdx{*anatomie}{f. 2\hoch{o} `action de disséquer,
de séparer méthodiquement les différentes parties d'un corps organisé; dissection'}{anathomie}
et\wdx{et}{conj. de coordination `et'}{}
es accions\wdx{accion}{f. `tout ce que l'on fait (dit aussi de choses)'}{} et
utilités\wdx{utilité}{f.
`caractère
de ce qui est utile; utilité'}{}
d'icelles. Et de ce lieu\wdx{lieu}{m. `portion
déterminée de l'espace; lieu'}{} ci est prins le
principes\wdx{principe}{m.
`commencement, première manifestation (d'une
chose); origine'}{}
de tout
le continent\wdx{continent}{m. `ce qui
est dans un contenant; contenu'}{}.
Et dit qu'il\wdx{il}{troisième personne, masculin ou féminin, singulier ou
pluriel, du pron. pers., aussi à valeur neutre}{}
est escript ou premier\wdx{premier}{adj.
`qui vient avant les
autres, dans un ordre; premier'}{}
du
livre\wdx{livre}{m. `assemblage d'un assez grand
nombre de
feuilles, portant des signes destinés à être lus'}{}
des me\emph{m}bres\wdx{membre}{m.
terme d'anat. `chacune des parties du
corps humain ou animal remplissant une fonction déterminée'}{}
que le
medicin\wdx{*medecin}{m. 1\hoch{o}
`personne habilitée à
exercer la médecine'}{medicin}
hardi\wdx{hardi}{adj. `qui manifeste un
tempérament, un esprit prompt à oser sans se
laisser intimider; hardi'}{}
doit\wdx{devoir}{v.tr. + inf. `être dans
l'obligation de (faire qch.); devoir'}{} estre sage en\wdx{en}{prép. marquant en général la position
à l'intérieur de limites spatiales, temporelles ou
notionelles `en'}{}
la cognoissance\wdx{*conoissance}{f.
`le fait de connaître'}{cognoissance} des
me\emph{m}bres\wdx{membre}{m.
terme d'anat. `chacune des parties du
corps humain ou animal remplissant une fonction déterminée'}{}
qui viennent\wdx{venir}{v.tr.indir.
1\hoch{o} `avoir
son origine dans'}{} en\wdx{en}{prép. marquant en général la position
à l'intérieur de limites spatiales, temporelles ou
notionelles `en'}{} ch\emph{acu}m
lieu\wdx{lieu}{m.
`portion
déterminée de l'espace; lieu'}{}. Et se c'est
chose\wdx{chose}{f. `toute réalité
concrète ou abstraite qu'on désigne d'une
manière déterminé'}{}
prouffitable\wdx{*porfitable}{adj.
`qui est
avantageux, utile; profitable'}{prouffitable} aux
phisiciens\wdx{*fisicïen}{m. `personne
habilitée à exercer la médecine'}{phisicien},
elle est plus necessaire\wdx{necessaire}{adj.
`dont l'existence, la
présence est requise pour répondre au besoin (de qn,
de qch.)'}{}
aux cirurgiens\wdx{*cirurgiien}{m. `celui qui exerce la chirurgie'}{cirurgien},
selon sa doctrine\wdx{doctrine}{f.
`ensemble de
notions qu'on affirme être vraies et par
lesquelles on veut fourner une interprétation
des faits, orienter ou diriger l'action'}{}
ou .vj.\hoch{e} de
\flq Terapeutique\frq\wdx{*therapeutique}{f. terme de méd.
`partie de la médecine qui étudie et
met en application les moyens propres à guerir et à
soulager les malades'}{terapeutique}
qui se intitule\wdx{intituler}{v.pron. `avoir
pour titre'}{}, en\wdx{en}{prép. marquant en général la position
à l'intérieur de limites spatiales, temporelles ou
notionelles `en'}{}
la
translacion\wdx{translacïon}{f.
`traduction d'une langue dans une
autre; traduction'}{translacion}
arabiq\emph{ue}\wdx{arabique}{adj. `qui appartient, est
relatif à l'Arabie et ses habitants'}{}, \flq De inge\emph{n}io
sanitatis\frq . Et les
cirurgiens\wdx{*cirurgiien}{m. `celui qui exerce la chirurgie'}{cirurgien}
ignorans\wdx{ignorer}{v.tr. `ne pas connaître, ne pas
savoir'}{} la
anatho\emph{m}ie\wdx{*anatomie}{f. 1\hoch{o} `structure et composition du corps
humain et animal, et, en parlant dans un sens abstrait, science
de cette structure'}{anathomie}
pechent\wdx{pechier}{v.tr.indir. `commettre une
faute'}{pechent \emph{3.p.pl. ind.prés.}}
maintes\wdx{maint}{adj. `plusieurs; maint'}{}
fois\wdx{*foiz}{f.
`cas
où un fait se produit, moment du temps où un
événement, conçu comme
identique à d'autres événements, se produit; fois'}{fois}
en\wdx{en}{prép. marquant en général la position
à l'intérieur de limites spatiales, temporelles ou
notionelles `en'}{}
incisio\emph{n}s\wdx{incision}{f. `action de fendre,
de couper avec un instrument tranchant, son
résultat (surtout en médecine)'}{} de
nerf\wdx{nerf}{m. terme d'anat.
`structure blanchâtre en forme de fil qui relie soit
un muscle à un os, soit un centre nerveux
(cerveau, moelle) à un organe ou une structure organique;
tendon ou nerf'}{} et de
liguemant\wdx{*liguement}{m. terme d'anat.
`faisceau de tissu fibreux blanchâtre,
résistant et peu extensible, unissant les éléments
d'une articulation ou maintenant en place un organe
ou une partie d'un organe; ligament'}{liguemant}. Mais
[15r\hoch{o}b] tu qui
sauras la nature\wdx{nature}{f.
1\hoch{o} `ensemble des
caractères, des propriétés qui définissent un être,
une chose concrète ou abstraite'}{} d'une chescune
petite\wdx{petit}{adj. 1\hoch{o}
dans l'ordre
physique, quantifiable `qui est d'une extension
au-dessous de la moyenne; petit (des choses)'}{}
partie\wdx{partie}{f.
`élément d'un tout;
partie'}{}
ou les posicions\wdx{posicïon}{f. `lieu où quelque chose est placée,
située'}{posicion}
et les plasmacions\wdx{plasmacïon}{f. `action de
donner une forme'}{plasmacion},
c'est a dire\wdx{dire}{v.tr.
`lire à haute voix;
réciter'}{\textbf{c'est a dire} \emph{loc.conj. qui
annonce une explication ou une précision}} co\emph{m}me
les me\emph{m}bres\wdx{membre}{m.
terme d'anat. `chacune des parties du
corps humain ou animal remplissant une fonction déterminée'}{}
sont fourmés\wdx{*former}{v.tr. `donner
une certaine forme (à qch.)'}{fourmé \emph{p.p.}}
par tout
le corps\wdx{cors}{m. `ce qui fait l'existence
matérielle d'un homme ou d'un animal'}{corps} et
selon chescum me\emph{m}bre\wdx{membre}{m.
terme d'anat. `chacune des parties du
corps humain ou animal remplissant une fonction déterminée'}{}
lors, quant une plaie\wdx{plaie}{f. `ouverture
dans les chairs, les tissus, due à une cause
externe (traumatisme, intervention chirurgicale) et
présentant une solution de continuité des téguments;
plaie'}{} sera faite\wdx{faire}{v.tr. `réaliser
ou effectuer (qch.)'}{}
au me\emph{m}bre, tu pourras cognoistre\wdx{*conoistre}{v.tr.
2\hoch{o} `saisir
(qch.) par la pensée; reconnaître'}{cognoistre
\emph{inf.}}
appertemant\wdx{*apertement}{adv. `d'une manière
évidente'}{appertemant}, se le
nerf\wdx{nerf}{m. terme d'anat.
`structure blanchâtre en forme de fil qui relie soit
un muscle à un os, soit un centre nerveux
(cerveau, moelle) à un organe ou une structure organique;
tendon ou nerf'}{} est
coupé\wdx{coper}{v.tr. `diviser (qch.) avec
un instrument tranchant; couper'}{coupé \emph{p.p.}}
ou
le tenant\wdx{*tendant}{m. terme d'anat. `structure conjonctive fibreuse par laquelle
un muscle s'insère sur un os'}{tenant} ou la
colligance\wdx{colligance}{f. 2\hoch{o}
terme d'anat. `faisceau de tissu blanchâtre, résistant et peu extensible, unissant les éléments
d'une articulation ou maintenant en place un organe ou une partie d'un
organe; ligament'}{},
c'est a dire\wdx{dire}{v.tr.
`lire à haute voix;
réciter'}{\textbf{c'est a dire} \emph{loc.conj. qui
annonce une explication ou une précision}}
le
liguemant\wdx{*liguement}{m. terme d'anat.
`faisceau de tissu fibreux blanchâtre,
résistant et peu extensible, unissant les éléments
d'une articulation ou maintenant en place un organe
ou une partie d'un organe; ligament'}{liguemant}. Et c'est ce
que
Henry de Mondeville\adx{Henri de Mondeville}{}{Henry}
argue\wdx{argüer}{v.tr. `prouver (qch.) par
des arguments'}{arguer} ou
premier\wdx{premier}{adj.
`qui vient avant les
autres, dans un ordre; premier'}{}
de sa \flq Cirurgie\frq\wdx{cirurgie}{f. `partie de l'art médical qui comporte
une intervention manuelle et instrumentale'}{} par ceste
maniere\wdx{maniere}{f. 2\hoch{o} `forme particulière
que revêt l'accomplissement d'une action, le
déroulement d'un fait, l'être ou
l'existence'}{}
ci: tout mestre\wdx{*maistre}{m.
1\hoch{o}
`celui qui est expert dans le domaine des
arts ou des sciences'}{mestre} doit\wdx{devoir}{v.tr. + inf. `être dans
l'obligation de (faire qch.); devoir'}{}
savoir\wdx{savoir}{v.tr.
`avoir présent à l'esprit
(un objet de pensée qu'on identifie et qu'on tient
pour réel); savoir'}{} et
cognoistre\wdx{*conoistre}{v.tr. 1\hoch{o}
`avoir une idée de (qch.); connaître'}{cognoistre
\emph{inf.}} le
subjet\wdx{subjet}{m. `ce qui est soumis à
l'esprit, à la pensée, sur quoi s'exerce la
réflexion; sujet'}{}
en\wdx{en}{prép. marquant en général la position
à l'intérieur de limites spatiales, temporelles ou
notionelles `en'}{} quoy il\wdx{il}{troisième personne, masculin ou féminin, singulier ou
pluriel, du pron. pers., aussi à valeur neutre}{}
fait\wdx{faire}{v.tr.
`réaliser
ou effectuer (qch.)'}{}
son
eovre\wdx{*uevre}{f. 1\hoch{o} `activité,
travail'}{eovre},
car\wdx{car}{conj. qui unit à une proposition une
proposition suivante qui donne
la raison de ce qu'affirme la
première}{}
autremant\wdx{*autrement}{adv. `d'une manière
différente; autrement'}{autremant}, en
ouvrant\wdx{*ovrer}{v.tr. `agir d'une manière suivi pour obtenir un
résultat utile; travailler'}{ouvrant \emph{p.prés.}}, il\wdx{il}{troisième personne, masculin ou féminin, singulier ou
pluriel, du pron. pers., aussi à valeur neutre}{}
erre\wdx{errer}{v.intr.
`s'écarter de la vérité; errer'}{}. Mais
\text{le
cirurgien}\fnb{Ms. \emph{les cirurgien}, \emph{s}
expungiert.}/\wdx{*cirurgiien}{m. `celui qui exerce la chirurgie'}{cirurgien} est
mestre\wdx{*maistre}{m.
1\hoch{o}
`celui qui est expert dans le domaine des
arts ou des sciences'}{mestre} de
santé\wdx{santé}{f. `bon état physiologique d'un
être vivant; santé'}{} de corps\wdx{cors}{m.
`ce qui fait l'existence
matérielle d'un homme ou d'un animal'}{corps}
humain\wdx{*umain}{adj. `qui appartient ou qui
est propre à l'homme; humain'}{humain}, donc le
cirurgien\wdx{*cirurgiien}{m. `celui qui exerce la chirurgie'}{cirurgien}
doit\wdx{devoir}{v.tr. + inf. `être dans
l'obligation de (faire qch.); devoir'}{}
savoir\wdx{savoir}{v.tr. `avoir présent à l'esprit
(un objet de pensée qu'on identifie et qu'on tient
pour réel); savoir'}{}
la nature\wdx{nature}{f.
1\hoch{o} `ensemble des
caractères, des propriétés qui définissent un être,
une chose concrète ou abstraite'}{}
et la composicion\wdx{composicion}{f.
`manière dont une chose est formée, par l'assemblage de
plusieurs éléments'}{} de
corps\wdx{cors}{m. `ce qui fait l'existence
matérielle d'un homme ou
d'un animal'}{corps} humain\wdx{*umain}{adj.
`qui appartient ou qui
est propre à l'homme; humain'}{humain} et,
par consequent\wdx{consequent}{adj.}{\textbf{par
consequent}
\emph{loc.adv. `comme suite logique'}},
\text{il}\fnb{Über der Zeile
nachgetragen.}/ doit\wdx{devoir}{v.tr. + inf. `être dans
l'obligation de (faire qch.); devoir'}{}
savoir\wdx{savoir}{v.tr. `avoir présent à l'esprit
(un objet de pensée qu'on identifie et qu'on tient
pour réel); savoir'}{}
la
anathomie\wdx{*anatomie}{f. 1\hoch{o} `structure et composition du corps
humain et animal, et, en parlant dans un sens abstrait, science
de cette structure'}{anathomie}.
Vecy la seconde rayson\wdx{raison}{f.
2\hoch{o} `ce qui permet d'expliquer qch.;
cause'}{rayson} par similitude\wdx{similitude}{f.
terme de rhétor. `comparaison prolongée'}{}: car\wdx{car}{conj. qui
unit à une proposition une
proposition suivante qui donne
la raison de ce qu'affirme la
première}{}
c'est comparacion\wdx{*comparation}{f. `fait
d'envisager ensemble (deux ou plusieurs choses) pour
en chercher
les différences ou les ressemblances; comparaison'}{comparacion}
semblable\wdx{semblable}{adj. `qui ressemble;
semblable'}{} d'ung
aveugle\wdx{*avugle}{m. `personne privée du sens de
la vue'}{aveugle} qui coupe et
tra\emph{n}che\wdx{*trenchier}{v.tr. `séparer (une chose
en parties, deux choses unies) d'une manière nette, au
moyen d'un instrument dur et fin; trancher'}{tranche
\emph{3.p.sg. ind.prés.}} bois
-- ainsi que\wdx{ainsi}{adv. `de cette façon'}{\textbf{ainsi que}
\emph{loc.conj.
`de la même façon que'}}
\text{por}\fnb{\emph{o} überschreibt \emph{a}.}/
fere\wdx{faire}{v.tr.
`réaliser
ou effectuer (qch.)'}{fere} une
ymage\wdx{*image}{f. 2\hoch{o}
`figure d'expression fondée sur la similitude et qui
consiste dans un transfert de sens par substitution
analogique'}{ymage}
-- et d'ung
cirurgien\wdx{*cirurgiien}{m. `celui qui exerce la chirurgie'}{cirurgien}
qui veult coper\wdx{coper}{v.tr.
`diviser (qch.) avec
un instrument tranchant; couper'}{}
ou tranchier\wdx{*trenchier}{v.tr.
`séparer (une chose
en parties, deux choses unies) d'une manière nette, au
moyen d'un instrument dur et fin; trancher'}{tranchier \emph{inf.}}
en\wdx{en}{prép. marquant en général la position
à l'intérieur de limites spatiales, temporelles ou
notionelles `en'}{}
corps\wdx{cors}{m. `ce qui fait l'existence
matérielle d'un homme ou d'un animal'}{corps}
humain\wdx{*umain}{adj.
`qui appartient ou qui
est propre à l'homme; humain'}{humain}
quant il\wdx{il}{troisième personne, masculin ou féminin, singulier ou
pluriel, du pron. pers., aussi à valeur neutre}{}
ne scet la anathomie\wdx{*anatomie}{f. 1\hoch{o} `structure et composition du corps
humain et animal, et, en parlant dans un sens abstrait, science
de cette structure'}{anathomie}.
Car\wdx{car}{conj. qui unit à une proposition une
proposition suivante qui donne
la raison de ce qu'affirme la
première}{}
l'aveugle\wdx{*avugle}{m. `personne privée du sens
de
la vue'}{aveugle} qui coupe le bois, il\wdx{il}{troisième personne, masculin ou féminin, singulier ou
pluriel, du pron. pers., aussi à valeur neutre}{}
en coupe
souvent ou plus ou moins
qu'il n'ap\emph{ar}tient, ainsi\wdx{ainsi}{adv. `de cette
façon'}{} fait\wdx{faire}{v.tr.
`réaliser
ou effectuer (qch.)'}{}
le cirurgien\wdx{*cirurgiien}{m. `celui qui exerce la chirurgie'}{cirurgien}
qui
ignore\wdx{ignorer}{v.tr. `ne pas connaître, ne pas
savoir'}{} la
anathomie\wdx{*anatomie}{f. 1\hoch{o} `structure et composition du corps
humain et animal, et, en parlant dans un sens abstrait, science
de cette structure'}{anathomie},
a la semblance\wdx{semblance}{f. `rapport entre
des objets quelquonques, présentant des éléments
identiques suffisamment nombreux et apparents;
ressemblance'}{} de aucuns
mauvais keux\wdx{*queu}{m. `personne qui a pour
fonction de faire la cuisine; cuisinier'}{keux
\emph{pl.}}
et bochiers\wdx{bochier}{m. `celui qui tue les
b\oe ufs, les moutons, etc., et en vend la chair
crue; boucher'}{}
[15v\hoch{o}a] -- desqueulx parle
Galien\adx{Galien}{}{} ou second de
\flq Terapeutique\frq\wdx{*therapeutique}{f.
terme de méd.
`partie de la médecine qui étudie et
met en application les moyens propres à guerir et à
soulager les malades'}{terapeutique} -- qui ne
coupent ne tranchent\wdx{*trenchier}{v.tr.
`séparer (une chose
en parties, deux choses unies) d'une manière nette, au
moyen d'un instrument dur et fin;
trancher'}{tranchent \emph{3.p.pl. ind.prés.}}
pas selon les articles\wdx{article}{m. et f. terme
d'anat.
`partie du corps formée par la jointure entre
deux ou plusieurs os; articulation'}{}, mais
ilz les rompent\wdx{rompre}{v.tr. `séparer en deux ou
plusieurs parties par un effort brusque; rompre'}{}
et cassent\wdx{casser}{v.tr. `mettre en morceaux
d'une manière soudaine, par coup ou pression;
casser'}{} et arragent\wdx{arrachier}{v.tr.
`détacher (qch.) en tirant et avec effort;
arracher'}{arragent
\emph{3.p.pl. ind.prés.}}
indeuema\emph{n}t\wdx{*indeüement}{adv. `d'une manière
qui va à l'encontre des exigences de la raison, de la
règle; indûment'}{indeuemant}. Donc y
s'ensuit\wdx{*ensivre}{v.pron.
2\hoch{o}
`penser ou agir selon (les idées, la
conduite de qn)'}{\textbf{il
s'ensuit que} \emph{`il résulte (d'un fait) que'}} que c'est
necessité\wdx{necessité}{f. `caractère nécessaire
(d'une chose, d'une action); nécessité'}{} aux
medicins\wdx{*medecin}{m. 1\hoch{o}
`personne habilitée à
exercer la médecine'}{medicin}
et maiemant\wdx{*maismement}{adv. `plus que tout
autre chose; surtout'}{maiemant} aux
cirurgie\emph{n}s\wdx{*cirurgiien}{m. `celui qui exerce la chirurgie'}{cirurgien}
savoir\wdx{savoir}{v.tr. `avoir présent à l'esprit
(un objet de pensée qu'on identifie et qu'on tient
pour réel); savoir'}{}
la anathomie.
Et ja ssoit
ce que\wdx{ja soit ce que}{loc.conj. `bien que assurément'}{} ce
fust necessité\wdx{necessité}{f.
`caractère nécessaire
(d'une chose, d'une action); nécessité'}{} a\wdx{a}{prép.
marquant des rapports de direction, de position `à'}{}
eulx de savoir\wdx{savoir}{v.tr. `avoir présent à l'esprit
(un objet de pensée qu'on identifie et qu'on tient
pour réel); savoir'}{}, avec la
anathomie, les
accions\wdx{accion}{f. `tout ce que l'on fait (dit aussi de choses)'}{} et
utilités\wdx{utilité}{f.
`caractère
de ce qui est utile; utilité'}{}
des me\emph{m}bres\wdx{membre}{m.
terme d'anat. `chacune des parties du
corps humain ou animal remplissant une fonction déterminée'}{}
qui sont les trois racines\wdx{racine}{f. 2\hoch{o}
`origine ou principe d'une chose'}{} et
ellemens\wdx{*element}{m. `partie constitutive d'une
chose; élément'}{ellemens \emph{pl.}} de toute
medicacion\wdx{medicacïon}{f. 1\hoch{o}
`emploi de médicaments dans un but
thérapeutique déterminé'}{medicacion},
si co\emph{m}me il\wdx{il}{troisième personne, masculin ou féminin, singulier ou
pluriel, du pron. pers., aussi à valeur neutre}{}
est monstré\wdx{*mostrer}{v.tr.
`faire connaître (qch.)'}{monstré
\emph{p.p.}} deuemant\wdx{*deuement}{adv. `selon ce
qu'on doit; dûment'}{deuemant} ou
p\emph{re}mier\wdx{premier}{adj.
`qui vient avant les
autres, dans un ordre; premier'}{}
de \flq Interioribus\frq . Et sans fere long\wdx{lonc}{adj.
2\hoch{o} `qui a une durée très étendue, qui dure
longtemps'}{long \emph{m.sg.}} s\emph{er}mo\emph{n}\wdx{sermon}{m. `relation
orale ou écrite; récit'}{}, il\wdx{il}{troisième personne, masculin ou féminin, singulier ou
pluriel, du pron. pers., aussi à valeur neutre}{}
me samble que les docteurs\wdx{*doctor}{m. `celui qui
enseigne des livres de doctrine'}{docteur}
q\emph{ue} s'ensuivent\wdx{*ensivre}{v.pron. 2\hoch{o}
`penser ou agir selon (les idées, la
conduite de qn)'}{ensuivent
\emph{3.p.pl. ind.prés.}}
en ont traitté\wdx{*traitier}{v.tr.indir.
`exposer ses vues sur (un sujet, une
science, etc.)'}{traitté
\emph{p.p.}} plainemant\wdx{*pleinement}{adv. `d'une
manière
pleine, totale; pleinement'}{plainemant}, c'est
assavoir\wdx{assavoir}{v.tr.}{\textbf{c'est
assavoir} \emph{loc.
`c'est-à-dire'}}
Galien\adx{Galien}{}{}
en .xvij. traictié\wdx{*traitié}{m.
`ouvrage
didactique, où est exposé d'une manière
systématique un sujet ou un ensemble de
sujets concernant une matière; traité'}{traictié} de
\flq Utilitate particula\emph{rum}\frq , car\wdx{car}{conj. qui
unit à une proposition une
proposition suivante qui donne
la raison de ce qu'affirme la
première}{}
autres\wdx{autre}{adj. `qui n'est
pas le même; autre'}{}
.xv.
tractiés\wdx{*traitié}{m.
`ouvrage
didactique, où est exposé d'une manière
systématique un sujet ou un ensemble de
sujets concernant une matière; traité'}{tractié} que il\wdx{il}{troisième personne, masculin ou féminin, singulier ou
pluriel, du pron. pers., aussi à valeur neutre}{}
ha fait de curacion\wdx{curacïon}{f.
`action de soigner'}{curacion} de anathomie
-- ce dit Hali\adx{Haly Rodoan}{}{Hali}
en\wdx{en}{prép. marquant en général la position
à l'intérieur de limites spatiales, temporelles ou
notionelles `en'}{} la
fin de \flq Tegny\frq\ --, encores ne sont il
pas translatés\wdx{translater}{v.tr. `traduire
d'une langue dans une autre; traduire'}{}. Item\wdx{item}{adv.
terme d'admin. `et de même, et aussi (introduction d'unités
traitées l'une après l'autre dans un traité, une ordonnance,
un doc. coutumier, etc.)'}{},
Haly\adx{Haliabas}{}{Haly} en parle en la p\emph{re}miere\wdx{premier}{adj.
`qui vient avant les
autres, dans un ordre; premier'}{}
partie\wdx{partie}{f.
`élément d'un tout;
partie'}{}
du livre\wdx{livre}{m. `assemblage d'un assez grand nombre
de feuilles, portant des signes destinés à être lus'}{}  \flq De
disposicione regali\frq , en la seconde et en la tierce
paraule\wdx{parole}{f. `ensemble de mots
qui expriment une idée'}{paraule},
et Avicene\adx{Avicene}{}{} \text{aussi}\fnb{Über der
Zeile nachgetragen.}/, en son \flq Canon\frq , ou
p\emph{re}mier\wdx{premier}{adj.
`qui vient avant les
autres, dans un ordre; premier'}{}
 livre\wdx{livre}{m. `assemblage d'un assez grand nombre
de feuilles, portant des signes destinés à être lus'}{}.
Toutesvoies\wdx{*totes voies}{loc.adv.
`en
considérant toutes les raisons, toutes les
circonstances qui pourraient s'y opposer, et
malgré elles; toutefois'}{toutesvoies},
nous ne metrons\wdx{metre}{v.tr.
2\hoch{o} `présenter ou
exposer (un fait, une idée, un ensemble de faits ou
d'idées)'}{} ci que la grosse\wdx{gros}{adj. 6\hoch{o} `?'}{}
et material\wdx{materïel}{adj. `qui est de
la nature de la matière, constitué par de la
matière'}{material
\emph{f.sg.}} anathomie
qui pourra adresser\wdx{*adrecier}{v.tr.
`diriger (qn) par des conseils'}{adresser} le
cirurgien\wdx{*cirurgiien}{m. `celui qui exerce la chirurgie'}{cirurgien}
qui fait operacion\wdx{operacïon}{f.
`action d'un pouvoir, d'une fonction,
d'un organe qui produit un effet selon sa
nature; opération'}{operacion}
en incisions\wdx{incision}{f. `action de fendre,
de couper avec un instrument tranchant, son
résultat (surtout en médecine)'}{} et en
reducion\wdx{reducion}{f. terme de méd. `opération
qui consiste à remettre en place (un os luxé,
fracturé; un organe déplacé); réduction'}{}
de me\emph{m}bres\wdx{membre}{m.
terme d'anat. `chacune des parties du
corps humain ou animal remplissant une fonction déterminée'}{}.
\pend
\pstart
Donc anatho[15v\hoch{o}b]mie, c'est
droite\wdx{droit}{adj. 2\hoch{o} `qui suit un
raisonnement correct'}{} division\wdx{*devisïon}{f.
`action de diviser (qch.) en parties, le résultat de
cette action'}{division}
et determinacion\wdx{determinacïon}{f. `action de
déterminer, le résultat de cette action'}{determinacion}
des me\emph{m}bres\wdx{membre}{m.
terme d'anat. `chacune des parties du
corps humain ou animal remplissant une fonction déterminée'}{}\linebreak
d'ung ch\emph{acu}m
corps\wdx{cors}{m. `ce qui fait l'existence
matérielle d'un homme
ou d'un animal'}{corps},
maiemant\wdx{*maismement}{adv. `plus que tout
autre chose; surtout'}{maiemant}
de corps humain\wdx{*umain}{adj.
`qui appartient ou qui
est propre à l'homme; humain'}{humain},
car\wdx{car}{conj. qui unit à une proposition une
proposition suivante qui donne
la raison de ce qu'affirme la
première}{}
l'entencion\wdx{entencïon}{f. 1\hoch{o} `manière de
penser; opinion'}{entencion} de toute
ceste science\wdx{scïence}{f.
`connaissance
exacte et approfondie; science'}{science} ci si est de
li\wdx{il}{troisième personne, masculin ou féminin, singulier ou
pluriel, du pron. pers., aussi à valeur neutre}{li
\emph{m.sg. c.r. (objet indirect)}}. Et est dicte anathomie de
\flq ana\frq\ en
grec\wdx{grec}{m. `la langue parlée par
les grecs'}{}, qui
signifie\wdx{*senefiier}{v.tr. `avoir le
sens de; signifier'}{signifie \emph{3.p.sg.
ind.prés.}}
\flq droit\frq\wdx{droit}{adj. 2\hoch{o}
`qui suit un
raisonnement correct'}{}
en françois\wdx{françois}{m. `la langue
parlée par les français'}{}, et
\flq thomos\frq\
en grec\wdx{grec}{m. `la langue parlée par
les grecs'}{}, qui
signiffient\wdx{*senefiier}{v.tr.
`avoir le
sens de; signifier'}{signiffient \emph{3.p.pl.
ind.prés.}}
\flq division\frq\wdx{*devisïon}{f.
`action de
diviser (qch.) en parties, le résultat de
cette action'}{division} en
françois\wdx{françois}{m. `la langue
parlée par les français'}{}, donc c'est a dire\wdx{dire}{v.tr.
`lire à haute voix;
réciter'}{\textbf{c'est a dire} \emph{loc.conj. qui
annonce une explication ou une précision}}
\text{einsi q\emph{ue}}\fnb{Über
der Zeile nachgetragen.}/\wdx{ainsi}{adv.
`de cette façon'}{\textbf{ainsi que}
\emph{loc.conj.
`de la même façon que'} einsi que}
\flq droite\wdx{droit}{adj. 2\hoch{o}
`qui suit un
raisonnement correct'}{} division\frq\wdx{*devisïon}{f.
`action de
diviser (qch.) en parties, le résultat de
cette action'}{division}.
Anathomie est
enquise\wdx{enquerre}{v.tr. `chercher à savoir
(qch.) en examinant ou en interrogeant'}{enquise
\emph{p.p. f.sg.}}
en deux manieres\wdx{maniere}{f. 2\hoch{o} `forme
particulière
que revêt l'accomplissement d'une action, le
déroulement d'un fait, l'être ou
l'existence'}{}: premiers\wdx{premier}{adv. `en
premier lieu, d'abord; premièrement'}{premiers} par
doctrine\wdx{doctrine}{f.
`ensemble de
notions qu'on affirme être vraies et par
lesquelles on veut fourner une interprétation
des faits, orienter ou diriger l'action'}{}
de livres\wdx{livre}{m. `assemblage d'un assez grand
nombre de feuilles, portant des signes destinés à être lus'}{},
et ja soit ce que\wdx{ja soit ce que}{loc.conj.
`bien que assurément'}{} ceste
maniere\wdx{maniere}{f. 2\hoch{o} `forme particulière
que revêt l'accomplissement d'une action, le
déroulement d'un fait, l'être ou
l'existence'}{} soit
proufitable\wdx{*porfitable}{adj.
`qui est
avantageux, utile; profitable'}{proufitable},
toutesvoies\wdx{*totes voies}{loc.adv.
`en
considérant toutes les raisons, toutes les
circonstances qui pourraient s'y opposer, et
malgré elles; toutefois'}{toutesvoies}
elle n'est pas
soufisante\wdx{*sofisant}{adj. `qui suffit;
suffisant'}{soufisante \emph{f.sg.}}
pour raconter les choses\wdx{chose}{f. `toute réalité
concrète ou abstraite qu'on désigne d'une
manière déterminé'}{} qui sont
seulemant\wdx{*seulement}{adv. `sans rien d'autre que
ce qui est mentionné; seulement'}{seulemant}
cogneuez\wdx{*conoistre}{v.tr. 1\hoch{o}
`avoir une idée de (qch.); connaître'}{cogneuez
\emph{p.p. f.pl.}}
au sens\wdx{sens}{m.
1\hoch{o} `faculté d'éprouver
les impressions que font les objets matériels, i.e.
goût, odorat, ouïe, toucher, vue'}{},
si co\emph{m}me il est escript ou premier\wdx{premier}{adj.
`qui vient avant les
autres, dans un ordre; premier'}{}
livre\wdx{livre}{m. `assemblage d'un assez grand nombre de
feuilles, portant des signes destinés à être lus'}{}, ou
.xvij.\hoch{e} chappitre\wdx{chapitre}{m. `chacune des
parties qui se suivent dans un livre et en articulent la lecture;
chapitre'}{chappitre}.
Et c'est ce que disoit Averrois\adx{Averrois}{}{} ou
premier
de son \flq Colliget\frq : \emph{et nos non abreviamus \emph{et}
c\emph{etera}}. Mais autremant\wdx{*autrement}{adv. `d'une manière
différente; autrement'}{autremant}, on\wdx{on}{pron.pers. indéfini 3\hoch{e} personne}{}
\text{peult}\fnb{Zu den Formen von \emph{pooir}, cf.
\flqq Die Sprache\frqq .}/\wdx{*pooir}{v.tr.
+ inf. `avoir la possibilité de (faire qch.);
pouvoir'}{peult \emph{3.p.sg. ind.prés.}}
enquerre\wdx{enquerre}{v.tr.
`chercher
à savoir
(qch.) en examinant ou en interrogeant'}{enquerre \emph{inf.}}
anathomie par
experiance\wdx{*esperïence}{f. `le fait
d'éprouver qch., considéré comme un
élargissement ou un enrichissement de la
connaissance; expérience'}{experiance}
des corps\wdx{cors}{m. `ce qui fait l'existence
matérielle d'un
homme ou d'un animal'}{corps} qui sont
mors\wdx{mort}{adj.
`qui a cessé de vivre; mort'}{mors \emph{m.pl.}}.
Car\wdx{car}{conj. qui unit à une proposition une
proposition suivante qui donne
la raison de ce qu'affirme la
première}{}
no\emph{us} veons\wdx{*vëoir}{v.tr. `percevoir (qch.) par le
sens de la vue'}{veons \emph{1.p.pl. ind.prés.}}, quant
on les decole\wdx{decoler}{v.tr. `trancher la tête
(de qn); décapiter'}{} ou q\emph{ua}nt
on\wdx{on}{pron.pers. indéfini 3\hoch{e} personne}{} les
pent\wdx{*pendiier}{v.tr. `mettre à mort (qn) en
suspendant par le cou au moyen d'une corde'}{pent
\emph{3.p.sg. ind.prés.}}, la
anathomie au mains\wdx{moins}{adv.}{\textbf{au moins}
\emph{loc.adv. qui sert
à marquer une restriction} au mains}
des me\emph{m}bres
officiaulx\wdx{*oficïal}{adj. terme d'anat `qui
remplit une fonction au bénéfice de tout le corps'}{officiaulx
\emph{m.pl.}}
de dedans\wdx{dedans}{adv. `à l'intérieur'}{\textbf{de dedans}
\emph{loc.adj. `qui est
situé à l'intérieur'}}, de la char\wdx{char}{f.
`substance somatique de consistance relativement
molle
(surtout par opposition au tissu osseux et sa
dureté); chair'}{},
des muscules\wdx{muscle}{m. terme d'anat. `structure
organique contractile qui assure les mouvements;
muscle'}{muscule} et du
cuir\wdx{cuir}{m. `peau
de l'homme'}{}, et si veons aussi la
anathomie
de pluseurs\wdx{*plusor}{adj. `un certain
nombre'}{pluseurs \emph{pl.}} vaines\wdx{*veine}{f. terme d'anat.
`vaisseau sanguin ou, spécialement, vaisseau sanguin
qui part du foie et qui distribue le sang
nutritif du foie à tout le
corps'}{vaine} et de
pluseurs\wdx{*plusor}{adj. `un certain
nombre'}{pluseurs \emph{pl.}}
nerfz\wdx{nerf}{m. terme d'anat.
`structure blanchâtre en forme de fil qui relie soit
un muscle à un os, soit un centre nerveux
(cerveau, moelle) à un organe ou une structure organique;
tendon ou nerf'}{nerfz \emph{pl.}},
maiemant\wdx{*maismement}{adv. `plus que tout
autre chose; surtout'}{maiemant}
quant a\wdx{a}{prép.
marquant des rapports de direction, de position `à'}{}
la naissance\wdx{naissance}{f. terme d'anat. `endroit
où commence
qch. (en parlant des membres, organes ou structures organiques
du corps)'}{}, selon ce que
traicte\wdx{*traitier}{v.tr. `soumettre qch. à la
pensée en vue d'étudier, d'exposer;
traiter'}{traicte \emph{3.p.sg. ind.prés.}}
Mestre\wdx{*maistre}{m.
3\hoch{o}
appellation
devant le prénom en parlant à ou d'une personne}{Mestre}
Dimus de
\text{Bolongne}\fnb{Zweites \emph{bo}
expungiert.}/\adx{Mestre Mundin}{}{Mestre Dimus de Bologne}
qui sur ce ha escript et ha fait [16r\hoch{o}a] la
anathomie\wdx{*anatomie}{f. 2\hoch{o} `action de disséquer,
de séparer méthodiquement les différentes parties d'un corps
organisé; dissection'}{anathomie}
maintes\wdx{maint}{adj.
`plusieurs; maint'}{} fois\wdx{*foiz}{f. `cas
où un fait se produit, moment du temps où un
événement, conçu comme identique
à d'autres événements, se produit; fois'}{fois}.
Et mon mestre\wdx{*maistre}{m. 2\hoch{o}
`celui qui enseigne'}{mestre},
Mestre\wdx{*maistre}{m.
3\hoch{o} appellation
devant le prénom en parlant à ou d'une
personne}{Mestre} Bertuces\adx{Mestre Bertuces}{}{},
par ceste maniere\wdx{maniere}{f. 2\hoch{o} `forme
particulière
que revêt l'accomplissement d'une action, le
déroulement d'un fait, l'être ou
l'existence'}{}
il asseoit\wdx{asseoir}{v.tr. 1\hoch{o} `mettre
(qn) sur un siège; asseoir'}{asseoit \emph{3.p.sg.
ind.prés.}} ho\emph{m}me\wdx{*ome}{m.
`être appartenant à l'espèce
animale la plus évoluée de la terre; être
humain'}{homme} mort\wdx{mort}{adj.
`qui a cessé de vivre; mort'}{} sur ung banc et en faisoit
quatre lessons\wdx{*leçon}{f. `enseignement donné par
un professeur, à une classe, un auditoire'}{lesson}:
en la p\emph{re}miere lesson\wdx{*leçon}{f.
`enseignement donné par
un professeur, à une classe, un auditoire'}{lesson} il
tractoit\wdx{*traitier}{v.tr.
`soumettre qch. à la
pensée en vue d'étudier, d'exposer; traiter'}{tractoit \emph{3.p.sg. ind.imp.}}
les
me\emph{m}bres
nutritifz\wdx{nutritif}{adj. terme de méd.
`qui a rapport aux esprits qui contrôlent
l'alimentation, le grandissement et la génération de
l'homme'}{nutritifz \emph{m.pl.}},
car\wdx{car}{conj. qui unit à une proposition une
proposition suivante qui donne
la raison de ce qu'affirme la
première}{}
ilz sont
le plus tost\wdx{tost}{adv. `en
un temps bref; rapidement'}{}
putrifiés\wdx{*putrefïer}{v.tr.
`amener à un état de
décomposition, faire pourrir; putréfier'}{putrifié
\emph{p.p.}}. En la seconde, les me\emph{m}bres
espirituelz\wdx{*esperituel}{adj. terme de méd.
`qui contient les esprits vitaux, qui a rapport à la
transformation ou à la diffusion des esprits
vitaux'}{espirituelz
\emph{m.pl.}}. En la tierce,
les me\emph{m}bres qui ont
\text{ame}\fnb{Im Ms. \emph{ama}
zu \emph{ame} korrigiert,
voranstehend gestrichenes \emph{qui}.}/%
\wdx{ame}{f. `principe spirituel de l'homme;
âme'}{}.
En la quarte, il tratoit\wdx{*traitier}{v.tr.
`soumettre qch. à la
pensée en vue d'étudier, d'exposer; traiter'}{tratoit
\emph{3.p.sg. ind.imp.}} des
extremités\wdx{extremité}{f.
`partie extrème qui termine une chose'}{}. Et selon
le
\text{co\emph{m}menteur}\fnb{Ms.
\emph{co}m\emph{men\-seur}; cp. l.\,347
\emph{commenteur} und im Mlt.
GuiChaul\textsc{jl} 20,37 \emph{commentatorem};
es handelt sich um Alexandrin.}/\wdx{*comenteor}{m.
`auteur d'un ensemble des explications,
des remarques faites à propos d'un
texte; commentateur'}{commenteur}\adx{Alexandrin le commenteur}{}{}
du \flq Livre\wdx{livre}{m. `assemblage d'un assez grand nombre de
feuilles, portant des signes destinés à être lus'}{} des
sectes\frq\wdx{secte}{f. `doctrine religieuse ou
philosophique'}{}, en ch\emph{acu}m me\emph{m}bre l'en doit\wdx{devoir}{v.tr. + inf. `être dans
l'obligation de (faire qch.); devoir'}{}
regarder
.ix. choses\wdx{chose}{f. `toute réalité
concrète ou abstraite qu'on désigne d'une
manière déterminé'}{}, c'est
assavoir\wdx{assavoir}{v.tr.}{\textbf{c'est
assavoir} \emph{loc.
`c'est-à-dire'}}
la posicion\wdx{posicïon}{f. `lieu où quelque chose est placée,
située'}{posicion}, la
substance\wdx{sustance}{f. `matière dont un
corps est formé, et en vertu de laquelle
il y a des propriétés particulières;
substance'}{substance},
la complexion\wdx{*complessïon}{f. `ensemble des
éléments constituant la nature physique d'un
individu, d'une partie du corps ou d'une chose;
complexion'}{complexion},
le quantité\wdx{*cantité}{f.
`nombre d'unités ou mesure qui sert à déterminer une
portion de matière ou une collection de choses
considérées comme
homogènes'}{quantité}, le nombre\wdx{nombre}{m.
`mot servant à caractériser une pluralité de
choses ou de personnes; nombre'}{}, les
signes\wdx{signe}{m. `chose perçu qui permet de
conclure à l'existence ou à la vérité (d'une autre
chose, à laquelle elle est liée); signe'}{} et
colligances\wdx{colligance}{f.
1\hoch{o}
`force qui maintient réunis les éléments d'un système matériel;
liaison'}{}, les
accions\wdx{accion}{f. `tout ce que l'on fait (dit aussi de choses)'}{} et les
utilité\wdx{utilité}{f.
`caractère
de ce qui est utile; utilité'}{}. Et
quelles sont les maladies\wdx{maladie}{f.
 `altération organique ou
fonctionnelle considérée dans son évolution, et comme
une entité définissable; maladie'}{} qui
peullent\wdx{*pooir}{v.tr.  +
inf. `avoir la possibilité de (faire qch.);
pouvoir'}{peullent \emph{3.p.pl. ind.prés.}}
avenir\wdx{avenir}{v.tr.indir. `venir ou être sur le
point d'être; arriver'}{} ou dit
me\emph{m}bre,
\text{par}\fnb{L. \emph{pour}\,?}/
lesqueulx choses\wdx{chose}{f. `toute réalité
concrète ou abstraite qu'on désigne d'une
manière déterminé'}{} le medicin\wdx{*medecin}{m.
1\hoch{o}
`personne habilitée à
exercer la médecine'}{medicin}
peult\wdx{*pooir}{v.tr.  +
inf. `avoir la possibilité de (faire qch.);
pouvoir'}{peult \emph{3.p.sg. ind.prés.}}
estre aidiez\wdx{aidier}{v.tr.
`appuyer (qn ou qch.) en apportant son
aide'}{aidiez \emph{p.p. m.sg.}}
par
anathomie en cognoissant\wdx{*conoistre}{v.tr.
1\hoch{o}
`avoir une idée de (qch.); connaître'}{cognoissant
\emph{p.prés.}}, en
\text{pronosticant}\fnb{Zweites \emph{o} über der Zeile
nachgetragen.}/\wdx{pronostiquer}{v.tr. `émettre un
pronostic au sujet de l'évolution d'une maladie, de sa
gravité; pronostiquer'}{pronosticant \emph{p.prés.}}
et en curant\wdx{curer}{v.tr.
`soumettre à un traitement médical'}{}. Et aussi nous veons
par experience\wdx{*esperïence}{f.
`le fait
d'éprouver qch., considéré comme un
élargissement ou un enrichissement de la
connaissance; expérience'}{experience}
en corps\wdx{cors}{m. `ce qui fait l'existence
matérielle d'un
homme ou d'un animal'}{corps} qui sont
sechiés\wdx{sechier}{v.tr. `rendre sec'}{}
ou soleil\wdx{soleil}{m. `astre qui donne lumière et
chaleur à la terre et rythme la vie à sa surface;
soleil'}{} et
degastés\wdx{degaster}{v.tr. `mettre
une chose en mauvais état, de sorte qu'elle ne puisse
plus servir; détériorer'}{}
en terre\wdx{terre}{f. `matière qui forme la
couche superficielle de la croûte terrestre;
terre'}{}, ou qui sont remis\wdx{remetre}{v.tr.
`faire passer (qch.) de nouveau dans son ancien
état, son ancienne place'}{} ou
fondus\wdx{fondre}{v.tr. `détruire (qch.)'}{fondu
\emph{p.p.}}
en aigue\wdx{aigue}{f. `liquide incolore, inodore et
transparent; eau'}{} coura\emph{n}t\wdx{*corir}{v.intr.
`couler' (d'un liquide)}{courant \emph{p.prés.}} ou
boillant\wdx{*bolir}{v.intr.
`être en
ébullition, s'agiter en
formant des bulles sous l'action de la chaleur'}{boillant \emph{p.prés.}}:
nous
voions\wdx{*vëoir}{v.tr. `percevoir (qch.) par le
sens de la vue'}{voions \emph{1.p.pl. ind.prés.}} la
anathomie,
au moins\wdx{moins}{adv.}{\textbf{au moins}
\emph{loc.adv. qui sert
à marquer une restriction}}
des
os\wdx{os}{m. `chacune des pièces rigides qui constituent le
squelette de l'homme et des animaux vertébrés'}{},
des cartillaiges\wdx{*cartilage}{m. terme d'anat. `variété de tissu conjonctif,
translucide, résistant mais élastique, ne contenant ni vaisseaux
ni nerfs, qui recouvre les surfaces osseuses des articulations et qui constitue la charpente
de certaines organes et le squelette de certains vertébrés inférieurs;
cartilage'}{cartillaige},
des joinctures\wdx{jointure}{f.
`partie du corps formée par la jointure entre
deux ou plusieurs os; articulation'}{joincture},
des gros\wdx{gros}{adj. 1\hoch{o}
dans l'ordre physique, quantifiable `qui, dans
son genre, dépasse le volume ordinaire; gros (du corps
humain et de ses parties)'}{} nerfz\wdx{nerf}{m. terme d'anat.
`structure blanchâtre en forme de fil qui relie soit
un muscle à un os, soit un centre nerveux
(cerveau, moelle) à un organe ou une structure organique;
tendon ou nerf'}{nerfz
\emph{pl.}}, des
tenans\wdx{*tendant}{m. terme d'anat. `structure conjonctive fibreuse par laquelle
un muscle s'insère sur un os'}{tenans \emph{pl.}} et des
colligans\wdx{colligance}{f. 2\hoch{o}
terme d'anat. `faisceau de tissu blanchâtre, résistant et peu extensible, unissant les éléments
d'une articulation ou maintenant en place un organe ou une partie d'un
organe; ligament'}{colligans \emph{pl.}}. Et
par ceste maniere\wdx{maniere}{f. 2\hoch{o} `forme
particulière
que revêt l'accomplissement d'une action, le
déroulement d'un fait, l'être ou
l'existence'}{}
ci, Galien\adx{Galien}{}{} eust la
cognoissance\wdx{*conoissance}{f.
`le fait de connaître'}{cognoissance}
de anathomie en
corps
des ho\emph{m}mes\wdx{*ome}{m. `être appartenant à l'espèce
animale
la plus évoluée de la terre; être humain'}{homme}, de
singes\wdx{singe}{m. `mammifère primate,
caractérisé par une face nue, un cerveau développé
et des membres inférieurs plus petits que les
membres supérieurs; singe'}{}, de
pourceaux\wdx{*porcel}{m.
`mammifère ongulé omnivore, dont la tête est terminée
par un groin et qui est domestiqué et élevé pour sa
chair; cochon'}{pourceaux \emph{pl.}} et d'aultres\wdx{autre}{adj. `qui n'est
pas le même; autre'}{aultre}
[16r\hoch{o}b]
bestes\wdx{beste}{f. `être vivant non végétal et non
humain; animal'}{}. Et ne vint pas a\wdx{a}{prép.
marquant des rapports de direction, de position `à'}{}
la
cognoissance\wdx{*conoissance}{f.
`le fait de connaître'}{cognoissance}
par poincture\wdx{*peinture}{f.
`représentation, suggestion du monde
visible ou imaginaire sur une surface plane
au moyen de couleurs'}{poincture}, ainsi
q\emph{ue}\wdx{ainsi}{adv. `de cette façon'}{\textbf{ainsi que}
\emph{loc.conj.
`de la même façon que'}}
fit le dit
Henri\adx{Henri de Mondeville}{}{}, lequel se
efforsa\wdx{*esforcier}{v.pron. `faire tous
ses efforts, employer toute force, son
adresse ou son intelligence en vue de qch.'}{efforsa
\emph{3.p.sg. passé
simple}} a demonstrer\wdx{demonstrer}{v.tr. `faire
voir, mettre devant les yeux; montrer'}{} la anathomie par .xiij.
poinctures\wdx{*peinture}{f.
`représentation, suggestion du monde
visible ou imaginaire sur une surface plane
au moyen de couleurs'}{poincture} ou
\text{picures}\fnb{Ms.
\emph{pictures} mit expungiertem
\emph{t}.}/\wdx{*peinture}{f.
`représentation, suggestion du monde
visible ou imaginaire sur une surface plane
au moyen de couleurs'}{picure},
c'est a dire\wdx{dire}{v.tr.
`lire à haute voix;
réciter'}{\textbf{c'est a dire} \emph{loc.conj. qui
annonce une explication ou une précision}}
par
.xiij.
ymages\wdx{*image}{f. 1\hoch{o} `reproduction
artisanale d'une personne, d'un
objet'}{ymage}.
\pend
\pstart
Donc il appert\wdx{*aparoir}{v.intr. `se montrer aux yeux; se
manifester'}{appert \emph{3.p.sg. ind.prés.}} par les
choses\wdx{chose}{f. `toute réalité
concrète ou abstraite qu'on désigne d'une
manière déterminé'}{} devant\wdx{devant}{adv.
2\hoch{o} qui
marque la priorité dans le temps `auparavant'}{}
dictes que corps humain\wdx{*umain}{adj.
`qui appartient ou qui
est propre à l'homme; humain'}{humain}
est une chose\wdx{chose}{f. `toute réalité
concrète ou abstraite qu'on désigne d'une
manière déterminé'}{}
aourné\wdx{*aorner}{v.tr. `rendre beau;
orner'}{aourné \emph{p.p.}} par raison\wdx{raison}{f.
1\hoch{o} `ce qui permet d'agir conformément à
des principes'}{\textbf{par raison} \emph{`en
consultant la raison'}}, de pluseurs\wdx{*plusor}{adj. `un certain
nombre'}{pluseurs \emph{pl.}}
et divers\wdx{divers}{adj.
`qui présente une différence par rapport à une autre
chose ou une autre personne; différent'}{divers
\emph{m.pl.}} me\emph{m}bres
et p\emph{ar}ties\wdx{partie}{f.
`élément d'un tout;
partie'}{}
composte\wdx{compost}{adj. terme d'anat. `qui est
formé
de plusieurs membres simples (dit d'un
membre du corps)'}{}.
Mais le me\emph{m}bre ou la partie\wdx{partie}{f.
`élément d'un tout;
partie'}{}, selon Galien\adx{Galien}{}{} ou
p\emph{re}mier livre\wdx{livre}{m. `assemblage d'un assez grand nombre
de feuilles, portant des signes destinés à être lus'}{} qui se
intitule\wdx{intituler}{v.pron. `avoir
pour titre'}{}
\flq De utilitate
particula\emph{rum}\frq ,
c'est ung corps qui n'est pas du
tout separé\wdx{separer}{v.tr. `mettre à part les unes
des autres des choses, des personnes
réunies; séparer'}{} ne
co\emph{n}joing\wdx{conjoindre}{v.tr. `mettre des choses
ensemble de façon qu'elles se touchent ou tiennent
ensemble; joindre'}{conjoing \emph{p.p. m.sg.}}
a\wdx{a}{prép.
marquant des rapports de direction, de position `à'}{}
aultre\wdx{autre}{adj. `qui n'est
pas le même; autre'}{aultre}.
Et est escript
ou dit livre\wdx{livre}{m. `assemblage d'un assez grand nombre
de feuilles, portant des signes destinés à être lus'}{} que
aucunes d'icelles
parties\wdx{partie}{f.
`élément d'un tout;
partie'}{}
sont grandes\wdx{grant}{adj. 1\hoch{o}
dans l'ordre
physique, quantifiable `qui est d'une extension
au-dessus de la moyenne; grand (des
choses)'}{grandes \emph{f.pl.}}
et aucunes sont moindres\wdx{*menor}{adj. `qui est
plus petit en étendue, en quantité, en qualité;
moindre'}{moindre}
et divisibles\wdx{divisible}{adj. `qui peut être
divisé; divisible'}{}
en autres\wdx{autre}{adj. `qui n'est
pas le même; autre'}{}
especes\wdx{espece}{f. `classe (de choses) définie par un
ensemble particulier de caractères communs'}{}.
Et c'est ce que disoit Avicene\adx{Avicene}{}{} par autres\wdx{autre}{adj. `qui n'est
pas le même; autre'}{}
parolles\wdx{parole}{f.
`ensemble de mots
qui expriment une idée'}{parolle}
en son \flq Canon\frq\ ou premier livre qui dit: les  me\emph{m}bres
sont corps qui sont engendrés\wdx{engendrer}{v.tr.
au fig. `faire naître, faire exister; produire'}{} de la
premiere co\emph{m}mixtion\wdx{*comistïon}{f. `mélange de
plusieurs choses différents'}{commixtion}
des humeurs\wdx{humeur}{f. 2\hoch{o} terme de méd. `dans l'humorisme,
un des quatre liquides du corps qui gouvernent son
équilibre (le sang, le flegme, la bile,
l'atrabile)'}{}.
Et ap\emph{ré}s il dit q\emph{ue}, des me\emph{m}bres, les aucuns sont
simples\wdx{simple}{adj.
terme d'anat.
`qui n'est pas
composé de plusieurs parties à distinguer
(dit d'un membre du
corps considéré comme un tout)'}{} et les
aultres\wdx{autre}{adj.
`qui n'est pas le même; autre'}{aultre}
co\emph{m}post\wdx{compost}{adj. terme d'anat. `qui est
formé
de plusieurs membres simples (dit d'un
membre du corps)'}{compost \emph{m.pl.}}, en
parlant de simple\wdx{simple}{adj.
terme d'anat.
`qui n'est pas
composé de plusieurs parties à distinguer
(dit d'un membre du
corps considéré comme un tout)'}{}
et de compost\wdx{compost}{adj.
terme d'anat.
`qui est
formé
de plusieurs membres simples (dit d'un
membre du corps)'}{}
largemant\wdx{*largement}{adv. `au sens
large'}{largemant} selon la
co\emph{n}sideracion\wdx{consideracïon}{f. `action
d'examiner ou d'observer (qch.) avec attention, le
résultat de cette action'}{} du
miedicin\wdx{*medecin}{m. 1\hoch{o}
`personne habilitée à
exercer la médecine'}{miedicin}.
\pend
\pstart
Les simples\wdx{simple}{adj.
terme d'anat.
`qui n'est pas
composé de plusieurs parties à distinguer
(dit d'un membre du
corps considéré comme un tout)'}{},
ce sont les me\emph{m}bres
co\emph{n}similes\wdx{consimile}{adj.
`dont les éléments constitutifs sont de même nature
ou répartis de façon uniforme; homogène'}{} qui ne peuent\wdx{*pooir}{v.tr.  +
inf. `avoir la possibilité de (faire qch.);
pouvoir'}{peuent \emph{3.p.pl. ind.prés.}}
estre
divisez\wdx{deviser}{v.tr. `séparer (qch.) en
plusieurs parties; diviser'}{divisé \emph{p.p.}} en
autres\wdx{autre}{adj. `qui n'est
pas le même; autre'}{}
especes\wdx{espece}{f. `classe (de choses) définie par un
ensemble particulier de caractères communs'}{}, car,
quelco\emph{n}ques partie\wdx{partie}{f.
`élément d'un tout;
partie'}{}
que tu
en prendras, elle aura le \text{nom}\fnb{Voranstehend
expungiertes
\emph{mor}.}/\wdx{nom}{m.
`mot
servant à désigner les êtres, les choses
qui appartiennent à une même catégorie
logique'}{} et la
raison\wdx{raison}{f.
1\hoch{o} `ce qui permet d'agir conformément à
des principes'}{}
[16v\hoch{o}a]
du tout. Et sont .xiij. telz\wdx{tel}{adj. `qui
est
semblable, du même genre; tel'}{telz \emph{m.pl.}}
me\emph{m}bres, c'est
assavoir\wdx{assavoir}{v.tr.}{\textbf{c'est
assavoir} \emph{loc.
`c'est-à-dire'}}
le cartillage\wdx{*cartilage}{m. terme d'anat. `variété de tissu conjonctif,
translucide, résistant mais élastique, ne contenant ni vaisseaux
ni nerfs, qui recouvre les surfaces osseuses des articulations et qui constitue la charpente
de certaines organes et le squelette de certains vertébrés inférieurs;
cartilage'}{cartillage}, les
os\wdx{os}{m. `chacune des pièces rigides qui constituent le
squelette de l'homme et des animaux vertébrés'}{},
les nerfz,
les vaines\wdx{*veine}{f. terme d'anat. `vaisseau
sanguin ou, spécialement, vaisseau sanguin
qui part du foie et qui distribue le sang
nutritif du foie à tout le
corps'}{vaine}, les
arteres\wdx{artere}{f. terme d'anat.
 `vaisseau sanguin qui part du c\oe ur et
qui distribue le sang, qui contient les esprits
vitaux,
à tout le corps'}{}, les
panicles\wdx{*pannicle}{m. terme d'anat.
`couche de tissu musculaire ou cellulaire qui recouvre
une structure organique du corps humain (un
organe, un os, une articulation, un
muscle, etc.)'}{panicle}, les
liguemans\wdx{*liguement}{m. terme d'anat.
`faisceau de tissu fibreux blanchâtre,
résistant et peu extensible, unissant les éléments
d'une articulation ou maintenant en place un organe
ou une partie d'un organe; ligament'}{liguemans \emph{pl.}},
les cordes\wdx{corde}{f. terme d'anat. `structure
conjonctive fibreuse par laquelle un muscle
s'insère sur un os; tendon'}{}, le
cuir\wdx{cuir}{m.
`peau de l'homme'}{} et
la char\wdx{char}{f.
`substance somatique de consistance relativement
molle
(surtout par opposition au tissu osseux et sa
dureté); chair'}{}. Et avec yceulx
on\wdx{on}{pron.pers. indéfini 3\hoch{e} personne}{}
y adjouste\wdx{*ajoster}{v.tr. `mettre
avec (qch.)'}{adjouste \emph{3.p.sg. ind.prés.}}
la graisse\wdx{*craisse}{f. `substance onctueuse, de fusion facile,
répartie en diverses parties du corps de l'homme et des mammifères;
graisse'}{graisse}, le
poil\wdx{poil}{m. `chacune des productions filiformes qui naissent
du tégument de l'homme et des mammifères; poil'}{}
et les ungles\wdx{ongle}{m. et f. `lame cornée, implantée sur l'extremité dorsale
des doigts et des orteils de l'homme;
ongle'}{ungle}, lesqueulx,
ja soit ce qu\wdx{ja soit ce que}{loc.conj.
`bien que assurément'}{}'ilz ne soient
pas me\emph{m}bres p\emph{ro}pres\wdx{propre}{adj. 2\hoch{o} `qui
est particulier à (qn, qch.)'}{},
\text{ains ce sont
sup\emph{er}fluités}\lemma{ains{\dots} superfluités}\fnb{Über
der Zeile
nachgetragen.}/\wdx{*ainz}{conj. `plutôt, de
préférence'}{ains}\wdx{superfluité}{f. terme de méd.
`sécrétion abondante du corps'}{},
toutevoies\wdx{*totes voies}{loc.adv.
`en
considérant toutes les raisons, toutes les
circonstances qui pourraient s'y opposer, et
malgré elles; toutefois'}{toutevoies}
il ont
une utilité\wdx{utilité}{f.
`caractère
de ce qui est utile; utilité'}{}
et regeneracion\wdx{regeneracïon}{f.
terme de méd. `reconstitution d'une partie du
corps
humain détruite naturellement ou
accidentellement'}{regeneracion} ainsi
q\emph{ue}\wdx{ainsi}{adv. `de cette façon'}{\textbf{ainsi que}
\emph{loc.conj.
`de la même façon que'}}
les autres\wdx{autre}{adj. `qui n'est
pas le même; autre'}{}
me\emph{m}bres,
si co\emph{m}me il est escript ou second de \flq Tegny\frq . Et de ces
simples\wdx{simple}{adj.
terme d'anat.
`qui n'est pas
composé de plusieurs parties à distinguer
(dit d'un membre du
corps considéré comme un tout)'}{} me\emph{m}bres,
les aucuns sont
sanguins\wdx{sanguin}{adj. 1\hoch{o}
`qui a rapport au, qui contient du sang'}{},
desqueulx peult\wdx{*pooir}{v.tr.  +
inf. `avoir la possibilité de (faire qch.);
pouvoir'}{peult \emph{3.p.sg. ind.prés.}}
estre faicte
vraie\wdx{*verai}{adj. `qui présente un caractère de
vérité; vrai'}{vraie \emph{f.sg.}}
generacion\wdx{generacïon}{f. `action ou faculté
d'engendrer'}{generacion} et
co\emph{n}solidacion\wdx{consolidacïon}{f.
terme de méd. `processus organique de rapprochement
et de soudure par lequel est réparé un état
pathologique d'une partie du corps (une lésion,
une plaie, une fracture, etc.)'}{consolidacion},
car\wdx{car}{conj. qui
unit à une proposition une
proposition suivante qui donne
la raison de ce qu'affirme la
première}{}
y sont engendrés\wdx{engendrer}{v.tr. au fig. `faire
naître, faire exister; produire'}{}
de sang\wdx{*sanc}{m.
terme de méd.
`liquide visqueux, de couleur
rouge, qui est porté par les vaisseaux dans tout
l'organisme où il joue des rôles multiples (l'une
des quatre humeurs de l'humorisme)'}{sang}, si
co\emph{m}me la
char\wdx{char}{f.
`substance somatique de consistance relativement
molle
(surtout par opposition au tissu osseux et sa
dureté); chair'}{} et la graisse\wdx{*craisse}{f. `substance onctueuse, de fusion facile,
répartie en diverses parties du corps de l'homme et des mammifères;
graisse'}{graisse}.
Les
aucuns sont spermatiques\wdx{spermatique}{adj.
terme de méd. `qui a rapport au sperme'}{},
car\wdx{car}{conj. qui
unit à une proposition une
proposition suivante qui donne
la raison de ce qu'affirme la
première}{}
ilz ont leur naissance\wdx{naissance}{f.
terme d'anat. `endroit où commence
qch. (en parlant des membres, organes ou structures organiques
du corps)'}{} du
sp\emph{er}me\wdx{esperme}{m. terme de méd.
`semence, cellule ou groupe de cellules
dont se forme un organisme'}{sperme},
qui ne se peuvent\wdx{*pooir}{v.tr. +
inf. `avoir la possibilité de (faire qch.);
pouvoir'}{peuvent \emph{3.p.pl. ind.prés.}}
regendrer\wdx{regendrer}{v.pron.
terme de méd. `se reconstituer (en parlant d'une
partie du corps humain détruite naturellement ou
accidentellement)'}{} ne
co\emph{n}solider\wdx{consolider}{v.pron.
terme de méd. `se fermer (en parlant d'une
lésion, d'une plaie, d'une fracture, etc.)'}{} par
vray\wdx{*verai}{adj. `qui présente un caractère de
vérité; vrai'}{vray \emph{f.sg.}}
consolidacion\wdx{consolidacïon}{f.
terme de méd. `processus organique de rapprochement
et de soudure par lequel est réparé un état
pathologique d'une partie du corps (une lésion,
une plaie, une fracture, etc.)'}{consolidacion},
si co\emph{m}me les \text{os}\fnb{Zweites
\emph{os} gestrichen.}/\wdx{os}{m. `chacune des pièces rigides qui
constituent le squelette de l'homme et des animaux vertébrés'}{}
et les aultres\wdx{autre}{adj. `qui n'est
pas le même; autre'}{aultre}
me\emph{m}bres
devant\wdx{devant}{adv. 2\hoch{o} qui marque la
priorité dans
le temps `auparavant'}{} dis. Item\wdx{item}{adv.
terme d'admin. `et de même, et aussi (introduction d'unités
traitées l'une après l'autre dans un traité, une ordonnance,
un doc. coutumier, etc.)'}{}, des me\emph{m}bres
simples\wdx{simple}{adj.
terme d'anat.
`qui n'est pas
composé de plusieurs parties à distinguer
(dit d'un membre du
corps considéré comme un tout)'}{},
aucuns sont chaulx\wdx{*chaut}{adj. terme de méd.
désignant une des qualités des quatre humeurs, celle qui gouverne
essentiellement l'équilibre du sang et de la bile
`chaut (comme terme de l'humorisme)'}{chaulx
\emph{m.pl.}} et moites\wdx{moiste}{adj. terme de méd.
désignant une des qualités des quatre humeurs, celle qui gouverne
essentiellement l'équilibre du sang et du flegme
`moite (comme terme de l'humorisme)'}{moites
\emph{m.pl.}}, aucuns frois\wdx{froit}{adj. terme de
méd.
désignant une des qualités des quatre humeurs, celle qui gouverne
essentiellement l'équilibre du flegme et de la mélancolie
`froit (comme terme de l'humorisme)'}{frois \emph{m.pl.}} et
moites\wdx{moiste}{adj. terme de méd.
désignant une des qualités des quatre humeurs, celle qui gouverne
essentiellement l'équilibre du sang et du flegme
`moite (comme terme de l'humorisme)'}{moites \emph{m.pl.}},
aucuns frois\wdx{froit}{adj. terme de méd.
désignant une des qualités des quatre humeurs, celle qui gouverne
essentiellement l'équilibre du flegme et de la mélancolie
`froit (comme terme de l'humorisme)'}{frois
\emph{m.pl.}}
et secs\wdx{sec}{adj. terme de méd.
désignant une des qualités des quatre humeurs, celle qui gouverne
essentiellement l'équilibre de la bile et de la mélancolie
`sec (comme terme de l'humorisme)'}{}. Mais nul me\emph{m}bre n'est
chault\wdx{*chaut}{adj. terme de méd.
désignant une des qualités des quatre humeurs, celle qui gouverne
essentiellement l'équilibre du sang et de la bile
`chaut (comme terme de l'humorisme)'}{chault \emph{m.sg.}} et
sec\wdx{sec}{adj. terme de méd.
désignant une des qualités des quatre humeurs, celle qui gouverne
essentiellement l'équilibre de la bile et de la mélancolie
`sec (comme terme de l'humorisme)'}{}, car, oultre la
nature\wdx{nature}{f.
1\hoch{o} `ensemble des
caractères, des propriétés qui définissent un être,
une chose concrète ou abstraite'}{}
du cuir\wdx{cuir}{m. `peau
de l'homme'}{} auquel tous les
\text{autres}\fnb{Über der Zeile nachgetragen.}/\wdx{autre}{adj. `qui n'est
pas le même; autre'}{}
me\emph{m}bres font\wdx{faire}{v.tr.
`réaliser
ou effectuer (qch.)'}{}
leur operacion\wdx{operacïon}{f.
`action d'un pouvoir, d'une fonction,
d'un organe qui produit un effet selon sa
nature; opération'}{operacion}, on ne
treuve\wdx{*trover}{v.tr. `rencontrer qn ou qch. qu'on cherche;
trouver'}{treuve \emph{3.p.sg. ind.prés.}}
nul me\emph{m}bre simple\wdx{simple}{adj.
terme d'anat.
`qui n'est pas
composé de plusieurs parties à distinguer
(dit d'un membre du
corps considéré comme un tout)'}{} qui
soit plus chault\wdx{*chaut}{adj. terme de méd.
désignant une des qualités des quatre humeurs, celle qui gouverne
essentiellement l'équilibre du sang et de la bile
`chaut (comme terme de l'humorisme)'}{chault
\emph{m.sg.}} et
plus sec\wdx{sec}{adj. terme de méd.
désignant une des qualités des quatre humeurs, celle qui gouverne
essentiellement l'équilibre de la bile et de la mélancolie
`sec (comme terme de l'humorisme)'}{} du cuir\wdx{cuir}{m. `peau
de l'homme'}{}.
Car le cuir\wdx{cuir}{m. `peau
de l'homme'}{} si est moien\wdx{moien}{m.
2\hoch{o} `limite extérieure d'une chose'}{}, non
pas
seulema\emph{n}t\wdx{*seulement}{adv. `sans rien d'autre que
ce qui est mentionné; seulement'}{seulemant}
des parties\wdx{partie}{f.
`élément d'un tout;
partie'}{}
humaines, mais aussi de toute
la substance\wdx{sustance}{f. `matière dont un
corps est formé, et en vertu de laquelle
il y a des propriétés particulières;
substance'}{substance} des
choses\wdx{chose}{f. `toute réalité
concrète ou abstraite qu'on désigne d'une
manière déterminé'}{} [16v\hoch{o}b] qui
se engendrent\wdx{engendrer}{v.pron. au fig.
`naître; se produire'}{}
et qui se
\text{corro\emph{m}pe\emph{n}t}\fnb{\emph{ro}m über der Zeile
nachgetragen.}/\wdx{corrompre}{v.pron.
`s'altérer par décomposition; se
pourrir'}{},
selon Galien\adx{Galien}{}{} ou premier livre des
\flq Co\emph{m}plexions\frq\wdx{*complessïon}{f.
`ensemble des
éléments constituant la nature physique d'un
individu, d'une partie du corps ou d'une chose;
complexion'}{complexion}, ou
darrier\wdx{derrier}{adj. `qui vient après tous
les autres, après lequel il n'y a pas
d'autre'}{darrier}
chappitre\wdx{chapitre}{m. `chacune des
parties qui se suivent dans un livre et en articulent la lecture;
chapitre'}{chappitre}.
Les me\emph{m}bres chaulx\wdx{*chaut}{adj. terme de méd.
désignant une des qualités des quatre humeurs, celle qui gouverne
essentiellement l'équilibre du sang et de la bile
`chaut (comme terme de l'humorisme)'}{chaulx
\emph{m.pl.}} et moites\wdx{moiste}{adj. terme de méd.
désignant une des qualités des quatre humeurs, celle qui gouverne
essentiellement l'équilibre du sang et du flegme
`moite (comme terme de l'humorisme)'}{moites
\emph{m.pl.}}, c'est le sang\wdx{*sanc}{m.
terme de méd.
`liquide visqueux, de couleur
rouge, qui est porté par les vaisseaux dans tout
l'organisme où il joue des rôles multiples (l'une
des quatre humeurs de l'humorisme)'}{sang}, au moins\wdx{moins}{adv.}{\textbf{au moins}
\emph{loc.adv. qui sert
à marquer une restriction}}
materiellema\emph{n}t\wdx{*materïelement}{adv. `par rapport
à la matière'}{materiellemant}
les esperiz\wdx{esperit}{m. terme de méd.
`ensemble de corpuscules subtils qui assurent
toutes les fonctions de la vie dans
l'organisme humain'}{esperiz
\emph{pl.}}, la cher\wdx{char}{f.
`substance somatique de consistance relativement
molle
(surtout par opposition au tissu osseux et sa
dureté); chair'}{cher};
et les h\emph{u}miditez\wdx{*umidité}{f.
`dans l'humorisme, qualité qui gouverne essentiellement
l'équilibre du sang et du flegme'}{humidité} vont par icelle
voie\wdx{voie}{f. 1\hoch{o} au fig. `conduite, suite
d'actes orientés vers une fin, considérée comme
un chemin que l'on peut suivre'}{},
co\emph{m}me dit Averrois\adx{Averrois}{}{} ou second de son
\flq Colliget\frq . Mais le fleume\wdx{fleume}{m. terme de
méd. `lymphe (l'une des quatre humeurs de
l'humorisme)'}{}, la
graisse\wdx{*craisse}{f. `substance onctueuse, de fusion facile,
répartie en diverses parties du corps de l'homme et des mammifères;
graisse'}{graisse} et les
medules\wdx{medulle}{f.
terme d'anat.
`substance moelleuse de l'intérieur d'une
structure osseuse'}{medule}
sont froides\wdx{froit}{adj. terme de méd.
désignant une des qualités des quatre humeurs, celle qui gouverne
essentiellement l'équilibre du flegme et de la mélancolie
`froit (comme terme de l'humorisme)'}{froides \emph{f.pl.}} et
moites\wdx{moiste}{adj. terme de méd.
désignant une des qualités des quatre humeurs, celle qui gouverne
essentiellement l'équilibre du sang et du flegme
`moite (comme terme de l'humorisme)'}{moites \emph{f.pl.}}. Mais
les aultres me\emph{m}bres
\text{sont froiz}\fnb{Ms. \emph{sont sont
froiz}.}/\wdx{froit}{adj. terme de méd.
désignant une des qualités des quatre humeurs, celle qui gouverne
essentiellement l'équilibre du flegme et de la mélancolie
`froit (comme terme de l'humorisme)'}{froiz \emph{m.pl.}} et
secs\wdx{sec}{adj. terme de méd.
désignant une des qualités des quatre humeurs, celle qui gouverne
essentiellement l'équilibre de la bile et de la mélancolie
`sec (comme terme de l'humorisme)'}{}
selon leur degré\wdx{degré}{m. 2\hoch{o} terme de méd.
`dans la
doctrine des quatre humeurs, un des quatre niveaux
d'intensité des quatre qualités (froid, chaud,
sec, humide)'}{},
si co\emph{m}me l'os\wdx{os}{m. `chacune des pièces rigides qui constituent
le squelette de l'homme et des animaux vertébrés'}{}, le
cartillage\wdx{*cartilage}{m. terme d'anat. `variété de tissu conjonctif,
translucide, résistant mais élastique, ne contenant ni vaisseaux
ni nerfs, qui recouvre les surfaces osseuses des articulations et qui constitue la charpente
de certaines organes et le squelette de certains vertébrés inférieurs;
cartilage'}{cartillage}, le
poil\wdx{poil}{m. `chacune des productions filiformes qui naissent
du tégument de l'homme et des mammifères; poil'}{}, les
cordes\wdx{corde}{f. terme d'anat. `structure
conjonctive fibreuse par laquelle un muscle
s'insère sur un os; tendon'}{},
les liguemans\wdx{*liguement}{m. terme d'anat.
`faisceau de tissu fibreux blanchâtre,
résistant et peu extensible, unissant les éléments
d'une articulation ou maintenant en place un organe
ou une partie d'un organe; ligament'}{liguemans \emph{pl.}},
les nerfz\wdx{nerf}{m. terme d'anat.
`structure blanchâtre en forme de fil qui relie soit
un muscle à un os, soit un centre nerveux
(cerveau, moelle) à un organe ou une structure organique;
tendon ou nerf'}{nerfz \emph{pl.}},
les
voines\wdx{*veine}{f. terme d'anat. `vaisseau
sanguin ou, spécialement, vaisseau sanguin
qui part du foie et qui distribue le sang
nutritif du foie à tout le
corps'}{voine}
et les panicles\wdx{*pannicle}{m. terme d'anat.
`couche de tissu musculaire ou cellulaire qui recouvre
une structure organique du corps humain (un
organe, un os, une articulation, un
muscle, etc.)'}{panicle}.
Et yci, c'est une mer\wdx{mer}{f. `vaste étendue
d'eau salée qui couvre une grande partie de la
surface du globe; mer'}{} en laquelle le
medicin\wdx{*medecin}{m. 1\hoch{o}
`personne habilitée à
exercer la médecine'}{medicin}
ne doit\wdx{devoir}{v.tr. + inf. `être dans
l'obligation de (faire qch.); devoir'}{}
point\wdx{point}{m. `endroit fixé et déterminé (où qch. à
lieu)'}{\textbf{ne{\dots} point} \emph{adv. de la négation `ne{\dots}
pas'}}
nagier\wdx{nagier}{v.intr. `se déplacer sur
l'eau; naviguer'}{} ne
navier\wdx{naviier}{v.intr. `se déplacer sur
l'eau; naviguer'}{navier},
car c'est chose\wdx{chose}{f. `toute réalité
concrète ou abstraite qu'on désigne d'une
manière déterminé'}{}
co\emph{n}venable\wdx{*covenable}{adj.
`qui est approprié'}{convenable} au
miege\wdx{miege}{m. `personne habilitée à exercer
la médecine; médecin'}{} de revoir la
co\emph{m}plexion\wdx{*complessïon}{f.
`ensemble des
éléments constituant la nature physique d'un
individu, d'une partie du corps ou d'une chose;
complexion'}{complexion}
des dis me\emph{m}bres du
philosophe\wdx{*filosofe}{m. `personne
qui s'adonne à l'étude rationnelle de la
nature et de la morale'}{philosophe}
naturel\wdx{naturel}{adj. 2\hoch{o} `qui est
adepte du naturalisme; naturaliste'}{}.
\pend
\pstart
Les me\emph{m}bres compos\wdx{compost}{adj.
terme d'anat.
`qui est
formé
de plusieurs membres simples (dit d'un
membre du corps)'}{compos \emph{m.pl.}},
ce sont ceulx qui sont co\emph{m}posés\wdx{composer}{v.tr.
`former par l'assemblage, par la combinaison de
parties'}{composé \emph{p.p.}}
des dis me\emph{m}bres simples\wdx{simple}{adj.
terme d'anat.
`qui n'est pas
composé de plusieurs parties à distinguer
(dit d'un membre du
corps considéré comme un tout)'}{}
\text{ou}\fnb{Ms. \emph{au}.}/
co\emph{n}similes\wdx{consimile}{adj.
`dont les éléments constitutifs sont de même nature
ou répartis de façon uniforme; homogène'}{}, et pour
ce il sont
etherogenees\wdx{*heterogené}{adj. `qui est composé
d'éléments de nature différente'}{etherogené},
car on les peult\wdx{*pooir}{v.tr.  +
inf. `avoir la possibilité de (faire qch.);
pouvoir'}{peult \emph{3.p.sg. ind.prés.}}
diviser\wdx{deviser}{v.tr. `séparer
(qch.) en plusieurs parties; diviser'}{diviser
\emph{inf.}} en autres
especes\wdx{espece}{f. `classe (de choses) définie par un
ensemble particulier de caractères communs'}{},
si co\emph{m}me en me\emph{m}bres
\text{\emph{con}similes}\fnb{Am
Zeilenrand nachgetragen.}/\wdx{consimile}{adj.
`dont les éléments constitutifs sont de même nature
ou répartis de façon uniforme; homogène'}{}.
Et chescune p\emph{ar}tie de yceulx ne
garde\wdx{garder}{v.tr. 2\hoch{o}
`continuer à avoir qch.; conserver'}{}
pas le nom\wdx{nom}{m.
`mot
servant à désigner les êtres, les choses
qui appartiennent à une même catégorie
logique'}{} ou la raison\wdx{raison}{f.
1\hoch{o} `ce qui permet d'agir conformément à
des principes'}{}
du tout. Et les
appelle\wdx{apeler}{v.tr. `donner un nom à qn, qch.; appeler'}{appelle
\emph{3.p.sg. ind.prés.}}
l'en
organiques\wdx{organique}{adj. `qui sert d'instrument
(pour mettre en \oe uvre, etc.)'}{} ou
instrumentaulx\wdx{instrumental}{adj. `qui sert d'instrument
(pour mettre en \oe uvre, etc.)'}{instrumentaulx
\emph{m.pl.}}, car ce sont les
instrumens\wdx{instrument}{m. 1\hoch{o}
`ce qui sert à travailler
(qch.), à mettre en \oe uvre, à exécuter (qch., dans un art),
désignation d'outil ou dispositif non spécifié'}{} de
l'ame\wdx{ame}{f. `principe
spirituel de l'homme; âme'}{},
si co\emph{m}me les meins\wdx{main}{f. `partie du corps humain située à
l'extrémité du bras'}{mein}, la
face\wdx{face}{f. `partie antérieure de la tête;
visage'}{}, le
cuer\wdx{cuer}{m. terme d'anat. `viscère de forme de cône
renversé, situé entre les poumons, qui est l'organe central de la
distribution du sang dans le corps'}{}, le
foye\wdx{foie}{m. terme d'anat. `organe
situé dans la partie supérieure droite de
l'abdomen et qui sécrète la bile;
foie'}{foye}. Et
pour ce dit Galien\adx{Galien}{}{} ou
\text{seco\emph{n}d}\fnb{Nachfolgend expungiertes \emph{tiers}.}/
du livre qui se in\mbox{titu}le\wdx{intituler}{v.pron.
`avoir pour titre'}{}
\flq De utilitate
particula\emph{rum}\frq , ou
darnier\wdx{*derrenier}{adj. `qui vient
après tous les autres, après lequel il n'y a pas
d'autre' (temporel ou spatial)}{darnier} chappitre\wdx{chapitre}{m. `chacune des
parties qui se suivent dans un livre et en articulent la lecture;
chapitre'}{chappitre}:
nature\wdx{nature}{f. 2\hoch{o} `principe actif qui
anime, organise l'ensemble des choses existantes selon
un certain ordre'}{} ha ordonné\wdx{*ordener}{v.tr.
1\hoch{o} `disposer,
mettre dans un certain ordre; ordonner'}{ordonné
\emph{p.p.}}
[17r\hoch{o}a] toutes
les parties du corps pour
les meurs\wdx{*mors}{f.pl. `habitudes (d'une
société, d'un individu) relatives à la pratique du
bien et du mal; m\oe urs'}{} et pour les
vertus\wdx{vertu}{f.
`principe
qui, dans une chose, est considéré comme la cause
des effets qu'elle produit; faculté, pouvoir'}{} de
l'ame\wdx{ame}{f.
`principe spirituel de l'homme; âme'}{}.
Item\wdx{item}{adv.
terme d'admin. `et de même, et aussi (introduction d'unités
traitées l'une après l'autre dans un traité, une ordonnance,
un doc. coutumier, etc.)'}{}, des
me\emph{m}bres compos, les
aucuns sont
principaulx\wdx{principal}{adj. `qui est le plus
important; principal'}{principaulx
\emph{m.pl.}}, les autres
non
principaulx.
Les principaulx\wdx{principal}{adj. `qui est le plus
important; principal'}{principaulx
\emph{m.pl.}}
sont le
cuer\wdx{cuer}{m. terme d'anat. `viscère de forme de cône
renversé, situé entre les poumons, qui est l'organe central de la
distribution du sang dans le corps'}{}, le
foye\wdx{foie}{m. terme d'anat. `organe
situé dans la partie supérieure droite de
l'abdomen et qui sécrète la bile;
foie'}{foye}, le
cervel\wdx{cervel}{m. terme d'anat. `partie
antérieure et supérieure de
l'encéphale, considérée comme le siège de la pensée; cerveau'}{}
et les
coillons\wdx{coillon}{m. 1\hoch{o} `gonade mâle suspendue dans le scrotum, qui
produit les spermatozoïdes; testicule'}{}. Les non
principaulx\wdx{principal}{adj. `qui est le plus
important; principal'}{principaulx
\emph{m.pl.}},
ce sont tous les aultres. Et de iceulx, les aucuns
sont petis\wdx{petit}{adj. 2\hoch{o}
dans l'ordre
physique, quantifiable `qui est d'une extension
au-dessous de la moyenne; petit (du corps humain
et de ses parties)'}{}, si co\emph{m}me l'ueil\wdx{ueil}{m.
`organe de la vue'}{}, le
nés\wdx{nés}{m. `partie saillante du visage,
située dans son axe entre le front et la lèvre
supérieure, et qui abrite la partie antérieure des
fosses nasales; nez'}{}, la mein petite\wdx{petite main}{f. terme d'anat.
`membre situé à l'extrémité du bras; main'}{mein
petite}.
Et les aultres so\emph{n}t plus grans\wdx{grant}{adj.
2\hoch{o} dans l'ordre
physique, quantifiable `qui est d'une extension
au-dessus de la moyenne; grand (du corps humain
et de ses parties)'}{grans
\emph{m.pl.}}, si
co\emph{m}me la face\wdx{face}{f. `partie antérieure de la
tête; visage'}{}, le col\wdx{col}{m.
1\hoch{o} `partie du
corps de l'homme et de certains vertébrés qui unit la
tête au tronc'}{}, les
espaules\wdx{espaule}{f. `partie supérieure du bras à l'endroit
où il s'attache au thorax, pouvant
désigner aussi l'omoplate'}{} et les aultres
.viij.,
esqueulx -- quant est de present\wdx{present}{m.}{\textbf{de present}
\emph{loc.adv. `au moment où l'on parle'}}
-- tout
le corps est
divisé par grace
de\wdx{grace}{f.}{\textbf{par grace de} \emph{loc.prép. `pour cause de'}}
doctrine\wdx{doctrine}{f.
`ensemble de
notions qu'on affirme être vraies et par
lesquelles on veut fourner une interprétation
des faits, orienter ou diriger l'action'}{}
cirurgiale\wdx{*cirurgicale}{adj. `qui
est relatif à la chirurgie'}{cirurgiale}. Et ja soit
ce que\wdx{ja soit ce que}{loc.conj.
`bien que assurément'}{}
les dis membres organiq\emph{ue}s\wdx{organique}{adj.
`qui sert d'instrument
(pour mettre en \oe uvre, etc.)'}{} soient
composés\wdx{composer}{v.tr.
`former par l'assemblage, par la combinaison de
parties'}{composé \emph{p.p.}} de pluseurs\wdx{*plusor}{adj. `un certain
nombre'}{pluseurs \emph{pl.}}
me\emph{m}bres
par
grace\wdx{grace}{f.}{\textbf{par grace de} \emph{loc.prép. `pour cause de'}}
de l'accion\wdx{accion}{f. `tout ce que l'on fait (dit aussi de choses)'}{}
et
passion\wdx{passïon}{f. `souffrance physique'}{} de iceulx, avec
deue\wdx{deu}{adj. `que l'on doit; dû'}{}
q\emph{uan}tité\wdx{*cantité}{f. `nombre
d'unités ou mesure qui sert à déterminer une
portion de matière ou une collection de choses
considérées comme
homogènes'}{quantité} et qualité\wdx{*calité}{f.
`manière d'être d'une chose ou d'une
personne, ce qui fait qu'elle est ce qu'elle est;
qualité'}{qualité} et universelle\wdx{universel}{adj.
`qui
s'applique à la totalité des objets (personnes, choses) que l'on
considère'}{}
plasmacion\wdx{plasmacïon}{f. `action
de donner une forme'}{plasmacion}, toutesvoies\wdx{*totes voies}{loc.adv.
`en
considérant toutes les raisons, toutes les
circonstances qui pourraient s'y opposer, et
malgré elles; toutefois'}{toutesvoies}
entre
ycelles, l'une est simple\wdx{simple}{adj.
terme d'anat.
`qui n'est pas
composé de plusieurs parties à distinguer
(dit d'un membre du
corps considéré comme un tout)'}{}
et co\emph{n}sim\emph{i}le\wdx{consimile}{adj.
`dont les éléments constitutifs sont de même nature
ou répartis de façon uniforme; homogène'}{}, qui est
co\emph{m}mencemant\wdx{*comencement}{m. `première
partie de qch., celle que d'autres doivent suivre et
qu'aucune ne précède (dans le temps ou dans l'espace)'}{commencemant}
de toute
l'accion\wdx{accion}{f. `tout ce que l'on fait (dit aussi de choses)'}{}.
Les aultres sont par
grace\wdx{grace}{f.}{\textbf{par grace de} \emph{loc.prép. `pour cause de'}}
de utilité\wdx{utilité}{f.
`caractère
de ce qui est utile; utilité'}{}, et les autres
sont afin
que\wdx{afin que}{loc.conj. qui marque l'intention,
 le but `pour que'}{} l'accion\wdx{accion}{f. `tout ce que l'on fait (dit aussi de choses)'}{}
soit
mieux faicte, \text{et les aultres sont
faictes}\lemma{et{\dots} faictes}\fnb{Am Zeilenrand
nachgetragen.}/ pour garder\wdx{garder}{v.tr.
1\hoch{o} `préserver
qch. (d'un mal, d'un danger, etc.); protéger'}{}
toutes
ycelles, si co\emph{m}me toutes ces
choses ci sont
demenees\wdx{demener}{v.tr. `déplacer dans ses
parties; remuer'}{demené \emph{p.p.}} et
demonstrees\wdx{demonstrer}{v.tr.
`faire
voir, mettre devant les yeux; montrer'}{}
es inclins\wdx{inclin}{subst. `?'}{},
ou premier et ou se\mbox{cond} du livre qui se
intitule\wdx{intituler}{v.pron. `avoir
pour titre'}{}
\flq De
utilitate particula\emph{rum}\frq . Et, p\emph{ar}
consequent\wdx{consequent}{adj.}{\textbf{par
consequent}
\emph{loc.adv. `comme suite logique'}},
en
aultres livres qui s'ensuivent\wdx{*ensivre}{v.pron.
1\hoch{o} `être placé ou considéré après dans un
ordre donné'}{ensuivent \emph{3.p.pl.
ind.prés.}}
est demonstré\wdx{demonstrer}{v.tr.
`faire
voir, mettre devant les yeux; montrer'}{} es aultres
parties ensuivant\wdx{*ensivant}{adj. `qui vient
immédiatement après; suivant'}{ensuivant}, en telle
maniere que\wdx{maniere}{f. 2\hoch{o} `forme
particulière
que revêt l'accomplissement d'une action, le
déroulement d'un fait, l'être ou
l'existence'}{\textbf{en telle maniere que}
\emph{loc.conj. `de sorte que'}}
tu dois ente\emph{n}dre\wdx{entendre}{v.tr. `saisir par
l'intelligence'}{}
[17r\hoch{o}b]
-- si co\emph{m}me il dit ou quart livre
deva\emph{n}t\wdx{devant}{adv. 2\hoch{o} qui marque la
priorité dans le temps `auparavant'}{}
dit -- que selon la
bestialité\wdx{bestialité}{f. `ce qui a la nature
de l'animal' (?)}{}
nulz corps n'est vuizeux\wdx{*oisos}{adj.
`qui est inutile'}{vuizeux}
ne fait sans cause\wdx{cause}{f. `ce qui produit
un effet (considéré par rapport à cet
effet)'}{}, ains\wdx{*ainz}{conj. `plutôt, de
préférence'}{ains}
il
est fait selon necessité\wdx{necessité}{f.
`caractère nécessaire
(d'une chose, d'une action); nécessité'}{} avec
co\emph{n}venable\wdx{*covenable}{adj. `qui est
approprié'}{convenable}
complexion\wdx{*complessïon}{f.
`ensemble des
éléments constituant la nature physique d'un
individu, d'une partie du corps ou d'une chose;
complexion'}{complexion}
et co\emph{m}posicion\wdx{composicion}{f.
`manière dont une chose est formée, par l'assemblage de
plusieurs éléments'}{},
avec aucunes vertus\wdx{vertu}{f. `principe
qui, dans une chose, est considéré comme la cause
des effets qu'elle produit; faculté, pouvoir'}{}
divines\wdx{*devin}{adj. `qui appartient à Dieu;
divin'}{divines \emph{f.pl.}} que le
createur\wdx{*crïator}{m. `celui qui crée, qui tire
(qch.) du néant; créateur'}{createur}
y a loiés et mises\wdx{metre}{v.tr. 1\hoch{o}
`placer (qch.) dans
une position déterminée'}{},
lesqueulx
sortissent\wdx{sortir}{v.tr.indir. `aller hors (d'un lieu)'}{sortissent
\emph{3.p.pl. subj.imp.}} es me\emph{m}bres
compost\wdx{compost}{adj. terme d'anat.
`qui est
formé
de plusieurs membres simples (dit d'un
membre du corps)'}{compost
\emph{m.pl.}} des simples, et es
si\emph{m}ples des ellemens\wdx{*element}{m. `partie
constitutive d'une chose; élément'}{ellemens
\emph{pl.}}.
\pend
\pstart
Le cuer\wdx{cuer}{m. terme d'anat. `viscère de forme de cône
renversé, situé entre les poumons, qui est l'organe central de la
distribution du sang dans le corps'}{}, qui est le
premier me\emph{m}bre
organique\wdx{organique}{adj.
`qui sert d'instrument
(pour mettre en \oe uvre, etc.)'}{},
est composé\wdx{composer}{v.tr.
`former par l'assemblage, par la combinaison de
parties'}{composé \emph{p.p.}} de
liguemans\wdx{*liguement}{m. terme d'anat.
`faisceau de tissu fibreux blanchâtre,
résistant et peu extensible, unissant les éléments
d'une articulation ou maintenant en place un organe
ou une partie d'un organe; ligament'}{liguemans \emph{pl.}}
et
panicles\wdx{*pannicle}{m. terme d'anat.
`couche de tissu musculaire ou cellulaire qui recouvre
une structure organique du corps humain (un
organe, un os, une articulation, un
muscle, etc.)'}{panicle}
et de char\wdx{char}{f.
`substance somatique de consistance relativement
molle
(surtout par opposition au tissu osseux et sa
dureté); chair'}{}
dure\wdx{dur}{adj. `qui résiste à la pression, qui ne se
laisse pas déformer facilement'}{},
lacertouze\wdx{*lacertos}{adj. terme d'anat. `qui est de la
nature des muscles'}{lacertouze
\emph{f.sg.}}, et est sec\wdx{sec}{adj. terme de méd.
désignant une des qualités des quatre humeurs, celle qui gouverne
essentiellement l'équilibre de la bile et de la mélancolie
`sec (comme terme de l'humorisme)'}{}
pour cause\wdx{cause}{f. `ce qui produit
un effet (considéré par rapport à cet
effet)'}{}
de multitude\wdx{multitude}{f. `grande quantité
(d'êtres, d'objets); multitude'}{} des
esperiz\wdx{esperit}{m. terme de méd.
`ensemble de corpuscules subtils qui assurent
toutes les fonctions de la vie dans
l'organisme humain'}{esperiz \emph{pl.}}
qui sont en lui. Car il est ainsi que\wdx{ainsi}{adv. `de cette façon'}{\textbf{ainsi que}
\emph{loc.conj.
`de la même façon que'}}
ung \text{four
chault}\fnb{Ms. \emph{four de chault}.}/\wdx{*for}{m. 1\hoch{o}
`ouvrage de maçonnerie voûté, de forme
circulaire, où l'on met les aliments pour les cuire;
four'}{four}
qui eschauffe\wdx{*eschaufer}{v.tr. `rendre
chaud'}{eschauffe \emph{3.p.sg. ind.prés.}} tout le
corps,
et po\emph{ur} ce dient les medicins\wdx{*medecin}{m.
1\hoch{o}
`personne habilitée à
exercer la médecine'}{medicin}
que le cuer\wdx{cuer}{m. terme d'anat. `viscère de forme de cône
renversé, situé entre les poumons, qui est l'organe central de la
distribution du sang dans le corps'}{} est
chault\wdx{*chaut}{adj. terme de méd.
désignant une des qualités des quatre humeurs, celle qui gouverne
essentiellement l'équilibre du sang et de la bile
`chaut (comme terme de l'humorisme)'}{chault
\emph{m.sg.}}
et sec\wdx{sec}{adj. terme de méd.
désignant une des qualités des quatre humeurs, celle qui gouverne
essentiellement l'équilibre de la bile et de la mélancolie
`sec (comme terme de l'humorisme)'}{}.
Mais les philosophes\wdx{*filosofe}{m.
`personne
qui s'adonne à l'étude rationnelle de la
nature et de la morale'}{philosophe}
dient, pour ce que
il est co\emph{m}mencemant\wdx{*comencement}{m.
`première
partie de qch., celle que d'autres doivent suivre et
qu'aucune ne précède (dans le temps ou dans
l'espace)'}{commencemant} de
vie\wdx{vie}{f. `état d'activité propre à tous
les êtres organisés, aussi la durée, la succession des
phénomènes par lesquels cette activité se
manifeste; vie'}{},
que il est actrempé\wdx{*atemprer}{v.tr. `adoucir
l'intensité; tempérer'}{actrempé \emph{p.p.}} ou
qu'il est
enclins\wdx{enclin}{adj.
`qui est porté par un penchant pour qch. (de
choses)'}{}
et se trait\wdx{traire}{v.pron. `se diriger quelque
part'}{} a\wdx{a}{prép.
marquant des rapports de direction, de position `à'}{}
chaleur\wdx{*chalor}{f.
terme de méd. `dans l'humorisme, qualité qui gouverne essentiellement
l'équilibre du sang et de la bile'}{chaleur} et moiteur\wdx{*moistor}{f.
terme de méd.
`dans l'humorisme, qualité qui gouverne essentiellement
l'équilibre du sang et du flegme'}{moiteur}. Mais du
foye\wdx{foie}{m. terme d'anat. `organe
situé dans la partie supérieure droite de
l'abdomen et qui sécrète la bile;
foie'}{foye},
quant est de son essence\wdx{essence}{f. `ce qui
constitue la nature d'une être; essence'}{}, il
samble
que il soit chault\wdx{*chaut}{adj. terme de méd.
désignant une des qualités des quatre humeurs, celle qui gouverne
essentiellement l'équilibre du sang et de la bile
`chaut (comme terme de l'humorisme)'}{chault \emph{m.sg.}}
et
moyte\wdx{moiste}{adj. terme de méd.
désignant une des qualités des quatre humeurs, celle qui gouverne
essentiellement l'équilibre du sang et du flegme
`moite (comme terme de l'humorisme)'}{moyte \emph{m.sg.}}, car la
moindre\wdx{*menor}{adj. `qui est
plus petit en étendue, en quantité, en qualité;
moindre'}{moindre}
partie de luy est charneuse\wdx{*charnel}{adj. `qui
est essentiellement constitué de
chair'}{charneuse \emph{f.sg.}} et
sanguine\wdx{sanguin}{adj. 1\hoch{o}
`qui a rapport au, qui contient du sang'}{},
et avec \text{ce
vont}\fnb{Über der Zeile nachgetragen.}/ a\wdx{a}{prép.
marquant des rapports de direction, de position `à'}{}
lui
maintes\wdx{maint}{adj. `plusieurs; maint'}{}
arteires\wdx{artere}{f. terme d'anat.
 `vaisseau sanguin qui part du c\oe ur et
qui distribue le sang, qui contient les esprits
vitaux,
à tout le corps'}{arteire}. Le
cervel\wdx{cervel}{m.
terme d'anat.
`partie antérieure et supérieure de
l'encéphale, considérée comme le siège de la pensée; cerveau'}{}, il
est froit\wdx{froit}{adj. terme de méd.
désignant une des qualités des quatre humeurs, celle qui gouverne
essentiellement l'équilibre du flegme et de la mélancolie
`froit (comme terme de l'humorisme)'}{}
et moiste\wdx{moiste}{adj. terme de méd.
désignant une des qualités des quatre humeurs, celle qui gouverne
essentiellement l'équilibre du sang et du flegme
`moite (comme terme de l'humorisme)'}{}, ja soit
ce que\wdx{ja soit ce que}{loc.conj.
`bien que assurément'}{}
il aye substance\wdx{sustance}{f. `matière dont un
corps est formé, et en vertu de laquelle
il y a des propriétés particulières;
substance'}{substance}
meduleuse\wdx{*medulleux}{adj.
terme d'anat.
`qui est de la
nature de la moelle'}{meduleuse \emph{f.sg.}},
toutesvoies\wdx{*totes voies}{loc.adv.
`en
considérant toutes les raisons, toutes les
circonstances qui pourraient s'y opposer, et
malgré elles; toutefois'}{toutesvoies}
y a difference\wdx{*diference}{f. `caractère ou
ensemble de caractères
qui distingue une chose d'une autre, un être
d'un autre'}{difference}, car la
medule\wdx{medulle}{f.
terme d'anat.
`substance moelleuse de l'intérieur d'une
structure osseuse'}{medule}, elle est
\text{causé de humeur. Et le cervel est
causé du sparme}\lemma{causé
de humeur{\dots} causé du sparme}\fnb{Ms.
\emph{cause de humeur{\dots} cuause du
sparme}; cp.
GuiChaul\textsc{jl} 22,30 \emph{quia medulla est ex
humoribus, cerebrum ex spermate}.}/\wdx{esperme}{m. terme de méd.
`semence, cellule ou groupe de cellules
dont se forme un organisme'}{sparme}\wdx{humeur}{f. 1\hoch{o} terme
de méd.
`substance liquide qui se trouve dans un organisme
humain ou animal'}{}%
\wdx{cervel}{m.
terme d'anat.
`partie antérieure et supérieure
de l'encéphale, considérée comme le siège de la
pensée; cerveau'}{}\wdx{causer}{v.tr. `être
cause de qch.'}{}. Et est chault\wdx{*chaut}{adj.
terme de méd.
désignant une des qualités des quatre humeurs, celle qui gouverne
essentiellement l'équilibre du sang et de la bile
`chaut (comme terme de l'humorisme)'}{chault \emph{m.sg.}}
selon nature\wdx{nature}{f.
1\hoch{o} `ensemble des
caractères, des propriétés qui définissent un être,
une chose concrète ou abstraite'}{},
si co\emph{m}me il est escript ou \flq Livre des
[17v\hoch{o}a] parties de bestes\frq\wdx{beste}{f. `être vivant non
végétal et non humain; animal'}{}. Mais
l'esplein\wdx{esplein}{m. terme d'anat. `organe
lymphoïde situé sous la partie gauche du
diaphragme; rate'}{} et
les \text{reins}\fnb{Ms.
\emph{Reins}.}/\wdx{rein}{m. 1\hoch{o}
terme d'anat. `chacun des deux organes
sécréteurs
glandulaires situés symétriquement dans les fosses
lombaires et qui élaborent l'urine'}{}
sont des me\emph{m}bres cha\emph{u}lz\wdx{*chaut}{adj. terme de méd.
désignant une des qualités des quatre humeurs, celle qui gouverne
essentiellement l'équilibre du sang et de la bile
`chaut (comme terme de l'humorisme)'}{chaulz
\emph{m.pl.}} et
moistes\wdx{moiste}{adj. terme de méd.
désignant une des qualités des quatre humeurs, celle qui gouverne
essentiellement l'équilibre du sang et du flegme
`moite (comme terme de l'humorisme)'}{},
ja soit ce que\wdx{ja soit ce que}{loc.conj.
`bien que assurément'}{}
les
reins\wdx{rein}{m. 1\hoch{o}
terme d'anat. `chacun des deux organes
sécréteurs
glandulaires situés symétriquement dans les fosses
lombaires et qui élaborent l'urine'}{}
soient desoubz l'esplein\wdx{esplein}{m.
terme d'anat.
`organe
lymphoïde situé sous la partie gauche du
diaphragme; rate'}{} en aucu\emph{n}
degré\wdx{degré}{m. 1\hoch{o} `chacun des états dans
une série d'états réels ou possibles; degré'}{}
-- pour la
groisseur\wdx{*grossor}{f. `caractère de
ce qui est consistant, filant (en parlant
du sang)'}{groisseur} du sang\wdx{*sanc}{m.
terme de méd.
`liquide visqueux, de couleur
rouge, qui est porté par les vaisseaux dans tout
l'organisme où il joue des rôles multiples (l'une
des quatre humeurs de l'humorisme)'}{sang}
qui est en
l'esplein\wdx{esplein}{m. terme d'anat.
`organe
lymphoïde situé sous la partie gauche du
diaphragme; rate'}{} --, ainsi que\wdx{ainsi}{adv. `de cette façon'}{\textbf{ainsi que}
\emph{loc.conj.
`de la même façon que'}}
l'esplein\wdx{esplein}{m. terme d'anat.
`organe
lymphoïde situé sous la partie gauche du
diaphragme; rate'}{}
est assés plus bas
du foye\wdx{foie}{m. terme d'anat. `organe
situé dans la partie supérieure droite de
l'abdomen et qui sécrète la bile;
foie'}{foye}
en ce degré\wdx{degré}{m. 1\hoch{o}
`chacun des états dans une
série d'états réels ou possibles; degré'}{} cy. Item\wdx{item}{adv.
terme d'admin. `et de même, et aussi (introduction d'unités
traitées l'une après l'autre dans un traité, une ordonnance,
un doc. coutumier, etc.)'}{}, la
cher\wdx{char}{f.
`substance somatique de consistance relativement
molle
(surtout par opposition au tissu osseux et sa
dureté); chair'}{cher} du poulmon\wdx{*poumon}{m.
terme d'anat.
`chacun des deux viscères logés
symétriquement dans la cage thoracique qui sont les
organes de la respiration, aussi l'ensemble
des deux; poumon'}{poulmon}
est moins moiste\wdx{moiste}{adj. terme de méd.
désignant une des qualités des quatre humeurs, celle qui gouverne
essentiellement l'équilibre du sang et du flegme
`moite (comme terme de l'humorisme)'}{} de la
graisse\wdx{*craisse}{f. `substance onctueuse, de fusion facile,
répartie en diverses parties du corps de l'homme et des mammifères;
graisse'}{graisse},
car, quant on la chauffe\wdx{*chaufer}{v.tr.
`rendre chaud'}{chauffer}, selon
Galien\adx{Galien}{}{} ou livre
devant\wdx{devant}{adv. 2\hoch{o} qui marque la
priorité
dans le temps `auparavant'}{} dit, elle ne se
font\wdx{fondre}{v.pron. `se liquéfier (par
l'éffet de la chaleur)'}{font \emph{3.p.sg.
ind.prés.}}
point\wdx{point}{m. `endroit fixé et déterminé (où qch. à
lieu)'}{\textbf{ne{\dots} point} \emph{adv. de la négation `ne{\dots}
pas'}}, car le poulmon\wdx{*poumon}{m. terme d'anat.
`chacun des deux viscères logés
symétriquement dans la cage thoracique qui sont les
organes de la respiration, aussi l'ensemble
des deux; poumon'}{poulmon} est
nouriz\wdx{norrir}{v.tr. `entretenir, faire vivre en
donnant à
manger ou en procurant les aliments nécessaires à la subsistance;
nourrir'}{nouriz \emph{p.p.}} de sang\wdx{*sanc}{m.
terme de méd.
`liquide visqueux, de couleur
rouge, qui est porté par les vaisseaux dans tout
l'organisme où il joue des rôles multiples (l'une
des quatre humeurs de l'humorisme)'}{sang}
plus
subtil\wdx{subtil}{adj. 2\hoch{o}
`qui est fluide, qui coule aisément (en parlant
d'un liquide)'}{}
q\emph{ue} le cuer\wdx{cuer}{m. terme d'anat. `viscère de forme de cône
renversé, situé entre les poumons, qui est l'organe central de la
distribution du sang dans le corps'}{} luy
envoie\wdx{envoiier}{v.tr. `faire aller qn ou qch.
(quelque part)'
(le sujet n'étant pas personnel)}{envoie \emph{3.p.sg. ind.prés.}},
si co\emph{m}me
Galie\emph{n}\adx{Galien}{}{} le dit ou livre \flq De utilitate
particula\emph{rum}\frq . Et, par consequent\wdx{consequent}{adj.}{\textbf{par consequent}
\emph{loc.adv. `comme suite logique'}}, on doit\wdx{devoir}{v.tr. + inf. `être dans
l'obligation de (faire qch.); devoir'}{}
ainsi\wdx{ainsi}{adv. `de cette façon'}{}
sillogiser\wdx{sillogiser}{v.tr.indir. `raisonner par
syllogismes'}{} et arguer\wdx{argüer}{v.tr.
`prouver (qch.) par des
arguments'}{} des
complexions\wdx{*complessïon}{f.
`ensemble des
éléments constituant la nature physique d'un
individu, d'une partie du corps ou d'une chose;
complexion'}{complexion}
des aultres
me\emph{m}bres compos, que
ilz soient de telle\wdx{tel}{adj. `qui est
semblable, du même genre; tel'}{telle \emph{f.sg.}}
complexion\wdx{*complessïon}{f.
`ensemble des
éléments constituant la nature physique d'un
individu, d'une partie du corps ou d'une chose;
complexion'}{complexion}
qui resulte et
vient\wdx{venir}{v.tr.indir. 1\hoch{o} `avoir
son origine dans'}{} des membres ou des choses dont ilz
sont composés\wdx{composer}{v.tr.
`former par l'assemblage, par la combinaison de
parties'}{composé \emph{p.p.}}.
\pend
%
% \memorybreak
%
\pstartueber
Le second chappitre\wdx{chapitre}{m. `chacune des
parties qui se suivent dans un livre et en articulent la lecture;
chapitre'}{chappitre}
parle de l'anathomie du
cuir\wdx{cuir}{m. `peau
de l'homme'}{} et de la graisse\wdx{*craisse}{f. `substance onctueuse, de fusion facile,
répartie en diverses parties du corps de l'homme et des mammifères;
graisse'}{graisse}
et de la char\wdx{char}{f.
`substance somatique de consistance relativement
molle
(surtout par opposition au tissu osseux et sa
dureté); chair'}{} et
des muscules\wdx{muscle}{m. terme d'anat. `structure
organique contractile qui assure les mouvements;
muscle'}{muscule}.
\pendueber

%
\pstart
PREMIERS\wdx{premier}{adv.
`en
premier lieu, d'abord; premièrement'}{premiers}  nous
volons parler de\wdx{parler}{\textbf{\emph{parler de}} v.tr.indir.
`s'entretenir de; parler de'}{}
l'anathomie du
cuir\wdx{cuir}{m. `peau
de l'homme'}{}, car c'est ce que on treuve\wdx{*trover}{v.tr. `rencontrer qn ou qch. qu'on cherche;
trouver'}{treuve \emph{3.p.sg. ind.prés.}}
premier\wdx{premier}{adv.
`en
premier lieu, d'abord; premièrement'}{} qu\emph{an}t  l'en
fait aucune anathomie. Donc le cuir\wdx{cuir}{m. `peau
de l'homme'}{}, c'est  la
couverture\wdx{*coverture}{f. `ce qui sert à
couvrir'}{couverture} du corps
qui est fait et
tissu\wdx{tistre}{v.tr. terme de méd.
`former par entrelacement (de fibres ou
structures organiques)'}{tissu \emph{p.p.}} de
fil\wdx{fil}{m. `brin long et fin, habituellement
utilisé pour coudre'}{} de nerfz\wdx{nerf}{m. terme d'anat.
`structure blanchâtre en forme de fil qui relie soit
un muscle à un os, soit un centre nerveux
(cerveau, moelle) à un organe ou une structure organique;
tendon ou nerf'}{nerfz \emph{pl.}},
de vaines\wdx{*veine}{f. terme d'anat. `vaisseau
sanguin ou, spécialement, vaisseau sanguin
qui part du foie et qui distribue le sang
nutritif du foie à tout le
corps'}{vaine} et de
arteres\wdx{artere}{f. terme d'anat.
 `vaisseau sanguin qui part du c\oe ur et
qui distribue le sang, qui contient les esprits
vitaux,
à tout le corps'}{} pour
deffe\emph{n}[17v\hoch{o}b]dre\wdx{*defendre}{v.tr.
`protéger
(qn, qch.) contre (qn, qch.)'}{deffendre \emph{inf.}}
et largir\wdx{largir}{v.tr. `rendre plus large'}{}
le sens\wdx{sens}{m. 2\hoch{o} `idée ou ensemble
d'idées intelligible que représente un signe ou un
ensemble de signes'}{} creé\wdx{*crïer}{v.tr.
`donner l'être,
la vie, l'existence à'}{creé \emph{p.p.}}. Et so\emph{n}t
deux
especes\wdx{espece}{f. `classe (de choses) définie par un
ensemble particulier de caractères communs'}{} de
cuir\wdx{cuir}{m. `peau
de l'homme'}{}: l'ung
coevre\wdx{*covrir}{v.tr. `garnir qch.
en disposant qch. dessus'}{coevre \emph{3.p.sg.
ind.prés.}} les me\emph{m}bres de dehors\wdx{*defors}{adv.
`à l'extérieur'}{\textbf{de dehors} \emph{loc.adj.
`qui est situé à l'extérieur'}}, et ceste est appellee\wdx{apeler}{v.tr. `donner un nom à qn, qch.; appeler'}{appellé
\emph{p.p.}}
p\emph{ro}premant\wdx{*proprement}{adv. `d'une
manière précise'}{propremant} cuir\wdx{cuir}{m.
`peau de l'homme'}{}.
Et de cest cuir cy sont notees\wdx{noter}{v.tr.
`prêter attention à (qch.); remarquer'}{}
.v.
differe\emph{n}ces\wdx{*diference}{f. `caractère ou
ensemble de caractères
qui distingue une chose d'une autre, un être
d'un autre'}{difference}, si co\emph{m}me
il est escript ou .xj.\hoch{e} livre, ou .v.\hoch{e} chappitre
du
livre qui se intitule\wdx{intituler}{v.pron. `avoir
pour titre'}{}
\flq De utilitate particula\emph{rum}\frq .
L'autre cuir coevre\wdx{*covrir}{v.tr. `garnir qch.
en disposant qch. dessus'}{coevre
\emph{3.p.sg.
ind.prés.}}
les me\emph{m}bres de dedans\wdx{dedans}{adv. `à l'intérieur'}{\textbf{de
dedans} \emph{loc.adj. `qui est
situé à l'intérieur'}}, et le
appelle\wdx{apeler}{v.tr. `donner un nom à qn, qch.; appeler'}{appelle
\emph{3.p.sg. ind.prés.}}
 on
p\emph{ro}premant\wdx{*proprement}{adv.
`d'une
manière précise'}{propremant}
panicle\wdx{*pannicle}{m. terme d'anat.
`couche de tissu musculaire ou cellulaire qui recouvre
une structure organique du corps humain (un
organe, un os, une articulation, un
muscle, etc.)'}{panicle}, si
co\emph{m}me les toilles\wdx{*toile}{f. terme d'anat.
`membrane, mince couche de tissu qui enveloppe un
organe'}{toille}
du cervel, le p\emph{er}icrane\wdx{pericrane}{m. terme d'anat. `membrane qui tapisse la boîte osseuse
renfermant l'encéphale'}{}, qui
coevre\wdx{*covrir}{v.tr. `garnir qch.
en disposant qch. dessus'}{coevre
\emph{3.p.sg.
ind.prés.}} le
crane\wdx{crane}{m.
terme d'anat.
`boîte osseuse renfermant l'encéphale'}{},
et celle qui coevre\wdx{*covrir}{v.tr. `garnir qch.
en disposant qch. dessus'}{coevre
\emph{3.p.sg.
ind.prés.}} les autres
os\wdx{os}{m.
`chacune des pièces
rigides qui constituent le squelette de l'homme et des animaux
vertébrés'}{} du
corps, et pleura\wdx{pleura}{gr. terme d'anat. `membrane séreuse située à l'intérieur
de la cavité thoracique et qui tapisse les parois
internes de la cavité
thoracique et couvre la surface des
poumons; plèvre'}{}
et le sifac\wdx{sifac}{m. terme d'anat. `membrane séreuse
qui recouvre les organes contenus dans
la cavité abdominale et pelvienne, à l'exception de
l'ovaire; péritoine viscéral'}{}
et le p\emph{er}icarde\wdx{pericarde}{m. terme d'anat.
`membrane qui enveloppe le c\oe ur; péricarde'}{},
qui coevre\wdx{*covrir}{v.tr.
`garnir qch. en disposant qch. dessus'}{coevre
\emph{3.p.sg.
ind.prés.}}
le cuer\wdx{cuer}{m. terme d'anat. `viscère de forme de cône
renversé, situé entre les poumons, qui est l'organe central de la
distribution du sang dans le corps'}{},
et les aultres des aultres
entrailles\wdx{entrailles}{f.pl.
terme d'anat.
`organes enfermés dans
l'abdomen de l'homme ou des
animaux; intestins'}{}.
\pend
\pstart
Aprés
vient\wdx{venir}{v.tr.indir. 2\hoch{o} `se
produire; survenir'}{}
la graisse\wdx{*craisse}{f. `substance onctueuse, de fusion facile,
répartie en diverses parties du corps de l'homme et des mammifères;
graisse'}{graisse},
et est ainsi que\wdx{ainsi}{adv. `de cette façon'}{\textbf{ainsi que}
\emph{loc.conj.
`de la même façon que'}}
huille\wdx{*uile}{m. et f. `substance
grasse, onctueuse et inflammable, liquide à
la température ordinaire et insoluble dans
l'eau, d'origine végétale, animale ou
minérale; huile'}{huille},
eschauffa\emph{n}t\wdx{*eschaufer}{v.tr.
`rendre chaud'}{eschauffant \emph{p.prés.}} et
humectans\wdx{humecter}{v.tr. `rendre humide,
mouiller légèrement ou superficiellement'}{} les
me\emph{m}\-bres. Et sont deux
especes\wdx{espece}{f. `classe (de choses) définie par un
ensemble particulier de caractères communs'}{}
de graisse\wdx{*craisse}{f. `substance onctueuse, de fusion facile,
répartie en diverses parties du corps de l'homme et des mammifères;
graisse'}{graisse}:
l'une qui est dehors\wdx{*defors}{adv. `à
l'extérieur'}{dehors}, pres du cuir, et
l'apelle\wdx{apeler}{v.tr. `donner un nom à qn, qch.; appeler'}{apelle
\emph{3.p.sg. ind.prés.}}
on p\emph{ro}premant\wdx{*proprement}{adv.
`d'une
manière précise'}{propremant}
graisse\wdx{*craisse}{f. `substance onctueuse, de fusion facile,
répartie en diverses parties du corps de l'homme et des mammifères;
graisse'}{graisse}.
L'autre
est par dedans\wdx{dedans}{adv. `à
l'intérieur'}{},
pres du ventre\wdx{ventre}{m. 1\hoch{o} `partie
antérieure du
tronc formant une cavité qui contient l'estomac et les intestins'}{}
et des
reins\wdx{rein}{m. 2\hoch{o} au plur.
`la partie inférieure du dos au niveau des
vertèbres lombaires'}{}, et
le appelle\wdx{apeler}{v.tr. `donner un nom à qn, qch.; appeler'}{appelle
\emph{3.p.sg. ind.prés.}}
on propremant\wdx{*proprement}{adv. `d'une
manière précise'}{propremant}
axu\emph{n}ge\wdx{*axonge}{f.
terme d'anat. `substance grasse répartie en
diverses
parties du corps humain, à l'exception du tissu conjonctif
sous-cutané'}{axunge}.
\pend
\pstart
Et aprés vient\wdx{venir}{v.tr.indir.
2\hoch{o} `se produire; survenir'}{}
la char\wdx{char}{f.
`substance somatique de consistance relativement
molle
(surtout par opposition au tissu osseux et sa
dureté); chair'}{}, et en sont .iij.
especes\wdx{espece}{f.
`classe (de choses) définie par un
ensemble particulier de caractères communs'}{},
c'est assavoir\wdx{assavoir}{v.tr.}{\textbf{c'est
assavoir} \emph{loc. `c'est-à-dire'}}
char simple et pure\wdx{pur}{adj. `qui est sans
mélange'}{} qui est
seulemant\wdx{*seulement}{adv. `sans rien d'autre que
ce qui est mentionné; seulement'}{seulemant}
trouvee\wdx{*trover}{v.tr. `rencontrer qn ou qch. qu'on cherche;
trouver'}{trouvé \emph{p.p.}}
ou chief du vit\wdx{chief du vit}{m.
`renflement antérieur de la verge; gland'}{} et
entre les dens\wdx{dent}{m. et f. `organe de
la bouche, de couleur blanchâtre, dur, implanté
sur le maxillaire'}{dens \emph{pl.}}. L'autre,
c'est char
glandelleuse\wdx{*glandulos}{adj. `qui contient des
glandes'}{glandelleuse \emph{f.sg.}}
ou noilleuse\wdx{*nöeillos}{adj. `qui
présente, qui contient des n\oe uds'}{noilleuse
\emph{f.sg.}},
si co\emph{m}me la char des coillons\wdx{coillon}{m. 1\hoch{o} `gonade mâle suspendue dans le scrotum, qui
produit les spermatozoïdes; testicule'}{} et des
mamelles\wdx{*mamele}{f. `organe glanduleux
qui sécrète le lait (chez les mammifères);
mamelle'}{mamelle}
et des emu\emph{n}ctoires\wdx{*emomptoire}{m. terme d'anat. `organe
qui élimine les substances inutiles formées au
cours des processus de désassimilation (l'anus,
l'uretère, etc.)'}{emunctoire}. La tierce
char\wdx{char}{f.
`substance somatique de consistance relativement
molle (surtout par opposition au tissu osseux et sa
dureté); chair'}{} est
musculeuse\wdx{musculeux}{adj. terme d'anat. `qui est de la nature
des muscles'}{}
ou lacertouse\wdx{*lacertos}{adj. terme d'anat. `qui est de la
nature des muscles'}{lacertouse
\emph{f.sg.}}, et est moult de telle\wdx{tel}{adj. `qui est
semblable, du même genre; tel'}{telle \emph{f.sg.}}
char\wdx{char}{f.
`substance somatique de consistance relativement
molle (surtout par opposition au tissu osseux et sa
dureté); chair'}{}, et la treuve\wdx{*trover}{v.tr. `rencontrer qn ou qch. qu'on cherche;
trouver'}{treuve \emph{3.p.sg. ind.prés.}}
on par tout le corps,
[18r\hoch{o}a] en tout
\text{lieu}\fnb{\emph{u} über der
Zeile nachgetragen.}/\wdx{lieu}{m.
`portion
déterminée de l'espace; lieu'}{}
ou est mouvemant\wdx{*movement}{m. `changement de
position ou
de place effectué par un organisme ou une de ses
parties'}{mouvemant}
liquide\wdx{liquide}{adj.  `qui semble se faire
aisément, sans effort'}{} et
ellective\wdx{*electif}{adj.
`qui est choisit'}{ellective
\emph{m.sg.}}.
\pend
\pstart
Le muscule\wdx{muscle}{m. terme d'anat. `structure
organique contractile qui assure les mouvements;
muscle'}{muscule}, c'est le
instrument\wdx{instrument}{m. 2\hoch{o}
`partie du corps remplissant une fonction
particulière; organe'}{} de
mouvemant\wdx{*movement}{m. `changement de position
ou de place effectué par un organisme ou une de ses
parties'}{mouvemant}
liquide\wdx{liquide}{adj.
`qui semble se faire
aisément, sans effort'}{} et
elletif\wdx{*electif}{adj. `qui est
choisit'}{elletif
\emph{m.sg.}},
c'est a dire\wdx{dire}{v.tr.
`lire à haute voix;
réciter'}{\textbf{c'est a dire} \emph{loc.conj. qui
annonce une explication ou une précision}}
manifest\wdx{manifest}{adj. `dont l'existence ou la nature
est évidente; manifeste'}{}, si co\emph{m}me il
est dit ou livre
\flq De utilitate particula\emph{rum}\frq . Et ja soit ce que\wdx{ja soit ce que}{loc.conj.
`bien que assurément'}{}
les
muscules\wdx{muscle}{m. terme d'anat. `structure
organique contractile qui assure les mouvements;
muscle'}{muscule},
quant au sens\wdx{sens}{m. 2\hoch{o} `idée ou ensemble
d'idées intelligible que représente un signe ou un
ensemble de signes'}{}, soient me\emph{m}bres
si\emph{m}mples\wdx{simple}{adj.
terme d'anat.
`qui n'est pas
composé de plusieurs parties à distinguer
(dit d'un membre du
corps
 considéré comme un tout)'}{simmple}, toutesvoies\wdx{*totes voies}{loc.adv.
`en
considérant toutes les raisons, toutes les
circonstances qui pourraient s'y opposer, et
malgré elles; toutefois'}{toutesvoies},
selon la verité\wdx{verité}{f. `ce à quoi l'esprit
donne son assentiment, par suite d'un
rapport de conformité avec l'objet de pensée, d'une
cohérence interne de la pensée; vérité'}{}, ilz sont
compos\wdx{composer}{v.tr.
`former par l'assemblage, par la combinaison de
parties'}{compos \emph{p.p.}}
de nerfz\wdx{nerf}{m. terme d'anat.
`structure blanchâtre en forme de fil qui relie soit
un muscle à un os, soit un centre nerveux
(cerveau, moelle) à un organe ou une structure organique;
tendon ou nerf'}{nerfz \emph{pl.}},
de liguemans\wdx{*liguement}{m. terme d'anat.
`faisceau de tissu fibreux blanchâtre,
résistant et peu extensible, unissant les éléments
d'une articulation ou maintenant en place un organe
ou une partie d'un organe; ligament'}{liguemans \emph{pl.}}
et de leur villis\wdx{villis}{m.pl. terme d'anat.
`production organique longue et fine comme des
fils'}{}
emplis\wdx{emplir}{v.tr. `rendre plein; remplir'}{}
de char, et de
panicle\wdx{*pannicle}{m. terme d'anat.
`couche de tissu musculaire ou cellulaire qui recouvre
une structure organique du corps humain (un
organe, un os, une articulation, un
muscle, etc.)'}{panicle}
qui les coevre\wdx{*covrir}{v.tr. `garnir qch.
en disposant qch. dessus'}{coevre
\emph{3.p.sg.
ind.prés.}}.
Et ainsi\wdx{ainsi}{adv. `de cette façon'}{} le
veult
Avicene\adx{Avicene}{}{} ou premier livre de son \flq Canon\frq .
Item\wdx{item}{adv.
terme d'admin. `et de même, et aussi (introduction d'unités
traitées l'une après l'autre dans un traité, une ordonnance,
un doc. coutumier, etc.)'}{},
muscules\wdx{muscle}{m. terme d'anat. `structure
organique contractile qui assure les mouvements;
muscle'}{muscule}
et lacertes\wdx{lacerte}{m. terme d'anat. `structure
organique contractile qui assure les mouvements; muscle'}{}, c'est tout
ung. Mais on le appelle\wdx{apeler}{v.tr. `donner un nom à qn, qch.; appeler'}{appelle
\emph{3.p.sg. ind.prés.}}
muscule\wdx{muscle}{m. terme d'anat. `structure
organique contractile qui assure les mouvements;
muscle'}{muscule}
pour ce qu'il est fait en la maniere\wdx{maniere}{f.
2\hoch{o}
`forme particulière
que revêt l'accomplissement d'une action, le
déroulement d'un fait, l'être ou
l'existence'}{}
de
mus\wdx{mus}{lt. `petite mammifère rongeur, voisin du
rat, dont l'espèce le plus répandue a le pelage gris;
souris'}{}, c'est a dire\wdx{dire}{v.tr.
`lire à haute voix;
réciter'}{\textbf{c'est a dire} \emph{loc.conj. qui
annonce une explication ou une précision}}
d'une
rate\wdx{rate}{f. `petit mammifère rongeur à
museau pointu et à longe queue, plus grand que le
souris; rat'}{} ou suriz\wdx{*soriz}{f. `petit mammifère
rongeur, voisin du rat, dont l'espèce la plus répandue
à le pelage gris; souris'}{suriz}.
Et lacerte\wdx{lacerte}{m. terme d'anat. `structure organique
contractile qui assure les mouvements; muscle'}{} est
fourmé\wdx{*former}{v.tr.
`donner une certaine forme (à qch.)'}{fourmé
\emph{p.p.}}
en maniere\wdx{maniere}{f. 2\hoch{o} `forme
particulière
que revêt l'accomplissement d'une action, le
déroulement d'un fait, l'être ou
l'existence'}{} d'une
lizarde\wdx{*laisarde}{f.
`petit
reptile saurien à longue queue effilée, au corps
allongé; lézard'}{lizarde}.
Car ces deux bestes\wdx{beste}{f. `être vivant non végétal et
non humain; animal'}{} ci, elles sont
grailles\wdx{*graisle}{adj. `qui a des formes relativement étroites
pour leur longueur et qui donne une
impression de finesse, gracilité, minceur'}{graille} ou deux
chiefz\wdx{chief}{m. 2\hoch{o} `partie terminale de
qch.
(d'un os, etc.); bout'}{}
-- au moins\wdx{moins}{adv.}{\textbf{au moins}
\emph{loc.adv. qui sert
à marquer une restriction}}
\text{vers}\fnb{\emph{r} über der Zeile nachgetragen.}/ la queue\wdx{*cöe}{f.
`appendice qui
prolonge la colonne vertébrale de l'homme et
de nombreux mammifères'}{queue}
--, et au milieu\wdx{milieu}{m. `partie d'une chose
qui est à égale distance des extrémités de cette
chose'}{}
elles sont grosses\wdx{gros}{adj. 2\hoch{o} dans
l'ordre physique, quantifiable
`qui, dans son genre, dépasse le volume ordinaire;
gros (de l'animal et des parties de son corps)'}{},
et ainsi\wdx{ainsi}{adv. `de cette façon'}{}
so\emph{n}t les muscles\wdx{muscle}{m. terme d'anat. `structure
organique contractile qui assure les mouvements;
muscle'}{} et les
lacertes\wdx{lacerte}{m. terme d'anat. `structure organique
contractile qui assure les mouvements; muscle'}{}. Et doiz
noter\wdx{noter}{v.tr.
`prêter attention à (qch.); remarquer'}{},
selon la entencion\wdx{entencïon}{f.
1\hoch{o} `manière de penser; opinion'}{entencion}
de Galien\adx{Galien}{}{} par tout le livre \flq De
utilitate particula\emph{rum}\frq ,
que, quant le muscule\wdx{muscle}{m. terme d'anat. `structure
organique contractile qui assure les mouvements;
muscle'}{muscule}
est composé\wdx{composer}{v.tr.
`former par l'assemblage, par la combinaison de
parties'}{composé \emph{p.p.}}, de lui
descendent\wdx{descendre}{v.tr.indir. `aller du haut
en bas'}{}
cordes\wdx{corde}{f. terme d'anat. `structure
conjonctive fibreuse par laquelle un muscle
s'insère sur un os; tendon'}{}
et liguemans\wdx{*liguement}{m. terme d'anat.
`faisceau de tissu fibreux blanchâtre,
résistant et peu extensible, unissant les éléments
d'une articulation ou maintenant en place un organe
ou une partie d'un organe; ligament'}{liguemans \emph{pl.}}
rons\wdx{*rëont}{adj. `qui a la forme
circulaire'}{rons \emph{m.pl.}}.
Et quant ilz viennent\wdx{venir}{v.tr.indir.
2\hoch{o} `se produire; survenir'}{}
pres des
\text{join\emph{n}ctures}\fnb{Nach \emph{join}n\emph{c} am
Zeilenanfang \emph{j}
gestrichen.}/\wdx{jointure}{f.
`partie du corps formée par la jointure entre
deux ou plusieurs os; articulation'}{joinncture}, ilz se
delaissent\wdx{delaisser}{v.pron. `augmenter de
volume; s'étendre'}{delaissent \emph{3.p.pl.
ind.prés.}} et eslargissent\wdx{eslargir}{v.pron. `devenir plus large'}{eslargissent
\emph{3.p.pl. ind.prés.}}
et lient\wdx{*liier}{v.tr. `entourer plusieurs choses avec un lien pour
qu'elles tiennent ensemble'}{lient \emph{3.p.pl.
ind.prés.}}
tout autour\wdx{*au tor}{loc.adv.
`(dans l'espace) en environnant (qch.)'}{autour} la
join\emph{n}cture\wdx{jointure}{f.
`partie du corps formée par la jointure entre
deux ou plusieurs os; articulation'}{joinncture} avec le
panicle\wdx{*pannicle}{m. terme d'anat.
`couche de tissu musculaire ou cellulaire qui recouvre
une structure organique du corps humain (un
organe, un os, une articulation, un
muscle, etc.)'}{panicle}
qui coevre\wdx{*covrir}{v.tr. `garnir qch.
en disposant qch. dessus'}{coevre
\emph{3.p.sg.
ind.prés.}}
la joincture\wdx{jointure}{f.
`partie du corps formée par la jointure entre
deux ou plusieurs os; articulation'}{joincture}
et la font\wdx{faire}{v.tr.
`réaliser
ou effectuer (qch.)'}{}
mouvoir\wdx{*movoir}{v.tr. `mettre en
mouvement'}{mouvoir \emph{inf.}}.
Et
quant ilz sont hors de la [18r\hoch{o}b]
join\emph{n}cture\wdx{jointure}{f.
`partie du corps formée par la jointure entre
deux ou plusieurs os; articulation'}{joinncture}, ilz
se rassemblent\wdx{rassembler}{v.pron. `se
réunir'}{} et redeviennent\wdx{redevenir}{v.intr.
`commencer à être de nouveau ce qu'on était
auparavant et qu'on avait cessé d'être'}{}
rons\wdx{*rëont}{adj. `qui a la forme
circulaire'}{rons \emph{m.pl.}} co\emph{m}me
devant\wdx{devant}{adv. 2\hoch{o} qui marque la
priorité dans le temps `auparavant'}{},
en maniere\wdx{maniere}{f. 2\hoch{o} `forme
particulière
que revêt l'accomplissement d'une action, le
déroulement d'un fait, l'être ou
l'existence'}{}
de corde\wdx{corde}{f. terme d'anat. `structure
conjonctive fibreuse par laquelle un muscle
s'insère sur un os; tendon'}{}, et avec la cher ilz font\wdx{faire}{v.tr.
`réaliser
ou effectuer (qch.)'}{}
ung
aultre muscule. Et de cest muscule encores en
ist\wdx{issir}{v.tr.indir. `aller hors de' (dit d'une
chose)}{}
la corde\wdx{corde}{f. terme d'anat. `structure
conjonctive fibreuse par laquelle un muscle
s'insère sur un os; tendon'}{}
et le liguemant\wdx{*liguement}{m. terme d'anat.
`faisceau de tissu fibreux blanchâtre,
résistant et peu extensible, unissant les éléments
d'une articulation ou maintenant en place un organe
ou une partie d'un organe; ligament'}{liguemant}, et se
\text{dilate\emph{n}t}\fnb{Ms.
\emph{dilace}n\emph{t}.}/\wdx{dilater}{v.pron.
`augmenter de volume; s'étendre'}{} et
loient\wdx{*liier}{v.tr. `entourer plusieurs choses
avec un lien pour
qu'elles tiennent ensemble'}{loient \emph{3.p.pl.
ind.prés.}} tout
autour\wdx{*au tor}{loc.adv. `(dans l'espace) en
environnant (qch.)'}{autour}
la join\emph{n}cture\wdx{jointure}{f.
`partie du corps formée par la jointure entre
deux ou plusieurs os; articulation'}{joinncture}
ensuivant\wdx{*ensivant}{adj. `qui vient
immédiatement après; suivant'}{ensuivant}
et la font mouvoir\wdx{*movoir}{v.tr. `mettre en
mouvement'}{mouvoir \emph{inf.}}. Et ne
cesse\wdx{cesser}{v.intr. `ne pas continuer;
s'arrêter'}{} le muscle\wdx{muscle}{m. terme d'anat. `structure
organique contractile qui assure les mouvements;
muscle'}{} de faire
ainsi\wdx{ainsi}{adv. `de cette façon'}{}, jusques a tant
qu'il\wdx{jusques a tant que}{loc.conj. qui marque
le terme final, la limite que l'on ne dépasse
pas}{} vient\wdx{venir}{v.tr.indir. 2\hoch{o}
`se produire; survenir'}{}
aux
darnieres\wdx{*derrenier}{adj.
`qui vient
après tous les autres, après lequel il n'y a pas
d'autre' (temporel ou spatial)}{darnier}
parties du me\emph{m}bre. Et selon ce, tousjours le
muscule va devant\wdx{devant}{adv. 2\hoch{o} qui
marque la priorité dans le
temps `auparavant'}{}
la joincture\wdx{jointure}{f.
`partie du corps formée par la jointure entre
deux ou plusieurs os; articulation'}{joincture} et
devant\wdx{devant}{adv. 2\hoch{o} qui marque
la priorité dans le
temps `auparavant'}{}
le membre
que il fait mouvoir\wdx{*movoir}{v.tr. `mettre en
mouvement'}{mouvoir \emph{inf.}}. Et ce que dit est,
nous le povons declarcir\wdx{declarcir}{v.tr.
`faire connaître clairement'}{} es
bras\wdx{bras}{m.
`membre supérieur de l'homme comprenant le
segment soit entre l'épaule
et le coude, soit entre
le coude et
la main,
soit les deux segments ensemble avec la main'}{},
car
les nerfz q\emph{ue} viennent\wdx{venir}{v.tr.indir.
1\hoch{o} `avoir
son origine dans'}{}
du col\wdx{col}{m. 1\hoch{o}
`partie du corps de
l'homme et de certains vertébrés qui unit la
tête au tronc'}{}, qui sont
envoiés\wdx{envoiier}{v.tr. `faire aller qn ou qch.
(quelque part)'
(le sujet n'étant pas personnel)}{envoié \emph{p.p.}}
aux bras\wdx{bras}{m.
`membre supérieur
de l'homme comprenant le segment soit entre l'épaule et le coude,
soit entre le coude et la main, soit les deux segments ensemble
avec la main'}{},
ilz prengnent fourme\wdx{prendre}{v.tr. 2\hoch{o} `commencer à avoir (un mode d'être)'}{prendre fourme}
de muscule ou col\wdx{col}{m. 1\hoch{o} `partie du
corps de l'homme et de certains vertébrés qui unit la
tête au tronc'}{} et ou
pis\wdx{*piz}{m.
terme d'anat.
`partie du corps humain qui s'étend
des épaules à
l'abdomen et qui contient le c\oe ur et les
poumons; thorax'}{pis}. Et puis, en venant
a la joincture\wdx{jointure}{f.
`partie du corps formée par la jointure entre
deux ou plusieurs os; articulation'}{joincture} de
l'espaule\wdx{espaule}{f.
`partie supérieure
du bras à l'endroit où il s'attache au thorax, pouvant
désigner aussi l'omoplate'}{},
il se met en fourme\wdx{metre}{v.pron.
`devenir
(en un état physique)'}{}\wdx{*forme}{f. `apparence
extérieure donnant à un objet ou à un être sa
spécificité'}{fourme}
de corde\wdx{corde}{f. terme d'anat. `structure
conjonctive fibreuse par laquelle un muscle
s'insère sur un os; tendon'}{}
et puis si se
\text{dilate}\fnb{Ms.
\emph{dilace}.}/\wdx{dilater}{v.pron.
`augmenter de volume; s'étendre'}{}
et conprent\wdx{comprendre}{v.tr. 2\hoch{o} `être
autour de (qch.) de manière à enfermer ou embrasser partiellement ou
complètement; entourer'}{conprent \emph{3.p.sg.
ind.prés.}} toute
la joincture\wdx{jointure}{f.
`partie du corps formée par la jointure entre
deux ou plusieurs os; articulation'}{joincture} et se
pla\emph{n}te\wdx{planter}{v.pron. `s'appliquer (sur
quelque chose)'}{}
et fiche\wdx{fichier}{v.pron. `être attaché, noué'}{} en
l'os de l'adjuctoire\wdx{*ajutoire}{m. terme d'anat.
`os supérieur du bras, (par extension) la partie
supérieure du bras'}{adjuctoire}
et fait mouvoir\wdx{*movoir}{v.tr. `mettre en
mouvement'}{mouvoir \emph{inf.}}
l'adjuctoire\wdx{*ajutoire}{m. terme d'anat.
`os supérieur du bras, (par extension)  la partie
supérieure du bras'}{adjuctoire}.
Et quant il ist\wdx{issir}{v.tr.indir. `aller hors de' (dit d'une chose)}{}, a deux
dois\wdx{doi}{m. 3\hoch{o} `mesure approximative, équivalent à
un travers de doigt'}{} ou a\wdx{a}{prép.
marquant des rapports de direction, de position `à'}{}
trois hors de la dicte joincture il se
arondist\wdx{*arrondir}{v.pron. `devenir rond'}{} et
revient\wdx{revenir}{v.intr.
`commencer à être ce qu'on était auparavant
et qu'on avait cessé d'être'}{}
en fourme\wdx{*forme}{f. `apparence extérieure
donnant à un objet ou à un être sa
spécificité'}{fourme} de corde\wdx{corde}{f. terme d'anat. `structure
conjonctive fibreuse par laquelle un muscle
s'insère sur un os; tendon'}{}. Et puis, avec la
char et le
liguema\emph{n}t\wdx{*liguement}{m. terme d'anat.
`faisceau de tissu fibreux blanchâtre,
résistant et peu extensible, unissant les éléments
d'une articulation ou maintenant en place un organe
ou une partie d'un organe; ligament'}{liguemant}
qui ist\wdx{issir}{v.tr.indir. `aller hors de' (dit d'une chose)}{} du
bout\wdx{*bot}{m. `partie terminale de qch.'}{bout}
ou du chief\wdx{chief}{m. 2\hoch{o}
`partie terminale de qch. (d'un os, etc.); bout'}{}
de l'os de l'espaule\wdx{espaule}{f. `partie supérieure du bras à
l'endroit où il s'attache au thorax, pouvant désigner aussi
l'omoplate'}{}, il fait
aultres muscules sur le milieu\wdx{milieu}{m. `partie d'une chose
qui est à égale distance des extrémités de cette
chose'}{}
de l'adjuctoire\wdx{*ajutoire}{m. terme d'anat.
`os supérieur du bras, (par extension) la partie
supérieure du bras'}{adjuctoire}.
Desqueulx
muscules ist\wdx{issir}{v.tr.indir. `aller hors de' (dit d'une chose)}{} une
corde\wdx{corde}{f. terme d'anat. `structure
conjonctive fibreuse par laquelle un muscle
s'insère sur un os; tendon'}{}
q\emph{ue} va vers le coude\wdx{*cote}{m. `partie du membre
supérieur correspondant à l'articulation du bras et l'avant-bras'}{coude}
et
se eslargist\wdx{eslargir}{v.pron. `devenir plus large'}{eslargist
\emph{3.p.sg. ind.prés.}} a\wdx{a}{prép.
marquant des rapports de direction, de position `à'}{}
trois
dois\wdx{doi}{m. 3\hoch{o} `mesure approximative, équivalent à
un travers de doigt'}{} pres du coude\wdx{*cote}{m.
`partie du membre
supérieur correspondant à l'articulation du bras et
l'avant-bras'}{coude},
et puis
[18v\hoch{o}a] se
conprent\wdx{comprendre}{v.tr. 2\hoch{o} `être
autour de (qch.) de manière à enfermer ou embrasser partiellement ou
complètement; entourer'}{conprent \emph{3.p.sg.
ind.prés.}}
toute la joincture et fait mouvoir\wdx{*movoir}{v.tr.
`mettre en mouvement'}{mouvoir \emph{inf.}}
le petit
bras\wdx{petit bras}{m. terme d'anat. `segment du
membre supérieur de l'homme compris entre le coude et
la main; avant-bras'}{}.
Et puis, oultre le coude\wdx{*cote}{m. `partie du
membre supérieur
correspondant à l'articulation du bras et l'avant-bras'}{coude} par trois
dois\wdx{doi}{m. 3\hoch{o} `mesure approximative,
équivalent à un travers de doigt'}{},
le dit muscule se arondit\wdx{*arrondir}{v.pron.
`devenir rond'}{} et
remet\wdx{remetre}{v.pron. `revenir
à un état antérieur'}{} en
corde\wdx{corde}{f. terme d'anat. `structure
conjonctive fibreuse par laquelle un muscle
s'insère sur un os; tendon'}{}. Et celle corde,
elle entre\wdx{entrer}{v.intr.
`passer du dehors en dedans'}{} avec le
liguemant\wdx{*liguement}{m. terme d'anat.
`faisceau de tissu fibreux blanchâtre,
résistant et peu extensible, unissant les éléments
d'une articulation ou maintenant en place un organe
ou une partie d'un organe; ligament'}{liguemant}
qui
ist\wdx{issir}{v.tr.indir. `aller hors de' (dit d'une chose)}{} du coude\wdx{*cote}{m.
`partie du membre
supérieur correspondant à l'articulation du bras et l'avant-bras'}{coude},
et
avec la char il fait autres muscules sur le
milieu\wdx{milieu}{m. `partie d'une chose
qui est à égale distance des extrémités de cette
chose'}{}
du dit petit bras\wdx{petit bras}{m.
terme d'anat. `segment du
membre supérieur de l'homme compris entre le coude et
la main;
avant-bras'}{}, de quoy ist\wdx{issir}{v.tr.indir. `aller hors de' (dit d'une chose)}{}
une corde\wdx{corde}{f. terme d'anat. `structure
conjonctive fibreuse par laquelle un muscle
s'insère sur un os; tendon'}{}, laquelle
se eslargist\wdx{eslargir}{v.pron. `devenir plus large'}{eslargist
\emph{3.p.sg. ind.prés.}} a
trois dois\wdx{doi}{m. 3\hoch{o}
`mesure approximative, équivalent à un travers de
doigt'}{}
pres
de la main petite\wdx{petite main}{f. terme d'anat.
`membre situé à l'extrémité du bras; main'}{main
petite} et comprent\wdx{comprendre}{v.tr. 2\hoch{o} `être
autour de (qch.) de manière à enfermer ou embrasser partiellement ou
complètement; entourer'}{comprent \emph{3.p.sg.
ind.prés.}}
toute
la joincture de la petite main\wdx{petite main}{f.
terme d'anat.
`membre situé à l'extrémité du bras; main'}{}.
Et puis se arondit\wdx{*arrondir}{v.pron.
`devenir rond'}{} et
entre\wdx{entrer}{v.tr. `passer du dehors en dedans de (qch.) (de choses)'}{}
l'autre muscule
ou milieu\wdx{milieu}{m. `partie d'une chose
qui est à égale distance des extrémités de cette
chose'}{} de la mein\wdx{main}{f.
`partie du corps humain située
à l'extrémité du bras'}{mein}, duquel
issent\wdx{issir}{v.tr.indir. `aller hors de' (dit d'une chose)}{}
les cordes qui
mouvent\wdx{*movoir}{v.tr.
`mettre en mouvement'}{mouvent \emph{3.p.pl.
ind.prés.}} les
dois\wdx{doi}{m. 1\hoch{o} `chacun des cinq prolongements
qui terminent la main'}{}. Et pour ce il
appert\wdx{*aparoir}{v.intr. `se montrer aux yeux; se
manifester'}{appert \emph{3.p.sg. ind.prés.}}
q\emph{u}e plaies\wdx{plaie}{f.
`ouverture
dans les chairs, les tissus, due à une cause
externe (traumatisme, intervention chirurgicale) et
présentant une solution de continuité des téguments;
plaie'}{}, qui
sont faictes a trois dois\wdx{doi}{m. 3\hoch{o}
`mesure approximative, équivalent à un travers de
doigt'}{}
pres des joinctures,  elles
sont
pe\mbox{ri}lleuses\wdx{*perillos}{adj.
`qui constitue un danger, présente du
danger; dangereux'}{perilleuses
\emph{f.pl.}} pour ce que
les cordes nerveuses\wdx{*nervos}{adj. terme d'anat. `qui a le caractère des nerfs
ou des tendons'}{nerveuses
\emph{f.pl.}}
sont la, apparens\wdx{*aparent}{adj. `qui apparaît, se montre
clairement aux yeux; apparent'}{apparens
\emph{f.pl.}}
et toutes denuees\wdx{denüer}{v.tr. `priver de'}{}
de char, et s'il
y a poincture\wdx{*pointure}{f. `petite
blessure faite par ce qui pique; piqûre'}{poincture},
elle
peult\wdx{*pooir}{v.tr. +
inf. `avoir la possibilité de (faire qch.);
pouvoir'}{peult \emph{3.p.sg. ind.prés.}}
enge\emph{n}drer\wdx{engendrer}{v.tr. au fig. `faire
naître, faire
exister; produire'}{} spasme\wdx{spasme}{m. terme
de méd. `contraction brusque et violente,
involontaire, d'un ou de plusieurs muscles'}{} et,
p\emph{ar} consequent\wdx{consequent}{adj.}{\textbf{par
consequent}
\emph{loc.adv. `comme suite logique'}},
mort\wdx{morir}{v.intr. `cesser de vivre;
mourir'}{}, si co\emph{m}me Galien\adx{Galien}{}{} le dit
ou tiers livre de \flq Tegny\frq\ et ou quart de
\flq Terapeutique\frq\wdx{*therapeutique}{f.
terme de méd.
`partie de la médecine qui étudie et
met en application les moyens propres à guerir et à
soulager les malades'}{terapeutique}.
Item\wdx{item}{adv.
terme d'admin. `et de même, et aussi (introduction d'unités
traitées l'une après l'autre dans un traité, une ordonnance,
un doc. coutumier, etc.)'}{},
les lacertes\wdx{lacerte}{m. terme d'anat. `structure
organique contractile qui assure les mouvements; muscle'}{} ou
lé muscules se
different\wdx{*diferer}{v.pron. `être
différent'}{different \emph{3.p.pl. ind.prés.}} en
.v. choses, c'est assavoir\wdx{assavoir}{v.tr.}{\textbf{c'est
assavoir} \emph{loc.
`c'est-à-dire'}}
en
qua\emph{n}tité\wdx{*cantité}{f. `nombre
d'unités ou mesure qui sert à déterminer une
portion de matière ou une collection de choses
considérées comme
homogènes'}{quantité}, en
figure\wdx{figure}{f.
`forme extérieure d'un corps'}{},
en posicion\wdx{posicïon}{f. `lieu où quelque chose est placée,
située'}{posicion}, en
co\emph{m}posicion\wdx{composicion}{f.
`manière dont une chose est formée, par l'assemblage de
plusieurs éléments'}{}
et en origine\wdx{origine}{f. `endroit d'où
quelque chose provient'}{} et
naissa\emph{n}ce\wdx{naissance}{f. terme d'anat. `endroit où
commence
qch. (en parlant des membres, organes ou structures organiques
du corps)'}{}
de cordes, si co\emph{m}me le dit Haly\adx{Haliabas}{}{Haly}
en la p\emph{re}miere
p\emph{ar}tie, ou tiers s\emph{er}mo\emph{n}\wdx{sermon}{m. `relation
orale ou écrite; récit'}{} du livre \flq De
regali disposicione\frq . Et pour ce dit
Galien\adx{Galien}{}{} ou .vj.\hoch{e}
du livre \flq De utilitate\frq : il sont .iiij.
posicio\emph{n}s\wdx{posicïon}{f. `lieu où quelque chose est placée,
située'}{posicion}
[18v\hoch{o}b] de muscles, c'est assavoir\wdx{assavoir}{v.tr.}{\textbf{c'est
assavoir} \emph{loc.
`c'est-à-dire'}}
la
droite\wdx{droit}{adj. 1\hoch{o} `qui est sans
déviation d'un bout à l'autre'}{},
la traverce\wdx{*traverse}{adj. `qui est dans une
position transversale ou oblique par rapport à un axe
de position ou de direction habituel'}{traverce} et
deux obliques\wdx{oblique}{adj. `qui s'écarte de la
verticale et de l'horizontale'}{} ou
tortues\wdx{tortu}{adj. `qui présente des courbes
irrégulières; tortu'}{}, et se sont
.v.\hoch{c}xxxj. muscule,
si co\emph{m}me le dit Avicene\adx{Avicene}{}{} ou
premier livre de l'anathomie des
muscles.
\pend
%
% \memorybreak
%
%
\pstartueber
Le tiers chappitre, c'est de  l'anathomie des
nerfz, des liguemans et des cordes.
\pendueber
%
%
\pstart
POUR CE que les muscules sont
composees\wdx{composer}{v.tr.
`former par l'assemblage, par la combinaison de
parties'}{composé \emph{p.p.}}
de
nerfz, de liguemans et de char, aprés la anathomie
de la char musculeuse\wdx{musculeux}{adj. terme d'anat. `qui est de la nature
des muscles'}{}, il nous
co\emph{n}vient\wdx{*covenir}{v.tr.indir. `être convenable
pour
(qn)'}{convient \emph{3.p.sg. ind.prés.}} dire\wdx{dire}{v.tr. `lire à haute voix;
réciter'}{} la anathomie des
nerfz,
\text{des liguemans et des
cordes. Le nerf}\lemma{des liguemans{\dots}
nerf}\fnb{Am
Zeilenrand nachgetragen; ms. \emph{liguemas}.}/,
c'est ung me\emph{m}bre simple qui est fait
pour donner\wdx{*doner}{v.tr. `mettre (qch.) à la disposition de qn,
de qch.'}{donner \emph{inf.}}
sens\wdx{sens}{m. 1\hoch{o}
`faculté d'éprouver
les impressions que font les objets matériels, i.e.
goût, odorat, ouïe, toucher, vue'}{}
et mouvemant\wdx{*movement}{m.
`changement de position ou
de place effectué par un organisme ou une de ses
parties'}{mouvemant}
creés\wdx{*crïer}{v.tr. `donner l'être,
la vie, l'existence à'}{creé \emph{p.p.}} aux
\text{muscules}\fnb{Ms.
\emph{muscles} mit über der Zeile nachgetragenem
\emph{u}.}/
et aux aultres parties. Et pour ce Galien\adx{Galien}{}{}  disoit
ou quart livre de \flq Utilitate particula\emph{rum}\frq , ou
darnier\wdx{*derrenier}{adj.
`qui vient
après tous les autres, après lequel il n'y a pas
d'autre' (temporel ou spatial)}{darnier}
chappitre: y sont trois ente\emph{n}cions\wdx{entencïon}{f.
2\hoch{o} `fait de se proposer un certain but'}{entencion}
que
nature\wdx{nature}{f. 2\hoch{o}
`principe actif qui
anime, organise l'ensemble des choses existantes selon
un certain ordre'}{} ha
distribué\wdx{distribuer}{v.tr.
`diviser
(qch.)
entre plusieurs personnes, lieux, etc., en donnant
une part à chacun; distribuer'}{} aux
nerfz\wdx{nerf}{m. terme d'anat.
`structure blanchâtre en forme de fil qui relie soit
un muscle à un os, soit un centre nerveux
(cerveau, moelle) à un organe ou une structure organique;
tendon ou nerf'}{nerfz \emph{pl.}}.
L'une si est grace de\wdx{grace}{f.}{\textbf{grace de} \emph{loc.prép. `pour cause de'}}
sentemant\wdx{*sentement}{m.
`faculté d'éprouver
les impressions que font les objets matériels, i.e.
goût, odorat, ouïe, toucher, vue'}{sentemant} es
organes\wdx{*orgene}{m.
terme d'anat.
`partie du corps
remplissant une fonction particulière;
organe'}{organe}
sencitives\wdx{*sensitif}{adj.
2\hoch{o} `qui a rapport aux sens'}{sencitives
\emph{m.pl.}}. L'autre si est grace\wdx{grace}{f.}{\textbf{grace de} \emph{loc.prép. `pour cause de'}}
de mouvema\emph{n}t\wdx{*movement}{m. `changement de
position ou
de place effectué par un organisme ou une de ses
parties'}{mouvemant}
es motives\wdx{motif}{adj. `qui fait
mouvoir'}{motives \emph{f.pl.}}.
Et la tierce, qui est en tous les aultres
organes\wdx{*orgene}{m. terme d'anat.
`partie du corps
remplissant une fonction particulière;
organe'}{organe} ou
instrumens\wdx{instrument}{m. 2\hoch{o}
`partie du corps remplissant une fonction
particulière; organe'}{},
c'est a cognoistre\wdx{*conoistre}{v.tr. 1\hoch{o}
`avoir une idée de (qch.); connaître'}{cognoistre
\emph{inf.}} les choses tristes. Et fut  bien
dit ce qu'il disoit des organes\wdx{*orgene}{m.
terme d'anat.
`partie du corps
remplissant une fonction particulière;
organe'}{organe}
sencitifz\wdx{*sensitif}{adj.
2\hoch{o} `qui a rapport aux sens'}{sencitifz
\emph{m.pl.}}, car les nerfz ne sont point\wdx{point}{m. `endroit fixé et déterminé (où qch. à
lieu)'}{\textbf{ne{\dots} point} \emph{adv. de la négation `ne{\dots}
pas'}}
es
cartillages\wdx{*cartilage}{m. terme d'anat. `variété de tissu conjonctif,
translucide, résistant mais élastique, ne contenant ni vaisseaux
ni nerfs, qui recouvre les surfaces osseuses des articulations et qui constitue la charpente
de certaines organes et le squelette de certains vertébrés inférieurs;
cartilage'}{cartillage}
ne es os\wdx{os}{m. `chacune
des pièces rigides qui constituent le squelette de l'homme et des
animaux vertébrés'}{} ne en
pluseurs\wdx{*plusor}{adj. `un certain
nombre'}{pluseurs \emph{pl.}} chars
gla\emph{n}dellouses\wdx{*glandulos}{adj. `qui contient des
glandes'}{glandellouse
\emph{f.sg.}}. Mais y
semble\wdx{sembler}{v.tr.
`avoir des traits communs avec; ressembler'}{\textbf{semble que} \emph{v.impers.
`il paraît que'}}
que y soie\emph{n}t
[19r\hoch{o}a]
plantés\wdx{planter}{v.tr. `fixer (qch.)'}{} es
dens\wdx{dent}{m. et f.
`organe de la bouche, de
couleur blanchâtre, dur, implanté sur le
maxillaire'}{dens \emph{pl.}},
si co\emph{m}me dit est ou .vj.\hoch{e} du livre
devant\wdx{devant}{adv. 2\hoch{o} qui marque
la priorité dans le
temps `auparavant'}{}
dit et
allegué\wdx{*aleguer}{v.tr. `rapporter un
passage, un texte (écrit d'une autorité);
citer'}{allegué \emph{p.p.}}.
Item\wdx{item}{adv.
terme d'admin. `et de même, et aussi (introduction d'unités
traitées l'une après l'autre dans un traité, une ordonnance,
un doc. coutumier, etc.)'}{}, les nerfz\wdx{nerf}{m. terme d'anat.
`structure blanchâtre en forme de fil qui relie soit
un muscle à un os, soit un centre nerveux
(cerveau, moelle) à un organe ou une structure organique;
tendon ou nerf'}{nerfz \emph{pl.}}
naissent\wdx{naistre}{v.tr.indir. terme d'anat.
`émaner de (en parlant
des membres, organes ou structures organiques du
corps)'}{naissent \emph{3.p.pl. ind.prés.}} ou se
manifestent\wdx{manifester}{v.pron. `se révéler
clairement dans son existence ou sa nature'}{} tous
du cervel, par lui ou par la nuque\wdx{nuche}{f.
terme d'anat.
`substance moelleuse de l'intérieur
de l'épine dorsale; moelle épinière'}{nuque},
qui est vicaire\wdx{vicaire}{m. `ce qui exerce en
second les fonctions de qch. autre'}{} du cervel, et
a ce s'acorde\wdx{acorder}{v.pron. `se mettre d'accord ou en accord;
s'accorder'}{}
toute \emph{com}mune\wdx{*comun}{adj. `qui appartient
à plusieurs personnes ou
choses'}{commun}
escole\wdx{escole}{f. `établissement où l'on
enseigne'}{} des
philosophes\wdx{*filosofe}{m.
`personne
qui s'adonne à l'étude rationnelle de la
nature et de la morale'}{philosophe} et des
medicins\wdx{*medecin}{m. 1\hoch{o}
`personne habilitée à
exercer la médecine'}{medicin}. Item\wdx{item}{adv.
terme d'admin. `et de même, et aussi (introduction d'unités
traitées l'une après l'autre dans un traité, une ordonnance,
un doc. coutumier, etc.)'}{},
aucuns
nerfz naisse\emph{n}t\wdx{naistre}{v.tr.indir. terme d'anat.
`émaner de (en parlant
des membres, organes ou structures organiques du
corps)'}{naissent \emph{3.p.pl. ind.prés.}}
du cervel de la partie
de devant\wdx{devant}{adv. 1\hoch{o} `au côté du
visage, à la face'}{\textbf{de devant}
\emph{loc.adj. `qui est situé au côté du visage, de la
face'}} et sont les
\text{plus moulz}\fnb{Nach \emph{plus}
gestrichener Buchstabenansatz.}/\wdx{mol}{adj.
`qui cède facilement à la pression, au toucher; mou'}{moulz \emph{m.pl.}},
ainsi q\emph{ue}\wdx{ainsi}{adv. `de cette façon'}{\textbf{ainsi que}
\emph{loc.conj.
`de la même façon que'}}
est ycelle p\emph{ar}tie, et
sont les plus \text{appers}\fnb{\emph{r} über der Zeile
nachgetragen.}/\wdx{*apert}{adj. `qui est visible,
évident'}{appers \emph{m.pl.}}
pour donner\wdx{*doner}{v.tr. `mettre (qch.) à la disposition de qn,
de qch.'}{donner \emph{inf.}}
sens\wdx{sens}{m. 1\hoch{o}
`faculté d'éprouver
les impressions que font les objets matériels, i.e.
goût, odorat, ouïe, toucher, vue'}{}.
Les
aucuns vie\emph{n}nent de la partie de
darrier\wdx{derrier}{adv. `du côté
opposé au
visage, à la face'}{\textbf{de derrier}
\emph{loc.adj.
`qui est situé au côté opposé au visage, à la face'}
de darrier},
de la nuque\wdx{nuche}{f.
terme d'anat.
`substance moelleuse de l'intérieur
de l'épine dorsale; moelle épinière'}{nuque}
qui descend\wdx{descendre}{v.tr.indir. `aller du haut
en bas'}{} du cervel, et telz\wdx{tel}{adj. `qui est
semblable, du même genre; tel'}{telz \emph{m.pl.}}
nerfz
sont plus
durs\wdx{dur}{adj. `qui résiste à la pression, qui ne se
laisse pas déformer facilement'}{} et plus
manifés\wdx{manifest}{adj. `dont l'existence ou la nature
est évidente; manifeste'}{manifés
\emph{m.pl.}} a donner\wdx{*doner}{v.tr. `mettre (qch.) à la disposition de qn,
de qch.'}{donner \emph{inf.}}
mouvemant\wdx{*movement}{m. `changement de position
ou de place effectué par un organisme ou une de ses
parties'}{mouvemant}. Mais l'en
pourroit demander, se le
sentemant\wdx{*sentement}{m.
`faculté d'éprouver
les impressions que font les objets matériels, i.e.
goût, odorat, ouïe, toucher, vue'}{sentemant}
et le mouvemant\wdx{*movement}{m. `changement de
position ou
de place effectué par un organisme ou une de ses
parties'}{mouvemant}
sont portés\wdx{porter}{v.tr. `déplacer (qch.)
d'un lieu à un autre en le menant avec soi;
transporter'}{} par ung
nerf ou par pluseurs\wdx{*plusor}{adj. `un certain
nombre'}{pluseurs \emph{pl.}}
\text{nerfz}\fnb{Am Zeilenrand
nachgetragen.}/.
Et semble que\wdx{sembler}{v.tr.
`avoir des traits communs avec; ressembler'}{\textbf{semble
que} \emph{v.impers. `il paraît que'}}
Galien\adx{Galien}{}{} dise ou premier livre qui
se intitule\wdx{intituler}{v.pron. `avoir
pour titre'}{}
\flq De interiorib\emph{us}\frq\ et ou quart livre  qui
parle de maladie\wdx{maladie}{f.
 `altération organique ou
fonctionnelle considérée dans son évolution, et comme
une entité définissable; maladie'}{}, que aucune
fois\wdx{*foiz}{f. `cas où un fait se produit, moment du temps où un
événement, conçu comme identique à d'autres
événements, se produit; fois'}{fois} c'est par ung
nerf et aucunes fois\wdx{*foiz}{f. `cas où un fait se produit, moment du temps où un
événement, conçu comme identique à d'autres
événements, se produit; fois'}{fois}
par pluseurs\wdx{*plusor}{adj. `un certain
nombre'}{pluseurs \emph{pl.}}. Et ceste
opinion\wdx{opinïon}{f. `maniére de penser, de juger;
opinion'}{opinion} tient\wdx{tenir}{v.tr. 2\hoch{o}
`considérer (qch.)'}{}
l'escole\wdx{escole}{f.
`établissement où l'on
enseigne'}{} de Montpellier. Et ceste
mat\emph{ier}e\wdx{matiere}{f. 2\hoch{o} au fig. `ce qui
constitue l'objet, le point de départ ou
d'application de la pensée'}{}
ci est difficille\wdx{*dificile}{adj. `qui
n'est pas facile; difficile'}{difficille} et est
encores
plus difficile\wdx{*dificile}{adj.
`qui
n'est pas facile; difficile'}{difficile} chose
a investiguer\wdx{investiguer}{v.tr. `examiner
avec soin pour découvrir ce qui est caché'}{}, se le
mouvemant\wdx{*movement}{m. `changement de position
ou de place effectué par un organisme ou une de ses
parties'}{mouvemant}
et le sentemant\wdx{*sentement}{m.
`faculté d'éprouver
les impressions que font les objets matériels, i.e.
goût, odorat, ouïe, toucher, vue'}{sentemant} sont
portés\wdx{porter}{v.tr.
`déplacer (qch.)
d'un lieu à un autre en le menant avec soi;
transporter'}{}
substanciellemant\wdx{*substancïeument}{adv.
`quant à la substance'}{substanciellemant} ou
radialma\emph{n}t\wdx{*radialement}{adv.
`en manière de rayons'}{radialmant},
et pour ce il vault mieulx q\emph{ue} nous le laissons
poser\wdx{poser}{v.intr. `rester immobile de
manière à se délasser; reposer' (dit d'un
abstrait)}{}. Et quoy que l'en die, il sont
.vij. pareilz\wdx{pareil}{m.
`réunion de deux choses,
de deux êtres semblables
qui vont ensemble; paire'}{pareilz \emph{pl.}} de
nerfz qui naissent\wdx{naistre}{v.tr.indir. terme d'anat.
`émaner de (en parlant
des membres, organes ou structures organiques du
corps)'}{naissent \emph{3.p.pl. ind.prés.}}
sans
moyen\wdx{moien}{m. 1\hoch{o} `ce qui sert
d'intermédiaire'}{moyen} du cervel, et
.xxx. pere qui
naisse\wdx{paire}{f. `réunion de deux choses,
de deux êtres semblables
qui vont ensemble; paire'}{pere}
parmi la
[19r\hoch{o}b] nuque\wdx{nuche}{f. terme d'anat.
`substance moelleuse de l'intérieur
de l'épine dorsale; moelle épinière'}{nuque}, et ung nerf qui
est sans compaignon\wdx{*compagnon}{m.
`ce qui existe en relation étroite avec
une autre chose'}{compaignon}, qui naist parmi
la fin de l'ossaire\wdx{ossaire}{subst.
terme d'anat. `os formé par la réunion des cinq
vertèbres sacrées, a la partie inférieure de la
colonne vertébrale; sacrum' (?)}{}, si
co\emph{m}me le dit Haliabas\adx{Haliabas}{}{} en la seconde
parole\wdx{parole}{f.
`ensemble de mots
qui expriment une idée'}{} de la p\emph{re}miere partie du livre
qui parle
de \flq Disposicion\wdx{disposicïon}{f. 2\hoch{o}
`action de disposer, résultat de cette action'}{}
regal\frq\wdx{regal}{adj. `qui concerne le roi, qui
est du roi; royal'}{}.
\pend
\pstart
Les liguemans \text{sont}\fnb{Ms.
\emph{son}.}/ de la nature\wdx{nature}{f.
1\hoch{o} `ensemble des
caractères, des propriétés qui définissent un être,
une chose concrète ou abstraite'}{}
des nerfz. Toutesvoies\wdx{*totes voies}{loc.adv.
`en
considérant toutes les raisons, toutes les
circonstances qui pourraient s'y opposer, et
malgré elles; toutefois'}{toutesvoies}, ilz naissent\wdx{naistre}{v.tr.indir.
terme d'anat.
`émaner de (en parlant
des membres, organes ou structures organiques du
corps)'}{naissent \emph{3.p.pl. ind.prés.}}
des os\wdx{os}{m. `chacune des pièces rigides qui constituent le
squelette de l'homme et des animaux vertébrés'}{}; desquelx en
sont deux manieres\wdx{maniere}{f. 1\hoch{o} `nature
propre
à plusieurs personnes ou choses, qui permet de les
considérer comme appartenant à une catégorie
distincte'}{}. Aucuns
lient\wdx{*liier}{v.tr. `entourer plusieurs choses avec un lien pour
qu'elles tiennent ensemble'}{lient \emph{3.p.pl.
ind.prés.}} les os\wdx{os}{m.
`chacune des pièces rigides
qui constituent le squelette de l'homme et des animaux vertébrés'}{}
par dedans\wdx{dedans}{adv. `à
l'intérieur'}{}, les autres
lient\wdx{*liier}{v.tr. `entourer plusieurs choses avec un lien pour
qu'elles tiennent ensemble'}{lient \emph{3.p.pl.
ind.prés.}}
par dehors\wdx{*defors}{adv. `à
l'extérieur'}{dehors}
toute la joincture, et ainsi\wdx{ainsi}{adv. `de cette façon'}{} le
disoit Galien\adx{Galien}{}{}
ou
.xij. livre qui se intitule\wdx{intituler}{v.pron.
`avoir pour titre'}{}
\flq De utilitate
particula\emph{rum}\frq , ou \text{p\emph{re}mier}\fnb{Ms.
\emph{p}re\emph{mie}.}/
chappitre, ou il
dit:
\emph{ossiu\emph{m} articulacio \emph{et} c\emph{etera}}. C'est
a dire\wdx{dire}{v.tr.
`lire à haute voix;
réciter'}{\textbf{c'est a dire} \emph{loc.conj. qui
annonce une explication ou une précision}},
la
articulacion\wdx{*articulation}{f.
terme d'anat.
`partie du corps formée par la jointure entre
deux ou plusieurs os; articulation'}{articulacion}
des
os\wdx{os}{m. `chacune des pièces rigides qui constituent le
squelette de l'homme et des animaux vertébrés'}{} est
compri\emph{n}se\wdx{comprendre}{v.tr. 2\hoch{o} `être
autour de (qch.) de manière à enfermer ou embrasser partiellement ou
complètement; entourer'}{comprinse
\emph{p.p. f.sg.}}
tout
outour\wdx{*au tor}{loc.adv. `(dans l'espace)
en environnant (qch.)'}{outour} de
fors\wdx{fort}{adj. 1\hoch{o} `qui résiste; fort (de
choses)'}{fors \emph{m.pl.}}
liguemans et remisses\wdx{remisse}{adj. `qui est
relaxé, décontracté' (d'un muscle)}{}.
\pend
\pstart
Les cordes ou
les
tenans\wdx{*tendant}{m. terme d'anat. `structure conjonctive fibreuse par laquelle
un muscle s'insère sur un os'}{tenans \emph{pl.}} -- qui
sont tout ung
--, ilz sont aussi de
la nature\wdx{nature}{f.
1\hoch{o} `ensemble des
caractères, des propriétés qui définissent un être,
une chose concrète ou abstraite'}{}
des nerfz et pl\emph{us} que ne sont les liguemans,
\text{car}\fnb{Am Foliorand \emph{Nota}
(und Kürzel\,?).}/ tout
ainsi que\wdx{ainsi}{adv. `de cette façon'}{\textbf{ainsi que}
\emph{loc.conj.
`de la même façon que'}}
les liguemans so\emph{n}t
moiens\wdx{moien}{adj. `qui est au milieu'}{} entre
les nerfz et les os\wdx{os}{m. `chacune des pièces rigides qui
constituent le squelette de l'homme et des animaux vertébrés'}{},
ainsi les
cordes et les tenans
so\emph{n}t moiens\wdx{moien}{adj.
`qui est au milieu'}{}
entre les liguemans et les nerfz; et
naissent\wdx{naistre}{v.tr.indir. terme d'anat.
`émaner de (en parlant
des membres, organes ou structures organiques du
corps)'}{naissent \emph{3.p.pl. ind.prés.}}
des muscules et ont leur sens\wdx{sens}{m. 1\hoch{o}
`faculté d'éprouver
les impressions que font les objets matériels, i.e.
goût, odorat, ouïe, toucher, vue'}{}
et leur
mouvema\emph{n}t\wdx{*movement}{m. `changement de position
ou de place effectué par un organisme ou une de ses
parties'}{mouvemant}
des
nerfz, par lequel sens\wdx{sens}{m. 1\hoch{o}
`faculté d'éprouver
les impressions que font les objets matériels, i.e.
goût, odorat, ouïe, toucher, vue'}{}
et
mouvemant\wdx{*movement}{m. `changement de position
ou de place effectué par un organisme ou une de ses
parties'}{mouvemant} ilz
mouvent  les
me\emph{m}bres. Et si co\emph{m}me dit est,
nonobstant\wdx{nonobstant}{adj.}{\textbf{nonobstant
que} \emph{loc.conj. `malgré que'}}
qu'ilz soient rons\wdx{*rëont}{adj. `qui a la forme
circulaire'}{rons \emph{m.pl.}} quant
issent\wdx{issir}{v.tr.indir. `aller hors de' (dit d'une chose)}{}
des muscules,
toutesvoies\wdx{*totes voies}{loc.adv.
`en
considérant toutes les raisons, toutes les
circonstances qui pourraient s'y opposer, et
malgré elles; toutefois'}{toutesvoies}
ilz se
eslargissent\wdx{eslargir}{v.pron. `devenir plus large'}{eslargissent
\emph{3.p.pl. ind.prés.}}
quant il viennent aux articles\wdx{article}{m. et f.
terme d'anat.
`partie du corps formée par la jointure entre
deux ou plusieurs os; articulation'}{}, et sont
assis\wdx{asseoir}{v.tr. 2\hoch{o} `placer, poser
(qch.)'}{assis \emph{p.p.}}
tout
autour\wdx{*au tor}{loc.adv. `(dans l'espace)
en environnant (qch.)'}{autour} du membre. Et quant
ceulx qui so\emph{n}t
[19v\hoch{o}a]
dedans\wdx{dedans}{adv. `à
l'intérieur'}{}
traient\wdx{traire}{v.tr. `faire venir dans une
certaine direction (qn, qch.)'}{} le me\emph{m}bre, ceulx
\text{qui}\fnb{Nachfolgendes \emph{qi} am Zeilenbeginn
expungiert.}/ sont dehors\wdx{*defors}{adv. `à
l'extérieur'}{dehors} se
estendent\wdx{estendre}{v.pron. `augmenter en
longueur ou en largeur'}{} et ainsi, quant l'ung
tire\wdx{tirer}{v.tr. empl.abs. `exercer un
effort sur qch. de manière à allonger, à tendre
ou à faire mouvoir; tirer'}{},
l'autre
lache\wdx{*laschier}{v.intr. `devenir moins tendu ou
moins serré'}{lache \emph{3.p.sg. ind.prés.}}. Et
p\emph{ou}r ce, quant on
coupe ung tel\wdx{tel}{adj. `qui est
semblable, du même genre; tel'}{}
membre par dehors\wdx{*defors}{adv. `à
l'extérieur'}{dehors},  il ne se
peult\wdx{*pooir}{v.tr. +
inf. `avoir la possibilité de (faire qch.);
pouvoir'}{peult \emph{3.p.sg. ind.prés.}}
ploier\wdx{ploier}{v.pron. `se courber'}{}, et
quant on le coupe par dedens,
il ne se peult estendre\wdx{estendre}{v.pron.
`augmenter en
longueur ou en largeur'}{}.
Ainsi le dit
Galie\emph{n}\adx{Galien}{}{} ou
.xij.\hoch{e} livre qui se intitule\wdx{intituler}{v.pron. `avoir
pour titre'}{}
\flq De utilitate
particula\emph{rum}\frq .
\pend
%
% \memorybreak
%
\pstartueber
Le \text{quart}\fnb{\emph{t} über der Zeile nachgetragen.}/
chappitre parle de l'anathomie des
vaines\wdx{*veine}{f. terme d'anat. `vaisseau
sanguin ou, spécialement, vaisseau sanguin
qui part du foie et qui
distribue
le sang nutritif du foie à tout le
corps'}{vaine}
et arteres\wdx{artere}{f. terme d'anat.
 `vaisseau sanguin qui part du c\oe ur et
qui distribue le sang, qui contient les esprits
vitaux,
à tout le corps'}{}.
\pendueber
%
%
\pstart
JA SOIT CE que\wdx{ja soit ce que}{loc.conj.
`bien que assurément'}{}
les vaines\wdx{*veine}{f. terme d'anat. `vaisseau
sanguin ou, spécialement, vaisseau sanguin
qui part du foie et qui distribue le sang
nutritif du foie à tout le
corps'}{vaine}
et les arteres\wdx{artere}{f. terme d'anat.
 `vaisseau sanguin qui part du c\oe ur et
qui distribue le sang, qui contient les esprits
vitaux,
à tout le corps'}{}
-- selon l'entencion\wdx{entencïon}{f.
1\hoch{o} `manière de penser; opinion'}{entencion}
de Galien\adx{Galien}{}{} ou
.xvj.\hoch{e} du livre \flq De
utilitate particula\emph{rum}\frq\ -- se
different\wdx{*diferer}{v.pron.
`être différent'}{different \emph{3.p.pl. ind.prés.}} quant
aux principes\wdx{principe}{m.
`commencement, première manifestation (d'une
chose); origine'}{}, car
les vaines\wdx{*veine}{f. terme d'anat. `vaisseau
sanguin ou, spécialement, vaisseau sanguin
qui part du foie et qui distribue le sang
nutritif du foie à tout le
corps'}{vaine}
naisse\emph{n}t\wdx{naistre}{v.tr.indir. terme d'anat.
`émaner de (en parlant
des membres, organes ou structures organiques du
corps)'}{naissent \emph{3.p.pl. ind.prés.}}
du foie\wdx{foie}{m. terme d'anat. `organe
situé dans la partie supérieure droite de
l'abdomen et qui sécrète la bile;
foie'}{}
et les arteres\wdx{artere}{f. terme d'anat.
 `vaisseau sanguin qui part du c\oe ur et
qui distribue le sang, qui contient les esprits
vitaux,
à tout le corps'}{}
du cuer\wdx{cuer}{m. terme d'anat. `viscère de forme de cône
renversé, situé entre les poumons, qui est l'organe central de la
distribution du sang dans le corps'}{}. Et en aucuns
lieux\wdx{lieu}{m.
`portion
déterminée de l'espace; lieu'}{}, la
vaine\wdx{*veine}{f. terme d'anat. `vaisseau
sanguin ou, spécialement, vaisseau sanguin
qui part du foie et qui distribue le sang
nutritif du foie à tout le
corps'}{vaine}
est separee\wdx{separer}{v.tr. `mettre à part les unes
des autres des choses, des personnes
réunies; séparer'}{} de
l'artere\wdx{artere}{f. terme d'anat.
 `vaisseau sanguin qui part du c\oe ur et
qui distribue le sang, qui contient les esprits
vitaux,
à tout le corps'}{},
si co\emph{m}me au bras\wdx{bras}{m. `membre supérieur de
l'homme
comprenant le segment soit entre l'épaule et le coude, soit entre
le coude et la main, soit les deux segments ensemble avec la main'}{}
et la rethine\wdx{*retine}{f. 2\hoch{o}
terme d'anat. `lacis rétiforme formé de
vaisseaux sanguins,
de fibres, de nerfs, etc.; réseau'}{rethine},
et en nul aultre lieu\wdx{lieu}{m.
`portion
déterminée de l'espace; lieu'}{}
on ne treuve\wdx{*trover}{v.tr. `rencontrer qn ou qch. qu'on cherche;
trouver'}{treuve \emph{3.p.sg. ind.prés.}}
l'artere\wdx{artere}{f. terme d'anat.
 `vaisseau sanguin qui part du c\oe ur et
qui distribue le sang, qui contient les esprits
vitaux,
à tout le corps'}{}
sans vaine\wdx{*veine}{f. terme d'anat. `vaisseau
sanguin ou, spécialement, vaisseau sanguin
qui part du foie et qui distribue le sang
nutritif du foie à tout le
corps'}{vaine}.
Et ont ainsi que\wdx{ainsi}{adv. `de cette façon'}{\textbf{ainsi que}
\emph{loc.conj.
`de la même façon que'}}
une
co\emph{m}munauté\wdx{*comunauté}{f. `caractère de ce qui est
commun'}{communauté}
et une distribucion\wdx{distribucïon}{f. `action
de distribuer, résultat de cette action;
distribution'}{distribucion}
semblable\wdx{semblable}{adj. `qui ressemble;
semblable'}{} enmi\wdx{enmi}{prép.
`au milieu de'}{} le corps. Et
pour ce il souffit\wdx{*sofire}{v.intr.
`avoir la quantité, la qualité, la force
etc. nécessaire pour (qch.); suffire'}{souffit
\emph{3.p.sg. ind.prés.}}, quant
est au cirurgien\wdx{*cirurgiien}{m. `celui qui exerce la chirurgie'}{cirurgien}, a parler
des\wdx{parler}{\textbf{\emph{parler de}} v.tr.indir. `s'entretenir de; parler de'}{}
deux ensamble\wdx{ensemble}{adv.
`l'un avec l'autre'}{ensamble}. Et
ainsi fit Galien\adx{Galien}{}{} ou
dit
livre devant\wdx{devant}{adv. 2\hoch{o} qui marque
la priorité dans le
temps `auparavant'}{}
allegué\wdx{*aleguer}{v.tr. `rapporter un
passage, un texte (écrit d'une autorité); citer'}{allegué \emph{p.p.}}.
Donc la vaine\wdx{*veine}{f. terme d'anat. `vaisseau
sanguin ou, spécialement, vaisseau sanguin
qui part du foie et qui distribue le sang
nutritif du foie à tout le
corps'}{vaine} n'est aultre
chose mais que le lieu\wdx{lieu}{m.
`portion
déterminée de l'espace; lieu'}{}
ou est le sang\wdx{*sanc}{m. terme de méd.
`liquide visqueux, de couleur
rouge, qui est porté par les vaisseaux dans tout
l'organisme où il joue des rôles multiples (l'une
des quatre humeurs de l'humorisme)'}{sang}
nutritif\wdx{nutritif}{adj. terme de méd. `qui a
rapport
aux esprits qui contrôlent
l'alimentation, le grandissement et la génération de
l'homme'}{},
et l'artere\wdx{artere}{f. terme d'anat.
 `vaisseau sanguin qui part du c\oe ur et
qui distribue le sang, qui contient les esprits
vitaux,
à tout le corps'}{}
n'est autre chose que le sang\wdx{*sanc}{m.
terme de méd.
`liquide visqueux, de couleur
rouge, qui est porté par les vaisseaux dans tout
l'organisme où il joue des rôles multiples (l'une
des quatre humeurs de l'humorisme)'}{sang}
espirituel\wdx{*esperituel}{adj. terme de méd.
`qui contient les esprits vitaux, qui a rapport à la
transformation ou à la diffusion des esprits
vitaux'}{espirituel \emph{m.sg.}},
ch\emph{acu}m le scet. Et quant elles sont nees
[19v\hoch{o}b] de leurs
co\emph{m}mencemans\wdx{*comencement}{m. `première partie de qch.,
celle que d'autres doivent suivre et qu'aucune ne précède
(dans le temps ou dans l'espace)'}{commencemant}, elles se
fourchent\wdx{*forchier}{v.pron. `se diviser en
forme de fourche'}{fourchent \emph{3.p.pl.
ind.prés.}}. Et
l'une partie va en hault\wdx{hault}{adv.
`en un endroit qui est d'une certaine
dimension dans le sens vertical; haut'}{\textbf{en
hault} \emph{loc.adv. `vers la partie haute'}} et
l'autre en bas\wdx{bas}{adv.
`à faible hauteur; bas'}{\textbf{en bas}
\emph{loc.adv. `vers la terre; vers le bas'}}, et
aprés chescu\emph{n}e
partie se divise\wdx{deviser}{v.pron. `se séparer en parties;
se diviser'}{divise \emph{3.p.sg. ind.prés.}}
en aultres p\emph{ar}ties et se
multiplient\wdx{*moutepliier}{v.pron.
`augmenter en nombre, en quantité; se
multiplier'}{multiplient \emph{3.p.pl. ind.prés.}}
ainsi,
jusques a ta\emph{n}t que\wdx{jusques a tant que}{loc.conj.
qui marque
le terme final, la limite que l'on ne dépasse
pas}{} elles
viennent
aux extremités\wdx{extremité}{f.
`partie extrème qui termine une chose'}{} du corps, pour
\text{norrir}\fnb{\emph{norrir} über der Zeile
ersetzt expungiertes \emph{unir}.}/%
\wdx{norrir}{v.tr. `entretenir, faire vivre en
donnant
à manger
ou en procurant les aliments nécessaires à la subsistance;
nourrir'}{} et pour
vivifier\wdx{vivifier}{v.tr. `donner la vie,
l'énérgie vitale à, entretenir la vie, l'énérgie
vitale de'}{\emph{inf.}} tous les membres.
Les vaines
particulieres\wdx{*particuler}{adj. `qui
appartient en propre (à qn, à qch., ou à une catégorie
de personnes, de choses); particulier'}{particulier}
qui, par leur grandesse\wdx{*grandece}{f. dans l'ordre
quantitatif `taille, grandeur'}{grandesse},
font emoragies\wdx{*hemorrhagie}{f. terme
de méd. `écoulement de sang hors d'un
vaisseau'}{emoragie}
[de]
petite\wdx{petit}{adj. 3\hoch{o}
dans
l'ordre qualitatif, non quantifiable `qui est d'un
degré inférieur à la moyenne en ce qui concerne la
qualité, l'intensité,
l'importance'}{}
douleur\wdx{*dolor}{f. `sensation pénible en un
point ou dans une région du corps;
douleur'}{douleur},
elles s\emph{er}ont dictes en la anathomie des grans membres.
\pend
%
% \memorybreak
%
\pstartueber
Veci le quint chappitre: De
l'anathomie des os
et des cartillages\wdx{*cartilage}{m. terme d'anat. `variété de tissu conjonctif,
translucide, résistant mais élastique, ne contenant ni vaisseaux
ni nerfs, qui recouvre les surfaces osseuses des articulations et qui constitue la charpente
de certaines organes et le squelette de certains vertébrés inférieurs;
cartilage'}{cartillage},
des ongles\wdx{ongle}{m. et f. `lame cornée, implantée sur l'extremité dorsale
des doigts et des orteils de l'homme;
ongle'}{}
et des poilz\wdx{poil}{m. `chacune des productions filiformes qui naissent
du tégument de l'homme et des mammifères; poil'}{poilz \emph{pl.}}.
\pendueber
%
%
\pstart
FINAUMENT y nous co\emph{n}vient\wdx{*covenir}{v.tr.indir.
`être convenable pour (qn)'}{convient \emph{3.p.sg. ind.prés.}} dire\wdx{dire}{v.tr.
`lire à haute voix;
réciter'}{}
la anathomie des
os, car ilz sont dedans\wdx{dedans}{prép. `à
l'intérieur de'}{} le corps. Les os sont
la pl\emph{us} dure\wdx{dur}{adj. `qui résiste à la pression, qui ne se
laisse pas déformer facilement'}{} partie de tout le corps et se
sont
le fondemant\wdx{*fondement}{m. `partie d'un corps
sur laquelle il porte, il repose'}{fondemant} et
le soustentacle\wdx{*sustentacle}{m.
`ce qui sert à soutenir; appui'}{soustentacle} de
toutes
les aultres parties. Et avec ce, aucuns sont fais pour
garder\wdx{garder}{v.tr. 1\hoch{o} `préserver qch.
(d'un mal, d'un danger, etc.); protéger'}{} et
deffendre\wdx{*defendre}{v.tr. `protéger (qn, qch.)
contre (qn, qch.)'}{deffendre \emph{inf.}} les
parties de dedans\wdx{dedans}{adv. `à l'intérieur'}{\textbf{de dedans} \emph{loc.adj. `qui est
situé à l'intérieur'}}, si co\emph{m}me
le crane\wdx{crane}{m.
terme d'anat.
`boîte osseuse renfermant l'encéphale'}{}, le
pis\wdx{*piz}{m. terme d'anat. `partie du corps
humain qui s'étend des épaules à l'abdomen et qui contient le c\oe ur
et les poumons; thorax'}{pis} et
le dos\wdx{dos}{m. `partie du corps de l'homme
qui s'étend des épaules jusqu'aux reins, de chaque
côté de la colonne vertebrale; dos'}{}. Donc
selon Avicene\adx{Avicene}{}{} il ha en nostre corps
.ij.\hoch{c}xlvj. os   --
s'ilz so\emph{n}t bien contés\wdx{conter}{v.tr. `déterminer
la quantité (de qch.) par le calcul'}{conté
\emph{p.p.}}
--, si co\emph{m}me il le dit ou
\text{premier}\fnb{Ms. \emph{premieer},
letztes \emph{e} gestrichen.}/ livre de son
\flq Canon\frq , excepté\wdx{excepté}{prép. `à la réserve
de'}{} encores ung
os qui se appelle sizamina\wdx{sizamina}{subst.
terme d'anat. `os qui a la forme d'un grain de
sésame'}{}, et
\text{l'os}\fnb{Voranstehendes \emph{los}
gestrichen.}/ de la lande\wdx{os de la lande}{f.
`os médian impair, en forme de fer de cheval, situé à
la partie antérieure du cou au niveau de l'angle
que forme celui-ci avec le plancher de la
bouche; os hyoïde' (?)}{}, ouquel la
langue\wdx{langue}{f. `organe charnu,
musculeux, allongé et mobile, placé dans la bouche;
langue'}{}
est fondee\wdx{fonder}{v.tr. `appuyer
(qch.)'}{}.
Les os du cors\wdx{cors}{m. `ce qui fait l'existence matérielle d'un
homme ou d'un animal'}{} ont
\text{divercité}\fnb{\emph{r} über der Zeile nachgetragen.}/%
\wdx{*diverseté}{f. `caractère,
état de ce qui est divers'}{divercité} en
nombre\wdx{nombre}{m. `mot servant à caractériser une pluralité de
choses ou de personnes; nombre'}{}
et en fourme\wdx{*forme}{f. `apparence extérieure
donnant à un objet ou à un être sa
spécificité'}{fourme}: aucuns pour
cause\wdx{cause}{f.
`ce qui produit un effet (considéré par rapport à cet
effet)'}{}
de eulx [20r\hoch{o}a] mesmes, aucuns pour
cause\wdx{cause}{f. `ce qui produit
un effet (considéré par rapport à cet
effet)'}{}
des
joinctures, aucuns sont qui sont
plains\wdx{*plein}{adj. `qui contient toute
la quantité possible; plein'}{plain} de
medule\wdx{medulle}{f.
terme d'anat.
`substance moelleuse de l'intérieur d'une
structure osseuse'}{medule},
aucuns n'e\emph{n} ont point\wdx{point}{m. `endroit fixé et déterminé (où qch. à
lieu)'}{\textbf{ne{\dots} point} \emph{adv. de la négation `ne{\dots}
pas'}}.
Et aucuns sont drois,
et aucuns tortus\wdx{tortu}{adj.
`qui présente des courbes
irrégulières; tortu'}{}, et aucuns sont grans,
et aucuns sont petis. Tous les gros\wdx{gros}{adj.
1\hoch{o}
dans l'ordre physique, quantifiable `qui, dans
son genre, dépasse le volume ordinaire; gros (du corps
humain et de ses parties)'}{}
os sont en
joincture, et aucuns ont
divercité\wdx{*diverseté}{f. `caractère,
état de ce qui est divers'}{divercité} au
milieu\wdx{milieu}{m. `partie d'une chose
qui est à égale distance des extrémités de cette
chose'}{}
de la joincture, pour cause\wdx{cause}{f. `ce qui produit
un effet (considéré par rapport à cet
effet)'}{}
de icelle joincture,  car
aucuns ont vacuités\wdx{vacuïté}{f. `espace qui
est vide'}{vacuité} en laquelle ilz ressoyvent\wdx{recevoir}{v.tr. `faire entrer
(qch.)'}{ressoyvent \emph{3.p.pl. ind.prés.}}
les
addicio\emph{n}s\wdx{addicion}{f. `éminence à la surface d'une structure
osseuse ou cartilagineuse'}{}
des aultres. Et aucuns ont
addicions\wdx{addicion}{f. `éminence à la surface d'une structure
osseuse ou cartilagineuse'}{},
et aucuns ont et
l'ung et l'autre, et les aucuns n'ont ne l'um ne
l'autre. Et aucuns ont addicions\wdx{addicion}{f. `éminence à la
surface d'une structure osseuse ou cartilagineuse'}{}
et
vacuités\wdx{vacuïté}{f.
`espace qui
est vide'}{vacuité}
que \text{on appelle}\fnb{Ms. \emph{ont appelle}.}/\wdx{apeler}{v.tr. `donner un nom à qn, qch.; appeler'}{appelle
\emph{3.p.sg. ind.prés.}}
clos\wdx{clo}{m. `ce qui a la forme d'un
petit tige de fer pointu qui sert à fixer qch.'
(par analogie de forme)}{} ou
clavelz\wdx{clavel}{m. 2\hoch{o}
`ce qui a la forme d'un petit
tige de fer pointu qui sert à fixer qch.' (par analogie de forme)}{},
si co\emph{m}me sont les de\emph{n}s\wdx{dent}{m. et f.
`organe de la bouche, de
couleur blanchâtre, dur, implanté sur le
maxillaire'}{dens \emph{pl.}}. Et les
aultres sont appellés
sarrailles\wdx{sarraille}{adj. `qui est en forme de scie'}{},
co\emph{m}me l'os du chief\wdx{chief}{m. 1\hoch{o} `partie
supérieure du corps de l'homme; tête'}{} ou
le cra\emph{n}ne\wdx{crane}{m.
terme d'anat.
`boîte osseuse renfermant l'encéphale'}{cranne}. Et aucuns ont
neux\wdx{*no}{m. terme d'anat. `partie
osseuse saillante dans une
articulation' (par analogie de forme)}{neux \emph{pl.}} ou chief\wdx{chief}{m.
2\hoch{o} `partie
terminale de qch. (d'un os, etc.); bout'}{}, si co\emph{m}me l'os
du bras\wdx{bras}{m. `membre supérieur de l'homme
comprenant
le segment soit entre l'épaule et le coude, soit entre le coude et
la main, soit les deux segments ensemble avec la main'}{} et de
la cuisse\wdx{cuisse}{f.
`partie
de la jambe qui s'articule à la hanche et s'étend
jusqu'au genou; cuisse'}{}. Et aucuns
ont fosses\wdx{fosse}{f. terme d'anat. `concavité
d'assez grandes
dimensions, le plus souvent osseuse, mais pouvant se
trouver
aussi dans d'autres structures anatomiques'}{},
si co\emph{m}me les os focilles\wdx{focile}{m. terme d'anat. `chacun des
deux os de l'avant-bras ou de la jambe; cubitus, radius, tibia,
péroné'}{focille}. Et aucuns ont l'un et l'aut\emph{re},
si co\emph{m}me les os du doy\wdx{doi}{m. 1\hoch{o} `chacun des cinq prolongements
qui terminent la main'}{doy}. Et aucuns
\text{n'ont}\fnb{Nachfolgend gestrichener
Buchstabenansatz.}/ ne l'ung ne l'autre et sont
joins\wdx{joindre}{v.tr. `mettre des choses
ensemble, de façon qu'elles se touchent ou tiennent
ensemble; joindre'}{joins \emph{p.p. m.pl.}} et
soudés\wdx{soudé}{p.p. comme adj. `qui est joint de
manière constante et durable' (de deux parties
organiques)}{} l'ung avec l'autre. Et telz\wdx{tel}{adj. `qui est
semblable, du même genre; tel'}{telz \emph{m.pl.}}
os qui ont ces rondesses\wdx{*rëondece}{f. `état de ce qui est rond;
rondeur'}{rondesse}
et \text{ces}\fnb{Voranstehendes \emph{ces}
gestrichen.}/ fosses\wdx{fosse}{f. terme d'anat.
`concavité d'assez
grandes dimensions, le plus souvent osseuse, mais pouvant se trouver
aussi dans d'autres structures anatomiques'}{},
ilz font p\emph{ro}premant\wdx{*proprement}{adv. `d'une
manière précise'}{propremant}
la joincture,
et en telz\wdx{tel}{adj. `qui est
semblable, du même genre; tel'}{telz \emph{m.pl.}}
os se fait
dislocacion\wdx{dislocacïon}{f.
`déplacement violent d'une partie du
corps (d'un membre, d'un os, d'une
articulation)'}{dislocacion} et es aultres
se fait p\emph{ro}premant\wdx{*proprement}{adv. `d'une
manière précise'}{propremant}
separacion\wdx{separacïon}{f. `action
de séparer, le résultat de cette
action; séparation'}{separacion}.
\pend
\pstart
Le cartillage\wdx{*cartilage}{m. terme d'anat. `variété de tissu conjonctif,
translucide, résistant mais élastique, ne contenant ni vaisseaux
ni nerfs, qui recouvre les surfaces osseuses des articulations et qui constitue la charpente
de certaines organes et le squelette de certains vertébrés inférieurs;
cartilage'}{cartillage}
est ainsi que\wdx{ainsi}{adv. `de cette façon'}{\textbf{ainsi que}
\emph{loc.conj.
`de la même façon que'}}
de la nat\emph{u}re\wdx{nature}{f.
1\hoch{o} `ensemble des
caractères, des propriétés qui définissent un être,
une chose concrète ou abstraite'}{}
de l'os, mais il est
\text{plus}\fnb{Nachfolgend gestrichener
Buchstabenansatz.}/
molz\wdx{mol}{adj.
`qui cède facilement à la pression, au toucher; mou'}{molz
\emph{m.sg.}} et est fait
pour acomplir\wdx{acomplir}{v.tr. `rendre (qch.)
complet'}{}
le deffault\wdx{*defaut}{m. `absence de ce qui serait
nécessaire ou désirable; défaut'}{deffault} de l'os, si
co\emph{m}me es \text{paupieres}\fnb{\emph{u} über der Zeile
nachgetragen.}/%
\wdx{paupiere}{f. `chacune des deux
membranes mobiles qui en se rapprochant recouvrent le
globe de l'\oe il; paupière'}{}, es
nazilles\wdx{*nasilles}{f.pl. `les deux cavités
nasales'}{nazilles} et es
oreilles\wdx{oreille}{f. 1\hoch{o} `l'un des deux organes
constituant l'appareil
auditif, aussi sa partie visible; oreille'}{},
[20r\hoch{o}b]
afin q\emph{ue}\wdx{afin que}{loc.conj. qui marque l'intention,
le but `pour que'}{} la
co\emph{n}ju\emph{n}ction\wdx{*conjoncïon}{f. `action de joindre,
le résultat de cette action'}{conjunction}
des os soit miudre\wdx{*meillor}{adj.comp.
`qui est d'un degré supérieur à bon;
meilleur'}{miudre}
avec les parties voisines\wdx{voisin}{adj. `qui
est à côté; voisin'}{},
si co\emph{m}me au
pis\wdx{*piz}{m.
terme d'anat.
`partie du corps humain qui s'étend
des épaules à
l'abdomen et qui contient le c\oe ur et les
poumons; thorax'}{pis} et es hanches\wdx{hanche}{f.
`chacune des deux parties
du corps formant saillie au-dessous des flancs,
entre la fesse en arrière et le pli de l'aine
en avant; hanche'}{}
et es extremités\wdx{extremité}{f.
`partie extrème qui termine une chose'}{} d'iceulx, afin
que\wdx{afin que}{loc.conj. qui marque l'intention,
 le but `pour que'}{} ilz
ne se cassent\wdx{casser}{v.pron. `se briser'}{} en
leur
mouvemant\wdx{*movement}{m. `changement de position
ou de place effectué par un organisme ou une de ses
parties'}{mouvemant}.
Item\wdx{item}{adv.
terme d'admin. `et de même, et aussi (introduction d'unités
traitées l'une après l'autre dans un traité, une ordonnance,
un doc. coutumier, etc.)'}{}, les ungles\wdx{ongle}{m. et f. `lame cornée, implantée sur l'extremité dorsale
des doigts et des orteils de l'homme;
ongle'}{ungle}
sont fais et assis\wdx{asseoir}{v.tr. 2\hoch{o} `placer, poser
(qch.)'}{assis
\emph{p.p.}}
es extremités\wdx{extremité}{f.
`partie extrème qui termine une chose'}{}
des dois\wdx{doi}{m. 1\hoch{o} `chacun des cinq prolongements
qui terminent la main'}{}
et des me\emph{m}bres po\emph{ur} mieulx
prendre\wdx{prendre}{v.tr. 1\hoch{o} `saisir (avec la main, etc.)'}{} ce
q\emph{ue} l'en
vouldra prendre\wdx{prendre}{v.tr. 1\hoch{o} `saisir (avec la main, etc.)'}{}. Mais les
poilz\wdx{poil}{m.
`chacune des productions filiformes qui naissent
du tégument de l'homme et des mammifères; poil'}{poilz \emph{pl.}}
sont faiz pour
embellir\wdx{*embelir}{v.tr. `rendre (qch., qn)
beau ou plus beau; embellir'}{}
et purger\wdx{purgier}{v.tr. 1\hoch{o} `débarasser un
corps ou une substance d'éléments nuisibles
qui l'altèrent'}{purger \emph{inf.}} le me\emph{m}bre et le
corps.
\pendueber   % war nur pend
%
%
\pstartueber
La seconde doctrine\wdx{doctrine}{f.
`ensemble de
notions qu'on affirme être vraies et par
lesquelles on veut fourner une interprétation
des faits, orienter ou diriger l'action'}{}
parle de
l'anathomie
des me\emph{m}bres compost\wdx{compost}{adj.
terme d'anat.
`qui est
formé
de plusieurs membres simples (dit d'un
membre du corps)'}{compost \emph{m.pl.}}
et p\emph{ro}pres\wdx{propre}{adj. 2\hoch{o} `qui
est particulier à (qn, qch.)'}{}.
\pendueber
%
%
\pstart%ueber
Le premier chappitre est de l'oule\wdx{*ole}{f.
terme d'anat.
`boîte osseuse renfermant l'encéphale'}{oule}
du chief\wdx{chief}{m. 1\hoch{o} `partie
supérieure du corps de l'homme; tête'}{}.
\pendueber
%
%
\pstart
DIT EST devant\wdx{devant}{adv. 2\hoch{o} qui marque
la priorité dans le
temps `auparavant'}{}
de
l'anathomie
des me\emph{m}bres simples
et co\emph{m}muns\wdx{*comun}{adj. `qui appartient
à plusieurs personnes ou
choses'}{commun} a tout le
corps; il nous co\emph{n}vient\wdx{*covenir}{v.tr.indir.
`être
convenable pour (qn)'}{convient \emph{3.p.sg. ind.prés.}}
venir a l'anathomie
des me\emph{m}bres compost\wdx{compost}{adj.
terme d'anat.
`qui est
formé
de plusieurs membres simples (dit d'un
membre du corps)'}{compost
\emph{m.pl.}} et singuliers\wdx{*singuler}{adj. `qui
se rapporte à ou qui concerne un seul;
singulier'}{singulier}.
Et non obstant que\wdx{nonobstant}{adj.}{\textbf{nonobstant
que} \emph{loc.conj. `malgré que'}}
aucuns sont grans
et les aultres petis,
neantmoins\wdx{neantmoins}{adv. `malgré ce qui
vient d'être dit; néanmoins'}{} la ana\mbox{tho}mie
d'iceulx en s\emph{er}a
tractié\wdx{*traitier}{v.tr.indir.
`exposer ses vues sur (un sujet, une
science, etc.)'}{tractié \emph{p.p.}} en
.viij. chappitres
selon la divizion\wdx{*devisïon}{f.
`action de
diviser (qch.) en parties, le résultat de
cette action'}{divizion}
des grans parties. Car telle\wdx{tel}{adj. `qui est
semblable, du même genre; tel'}{telle \emph{f.sg.}}
division\wdx{*devisïon}{f.
`action de
diviser (qch.) en parties, le résultat de
cette action'}{division} est plus
sencible\wdx{*sensible}{adj. `qui peut
être perçu par les sens'}{sencible}
et plus manifeste\wdx{manifest}{adj. `dont l'existence ou la nature
est évidente; manifeste'}{}
et avec ce, la
maniere\wdx{maniere}{f. 2\hoch{o} `forme particulière
que revêt l'accomplissement d'une action, le
déroulement d'un fait, l'être ou
l'existence'}{}
de miegier\wdx{*megier}{v.tr. empl.abs. `traiter
médicalement'}{miegier \emph{inf.}}
se varie\wdx{*variier}{v.pron. `être différent;
différer'}{varie \emph{3.p.sg. ind.prés.}} selon
\text{icelle}\fnb{Voranstehendes
\emph{la} gestrichen.}/ division\wdx{*devisïon}{f.
`action de
diviser (qch.) en parties, le résultat de
cette action'}{division}.
Donc nous co\emph{m}mencerons\wdx{*comencier}{v.tr.indir.
`faire d'abord (qch.)'}{commencerons
\emph{1.p.pl. ind.futur simple}} ou
chief\wdx{chief}{m. 1\hoch{o} `partie supérieure du
corps de l'homme; tête'}{}, maiemant\wdx{*maismement}{adv. `plus que tout
autre chose; surtout'}{maiemant}
au cervel et a l'oule\wdx{*ole}{f.
terme d'anat.
`boîte osseuse
renfermant l'encéphale'}{oule} qui
contient\wdx{contenir}{v.tr. `comprendre en soi,
dans sa capacité, son étendue, sa substance;
contenir'}{contient \emph{3.p.sg. ind.prés.}}
le cervel, pour ce
que icellui lieu\wdx{lieu}{m.
`portion
déterminée de l'espace; lieu'}{}, c'est
la habitacion\wdx{*abitacion}{f. `endroit où siège
qch.' (ici: dit de l'âme rationelle)}{habitacion}
de l'ame racionelle\wdx{racionel}{adj. `qui
appartient à la raison'}{}, si co\emph{m}me
Galien\adx{Galien}{}{} le dit
%
[20v\hoch{o}a]
ou .ix.\hoch{e} livre et ou
quart chappitre du
livre qui se intitule \flq P\emph{ar}ticula\emph{rum}\frq , et ou premier
livre, ou .ix.\hoch{e}
chapitre\wdx{chapitre}{m. `chacune des
parties qui se suivent dans un livre et en articulent la lecture;
chapitre'}{}
\text{du livre}\fnb{Über der Zeile nachgetragen.}/ qui
se intitule \flq De
custodia sanitatis\frq . Et aussi il est escript ou tiers
du livre qui se appelle \flq De interioribus\frq , et ou
\emph{com}ment\wdx{*coment}{m. `ensemble des
explications, des remarques que l'on fait à
propos d'un texte; commentaire'}{comment} du tiers
qui se appelle
\flq De regimine
acuto\emph{rum}\frq .
\pend
\pstart
Item\wdx{item}{adv.
terme d'admin. `et de même, et aussi (introduction d'unités
traitées l'une après l'autre dans un traité, une ordonnance,
un doc. coutumier, etc.)'}{}, en l'oule\wdx{*ole}{f.
terme d'anat.
`boîte osseuse renfermant l'encéphale'}{oule} du
chief\wdx{chief}{m. 1\hoch{o} `partie
supérieure du corps de l'homme; tête'}{}
et en ces
p\emph{ar}ties, \text{il fault y
enquerrir}\fnb{Ms. \emph{il fault y fault
enquerrir}.}/\wdx{*enquerir}{v.tr. `chercher à savoir
(qch.) en examinant ou en interrogeant'}{enquerrir
\emph{inf.}} les
.ix. choses qui so\emph{n}t  dictes ci  devant par le
\text{co\emph{m}menteur}\fnb{Ms. \emph{co}m\emph{menceur}\,? (cp.
GuiChaul\textsc{jl} 26,34 \emph{comment.});
cp. l.\,74.}/\wdx{*comenteor}{m.
`auteur d'un ensemble des explications,
des remarques faites à propos d'un
texte; commentateur'}{commenteur}
Alexandrin\adx{Alexandrin le commenteur}{}{} ou
\flq Livre des sectes\frq\wdx{secte}{f.
`doctrine religieuse ou
philosophique'}{}, lesquelles
on doit \text{enquerir}\fnb{Voranstehend
gestrichenes
\emph{q}.}/\wdx{*enquerir}{v.tr. `chercher à savoir
(qch.) en examinant ou en interrogeant'}{} aussi en
chescum
me\emph{m}bre, c'est assavoir\wdx{assavoir}{v.tr.}{\textbf{c'est
assavoir} \emph{loc.
`c'est-à-dire'}}
la aide\wdx{aide}{f. r\hoch{o} `caractère de ce qui est
utile, qui satisfait un besoin; utilité'}{} du
me\emph{m}bre,
la posicion\wdx{posicïon}{f. `lieu où quelque chose est placée,
située'}{posicion},
la colligance\wdx{colligance}{f. 1\hoch{o}
`force qui maintient réunis les éléments d'un système matériel;
liaison'}{}, la quantité\wdx{*cantité}{f. `nombre
d'unités ou mesure qui sert à déterminer une
portion de matière ou une collection de choses
considérées comme
homogènes'}{quantité}, la
figure\wdx{figure}{f. `forme extérieure d'un corps'}{},
la substance\wdx{sustance}{f. `matière dont un
corps est formé, et en vertu de laquelle
il y a des propriétés particulières;
substance'}{substance},
la complexion\wdx{*complessïon}{f.
`ensemble des
éléments constituant la nature physique d'un
individu, d'une partie du corps ou d'une chose;
complexion'}{complexion}, le nombre\wdx{nombre}{m.
`mot servant à caractériser une pluralité de
choses ou de personnes; nombre'}{}
des parties
et les maladies\wdx{maladie}{f.
 `altération organique ou
fonctionnelle considérée dans son évolution, et comme
une entité définissable; maladie'}{}
qui y peulent
venir\wdx{venir}{v.tr.indir.
1\hoch{o} `avoir
son origine dans'}{}.
L'oule\wdx{*ole}{f. terme d'anat. `boîte osseuse
renfermant l'encéphale'}{oule}
du chief\wdx{chief}{m. 1\hoch{o} `partie
supérieure du corps de l'homme; tête'}{}
selon le
philosophe\wdx{*filosofe}{\textbf{\emph{le
philosophe}} désigne Aristote}{}
est appellé la
partie
capillaire\wdx{capillaire}{adj. `qui est garnie
de cheveux; chevelu'}{},
en laquelle sont co\emph{n}tenus\wdx{contenir}{v.tr.
`comprendre en soi,
dans sa capacité, son étendue, sa substance;
contenir'}{contenu \emph{p.p.}} les
me\emph{m}bre
qui ont ame\wdx{ame}{f. `principe spirituel de l'homme; âme'}{}
et vie\wdx{vie}{f.
`état d'activité propre à tous
les êtres organisés, aussi la durée, la succession des
phénomènes par lesquels cette activité se
manifeste; vie'}{}, et en ce est
demonstree\wdx{demonstrer}{v.tr.
`faire
voir, mettre devant les yeux; montrer'}{}
son aide\wdx{aide}{f. 2\hoch{o}
`caractère de ce qui est
utile, qui satisfait un besoin; utilité'}{}.
Item, sa posicion\wdx{posicïon}{f. `lieu où quelque chose est placée,
située'}{posicion}
ou so\emph{n} siege\wdx{siege}{m. `lieu où qch. réside'}{}
est ou plus hault\wdx{hault}{adj. `qui est d'une
certaine dimension dans le sens vertical; haut'}{}
lieu\wdx{lieu}{m.
`portion
déterminée de l'espace; lieu'}{}
de tout le corps.  Et c'est ce pour
cause\wdx{cause}{f. `ce qui produit
un effet (considéré par rapport à cet
effet)'}{}
des
yeulx\wdx{ueil}{m. `organe de la vue'}{yeulx
\emph{pl.}} ou pour  aultre chose y\wdx{et}{conj. de
coordination `et'}{y} n'ap\emph{ar}tient pas a la
consideracion\wdx{consideracïon}{f. `action
d'examiner ou d'observer (qch.) avec attention, le
résultat de cette action'}{} du
cirurgien\wdx{*cirurgiien}{m. `celui qui
exerce la chirurgie'}{cirurgien}.
Item, sa colligance\wdx{colligance}{f. 1\hoch{o}
`force qui maintient réunis les éléments d'un système matériel;
liaison'}{}
est
manifeste\wdx{manifest}{adj. `dont l'existence ou la nature
est évidente; manifeste'}{}, car c'est avec la face\wdx{face}{f. `partie antérieure
de la tête; visage'}{} et
avec le col\wdx{col}{m. 1\hoch{o} `partie du corps de
l'homme et de certains vertébrés qui unit la
tête au tronc'}{}, car
de la vie\emph{n}nent toutes les p\emph{ar}ties de la face\wdx{face}{f. `partie
antérieure de la tête; visage'}{}, et les
muscules qui mouvent\wdx{*movoir}{v.tr. `mettre en
mouvement'}{mouvent \emph{3.p.pl. ind.prés.}}
le chief\wdx{chief}{m. 1\hoch{o}
`partie supérieure du corps de l'homme; tête'}{}
sont plantés\wdx{planter}{v.tr.
`fixer (qch.)'}{} ou col\wdx{col}{m. 1\hoch{o}
`partie du corps de
l'homme et de certains vertébrés qui unit la
tête au tronc'}{}.
Pour ce est
%
[20v\hoch{o}b]
il escript ou livre de
Hali\adx{Haliabas}{}{Hali}, en la tierce
p\emph{ar}olle\wdx{parole}{f.
`ensemble de mots
qui expriment une idée'}{parolle}
de la premiere partie:  les lacertes\wdx{lacerte}{m.
terme d'anat. `structure organique contractile qui assure
les mouvements; muscle'}{}
qui
mouvent\wdx{*movoir}{v.tr. `mettre en
mouvement'}{mouvent \emph{3.p.pl. ind.prés.}}
le chief\wdx{chief}{m. 1\hoch{o} `partie
supérieure du corps de l'homme; tête'}{}
sont doubles\wdx{*doble}{adj. `qui est
répété deux fois, qui vaut deux fois (la
chose désignée) ou qui existe deux fois'}{double},
car les aucuns mouvent\wdx{*movoir}{v.tr. `mettre en
mouvement'}{mouvent \emph{3.p.pl. ind.prés.}}
p\emph{ro}premant\wdx{*proprement}{adv. `d'une
manière précise'}{propremant}
le chief\wdx{chief}{m. 1\hoch{o} `partie
supérieure du corps de l'homme; tête'}{}
sans les aultres, et ont leur
naissance\wdx{naissance}{f. terme d'anat. `endroit où
commence
qch. (en parlant des membres, organes ou structures organiques
du corps)'}{}
de pres les oreilles\wdx{oreille}{f. 1\hoch{o} `l'un
des deux organes constituant l'appareil
auditif, aussi sa partie visible; oreille'}{} et
vie\emph{n}nent jusques\wdx{jusques a}{loc.prép.
qui marque
le terme final, la limite que l'on ne dépasse
pas}{}
a lla
forcelle\wdx{*forcele}{f. 1\hoch{o} terme d'anat.
`os long, en
forme d'un S allongé, formant la partie antérieure de
la ceinture scapulaire; clavicule'}{forcelle}.
Mais les autres sont co\emph{m}muns\wdx{*comun}{adj. `qui appartient
à plusieurs personnes ou
choses'}{commun}
au chief\wdx{chief}{m. 1\hoch{o} `partie
supérieure du corps de l'homme; tête'}{}
et au col\wdx{col}{m. 1\hoch{o} `partie du corps de
l'homme et de certains vertébrés qui unit la
tête au tronc'}{}, desquelx il s\emph{er}a dit
ou chappitre
du col\wdx{col}{m. 1\hoch{o} `partie du corps de
l'homme et de certains vertébrés qui unit la
tête au tronc'}{}. Item,
la quantité\wdx{*cantité}{f. `nombre
d'unités ou mesure qui sert à déterminer une
portion de matière ou une collection de choses
considérées comme
homogènes'}{quantité} de l'oule\wdx{*ole}{f.
terme d'anat.
 `boîte osseuse
renfermant l'encéphale'}{oule} en
l'o\emph{m}me\wdx{*ome}{m.
`être appartenant à l'espèce
animale la plus évoluée de la terre; être humain'}{omme}
est de plus grant\wdx{grant}{adj. 1\hoch{o}
dans l'ordre
physique, quantifiable `qui est d'une extension
au-dessus de la moyenne; grand (des choses)'}{grant
\emph{f.sg.}} capacité\wdx{capacité}{f. `propriété de
contenir une certaine quantité de substance;
capacité'}{} que de nulle
aultre selon sa quantité\wdx{*cantité}{f. `nombre
d'unités ou mesure qui sert à déterminer une
portion de matière ou une collection de choses
considérées comme
homogènes'}{quantité}.
Item, sa fourme\wdx{*forme}{f. `apparence extérieure
donnant à un objet ou à un être sa
spécificité'}{fourme} est ronde\wdx{*rëont}{adj.
`qui a la forme circulaire'}{ronde \emph{f.sg.}}
en maniere\wdx{maniere}{f. 2\hoch{o} `forme
particulière
que revêt l'accomplissement d'une action, le
déroulement d'un fait, l'être ou
l'existence'}{}
d'une espere\wdx{espere}{f. `surface constituée
par le lieu des points situés à une même distance
d'un point donné; sphère'}{} et
est comp\emph{a}niee\wdx{*compagnier}{v.tr. `se joindre à'
(dit de choses)}{companié \emph{p.p.}} ou
co\emph{m}posee\wdx{composer}{v.tr.
`former par l'assemblage, par la combinaison de
parties'}{composé \emph{p.p.}}
d'une
partie et d'aultre,
auteille\wdx{*autel comme}{adj. `qui est pareil
à'}{auteille comme}
devant\wdx{devant}{adv. 1\hoch{o} `au côté du visage,
à la face'}{} co\emph{m}me darrier\wdx{derrier}{adv. `du
côté opposé au visage, à la face'}{darrier}, si co\emph{m}me
il est escript ou second de \flq Tegny\frq .
Et la cause\wdx{cause}{f. `ce qui produit
un effet (considéré par rapport à cet
effet)'}{}
de ceste figure\wdx{figure}{f. `forme extérieure
d'un corps'}{} disoit
Galien\adx{Galien}{}{} ou
.ix.\hoch{e} du
livre qui se intitule \flq De utilitate part\emph{icu}la\emph{rum}\frq , ou
darnier\wdx{*derrenier}{adj.
`qui vient
après tous les autres, après lequel il n'y a pas
d'autre' (temporel ou spatial)}{darnier}
chapp\emph{itre}, ou il dit ainsi: \emph{o\emph{m}nium
figurarum
vix passibilis \emph{et} c\emph{etera}}. C'est a dire\wdx{dire}{v.tr.
`lire à haute voix;
réciter'}{\textbf{c'est a dire} \emph{loc.conj. qui
annonce une explication ou une précision}}
que, de toutes les fi\mbox{gu}res\wdx{figure}{f.
`forme extérieure
d'un corps'}{}, la ronde\wdx{*rëont}{adj. `qui a la
forme circulaire'}{ronde \emph{f.sg.}}
c'est la moins passible\wdx{passible}{adj. `qui
réagit au contact avec peu de résistance'}{}
et c'est celle q\emph{ue} co\emph{n}tient\wdx{contenir}{v.tr.
`comprendre en soi,
dans sa capacité, son étendue, sa substance;
contenir'}{contient \emph{3.p.sg. ind.prés.}} le
plus. Item, sa
substance\wdx{sustance}{f. `matière dont un
corps est formé, et en vertu de laquelle
il y a des propriétés particulières;
substance'}{substance}
est ossue\wdx{ossu}{adj. `qui est de la nature des
os; osseux'}{ossue \emph{f.sg.}} et y a
panicle\wdx{*pannicle}{m. terme d'anat.
`couche de tissu musculaire ou cellulaire qui recouvre
une structure organique du corps humain (un
organe, un os, une articulation, un
muscle, etc.)'}{panicle}
et medule\wdx{medulle}{f.
terme d'anat.
`substance moelleuse de l'intérieur d'une
structure osseuse'}{medule} \text{et est}\fnb{Ms. \emph{et cest},
\emph{c} getilgt.}/
sa complexion\wdx{*complessïon}{f.
`ensemble des
éléments constituant la nature physique d'un
individu, d'une partie du corps ou d'une chose;
complexion'}{complexion}
froide\wdx{froit}{adj. terme de méd.
désignant une des qualités des quatre humeurs, celle qui gouverne
essentiellement l'équilibre du flegme et de la mélancolie
`froit (comme terme de l'humorisme)'}{froide \emph{f.sg.}}.
\pend
\pstart
Item, les parties du chief\wdx{chief}{m. 1\hoch{o} `partie
supérieure du corps de l'homme; tête'}{}
--
selon Avicene\adx{Avicene}{}{} en son \flq Canon\frq\ ou tiers livre et au
premier chapitre\wdx{chapitre}{m. `chacune des
parties qui se suivent dans un livre et en articulent la lecture;
chapitre'}{} -- sont .x. ou
.xj., c'est assavoir\wdx{assavoir}{v.tr.}{\textbf{c'est
assavoir} \emph{loc.
`c'est-à-dire'}}
.v. parties qui contiennent\wdx{contenir}{v.tr. `comprendre en soi,
dans sa capacité, son étendue, sa substance;
contenir'}{\emph{empl.abs.}}
et .v. qui
sont
\text{contenues}\fnb{Ms. \emph{contenuees}.}/\wdx{contenir}{v.tr. `comprendre en soi,
dans sa capacité, son étendue, sa substance;
contenir'}{contenu \emph{p.p.}}.
Premiers\wdx{premier}{adv.
`en
premier lieu, d'abord; premièrement'}{premiers}  par
dehors\wdx{*defors}{adv. `à l'extérieur'}{dehors}
sont
\text{les cheveux et}\fnb{Nachfolgend gestrichenes
\emph{q}.}/\wdx{*chevel}{m. `poil qui
recouvre le crâne de l'homme; cheveu'}{cheveux
\emph{pl.}} aprés le cuir et aprés la char
%
[21r\hoch{o}a]
musculeuse\wdx{musculeux}{adj. terme d'anat. `qui est de la nature
des muscles'}{} et aprés le
gros panicle\wdx{*gros pannicle}{m.
`membrane qui tapisse la boîte osseuse
renfermant l'encéphale'}{gros panicle} et
aprés le cra\emph{n}ne\wdx{crane}{m.
terme d'anat.
`boîte osseuse renfermant l'encéphale'}{cranne}. Et ap\emph{ré}s
vient\wdx{venir}{v.tr.indir. 2\hoch{o} `se
produire; survenir'}{}
par
dedans\wdx{dedans}{adv. `à l'intérieur'}{} la dure
mere\wdx{dure mere}{f. terme d'anat. `la plus
superficielle et la plus résistante des trois
méninges; dure mère'}{} et la pie mere\wdx{pie
mere}{f. terme d'anat. `la plus profonde des
méninges, mince et transparente qui enveloppe
directement le cerveau et la moelle épinière;
pie-mère'}{}, aprés la substance\wdx{sustance}{f.
`matière dont un
corps est formé, et en vertu de laquelle
il y a des propriétés particulières;
substance'}{substance} du cervel.
Et aprés au dessoubz\wdx{*desoz}{prép.
qui marque la position en bas par rapport
à ce qui est en haut `sous'}{\textbf{au
dessoubz de} \emph{loc.prép. `en bas de'}} du cervel
vient\wdx{venir}{v.tr.indir. 2\hoch{o} `se
produire; survenir'}{}
la pie \text{mere}\fnb{Erstes \emph{e} über der Zeile
nachgetragen.}/\wdx{pie mere}{f.
terme d'anat. `la plus profonde des
méninges, mince et transparente qui enveloppe
directement le cerveau et la moelle épinière;
pie-mère'}{}
encores et la dure maire\wdx{dure mere}{f.
terme d'anat. `la plus
superficielle et la plus résistante des trois
méninges; dure mère'}{dure maire},
et finalmant\wdx{finalment}{adv. `à la
fin'}{finalmant} vient\wdx{venir}{v.tr.indir.
2\hoch{o} `se produire;
survenir'}{}
aprés
la \text{roith ou rethine}\fnb{Ms.
\emph{Roich ou rechine}.}/\wdx{*roit
merveilleuse}{f. terme d'anat.
`lacis rétiforme situé dans la tête,
formé par les arteres du
cerveau'}{roith merveilleuse}%
\wdx{*retine}{f. 2\hoch{o}
terme d'anat. `lacis rétiforme formé de
vaisseaux sanguins,
de fibres, de nerfs, etc.; réseau'}{rethine} qui est
merveilleuse. Et en
ap\emph{ré}s
vient\wdx{venir}{v.tr.indir. 2\hoch{o} `se
produire; survenir'}{}
le \text{os}\fnb{Ms. \emph{cis}, fehlerhaft für
\emph{os}.}/\wdx{os}{m.
`chacune des pièces rigides qui constituent le
squelette de l'homme et des animaux
vertébrés'}{} qui est
le fondema\emph{n}t\wdx{*fondement}{m.
`partie d'un corps
sur laquelle il porte, il repose'}{fondemant} du
cervel, et la sont plantés\wdx{planter}{v.tr.
`fixer (qch.)'}{} les nerfz et
en naissent\wdx{naistre}{v.tr.indir. terme d'anat.
`émaner de (en parlant
des membres, organes ou structures organiques du
corps)'}{naissent \emph{3.p.pl. ind.prés.}},
desqueulx il s\emph{er}a dit par
ordre\wdx{ordre}{m. 1\hoch{o} `relation
intelligible entre une pluralité de
termes'}{\textbf{par ordre} \emph{`selon la
relation intelligible entre une pluralité de
termes'}} l'un
aprés l'autre. Mais des
poilz\wdx{poil}{m. `chacune des productions filiformes qui naissent
du tégument de l'homme et des mammifères; poil'}{poilz \emph{pl.}}
et
du cuir et de
la char musculeuse\wdx{musculeux}{adj. terme d'anat. `qui est de la nature
des muscles'}{}
dit est assez\wdx{assez}{adv. `en suffisance;
assez'}{} si devant.
Mais du gros panicle\wdx{*gros pannicle}{m.
`membrane qui tapisse la boîte osseuse
renfermant l'encéphale'}{gros panicle} que
Galien\adx{Galien}{}{} appelle\wdx{apeler}{v.tr. `donner un nom à qn, qch.; appeler'}{appelle
\emph{3.p.sg. ind.prés.}}
pericraneum\wdx{pericraneum}{mlt. terme d'anat. `membrane qui tapisse la boîte osseuse
renfermant l'encéphale'}{}, q\emph{ue}
coevre\wdx{*covrir}{v.tr. `garnir qch.
en disposant qch. dessus'}{coevre
\emph{3.p.sg.
ind.prés.}}
tout le cra\emph{n}ne\wdx{crane}{m.
terme d'anat.
`boîte osseuse renfermant l'encéphale'}{cranne}, tu dois
savoir\wdx{savoir}{v.tr. `avoir présent à l'esprit
(un objet de pensée qu'on identifie et qu'on tient
pour réel); savoir'}{}
que il est nerveux\wdx{*nervos}{adj. terme d'anat. `qui a le caractère des nerfs
ou des tendons'}{nerveux
\emph{m.sg.}} et naist\wdx{naistre}{v.tr.indir. terme
d'anat.
`émaner de (en parlant
des membres, organes ou structures organiques du
corps)'}{naist \emph{3.p.sg. ind.prés.}}
de la dure
mere\wdx{dure mere}{f.
terme d'anat. `la plus
superficielle et la plus résistante des trois
méninges; dure mère'}{}. Et est loiés a la
dure maire\wdx{dure mere}{f.
terme d'anat. `la plus
superficielle et la plus résistante des trois
méninges; dure mère'}{dure maire}
par liguemans, par nerfz et par vaines qui
entrent\wdx{entrer}{v.intr. `passer du dehors en
dedans'}{} et qui
issent\wdx{issir}{v.tr.indir. `aller hors de' (dit d'une chose)}{}
par les co\emph{m}missures\wdx{*comissure}{f.
terme d'anat. `articulation immobile
caractérisée par deux surfaces
articulaires; suture' (souvent dit des
os du crâne)}{commissure} du
cra\emph{n}ne\wdx{crane}{m.
terme d'anat.
`boîte osseuse renfermant l'encéphale'}{cranne}.
\pend
\pstart
Et en aprés
vient\wdx{venir}{v.tr.indir. 2\hoch{o} `se
produire; survenir'}{}
l'os
que on appelle\wdx{apeler}{v.tr. `donner un nom à qn, qch.; appeler'}{appelle
\emph{3.p.sg. ind.prés.}}
cra\emph{n}ne, et n'est pas ung os seulemant\wdx{*seulement}{adv. `sans rien d'autre que
ce qui est mentionné; seulement'}{seulemant}
continuel\wdx{continuel}{adj. `qui n'est pas
interrompu dans l'espace'}{},
ains\wdx{*ainz}{conj. `plutôt, de préférence'}{ains}
il est fait de
.vij. os
contigues\wdx{contigu}{adj. `qui touche à autre
chose; contigu'}{}, tenans l'ung a l'autre.
Et sont ordo\emph{n}nés\wdx{*ordener}{v.tr. 1\hoch{o}
`disposer, mettre
dans un certain ordre; ordonner'}{ordonné
\emph{p.p.}} ainsi,
afin que\wdx{afin que}{loc.conj. qui marque l'intention,
le but `pour que'}{}, se
lezion\wdx{*lesïon}{f. `modification de la
structure normale d'une partie de l'organisme;
lésion'}{}
venoit en l'ung, que elle ne passa\wdx{passer}{v.tr.indir. `aller d'un
lieu à un autre; passer'}{}
pas en l'autre. Et
sont co\emph{n}joings\wdx{conjoindre}{v.tr.
`mettre des choses
ensemble de façon qu'elles se touchent ou tiennent
ensemble; joindre'}{conjoings \emph{p.p. m.pl.}} par
les co\emph{m}missures\wdx{*comissure}{f.
terme d'anat. `articulation immobile
caractérisée par deux surfaces
articulaires; suture' (souvent dit des
os du crâne)}{commissure}
sarratilles\wdx{serratil}{adj. `qui est en forme de
scie'}{sarratilles \emph{f.pl.}}, afin
que\wdx{afin que}{loc.conj. qui marque l'intention,
le but `pour que'}{}
les vapeurs\wdx{*vapor}{f. terme de physiol.
`liquide du corps humain à l'état gazeux'}{vapeur}
en puissent
issir\wdx{issir}{v.tr.indir. `aller hors de' (dit d'une chose)}{}.
Le p\emph{re}mier
os de l'oule\wdx{*ole}{f.
terme d'anat.
`boîte osseuse
renfermant l'encéphale'}{oule} du chief, on le
appelle coronale\wdx{coronal}{m.
terme d'anat. `os qui forme la partie
antérieure du crâne; os frontal'}{} et
est en la p\emph{ar}tie
%
[21r\hoch{o}b]
du devant et ainsi que
du milieu\wdx{milieu}{m. `partie d'une chose
qui est à égale distance des extrémités de cette
chose'}{} des
orbites\wdx{orbite}{m. ou f. terme d'anat.
`structure osseuse en forme d'une concavité
sphérique'
(dit de la concavité dans laquelle se trouve placé
l'\oe il ou dit du conque de l'oreille)}{} ou du
visage
jusq\emph{ue}s a\wdx{jusques a}{loc.prép.
qui marque
le terme final, la limite que l'on ne dépasse
pas}{}
la
co\emph{m}missure\wdx{*comissure}{f.
terme d'anat. `articulation immobile
caractérisée par deux surfaces
articulaires; suture' (souvent dit des
os du crâne)}{commissure} qui
va de travers\wdx{travers}{prép.}{\textbf{de travers}
\emph{loc.prép. `dans une direction transversale de'}} du
crane\wdx{crane}{m.
terme d'anat.
`boîte osseuse renfermant l'encéphale'}{}. Et en cest os
la sont les troux\wdx{*tro}{m. `ouverture au travers
d'un corps ou qui y pénètre à une certaine
profondeur; trou'}{troux \emph{pl.}} des
yeulx\wdx{ueil}{m. `organe de la vue'}{yeulx
\emph{pl.}}
et les collectoires\wdx{collectoire}{m. terme d'anat. `partie du
nez qui sert de réceptacle; fosse du nez (?)'}{} des nazilles\wdx{*nasilles}{f.pl.
`les deux cavités
nasales'}{nazilles} qui sont
partis\wdx{partir}{v.tr. `diviser en plusieurs parties'}{parti \emph{p.p.}}
et divisés par une
addiccion\wdx{addicion}{f. `éminence à la surface d'une structure
osseuse ou cartilagineuse'}{addiccion}
osseuze\wdx{*ossos}{adj. `qui est de la
nature des os; osseux'}{osseuze
\emph{f.sg.}} en maniere de\wdx{maniere}{f.
2\hoch{o} `forme particulière
que revêt l'accomplissement d'une action, le
déroulement d'un fait, l'être ou
l'existence'}{\textbf{a\,/\,en maniere de}
\emph{loc.prép. `comme'}}
la creste\wdx{creste}{f.
`excroissance charnue, rouge, dentelée, sur la
tête de certains gallinacés; crête'}{} d'une
gelline\wdx{*geline}{f. `femelle du coq; poule'}{gelline},
et la est planté\wdx{planter}{v.tr.
`fixer (qch.)'}{} le
cartillage\wdx{*cartilage}{m. terme d'anat. `variété de tissu conjonctif,
translucide, résistant mais élastique, ne contenant ni vaisseaux
ni nerfs, qui recouvre les surfaces osseuses des articulations et qui constitue la charpente
de certaines organes et le squelette de certains vertébrés inférieurs;
cartilage'}{cartillage} qui
devise\wdx{deviser}{v.tr. `séparer (qch.) en
plusieurs parties; diviser'}{}
les narilles\wdx{narille}{f. `chacune des deux
cavités nasales'}{}. Toutesvoies\wdx{*totes voies}{loc.adv.
`en
considérant toutes les raisons, toutes les
circonstances qui pourraient s'y opposer, et
malgré elles; toutefois'}{toutesvoies},
tu dois savoir\wdx{savoir}{v.tr. `avoir présent à l'esprit
(un objet de pensée qu'on identifie et qu'on tient
pour réel); savoir'}{}
que
aucune
fois\wdx{*foiz}{f. `cas où un fait se produit, moment du temps où un
événement, conçu comme identique à d'autres
événements, se produit; fois'}{fois}
l'os coronal\wdx{os coronal}{m. terme d'anat. `os qui forme la partie
antérieure du crâne; os frontal'}{} est
divisé ou milieu\wdx{milieu}{m. `partie d'une chose
qui est à égale distance des extrémités de cette
chose'}{} du
front\wdx{front}{m.
`partie supérieure de la
face humaine, comprise entre les sourcils et la
racine des cheveux'}{} de
travers\wdx{travers}{adv.}{\textbf{de travers}
\emph{loc.adv. `dans une direction transversale'}}
et le voit\wdx{*vëoir}{v.tr. `percevoir (qch.) par le sens de la vue'}{voit
\emph{3.p.sg. ind.prés.}}
\text{co\emph{m}munemant}\fnb{Ms.
\emph{co}m\emph{munenemant}.}/\wdx{*comunement}{adv. `en
général'}{communemant} ou coronal\wdx{coronal}{m.
terme d'anat. `os qui forme la partie
antérieure du crâne; os frontal'}{}
des fames\wdx{*feme}{f. `être humain du sexe
féminin; femme'}{fame}. Le second os est en la p\emph{ar}tie
de derrier\wdx{derrier}{adv. `du côté
opposé au visage, à la face'}{\textbf{de
derrier} \emph{loc.adj.
`qui est situé au côté opposé
au visage, à la face'} de darrier}
que on appelle occipital\wdx{occipital}{adj. terme
d'anat. `qui a rapport à l'occiput, i.e. à la partie
postérieure et inférieure médiane de la tête;
occipital'}{}, et est conclus\wdx{conclure}{v.tr.
`enfermer (qch.)'}{conclus \emph{p.p.}} et
comprins\wdx{comprendre}{v.tr. 2\hoch{o} `être
autour de (qch.) de manière à enfermer ou embrasser partiellement ou
complètement; entourer'}{comprins \emph{p.p. m.sg.}}
par une
co\emph{m}missure\wdx{*comissure}{f.
terme d'anat. `articulation immobile
caractérisée par deux surfaces
articulaires; suture' (souvent dit des
os du crâne)}{commissure}
qui descend\wdx{descendre}{v.tr.indir. `aller du haut
en bas'}{} de
travers\wdx{travers}{adv.}{\textbf{de travers}
\emph{loc.adv. `dans une direction transversale'}} en maniere
d'une figure\wdx{figure}{f. `forme extérieure d'un corps'}{} de
lande\wdx{lande}{f. `11\hoch{e}
lettre de l'alphabet grec correspondant à l;
lambda'}{} qui est ainsi que
une figure\wdx{figure}{f. `forme extérieure d'un corps'}{}
de .vij.,
7,
et est dur\wdx{dur}{adj. `qui résiste à la pression, qui ne se
laisse pas déformer facilement'}{} et
\text{fores}\fnb{\emph{e} fehlerhaft über der
Zeile nachgetragen.}/\wdx{fort}{adj.
1\hoch{o} `qui résiste; fort (de
choses)'}{fores \emph{f.pl.}}
en bas
et par ou la nuque\wdx{nuche}{f. terme d'anat.
`substance moelleuse de l'intérieur
de l'épine dorsale; moelle épinière'}{nuque}
passe\wdx{passer}{v.tr.indir. `aller d'un
lieu à un autre; passer'}{}, qui vie\emph{n}t\wdx{venir}{v.tr.indir.
1\hoch{o} `avoir
son origine dans'}{}
du cervel
parmi les
spondilles\wdx{spondille}{f. terme d'anat. `chacun
des
os qui forment la colonne vertébrale; vertèbre'}{}
jusques\wdx{jusques a}{loc.prép. qui marque
le terme final, la limite que l'on ne dépasse
pas}{}
a lla
fin du dos\wdx{dos}{m. `partie du corps de l'homme
qui s'étend des épaules jusqu'aux reins, de chaque
côté de la colonne vertebrale; dos'}{}. Le tiers os et le quart sont
ou cousté
du cra\emph{n}ne\wdx{crane}{m.
terme d'anat.
`boîte osseuse renfermant
l'encéphale'}{cranne} et les appellent
perietalz\wdx{*os parietal}{m. terme d'anat. `chacun des deux os plats
constituant la partie moyenne et supérieure de la voûte
du crâne; os pariétal'}{perietal},
et so\emph{n}t divisés par une co\emph{m}missure\wdx{*comissure}{f.
terme d'anat. `articulation immobile
caractérisée par deux surfaces
articulaires; suture' (souvent dit des
os du crâne)}{commissure}
selon la
longuesse\wdx{*longuece}{f. `dimension d'une
chose dans le sens de sa plus grande étendue;
longueur'}{longuesse} de l'oule\wdx{*ole}{f.
terme d'anat.
`boîte osseuse
renfermant l'encéphale'}{oule} et par deux autres
co\emph{m}missures\wdx{*comissure}{f.
terme d'anat. `articulation immobile
caractérisée par deux surfaces
articulaires; suture' (souvent dit des
os du crâne)}{commissure}
devant dictes, et vont
jusques aux\wdx{jusques a}{loc.prép.
qui marque
le terme final, la limite que l'on ne dépasse
pas}{}
os des
oreilles\wdx{oreille}{f. 1\hoch{o} `l'un des deux organes
constituant l'appareil
auditif, aussi sa partie visible; oreille'}{}
et sont quarrés\wdx{*carré}{adj.
`quadrangulaire et à côtés à peu près de
même dimension; carré'}{quarré}.
Le quint et le .vj.\hoch{e} sont appellés
petrous\wdx{*os petros}{m. terme d'anat. `partie massive du temporal
en forme de pyramide quadrangulaire; rocher'}{petrous},
pour ce
q\emph{u}'\emph{i}lz sont durs\wdx{dur}{adj. `qui résiste à la pression, qui ne se
laisse pas déformer facilement'}{}
co\emph{m}me pierre\wdx{piere}{f. 1\hoch{o}
`matière minérale solide et dure qui se
rencontre à l'intérieur ou à la surface de
l'écorce terrestre en masses compactes;
pierre'}{pierre}, et si sont appellés aussi
mendeux\wdx{*mentos}{adj. `qui n'est pas ce qu'on le nomme; faux'}{mendeux \emph{m.sg.}},
car il sont plains\wdx{*plein}{adj.
`qui contient toute
la quantité possible; plein'}{plain}
d'escames\wdx{*squame}{f. `lamelle
qui se détache de la peau ou d'une autre structure
organique'}{escame} et
%
[21v\hoch{o}a]
sont co\emph{n}joings\wdx{conjoindre}{v.tr.
`mettre des choses
ensemble de façon qu'elles se touchent ou tiennent
ensemble; joindre'}{conjoings \emph{p.p. m.pl.}} avec
les os p\emph{er}ietalz\wdx{*os parietal}{m. terme d'anat. `chacun des deux os plats
constituant la partie moyenne et supérieure de la voûte
du crâne; os pariétal'}{os perietal} ou
sont les trous\wdx{*tro}{m. `ouverture au travers
d'un corps ou qui y pénètre à une certaine
profondeur; trou'}{trous \emph{pl.}} des
oreilles\wdx{oreille}{f. 1\hoch{o} `l'un des deux organes
constituant l'appareil
auditif, aussi sa partie visible; oreille'}{} et les addiccions
mamillaires\wdx{mamillaire}{adj. terme d'anat.
`qui a la forme d'un mamelon; mamillaire'}{} des
emu\emph{n}ctoires\wdx{*emomptoire}{m. terme d'anat.
`organe
qui élimine les substances inutiles formées au
cours des processus de désassimilation (l'anus,
l'uretère, etc.)'}{emunctoire}, et vont
selon la
lo\emph{n}gue\emph{ur}\wdx{*longor}{f. `dimension d'une
chose dans le sens de sa plus grande
étendue; longueur'}{longueur} des ditz
p\emph{e}\mbox{\emph{r}ie}talz\wdx{*os parietal}{m. terme d'anat. `chacun des deux os plats
constituant la partie moyenne et supérieure de la voûte
du crâne; os pariétal'}{perietal} depuis\wdx{depuis}{prép. `à partir de
(en parlant de l'espace)'}{} la
co\emph{m}missure\wdx{*comissure}{f.
terme d'anat. `articulation immobile
caractérisée par deux surfaces
articulaires; suture' (souvent dit des
os du crâne)}{commissure}
de la lande\wdx{lande}{f. `11\hoch{e}
lettre de l'alphabet grec correspondant à l;
lambda'}{}
jusques au\wdx{jusques a}{loc.prép.
qui marque
le terme final, la limite que l'on ne dépasse
pas}{}
milieu\wdx{milieu}{m. `partie d'une chose
qui est à égale distance des extrémités de cette
chose'}{} des os des
temples\wdx{temple}{m. terme d'anat. `région
latérale de la tête, entre le coin de l'\oe il et le haut
de l'oreille'}{}. Le
.vij. os est appellé paxillaire\wdx{paxillaire}{adj.
terme
d'anat. `qui sert de base ou qui appartient à une
base
(dit d'un os)'}{}
ou bazillaire\wdx{*basilaire}{adj.
terme
d'anat. `qui sert de base ou qui appartient à une
base
(dit d'un os)'}{bazillaire},
qui est ainsi que
ung coing\wdx{coing}{m. `instrument en forme
prismatique pour fendre des matériaux, serrer et
assujettir certaines choses; coin'}{} qui
ferme\wdx{fermer}{v.tr. `faire tenir (à une chose) au
moyen d'une attache, d'un lien; attacher'}{}
et soustient\wdx{*sostenir}{v.tr. `tenir (qch.)
par-dessous en servant de support ou
d'appui; soutenir'}{} tous
les aultres os devant diz sur le
palais\wdx{*palé}{m.
terme d'anat.
`cloison qui forme
la partie supérieure de la cavité buccale et la sépare
des fosses nasales; palais'}{palais}, et en lui
sont traux\wdx{*tro}{m. `ouverture au travers d'un
corps ou qui y pénètre à une certaine
profondeur; trou'}{traux \emph{pl.}} et pluseurs\wdx{*plusor}{adj. `un certain
nombre'}{pluseurs \emph{pl.}}
spongiousités\wdx{*spongiosité}{f. `ce qui est
spongieux, aussi la qualité d'être
spongieux'}{spongiousité}
pour expurger\wdx{*espurgier}{v.tr.
terme de méd. `expulser (qch.) de l'organisme'
(des substances, des impuretés nuisibles à
la santé)}{expurger \emph{inf.}}
les grosses\wdx{gros}{adj. 5\hoch{o}
`qui manque
de finesse, qui est rudimentaire mais
solide' (employé dans l'humorisme)}{}
sup\emph{er}fluités\wdx{superfluité}{f. terme de méd.
`sécrétion abondante du corps'}{}, et ha
moult dure\wdx{dur}{adj. `qui résiste à la pression, qui ne se
laisse pas déformer facilement'}{}
substance\wdx{sustance}{f. `matière dont un
corps est formé, et en vertu de laquelle
il y a des propriétés particulières;
substance'}{substance}.
Donc en l'oule\wdx{*ole}{f. terme d'anat. `boîte
osseuse renfermant l'encéphale'}{oule} du chief
sont
.vij. os, et les peult
on nombrer\wdx{nombrer}{v.tr. `évaluer (qch.) en
nombre; nombrer'}{} es chiefz de ceulx qui sont
mors\wdx{mort}{adj.
`qui a cessé de vivre; mort'}{mors \emph{m.pl.}},
quant
ilz sont cuiz\wdx{cuire}{v.tr. `transformer un corps
en l'exposant à une grande chaleur, en général pour
le rendre propre à l'alimentation; cuire'}{cuiz
\emph{p.p. m.pl.}} et divisés en
\text{aigue}\fnb{In Foliomitte \emph{Nota}
und kleine
kleeblattähnliche Zeichnung.}/%
\wdx{aigue}{f. `liquide incolore, inodore et transparent; eau'}{}
boullié\wdx{*bolir}{v.tr.
`transformer un corps
en l'exposant à une grande chaleur, en général pour
le rendre propre à l'alimentation; cuire'}{boullié \emph{p.p.}}.
Et ainsi les
nombre\wdx{nombrer}{v.tr. `évaluer (qch.) en
nombre; nombrer'}{} Galien\adx{Galien}{}{} en le
.xj.\hoch{e} livre et  ou  .xx.\hoch{e}  chappitre de
\flq Utilitate
p\emph{ar}tic\emph{u}la\emph{rum}\frq . Item, avec ce -- oultre les deux os qui
sont appellés
bregatis\wdx{bregatis}{lt. pl. `les deux os plats
constituant la partie moyenne et supérieure de la
voûte du crâne; os pariétaux'}{},
qui se co\emph{n}tiennent\wdx{contenir}{v.pron. `agier
d'une certaine maniére; se comporter' (ici: dit d'un
os)}{contiennent \emph{3.p.pl. ind.prés.}} de toutes
pars a deux os durs\wdx{dur}{adj. `qui résiste à la pression, qui ne se
laisse pas déformer facilement'}{}
et espés\wdx{espés}{adj. `qui est gros considéré
dans son épaisseur; épais'}{}
--
est ung os darrier\wdx{derrier}{adv. `du côté opposé
au visage, à la face'}{darrier} et
devant\wdx{devant}{adv. 1\hoch{o} `au côté du visage,
à la face'}{}, ouquel se
appliquent\wdx{*apliquier}{v.pron. `se
placer sur (qch.) de manière à y adhérer'}{appliquent
\emph{3.p.pl. ind.prés.}} les os des
temples\wdx{temple}{m. terme d'anat.
`région
latérale de la tête, entre le coin de l'\oe il et le haut
de l'oreille'}{}.
Et c'est ce que dit Galien\adx{Galien}{}{} ou dit
livre, ou il dit:
\emph{ultra
duo \emph{et} c\emph{etera}}. Et pour
ce il appart que
Guilliaume\adx{Guillem de Saliceto}{}{Guilliaume} et
La\emph{n}frant\adx{Lanfrant}{}{} et
\text{aussi}\fnb{Voranstehend gestrichenes
\emph{au}.}/ Henri\adx{Henri de Mondeville}{}{Henri}
virent\wdx{*vëoir}{v.tr. `percevoir (qch.) par le
sens de la vue'}{virent \emph{3.p.pl. ind.passé
simple}} mal la anathomie, car ilz dient que l'os
paxillaire\wdx{paxillaire}{adj. terme
d'anat. `qui sert de base ou qui appartient à une
base
(dit d'un os)'}{} est
dessoubz\wdx{*desoz}{prép.
qui marque la position en bas par rapport
à ce qui est en haut `sous'}{dessoubz} l'os de
la lande\wdx{os de la lande}{f.
`os médian impair, en forme de fer de cheval, situé à
la partie antérieure du cou au niveau de l'angle
que forme celui-ci avec le plancher de la
bouche; os hyoïde' (?)}{} et q\emph{ue} c'est ung des os du
%
[21v\hoch{o}b]
col\wdx{col}{m. 1\hoch{o} `partie du corps de
l'homme et de certains vertébrés qui unit la
tête au tronc'}{}.  Et dient encores que les os
petrous\wdx{*os petros}{m. terme d'anat. `partie massive du temporal
en forme de pyramide quadrangulaire; rocher'}{os petrous} sont
adjoustez\wdx{*ajoster}{v.tr. `mettre
avec (qch.)'}{adjoustez \emph{p.p. m.pl.}}
dessus\wdx{*desus}{prép. qui marque
la position en haut par rapport à ce qui est en bas
`sur'}{dessus}
les os paritales\wdx{*os parietal}{m. terme d'anat. `chacun des deux os plats
constituant la partie moyenne et supérieure de la voûte
du crâne; os pariétal'}{os parital} et
ne touchent\wdx{*tochier}{v.tr. `entrer ou être en
contact avec (qch., qn)'}{touchent \emph{3.p.pl.
ind.prés.}} point\wdx{point}{m. `endroit fixé et déterminé (où qch. à
lieu)'}{\textbf{ne{\dots} point} \emph{adv. de la négation `ne{\dots}
pas'}} le
cervel \emph{et} qu'il
ne sont point\wdx{point}{m. `endroit fixé et déterminé (où qch. à
lieu)'}{\textbf{ne{\dots} point} \emph{adv. de la négation `ne{\dots}
pas'}}
des
os principaulx\wdx{principal}{adj. `qui est le plus
important; principal'}{principaulx
\emph{m.pl.}}, mais ce n'est pas
verité\wdx{verité}{f.
`ce à quoi l'esprit
donne son assentiment, par suite d'un
rapport de conformité avec l'objet de pensée, d'une
cohérence interne de la pensée; vérité'}{}. Donc il
s'ensuit\wdx{*ensivre}{v.pron.
2\hoch{o}
`penser ou agir selon (les idées, la
conduite de qn)'}{\textbf{il
s'ensuit que} \emph{`il résulte (d'un fait) que'}}
que au chief sont
.vij. os
principaulx\wdx{principal}{adj. `qui est le plus
important; principal'}{principaulx
\emph{m.pl.}}
qui contiennent le cervel.  Toutesvoies,
il y a aucuns autres petis os qui ne sont pas
principaulx, qui so\emph{n}t fais par aucunes
aides\wdx{aide}{f. 2\hoch{o}
`caractère de ce qui est
utile, qui satisfait un besoin; utilité'}{},
si co\emph{m}me l'os de la creste\wdx{os de la creste}{f.
`pièce rigide qui forme le squelette du nez; os
propre du nez'}{} q\emph{ue} devise\wdx{deviser}{v.tr. `séparer (qch.) en
plusieurs parties; diviser'}{}
et separe\wdx{separer}{v.tr. `mettre à part les unes
des autres des choses, des personnes
réunies; séparer'}{} lé
collactoires\wdx{collectoire}{m. terme d'anat.
`partie du
nez qui sert de réceptacle; fosse du nez (?)'}{collactoire} des
\text{nazilles}\fnb{Im Ms. \emph{r} in
\emph{z} korrigiert.}/\wdx{*nasilles}{f.pl.
`les deux cavités
nasales'}{nazilles}, qui
est dedans\wdx{dedans}{prép. `à
l'intérieur de'}{} l'os coronal\wdx{os coronal}{m. terme d'anat. `os qui forme la partie
antérieure du crâne; os frontal'}{},
et les os qui sont de la face\wdx{face}{f. `partie antérieure de la
tête; visage'}{} qui ne sont pas de
l'oule\wdx{*ole}{f.
terme d'anat.
`boîte osseuse renfermant
l'encéphale'}{oule}, et aucuns os qui
sont co\emph{m}me aguilles\wdx{aguille}{f. 1\hoch{o} `fine
tige
d'acier pointue à une extrémité et percée à l'autre
d'un chas où passe le fil; aiguille'}{} et clavelz\wdx{clavel}{m.
1\hoch{o} `petit
tige de fer pointu qui sert à fixer qch.'}{},
qui sont dessoubz\wdx{*desoz}{prép.
qui marque la position en bas par rapport
à ce qui est en haut `sous'}{dessoubz}
les deux oreilles\wdx{oreille}{f. 1\hoch{o} `l'un des deux
organes constituant l'appareil
auditif, aussi sa partie visible; oreille'}{}, ouquelz
sont plantés\wdx{planter}{v.tr. `fixer (qch.)'}{}
et fermés\wdx{fermer}{v.tr.
`faire tenir (à une chose) au
moyen d'une attache, d'un lien; attacher'}{} les muscules ou
les cordes
q\emph{ue} \text{oevrant}\fnb{Zur Form, cf.
\flq Die Sprache\frq .}/\wdx{ovrir}{v.tr.
`faire que ce qui
était fermé ne le soit plus;
ouvrir'}{oevrant \emph{3.p.pl. ind.prés.}} la
mandible\wdx{mandibule}{f. `chacun
des deux arcs osseux de la bouche dans lesquels
sont plantées les dents, qui sont articulés et qui
servent à mâcher; mâchoire'}{mandible}. Et ces os
ci
no\emph{m}bre\wdx{nombrer}{v.tr. `évaluer (qch.) en
nombre; nombrer'}{} Haliabas\adx{Haliabas}{}{}
en la seconde parole\wdx{parole}{f.
`ensemble de mots
qui expriment une idée'}{}
de la p\emph{re}miere
partie du livre qui se intitule \flq De disposicione
regali\frq\ ou chappitre ou il parle des os du chief. Et
po\emph{ur} ce il disoit q\emph{ue} tous les os du
cra\emph{n}ne\wdx{crane}{m.
terme d'anat.
`boîte osseuse renfermant l'encéphale'}{cranne}
so\emph{n}t .xv.
Mais Avicene\adx{Avicene}{}{}, qui n'e\emph{n} nombra que
.v., il entendoit\wdx{entendre}{v.tr.
`saisir par
l'intelligence'}{} des os qui
avoient vraies\wdx{*verai}{adj. `qui présente un
caractère de vérité; vrai'}{vraies \emph{f.pl.}}
co\emph{m}missures\wdx{*comissure}{f.
terme d'anat. `articulation immobile
caractérisée par deux surfaces
articulaires; suture' (souvent dit des
os du crâne)}{commissure}
serratilles\wdx{serratil}{adj. `qui est en forme de scie'}{serratilles
\emph{f.pl.}}, et sont
trois. Mais il n'en y a pas q\emph{ue} en ch\emph{acu}m
costé\wdx{costé}{m. `partie qui est à droite ou à gauche
(d'un corps); côté'}{},
il n'en y eust deux aultres qui avoient \emph{com}missures
scamouzes\wdx{*comissure
scameuse}{m. `suture entre l'écaille de
l'os temporal et l'os pariétal; suture
squameuse'}{commissure scamouze} et
mendouzes\wdx{*mentos}{adj. `qui n'est pas ce qu'on le nomme; faux'}{mendouze \emph{f.sg.}};
et vous souffise\wdx{*sofire}{v.intr.
`avoir la quantité, la qualité, la force
etc. nécessaire pour (qch.);
suffire'}{\textbf{*sofire a qn}
\emph{v.impers. + subj.prés.} vous souffise} ce que dit
est de l'anathomie du
%
[22r\hoch{o}a]
crane\wdx{crane}{m.
terme d'anat.
`boîte osseuse renfermant l'encéphale'}{} et, p\emph{ar}
consequent\wdx{consequent}{adj.}{\textbf{par
consequent}
\emph{loc.adv. `comme suite logique'}},
des .v. p\emph{ar}ties
qui sont contenues dehors\wdx{*defors}{prép. `à
l'extérieur de'}{dehors} le
cra\emph{n}ne.
\pend
\pstart
Mais les
parties qui sont contenues dedans\wdx{dedans}{prép. `à
l'intérieur de'}{}
le cra\emph{n}ne, tu ne les
pourras demonstrer\wdx{demonstrer}{v.tr.
`faire
voir, mettre devant les yeux; montrer'}{}
a l'ueil\wdx{ueil}{m. `organe de la vue'}{},
\text{se on}\fnb{Nachfolgend gestrichenes \emph{d}.}/ ne
divise\wdx{deviser}{v.tr. `séparer (qch.) en
plusieurs parties; diviser'}{divise
\emph{3.p.sg. ind.prés.}} le cra\emph{n}ne
par une cerre\wdx{*serre}{f. `outil dont la pièce
essentielle est une lame dentée et dont on se sert
pour couper des matières dures; scie'}{cerre} selon
la rondesse\wdx{*rëondece}{f. `état de ce qui est rond;
rondeur'}{rondesse}
de lui et \text{q\emph{ue} on eslieve}\fnb{\emph{qu}e über der Zeile
nachgetragen}/\wdx{eslever}{v.tr. `mettre ou porter
(qch.) plus haut'}{eslieve \emph{3.p.sg. ind.prés.}}
la partie de
\text{dessus}\fnb{Ms.
\emph{sessus}.}/\wdx{*desus}{adv. `au
côté supérieur'}{\textbf{de dessus}
\emph{loc.adj. `qui est situé au côté supérieur'}}.
Et le premier qui se
app\emph{er}t\wdx{*aparoir}{v.pron. `se montrer aux yeux; se
manifester'}{appert \emph{3.p.sg. ind.prés.}}
a la veue\wdx{veüe}{f. `sens visuel'}{}, c'est la
dure maire\wdx{dure mere}{f.
terme d'anat. `la plus
superficielle et la plus résistante des trois
méninges; dure mère'}{dure maire}
et la
pie maire\wdx{pie mere}{f.
terme d'anat. `la plus profonde des
méninges, mince et transparente qui enveloppe
directement le cerveau et la moelle épinière;
pie-mère'}{pie maire}, et sont deux
panicles\wdx{*pannicle}{m. terme d'anat.
`couche de tissu musculaire ou cellulaire qui recouvre
une structure organique du corps humain (un
organe, un os, une articulation, un
muscle, etc.)'}{panicle}
nerveux\wdx{*nervos}{adj. terme d'anat. `qui a le caractère des nerfs
ou des tendons'}{nerveux \emph{m.sg.}} --
l'ung vers la partie du cra\emph{n}ne et
l'autre devers\wdx{devers}{prép. `du côté de'}{} la partie
du cervel -- qui envelloupent\wdx{*envoleper}{v.tr.
`entourer (qch.) d'une chose souple qui couvre de
tous
côtés; envelopper'}{envelloupent \emph{3.p.pl. ind.prés.}}
et coevrent toute la substance\wdx{sustance}{f. `matière dont un
corps est formé, et en vertu de laquelle
il y a des propriétés particulières;
substance'}{substance} du cervel. Item, de
la dure maire\wdx{dure mere}{f.
terme d'anat. `la plus
superficielle et la plus résistante des trois
méninges; dure mère'}{dure maire}
naist, parmy les co\emph{m}missures\wdx{*comissure}{f.
terme d'anat. `articulation immobile
caractérisée par deux surfaces
articulaires; suture' (souvent dit des
os du crâne)}{commissure}
du
cra\emph{n}ne,
le p\emph{er}icra\emph{n}ne\wdx{pericrane}{m. terme d'anat. `membrane qui tapisse la boîte osseuse
renfermant l'encéphale'}{pericranne}. Et de la
pie maire\wdx{pie mere}{f.
terme d'anat. `la plus profonde des
méninges, mince et transparente qui enveloppe
directement le cerveau et la moelle épinière;
pie-mère'}{pie maire}
va
le nourricemant\wdx{*norrissement}{m.
`substance qui est assimilé par
l'organisme et sert à le nourrir;
nourriture'}{nourricemant}
au cervel.  Et y viennent vaines et arteres\wdx{artere}{f. terme d'anat.
 `vaisseau sanguin qui part du c\oe ur et
qui distribue le sang, qui contient les esprits
vitaux,
à tout le corps'}{}, par
dedans\wdx{dedans}{adv. `à
l'intérieur'}{} parmy les troux\wdx{*tro}{m.
`ouverture au travers d'un
corps ou qui y pénètre à une certaine
profondeur; trou'}{troux \emph{pl.}} des
os qui sont bas\wdx{bas}{adv. `à faible hauteur; bas'}{}, et par
dessus\wdx{*desus}{adv. `au côté
supérieur'}{dessus}
parmy les co\emph{m}missures\wdx{*comissure}{f.
terme d'anat. `articulation immobile
caractérisée par deux surfaces
articulaires; suture' (souvent dit des
os du crâne)}{commissure}
des os qui so\emph{n}t
hault\wdx{hault}{adv.
`en un endroit qui est d'une certaine dimension
dans le sens vertical; haut'}{}.
Et en aprés de ycelles vient\wdx{venir}{v.tr.indir. 2\hoch{o} `se produire;
survenir'}{}
la
sustance\wdx{sustance}{f. `matière dont un
corps est formé, et en vertu de laquelle
il y a des propriétés particulières; substance'}{} du cervel, et
est molle\wdx{mol}{adj.
`qui cède facilement à la pression, au toucher; mou'}{molle \emph{f.sg.}} et
blanche\wdx{blanc}{adj. `qui est de
la couleur de la neige; blanc'}{}
et en figure\wdx{figure}{f. `forme extérieure d'un corps'}{}
ronde\wdx{*rëont}{adj. `qui a la forme
circulaire'}{ronde \emph{f.sg.}} avec aucunes
addicions\wdx{addicion}{f. `éminence à la surface d'une structure
osseuse ou cartilagineuse'}{}
qui sont ou cervel, qui ne sont  pas
blanches\wdx{blanc}{adj. `qui est de
la couleur de
la neige; blanc'}{}. Et de la
moitié\wdx{moitié}{f. `l'une des deux parties
égales d'un tout; moitié'}{}
du cervel de lonc\wdx{lonc}{adv.}{\textbf{de lonc}
\emph{loc.adv. `dans le sens de la longueur'}},
depuis\wdx{depuis}{prép.
`à partir de
(en parlant de l'espace)'}{} le milieu
en venant
par devant\wdx{devant}{adv. 1\hoch{o} `au côté du
visage,
à la face'}{}, se
doublent\wdx{*dobler}{v.pron. `devenir
double'}{doublent
\emph{3.p.pl. ind.prés.}} les vertus\wdx{vertu}{f.
`principe
qui, dans une chose, est considéré comme la cause
des effets qu'elle produit; faculté, pouvoir'}{}
sencitives\wdx{*sensitif}{adj. 2\hoch{o} `qui a
rapport aux sens'}{sencitives \emph{f.pl.}} et
pluseurs\wdx{*plusor}{adj. `un certain
nombre'}{pluseurs \emph{pl.}}
autres organes et
instrumens\wdx{instrument}{m. 2\hoch{o}
`partie du corps remplissant une fonction
particulière; organe'}{},
afin que\wdx{afin que}{loc.conj. qui marque l'intention,
 le but `pour que'}{},
se l'ung estoit blessé\wdx{*blecier}{v.tr. 1\hoch{o}
`causer une blessure' (à une partie du
corps)}{blessé \emph{p.p.}},
que l'autre le peult s\emph{er}vir\wdx{servir}{v.tr. `aider en étant utile ou utilisé' (de
choses)}{}, si
%
[22r\hoch{o}b]
co\emph{m}me il est escript ou
.viij. livre de
\flq Utilitate partic\emph{u}la\emph{rum}\frq .
\pend
\pstart
Item, le cervel ha trois
ventres\wdx{ventre}{m. 2\hoch{o} terme d'anat.
`partie creuse
dans le corps'}{} qui vont du lonc du chief, et
ch\emph{acu}m
ve\emph{n}tre\wdx{ventre}{m.
2\hoch{o} terme d'anat. `partie creuse
dans le corps'}{} ha deux p\emph{ar}ties, et en
chescune partie organize\wdx{*organiser}{v.tr.
empl.abs. `donner la disposition qui rend des
substances aptes à vivre'}{organize \emph{3.p.sg.
ind.prés.}} et domine\wdx{dominer}{v.intr.
`exercer une autorité'}{} une
vertu\wdx{vertu}{f.
`principe
qui, dans une chose, est considéré comme la cause
des effets qu'elle produit; faculté, pouvoir'}{}. En la p\emph{re}miere
partie du ventre\wdx{ventre}{m.
2\hoch{o} terme d'anat. `partie creuse
dans le corps'}{}
de devant\wdx{devant}{adv. 1\hoch{o} `au côté du
visage, à la face'}{\textbf{de devant}
\emph{loc.adj. `qui est situé au côté du visage, de la
face'}}
est assigné\wdx{*assignier}{v.tr. `donner, attribuer
(une tâche, un bien, une qualité, etc.)'}{assigné \emph{p.p.}} le sens
co\emph{m}mun\wdx{sens}{m.
2\hoch{o} `idée ou ensemble
d'idées intelligible que représente un signe ou un
ensemble de signes'}{\textbf{*sens comun}
\emph{`capacité de bien juger; bon sens'} sens
commun},
en la seconde partie la vertu\wdx{vertu}{f. `principe
qui, dans une chose, est considéré comme la cause
des effets qu'elle produit; faculté, pouvoir'}{}
ymaginative\wdx{*imaginatif}{adj.
`qui est caractérisé par une forte
faculté de se représenter dans l'esprit des personnes,
des objets'}{ymaginatif}. Ou
moyen ventre\wdx{ventre}{m.
2\hoch{o} terme d'anat. `partie creuse
dans le corps'}{}
est assize\wdx{asseoir}{v.tr. 2\hoch{o} `placer, poser
(qch.)'}{assize \emph{p.p. f.sg.}}
la vertu\wdx{vertu}{f. `principe
qui, dans une chose, est considéré comme la cause
des effets qu'elle produit; faculté, pouvoir'}{}
cogitative\wdx{cogitatif}{adj. `qui a rapport à la
pensée'}{} et la racionelle\wdx{racionel}{adj. `qui
appartient à la raison'}{}. Et en la
darniere\wdx{*derrenier}{adj.
`qui vient
après tous les autres, après lequel il n'y a pas
d'autre' (temporel ou spatial)}{darnier}
est la vertu\wdx{vertu}{f.
`principe
qui, dans une chose, est considéré comme la cause
des effets qu'elle produit; faculté, pouvoir'}{}
s\emph{er}vative\wdx{servatif}{adj. `qui a rapport à la
préservation'}{} et
memorative\wdx{memoratif}{adj. `qui a rapport à
la mémoire'}{}. Et
saches que le ventre\wdx{ventre}{m.
2\hoch{o} terme d'anat. `partie creuse
dans le corps'}{}
de devant\wdx{devant}{adv. 1\hoch{o} `au côté du
visage, à la face'}{\textbf{de devant}
\emph{loc.adj. `qui est situé au côté du visage, de la
face'}}, c'est le plus grant\wdx{grant}{adj.
2\hoch{o} dans l'ordre
physique, quantifiable `qui est d'une extension
au-dessus de la moyenne; grand (du corps humain
et de ses parties)'}{}.
Et au milieu, c'est le
moindre\wdx{*menor}{m. `le plus petit,
le moins important ou remarquable'}{moindre}.
Et le darnier\wdx{*derrenier}{adj.
`qui vient
après tous les autres, après lequel il n'y a pas
d'autre' (temporel ou spatial)}{darnier}, c'est le
moien\wdx{moien}{adj. `qui est au milieu'}{} en q\emph{uan}tité\wdx{*cantité}{f. `nombre
d'unités ou mesure qui sert à déterminer une
portion de matière ou une collection de choses
considérées comme
homogènes'}{quantité}, et ha
conduis\wdx{conduit}{m. `canal
ou tuyau qui sert à l'ecoulement ou au
transport d'une matière (un liquide, l'air, un
gaz, etc.)'}{} de l'ung a l'autre
par lesqueulx passent\wdx{passer}{v.tr.indir. `aller d'un
lieu à un autre; passer'}{}
les
experiz\wdx{esperit}{m.
terme de méd.
`ensemble de corpuscules subtils qui assurent
toutes les fonctions de la vie dans
l'organisme humain'}{experiz \emph{pl.}}. Et en
cellui de darrier\wdx{derrier}{adv. `du côté
opposé au visage, à la face'}{\textbf{de
derrier} \emph{loc.adj.
`qui est situé au côté opposé au visage, à la face'}
de darrier} sont les
addiccions, en maniere de\wdx{maniere}{f. 2\hoch{o}
`forme particulière
que revêt l'accomplissement d'une action, le
déroulement d'un fait, l'être ou
l'existence'}{\textbf{a\,/\,en maniere de}
\emph{loc.prép. `comme'}}
mamelles\wdx{*mamele}{f.
`organe glanduleux
qui sécrète le lait (chez les mammifères);
mamelle'}{mamelle},
esquelles est fondé\wdx{fonder}{v.tr. `appuyer
(qch.)'}{}
\text{le}\fnb{Über der Zeile nachgetragen.}/
sentema\emph{n}t\wdx{*sentement}{m.
`faculté d'éprouver
les impressions que font les objets matériels, i.e.
goût, odorat, ouïe, toucher, vue'}{sentemant} de
\text{odorer}\fnb{Nachfolgend gestrichener
Buchstabenansatz.}/\wdx{odorer}{v.tr. empl.abs.
`avoir la sensation d'une odeur'}{};
et de lui, en la plus grant\wdx{grant}{adj. 1\hoch{o}
dans l'ordre
physique, quantifiable `qui est d'une extension
au-dessus de la moyenne; grand (des choses)'}{grant
\emph{f.sg.}} partie,
naissent .v. paire\wdx{paire}{f.
`réunion de deux choses,
de deux êtres semblables
qui vont ensemble; paire'}{} de nerfz
sencitiz\wdx{*sensitif}{adj.
2\hoch{o} `qui a rapport aux sens'}{sencitiz
\emph{m.pl.}} qui vont aux
yeulx\wdx{ueil}{m. `organe de la vue'}{yeulx
\emph{pl.}} et es
oreilles\wdx{oreille}{f. 1\hoch{o} `l'un des deux organes
constituant l'appareil
auditif, aussi sa partie visible; oreille'}{}, et y sont
troux\wdx{*tro}{m. `ouverture au travers d'un
corps ou qui y pénètre à une certaine
profondeur; trou'}{troux \emph{pl.}} par ou ilz
naissent
et passent\wdx{passer}{v.tr.indir. `aller d'un
lieu à un autre; passer'}{}, et sont envelloupés\wdx{*envoleper}{v.tr.
`entourer (qch.) d'une chose souple qui couvre de
tous côtés; envelopper'}{envelloupé \emph{p.p.}} de
panicles\wdx{*pannicle}{m. terme d'anat.
`couche de tissu musculaire ou cellulaire qui recouvre
une structure organique du corps humain (un
organe, un os, une articulation, un
muscle, etc.)'}{panicle},
si co\emph{m}me vous porrés voir\wdx{*vëoir}{v.tr.
`percevoir (qch.) par le sens de la vue'}{voir
\emph{inf.}}. Item, pres du moyen
ventre\wdx{ventre}{m.
2\hoch{o} terme d'anat. `partie creuse
dans le corps'}{},
note\wdx{noter}{v.tr.
`prêter attention à (qch.); remarquer'}{}
ung lieu\wdx{lieu}{m.
`portion
déterminée de l'espace; lieu'}{}
que on appelle
\text{lacuna}\fnb{Ms.
\emph{lacrima} (?) mit über der Zeile nachgetragenem \emph{r}.}/\wdx{lacuna}{lt. `espace vide à
l'intérieur d'un corps; lacune'}{} et uniformis\wdx{uniformis}{lt.
`qui a la même forme' (?)}{}
et
antraformis\wdx{antraformis}{lt. `qui présente
une concavité'}{}. Et y sont chars
glandellouses\wdx{*glandulos}{adj. `qui contient des
glandes'}{glandellouse \emph{f.sg.}}
qui remplissent\wdx{remplir}{v.tr. `rendre (un
espace disponible) plein de (qch.)'}{} le dit lieu\wdx{lieu}{m.
`portion
déterminée de l'espace; lieu'}{}. Et
dessoubz\wdx{*desoz}{prép.
qui marque la position en bas par rapport
à ce qui est en haut `sous'}{dessoubz}
%
[22v\hoch{o}a]
les panicles est assise\wdx{asseoir}{v.tr. 2\hoch{o} `placer, poser
(qch.)'}{assis \emph{p.p.}}
une \text{royt}\fnb{Ms.
\emph{Royt}.}/
merveilleuse\wdx{*roit merveilleuse}{f.
terme d'anat.
`lacis rétiforme
situé dans la tête,
formé par les arteres du
cerveau'}{royt merveilleuse}
qui est faicte et tissue\wdx{tistre}{v.tr.
terme de méd.
`former par entrelacement (de fibres ou
structures organiques)'}{tissu
\emph{p.p.}}
seullemant de arteres q\emph{ue} viennent \text{du}\fnb{Über
der Zeile nachgetragen.}/ cuer\wdx{cuer}{m. terme d'anat. `viscère
de forme de cône
renversé, situé entre les poumons, qui est l'organe central de la
distribution du sang dans le corps'}{},
par lesquelles passe\wdx{passer}{v.tr.indir. `aller d'un
lieu à un autre; passer'}{}
l'esperit vital\wdx{esperit
vital}{m. terme de méd.
`ensemble de corpuscules subtils qui assurent le maintien de la
chaleur des organes et des membres du corps, considéré
comme fonction essentielle de l'organisme; esprits
vitaux'}{}.
\pend
\pstart
Et
finalmant\wdx{finalment}{adv. `à la
fin'}{finalmant} tu
regarderas co\emph{m}me la nuque\wdx{nuche}{f. terme d'anat.
`substance moelleuse de l'intérieur
de l'épine dorsale; moelle épinière'}{nuque}
ou nuche\wdx{nuche}{f. terme d'anat.
`substance moelleuse de l'intérieur
de l'épine dorsale; moelle épinière'}{}
ou la
medulle du dos\wdx{medulle du dos}{f. terme
d'anat. `substance moelleuse de l'intérieur de
l'épine dorsale; moelle épinière'}{}
naist de la partie de
darrier\wdx{derrier}{adv. `du
côté opposé au visage, à la
face'}{\textbf{de derrier} \emph{loc.adj.
`qui est situé au côté opposé au visage, à la face'}
de darrier} du cervel, et est
envellopee\wdx{*envoleper}{v.tr.
`entourer
(qch.) d'une chose souple qui couvre de tous
côtés; envelopper'}{envellopé \emph{p.p.}}
[de]
deux panicles
ainsi que le
cervel, et descend\wdx{descendre}{v.tr.indir. `aller
du haut en bas'}{}
parmy les spondilles\wdx{spondille}{f. terme d'anat.
`chacun des os qui
forment la colonne vertébrale; vertèbre'}{} jusques
a\wdx{jusques a}{loc.prép.
qui marque
le terme final, la limite que l'on ne dépasse
pas}{}
la fin du dos\wdx{dos}{m.
`partie du corps de l'homme
qui s'étend des épaules jusqu'aux reins, de chaque
côté de la colonne vertebrale; dos'}{};
de laquelle
nuque\wdx{nuche}{f. terme d'anat.
`substance moelleuse de l'intérieur
de l'épine dorsale; moelle épinière'}{nuque}
naissent
principaumant\wdx{*principaument}{adv. `d'une manière principale'}{principaumant}
les nerfz motis\wdx{motif}{adj. `qui fait
mouvoir'}{motis \emph{m.pl.}}, si co\emph{m}me il s\emph{er}a dit
ci aprés. Car
elle est semblable\wdx{semblable}{adj.
`qui ressemble;
semblable'}{} au cervel et
semble que\wdx{sembler}{v.tr.
`avoir des traits communs avec; ressembler'}{\textbf{semble que}
\emph{v.impers. `il paraît que'}}
ce soit
une partie du cervel et pour ce, ses
sinthomates\wdx{*symptomates}{m.pl. terme
de méd. `phénomènes
caractéristiques perceptibles ou observables liés à
un état ou une évolution de
l'organisme'}{sinthomates},
ce sont les sieges\wdx{siege}{m.
`lieu où qch. réside'}{} du
cervel, si co\emph{m}me Galen\adx{Galien}{}{Galen} le dit ou
.xij. de \flq Utilitate partic\emph{u}la\emph{rum}\frq .  Et ainsi est
espediee\wdx{*expedier}{v.tr. `accomplir (qch.)
rapidement'}{espedié \emph{p.p.}} l'anathomie
de l'oule\wdx{*ole}{f.
terme d'anat.
`boîte osseuse renfermant
l'encéphale'}{oule} du chief qu\emph{an}t a
.vj.
choses devant dictes.
\pend
\pstart
Or nous
co\emph{n}vient\wdx{*covenir}{v.tr.indir. `être convenable
pour
(qn)'}{convient \emph{3.p.sg. ind.prés.}} veoir des
maladies\wdx{maladie}{f.
 `altération organique ou
fonctionnelle considérée dans son évolution, et comme
une entité définissable; maladie'}{}
qui y peuellent avenir\wdx{avenir}{v.tr.indir. `venir
ou
être sur le point d'être; arriver'}{}. La
oulle du chief
peult avoir plaies\wdx{plaie}{f.
`ouverture
dans les chairs, les tissus, due à une cause
externe (traumatisme, intervention chirurgicale) et
présentant une solution de continuité des téguments;
plaie'}{},
empostumes\wdx{empostume}{m. ou f. terme de
méd. `tumeur accompagné de suppuration'}{} et males
complexions\wdx{*complessïon}{f.
`ensemble des
éléments constituant la nature physique d'un
individu, d'une partie du corps ou d'une chose;
complexion'}{complexion}.
Et appert\wdx{*aparoir}{v.intr. `se montrer aux yeux; se
manifester'}{appert \emph{3.p.sg. ind.prés.}} que les plaies\wdx{plaie}{f.
`ouverture
dans les chairs, les tissus, due à une cause
externe (traumatisme, intervention chirurgicale) et
présentant une solution de continuité des téguments;
plaie'}{}
qui
penetrent\wdx{penetrer}{v.tr. `passer à travers
de (qch.)'}{} tout le cra\emph{n}ne, que elles sont
perilleuses\wdx{*perillos}{adj.
`qui constitue un danger, présente du
danger; dangereux'}{perilleuses \emph{f.pl.}}.
Et celles qui touchent\wdx{*tochier}{v.tr.
`entrer ou être en
contact avec (qch., qn)'}{touchent \emph{3.p.pl.
ind.prés.}}
les toilles\wdx{*toile}{f. terme d'anat.
`membrane, mince couche de tissu qui enveloppe un
organe'}{toille} du cervel
sont plus perilleuses\wdx{*perillos}{adj.
`qui constitue un danger, présente du
danger; dangereux'}{perilleuses \emph{f.pl.}}. Et
encores plus celles qui touchent\wdx{*tochier}{v.tr.
`entrer ou être en
contact avec (qch., qn)'}{touchent \emph{3.p.pl.
ind.prés.}}
la sustance\wdx{sustance}{f. `matière dont un
corps est formé, et en vertu de laquelle
il y a des propriétés particulières; substance'}{} du
cervel. Item, les operacions\wdx{operacïon}{f.
`action d'un pouvoir, d'une fonction,
d'un organe qui produit un effet selon sa
nature; opération'}{operacion} que
on fait entour les
%
[22v\hoch{o}b]
co\emph{m}missures, elles sont
\text{suspectes}\fnb{\emph{Nota} und Kürzel des
Schreibers am
Foliorand.}/\wdx{suspect}{adj. `qui éveille les
soupçons, dont la valeur, l'intérêt, la sûreté,
etc., sont douteux'}{}
et doubteuses\wdx{*dotos}{adj. `qui cause de la
peur'}{doubteuses
\emph{f.pl.}}, que
\text{la dure maire}\fnb{Am Zeilenrand
nachgetragen, ersetzt expungiertes
\emph{la pericrane}.}/\wdx{dure mere}{f.
terme d'anat. `la plus
superficielle et la plus résistante des trois
méninges; dure mère'}{dure maire}
ne chee\wdx{*chëoir}{v.intr. `tomber' (de choses)}{chee
\emph{3.p.sg.
subj.prés.}} sur la premiere et que le cervel ne
soit co\emph{m}primé\wdx{comprimer}{v.tr. `exercer une
pression sur (qch.) en diminuer le volume;
comprimer'}{}.
Ite\emph{m}, toutes incisions\wdx{incision}{f. `action de fendre,
de couper avec un instrument tranchant, son
résultat (surtout en médecine)'}{} du chief
doive\emph{n}t\wdx{devoir}{v.tr. + inf. `être dans
l'obligation de (faire qch.); devoir'}{}
estre faictes ainsi que vont
lé cheveux\wdx{*chevel}{m. `poil qui
recouvre le crâne de l'homme; cheveu'}{cheveux \emph{pl.}},
car ainsi vont les muscules. Item, la
\text{maniere}\fnb{Ms. \emph{manie}.}/ de
loier\wdx{*liier}{v.tr. `entourer plusieurs choses
avec un lien pour qu'elles tiennent ensemble'}{loier
\emph{inf.}} est p\emph{ro}pre\wdx{propre}{adj.
1\hoch{o} `qui
appartient exclusivement à (qn, qch.)'}{}
po\emph{ur} cause\wdx{cause}{f. `ce qui produit
un effet (considéré par rapport à cet
effet)'}{}
de la rondesse\wdx{*rëondece}{f. `état de ce qui est
rond; rondeur'}{rondesse}
du me\emph{m}bre, laquelle maniere s\emph{er}a
dicte cy aprés.
\pend
%
% \memorybreak
%
\pstartueber
Le second chappitre
parlle\wdx{parler}{\textbf{\emph{parler de}} v.tr.indir. `s'entretenir de;
parler de'}{parlle \emph{3.p.sg. ind.prés.}}
de
l'anathomie de la face\wdx{face}{f. `partie antérieure de la
tête; visage'}{} et de ces p\emph{ar}ties.
\pendueber
%
% \memorybreak
%
\pstart
LES PARTIES de la face\wdx{face}{f. `partie antérieure de la
tête; visage'}{} sont le
front\wdx{front}{m. `partie supérieure de la
face humaine, comprise entre les sourcils et la
racine des cheveux'}{}, les
surcilz\wdx{*sorcil}{m. `saillie arquée, garnie de
poils, au-dessus de l'orbite, désignant aussi ses
poils; sourcil'}{surcilz \emph{pl.}},
les yeulx\wdx{ueil}{m. `organe de la vue'}{yeulx
\emph{pl.}}, les nazilles\wdx{*nasilles}{f.pl.
`les deux cavités
nasales'}{nazilles}, les
temples\wdx{temple}{m. terme d'anat. `région
latérale de la tête, entre le coin de l'\oe il et le haut
de l'oreille'}{}, les
oreilles\wdx{oreille}{f. 1\hoch{o} `l'un des deux organes
constituant l'appareil
auditif, aussi sa partie visible; oreille'}{}, les
joues\wdx{*joë}{f. `partie latérale de la
face s'étendant entre le nez et l'oreille, du dessous
de l'\oe il au menton'}{joue}, la
bouche\wdx{*boche}{f.
`cavité située à la partie
inférieure du visage de l'homme, bordée
par les lèvres; bouche'}{bouche},
les mandibules\wdx{mandibule}{f. `chacun
des deux arcs osseux de la bouche dans lesquels
sont plantées les dents, qui sont articulés et qui
servent à mâcher; mâchoire'}{} avec les
dens\wdx{dent}{m. et f.
`organe de la bouche, de
couleur blanchâtre, dur, implanté sur le
maxillaire'}{dens \emph{pl.}}. Le
front\wdx{front}{m. `partie supérieure de la
face humaine, comprise entre les sourcils et la
racine des cheveux'}{} ne
contient q\emph{ue}
la pel\wdx{pel}{f. 1\hoch{o} `enveloppe extérieure du
corps de l'homme; peau'}{} et la char
musculeuse\wdx{musculeux}{adj. terme d'anat. `qui est de la nature
des muscles'}{}, car
l'os qui est dessoubz\wdx{*desoz}{adv. `à la face
inférieure'}{dessoubz},
c'est du coronal\wdx{coronal}{m.
terme d'anat. `os qui forme la partie
antérieure du crâne; os frontal'}{}.
Car selon la table\wdx{table}{f. `surface plane
sur laquelle on peut écrire, peindre, etc.'}{} de
dessus\wdx{*desus}{adv.
`au côté supérieur'}{\textbf{de
dessus} \emph{loc.adj. `qui est situé au côté
supérieur'}}
il se eslieve\wdx{eslever}{v.pron. `avoir une
certaine hauteur'}{eslieve \emph{3.p.sg. ind.prés.}}
et sa spongiousité\wdx{*spongiosité}{f.
`ce qui est
spongieux, aussi la qualité d'être
spongieux'}{spongiousité}
se eslargist\wdx{eslargir}{v.pron. `devenir plus large'}{eslargist
\emph{3.p.sg. ind.prés.}} ainsi que
ce
c'estoit ung os double\wdx{*doble}{adj.
`qui est
répété deux fois, qui vaut deux fois (la
chose désignée) ou qui existe deux fois'}{double} et
fait la fourme\wdx{*forme}{f. `apparence extérieure
donnant à un objet ou à un être sa
spécificité'}{fourme} des surcilz\wdx{*sorcil}{m.
`saillie arquée, garnie de poils,
au-dessus de l'orbite, désignant aussi ses
poils; sourcil'}{surcilz
\emph{pl.}}. Item, les
surcilz\wdx{*sorcil}{m. `saillie arquée, garnie de
poils,
au-dessus de l'orbite, désignant aussi ses
poils; sourcil'}{surcilz
\emph{pl.}} sont fais po\emph{ur}
beaulté\wdx{*biauté}{f. `caractère de ce qui fait
éprouver une émotion esthétique; beauté'}{beaulté},
et pour
garder\wdx{garder}{v.tr. 1\hoch{o} `préserver qch.
(d'un mal, d'un danger, etc.); protéger'}{} les
yeulx\wdx{ueil}{m. `organe de la vue'}{yeulx
\emph{pl.}}
et pour ce ilz sont adornés\wdx{*aorner}{v.tr.
`rendre beau; orner'}{adorné \emph{p.p.}} de
poilz\wdx{poil}{m. `chacune des productions filiformes qui naissent
du tégument de l'homme et des mammifères; poil'}{poilz \emph{pl.}}.
Item, les
incisions\wdx{incision}{f. `action de fendre,
de couper avec un instrument tranchant, son
résultat (surtout en médecine)'}{},
en
ces parties cy, doivent\wdx{devoir}{v.tr. + inf. `être dans
l'obligation de (faire qch.); devoir'}{}
estre faictes du lonc
des corps et des parties, car ainsi vont les
muscules qui
mouvent\wdx{*movoir}{v.tr. `mettre en
mouvement'}{mouvent \emph{3.p.pl. ind.prés.}}
les surcilz\wdx{*sorcil}{m. `saillie arquée, garnie
de poils,
au-dessus de l'orbite, désignant aussi ses
poils; sourcil'}{surcilz
\emph{pl.}} et non
pas ainsi q\emph{ue} les roies\wdx{roie}{f. `petit pli de la
peau, petit sillon cutané (le plus souvent au front, à
la face, au cou); ride'}{} du
%
[23r\hoch{o}a]
front\wdx{front}{m. `partie supérieure de la
face humaine, comprise entre les sourcils et la
racine des cheveux'}{} vont.
\pend
\pstart
Item, les
yeulx, ce sont les
instrumens\wdx{instrument}{m. 2\hoch{o}
`partie du corps remplissant une fonction
particulière; organe'}{} de la
veue\wdx{veüe}{f. `sens visuel'}{}, et sont
dedans\wdx{dedans}{prép. `à l'intérieur de'}{} le
orbite\wdx{orbite}{m. ou f. terme d'anat.
`structure osseuse en forme d'une concavité
sphérique' (dit de la concavité dans laquelle se
trouve
placé l'\oe il ou dit du conque de l'oreille)}{} ou la
rondesse\wdx{*rëondece}{f. `état de ce qui est rond; rondeur'}{rondesse}
qui est une partie du
coronal\wdx{coronal}{m.
terme d'anat. `os qui forme la partie
antérieure du crâne; os frontal'}{} et des
os des temples\wdx{temple}{m. terme d'anat. `région
latérale de la tête, entre le coin de l'\oe il et le haut
de l'oreille'}{}. Et
raconte Galien\adx{Galien}{}{} le\emph{ur}
naissance\wdx{naissance}{f. terme d'anat. `endroit où
commence
qch. (en parlant des membres, organes ou structures organiques
du corps)'}{}
ou
.x.\hoch{e}
livre ou darrier\wdx{derrier}{adj. `qui vient après tous
les autres, après lequel il n'y a pas
d'autre'}{darrier}
chapitre\wdx{chapitre}{m. `chacune des
parties qui se suivent dans un livre et en articulent la lecture;
chapitre'}{}
du
livre qui se intitule \flq De
utilitate partic\emph{u}la\emph{rum}\frq . Et dit Galien\adx{Galien}{}{}
q\emph{ue} les nerfz
obtiques\wdx{nerf obtique}{adj. terme d'anat.
`structure conjonctive qui relie le cervel
à l'\oe il'}{}, il
doivent estre perforés\wdx{perforé}{adj.
`qui est percé d'un ou de plusieurs
trous;
perforé'}{}, afin que\wdx{afin
que}{loc.conj. qui marque l'intention,
 le but `pour que'}{}
l'esperit\wdx{esperit}{m.
terme de méd.
`ensemble de corpuscules subtils qui assurent
toutes les fonctions de la vie dans
l'organisme humain'}{}
y puist passer\wdx{passer}{v.tr.indir. `aller d'un
lieu à un autre; passer'}{}, et doivent venir\wdx{venir}{v.tr.indir.
1\hoch{o} `avoir
son origine dans'}{}
des deux parties et
\text{se doivent aprés
unir}\fnb{Ms. \emph{se doivent apres a
unir}.}/\wdx{unir}{v.pron. `former un tout;
s'unir'}{} dedans\wdx{dedans}{prép. `à
l'intérieur de'}{} le cra\emph{n}ne, et puis si se
doivent
separer\wdx{separer}{v.pron. `cesser de former un
tout; se séparer'}{}
et aller a ch\emph{acu}m oeil\wdx{ueil}{m. `organe de la vue'}{oeil}, chescum a
son lés\wdx{*lez}{m. `partie qui est à droite
ou à gauche (d'un corps); côté'}{lés} d'ou il
vient\wdx{venir}{v.tr.indir.
1\hoch{o} `avoir
son origine dans'}{},
sa\emph{n}s eulx croiser\wdx{*croisier}{v.tr. `disposer
(deux choses) en croix'}{}. Car ceulx qui
vie\emph{n}ne\emph{n}t\wdx{venir}{v.tr.indir.
1\hoch{o} `avoir
son origine dans'}{}
de la partie dextre\wdx{*destre}{adj.
`qui est du côté droit; droit'}{dextre},
il doit
aller a l'ueil\wdx{ueil}{m.
`organe de la vue'}{} dextre\wdx{*destre}{adj. `qui est du
côté droit; droit'}{dextre}
et non pas a l'autre, ainsi que aucuns le
cuident\wdx{cuidier}{v.tr. `avoir pour pensée;
s'imaginer'}{cuident \emph{3.p.pl. ind.prés.}}.
Item, les yeulx sont
composés\wdx{composer}{v.tr.
`former par l'assemblage, par la combinaison de
parties'}{composé \emph{p.p.}}
et fais de .vij.
tuniques\wdx{tunique}{f.
terme d'anat.
`membrane qui enveloppe
certains organes ou qui constitue la paroi d'un
organ ou d'un vaisseau'}{}
et de trois ou de quatre
humeurs\wdx{humeur}{f. 1\hoch{o} terme de méd.
`substance liquide qui se trouve dans un organisme
humain ou animal'}{}.
La p\emph{re}miere tunique\wdx{tunique}{f.
terme d'anat.
`membrane qui enveloppe
certains organes ou qui constitue la paroi d'un
organ ou d'un vaisseau'}{},
elle est par dehors\wdx{*defors}{adv. `à
l'extérieur'}{dehors} et est blanche\wdx{blanc}{adj.
`qui est de la couleur de la neige; blanc'}{} et
grosse\wdx{gros}{adj.
1\hoch{o}
dans l'ordre physique, quantifiable `qui, dans
son genre, dépasse le volume ordinaire; gros (du corps
humain et de ses parties)'}{}
et environne\wdx{*environer}{v.tr. `être autour
de (qch., qn); environner'}{environne \emph{3.p.sg.
ind.prés.}} tout l'ueil, excepté\wdx{excepté}{prép.
`à la réserve
de'}{} ce qui appert\wdx{*aparoir}{v.intr. `se montrer aux yeux; se
manifester'}{appert \emph{3.p.sg. ind.prés.}} de la
cornee\wdx{cornee}{f.
terme d'anat. `membrane extérieure de l'\oe il située
sur la pupille, correspondante de la sclérotique;
cornée'}{}, et naist du panicle\wdx{*pannicle}{m. terme d'anat.
`couche de tissu musculaire ou cellulaire qui recouvre
une structure organique du corps humain (un
organe, un os, une articulation, un
muscle, etc.)'}{panicle}
qui
coevre le cra\emph{n}ne.
Mais les aultres tuniques\wdx{tunique}{f.
terme d'anat.
`membrane qui enveloppe
certains organes ou qui constitue la paroi d'un
organ ou d'un vaisseau'}{} sont
materialmant\wdx{*materïelement}{adv.
`par rapport
à la matière'}{materialmant}
trois
qui enviro\emph{n}nent\wdx{*environer}{v.tr. `être autour
de (qch., qn); environner'}{environner} tout l'ueil.
Mais pour cause\wdx{cause}{f. `ce qui produit
un effet (considéré par rapport à cet
effet)'}{}
des couleurs\wdx{*color}{f. `caractère de
lumière, de la surface d'un objet, selon
l'impression visuelle particulière qu'elles
produisent; couleur'}{couleur} qui se
varient\wdx{*variier}{v.pron.
`être différent;
différer'}{varient \emph{3.p.pl. ind.prés.}}
et diverciffient\wdx{*diversefiier}{v.pron. `être
différent; différer'}{diverciffient
\emph{3.p.pl. ind.prés.}} enmy\wdx{enmi}{prép.
`au milieu de'}{enmy}
l'ueil, ou lieu que
on appelle
yride\wdx{*iris}{m. terme d'anat. `lieu
diversement coloré qui occupe le
centre de l'\oe il'}{yride}, on dit qu'il
en y a
.vj.
fourmelemant\wdx{*formellement}{adv.
`par sa forme'}{fourmelemant}, trois qui sont vers
la partie du cervel, et trois par
dehors\wdx{*defors}{adv. `à
l'extérieur'}{dehors}. La premiere
%
[23r\hoch{o}b]
tunique\wdx{tunique}{f.
terme d'anat.
`membrane qui enveloppe
certains organes ou qui constitue la paroi d'un
organ ou d'un vaisseau'}{}
naist de la dure
maire\wdx{dure mere}{f.
terme d'anat. `la plus
superficielle et la plus résistante des trois
méninges; dure mère'}{dure maire},
et la p\emph{ar}tie de dedans\wdx{dedans}{adv. `à l'intérieur'}{\textbf{de dedans} \emph{loc.adj. `qui est
situé à l'intérieur'}}, on la appelle
sclirotica\wdx{sclirotica}{mlt.
terme d'anat. `membrane située au fond de l'\oe il
au-dessous de la face interne de la choroïde;
sclérotique'}{}, et
la partie de dehors\wdx{*defors}{adv. `à
l'extérieur'}{\textbf{de dehors} \emph{loc.adj.
`qui est situé à l'extérieur'}}, on l'appelle
cornea\wdx{cornea}{mlt.
terme d'anat. `membrane extérieure de l'\oe il située
sur la pupille, correspondante de la
sclérotique; cornée'}{}. La
seconde tunique\wdx{tunique}{f.
terme d'anat.
`membrane qui enveloppe
certains organes ou qui constitue la paroi d'un
organ ou d'un vaisseau'}{}
nait\wdx{naistre}{v.tr.indir. terme d'anat.
`émaner de (en parlant
des membres, organes ou structures organiques du
corps)'}{nait \emph{3.p.sg. ind.prés.}}
et vient\wdx{venir}{v.tr.indir.
1\hoch{o} `avoir
son origine dans'}{}
de la pie mere\wdx{pie mere}{f.
terme d'anat. `la plus profonde des
méninges, mince et transparente qui enveloppe
directement le cerveau et la moelle épinière;
pie-mère'}{}, et la partie
de dedans\wdx{dedans}{adv. `à l'intérieur'}{\textbf{de dedans} \emph{loc.adj. `qui est
situé à l'intérieur'}}, on la appelle
secondine\wdx{secondine}{f.
terme d'anat. `membrane située au fond de l'\oe il
entre la sclérotique et la rétine; choroïde'}{}, et celle de
dehors\wdx{*defors}{adv. `à
l'extérieur'}{\textbf{de dehors} \emph{loc.adj.
`qui est situé à l'extérieur'}},
on la appelle uvea\wdx{uvea}{mlt. terme d'anat.
`membrane de l'\oe il située au-dessous de la
cornée, correspondante de la face interne de la
choroïde'}{} et ha le trou\wdx{*tro}{m.
`ouverture au travers d'un
corps ou qui y pénètre à une certaine
profondeur; trou'}{trou}
de la pupille\wdx{pupille}{f.
terme d'anat.
`prunelle de l'\oe
il; pupille'}{}. La tierce
tunique\wdx{tunique}{f.
terme d'anat.
`membrane qui enveloppe
certains organes ou qui constitue la paroi d'un
organ ou d'un vaisseau'}{},
elle naist\wdx{naistre}{v.tr.indir.
terme d'anat.
`émaner de (en parlant
des membres, organes ou structures organiques du
corps)'}{naist \emph{3.p.sg. ind.prés.}}
du nerf
optic\wdx{nerf obtique}{adj.
terme d'anat.
`structure conjonctive qui relie le cervel à
l'\oe il'}{nerf optic}, et la
partie de dedans\wdx{dedans}{adv. `à l'intérieur'}{\textbf{de dedans}
\emph{loc.adj. `qui est
situé à l'intérieur'}}, on l'appelle
\text{rethine}\fnb{Ms. \emph{rechine}.}/\wdx{*retine}{f. 1\hoch{o} terme
d'anat.
`membrane formée au fond de l'\oe il par
l'épanouissement du nerf optique; rétine'}{rethine
\emph{(Ms.} rechine \emph{)}}, et la partie
de dehors\wdx{*defors}{adv.
`à l'extérieur'}{\textbf{de dehors} \emph{loc.adj.
`qui est situé à l'extérieur'}} qui est sur l'umeur
cristalline\wdx{humeur cristalline}{f.
terme d'anat. `humeur transparente de l'\oe il située
au milieu de l'\oe il entre l'\flq humeur
albuginee\frq\ et le corps vitré; cristallin'}{umeur
cristalline}, on la appelle
aranee\wdx{aranee}{f. terme d'anat. `membrane de
l'\oe il formant la partie antérieure de la
rétine, située derrière la cornée et sur le
cristallin'}{}. Et ainsi, en
l'ueil sont
.vij. tuniques\wdx{tunique}{f.
terme d'anat.
`membrane qui enveloppe
certains organes ou qui constitue la paroi d'un
organ ou d'un vaisseau'}{}
fourmelemant\wdx{*formellement}{adv. `par sa
forme'}{fourmelemant}
devisees\wdx{deviser}{v.tr. `séparer (qch.) en
plusieurs parties; diviser'}{}. Mais il
n'y a que trois
selon
co\emph{n}tinuacion\wdx{continuacïon}{f.
`action de continuer, le résultat de
cette action'}{continuacion}
materielle\wdx{materïel}{adj.
`qui est de
la nature de la matière, constitué par de la
matière'}{materielle \emph{f.sg.}}.
\pend
\pstart
Des
trois humeurs\wdx{humeur}{f. 1\hoch{o} terme de méd.
`substance liquide qui se trouve dans un organisme
humain ou animal'}{},
la premiere, c'est la
cristalline\wdx{humeur cristalline}{f.
terme d'anat. `humeur transparente de l'\oe il située
au milieu de l'\oe il entre l'\flq humeur
albuginee\frq\ et le corps vitré; cristallin'}{}
qui est assise\wdx{asseoir}{v.tr. 2\hoch{o} `placer, poser
(qch.)'}{assis \emph{p.p.}}
enmi\wdx{enmi}{prép. `au milieu de'}{}
l'ueil, de couleur\wdx{*color}{f.
`caractère de
lumière, de la surface d'un objet, selon
l'impression visuelle particulière qu'elles
produisent; couleur'}{couleur} de
cristail\wdx{*cristal}{m. `matière transparente
et dure qui ressemble à la glace;
cristal'}{cristail}, selon qui
est de fourme\wdx{*forme}{f. `apparence extérieure
donnant à un objet ou à un être sa
spécificité'}{fourme}
\text{de grezil}\fnb{\emph{de}
über der Zeile nachgetragen.}/\wdx{*gresil}{m.
`ensemble de gouttes de pluie transformées en
grains ou en granules de glace sous l'effet de la
congélation; aussi grêlon'}{grezil},
ouquel
principaulma\emph{n}t\wdx{*principaument}{adv. `d'une manière principale'}{principaulment}
la veue\wdx{veüe}{f. `sens visuel'}{} est
fondee\wdx{fonder}{v.tr. `appuyer
(qch.)'}{}. Et aprés, vers le cervel,
vient l'umeur
vitree\wdx{*humeur vitree}{f. terme d'anat. `humeur
de l'\oe il transparente comme du verre, située en
arrière
du cristallin; corps vitré'}{umeur vitree} qui
soustient\wdx{*sostenir}{v.tr. `tenir (qch.)
par-dessous en servant de support ou
d'appui; soutenir'}{}
et comprent\wdx{comprendre}{v.tr. 2\hoch{o} `être
autour de (qch.) de manière à enfermer ou embrasser partiellement ou
complètement; entourer'}{comprent \emph{3.p.sg.
ind.prés.}}
par darrier\wdx{derrier}{adv. `du côté
opposé au visage, à la face'}{darrier}
l'\text{umeur}\fnb{Ms. \emph{umeu}.}/
cristalline\wdx{humeur cristalline}{f.
terme d'anat. `humeur transparente de l'\oe il située
au milieu de l'\oe il entre l'`humeur albuginee'
et le corps vitré; cristallin'}{umeur
cristalline}. Et toutes
deux sont envelloupees\wdx{*envoleper}{v.tr.
`entourer
(qch.) d'une chose souple qui couvre de tous
côtés; envelopper'}{envelloupé \emph{p.p.}}
\text{ou}\fnb{Ms. \emph{ou} l. \emph{du}\,?}/ panicle
devant dit qui
est engendré\wdx{engendrer}{v.tr. au fig. `faire
naître, faire exister; produire'}{}
du nerf obtique\wdx{nerf obtique}{adj.
terme d'anat.
`structure conjonctive qui relie le cervel à l'\oe il'}{}.
Item, aprés, plus vers
la partie de devant\wdx{devant}{adv. 1\hoch{o} `au
côté du visage, à la face'}{\textbf{de devant}
\emph{loc.adj. `qui est situé au côté du visage, de la
face'}}, est l'umeur
albuginee\wdx{*humeur albuginee}{f.
terme d'anat. `humeur de l'\oe il qui est d'une
couleur blanche; humeur aqueuse'}{umeur albuginee}
qui est comprinse\wdx{comprendre}{v.tr. 2\hoch{o} `être
autour de (qch.) de manière à enfermer ou embrasser partiellement ou
complètement; entourer'}{comprinse \emph{p.p. f.sg.}}
entre ycelle toille\wdx{*toile}{f. terme d'anat.
`membrane, mince couche de tissu qui enveloppe un
organe'}{toille} devant
dicte et
l'autre toille\wdx{*toile}{f. terme d'anat.
`membrane, mince couche de tissu qui enveloppe un
organe'}{toille} qui est nee de
la pie maire\wdx{pie mere}{f.
terme d'anat. `la plus profonde des
méninges, mince et transparente qui enveloppe
directement le cerveau et la moelle épinière;
pie-mère'}{pie maire}. Item,
Galien\adx{Galien}{}{} y assigne\wdx{*assignier}{v.tr.
`donner, attribuer (une tâche, un bien, une qualité, etc.)'}{assigne \emph{3.p.sg. ind.prés.}} la quarte
humeur\wdx{humeur}{f. 1\hoch{o} terme de méd.
`substance liquide qui se trouve dans un organisme
humain ou animal'}{} et
le prouve\wdx{*prover}{v.tr. `faire apparaître ou
reconnaître (qch.) comme vrai; prouver'}{prouve
\emph{3.p.sg. ind.prés.}}
ou lieu devant allegué\wdx{*aleguer}{v.tr. `rapporter
un passage, un texte (écrit d'une autorité);
citer'}{allegué \emph{p.p.}}.
Et dit q\emph{ue} en la region
%
[23v\hoch{o}a]
de la pupille\wdx{pupille}{f.
terme d'anat.
`prunelle de l'\oe
il; pupille'}{} est une humeur\wdx{humeur}{f.
1\hoch{o} terme de méd.
`substance liquide qui se trouve dans un organisme
humain ou animal'}{}
de couleur\wdx{*color}{f.
`caractère de
lumière, de la surface d'un objet, selon
l'impression visuelle particulière qu'elles
produisent; couleur'}{couleur}
de
air\wdx{air}{m. `fluide gazeux, répandu autour de la terre, qui sert à la
respiration'}{} luisant\wdx{luisant}{adj.
`qui réfléchit la lumière'}{} et est
plaine\wdx{*plein}{adj.
`qui contient toute
la quantité possible; plein'}{plain}
de escume\wdx{escume}{f.
`mousse qui se forme à la surface d'un liquide qu'on
agite, qu'on chauffe ou qui fermente'}{}. Et telle est
la composicion\wdx{composicion}{f.
`manière dont une chose est formée, par l'assemblage de
plusieurs éléments'}{}
de l'ueil.
Neantmoins\wdx{neantmoins}{adv.
`malgré ce qui
vient d'être dit; néanmoins'}{}, l'ueil \text{ha
aucuns}\fnb{Ms. \emph{ha au aucuns}.}/ nerfz
motis\wdx{motif}{adj. `qui fait mouvoir'}{motis
\emph{m.pl.}}
qui descende\emph{n}t\wdx{descendre}{v.tr.indir. `aller du
haut en bas'}{} de la seconde pere\wdx{paire}{f.
`réunion de deux choses,
de deux êtres semblables
qui vont ensemble; paire'}{pere}
des nerfz du cervel, et si ha .vj. muscules qui le
mouvent, et si ha vaines et arteres et char
spongiouse\wdx{*spongios}{adj. `dont
la structure ressemble à celle de l'éponge;
spongieux'}{spongiouse
\emph{f.sg}} entour les coustes\wdx{coute}{f. `pièce
d'une matière souple, cousue et remplie d'un
rembourrage, servant à supporter quelque partie du
corps; coussin'}{couste}
qui remplissent\wdx{remplir}{v.tr.
`rendre (un
espace disponible) plein de (qch.)'}{} les
espaces\wdx{espace}{m. et f. `lieu, plus ou moins
bien délimité, où peut se situer qch.; espace'}{}, et si ha pres de
luy popieres\wdx{paupiere}{f.
`chacune des deux
membranes mobiles qui en se rapprochant recouvrent le
globe de l'\oe il; paupière'}{popiere}
cartillagineuses\wdx{*cartilaginos}{adj. terme d'anat. `qui est formé de cartilage;
cartilagineux'}{cartillagineuses \emph{f.pl.}}
avec certains
poilz\wdx{poil}{m. `chacune des productions filiformes qui naissent
du tégument de l'homme et des mammifères; poil'}{poilz \emph{pl.}}
determinés\wdx{determiner}{v.tr. `indiquer,
délimiter avec précision; déterminer'}{}, et celles
de dessus\wdx{*desus}{adv.
`au côté supérieur'}{\textbf{de
dessus} \emph{loc.adj. `qui est situé au côté
supérieur'}}
se
cloient\wdx{clore}{v.pron. `se fermer'}{cloient
\emph{3.p.pl. ind.prés.}} par ung muscle et se
\text{oevrant}\fnb{Zur Form, cf.
\flq Die Sprache\frq .}/\wdx{ovrir}{v.pron.
`devenir ouvert'}{oevrant \emph{3.p.pl. ind.prés.}}
par deux aultres q\emph{ue} vont de
travers\wdx{travers}{adv.}{\textbf{de travers}
\emph{loc.adv. `dans une direction transversale'}}.
Item, les aides\wdx{aide}{f. 2\hoch{o}
`caractère de ce qui est
utile, qui satisfait un besoin; utilité'}{} des
yelx\wdx{ueil}{m. `organe de la vue'}{yelx
\emph{pl.}}
et la maniere\wdx{maniere}{f. 1\hoch{o} `nature
propre
à plusieurs personnes ou choses, qui permet de les
considérer comme appartenant à une catégorie
distincte'}{} sont plus
speciffiees\wdx{*especefiier}{v.tr.
`exprimer en particulier, en détail;
spécifier'}{speciffié
\emph{p.p.}} es livres de Alcoatin\adx{Alcoatin}{}{}
et es livres especiaulx des yeulx; toutesvoies, il
souffit\wdx{*sofire}{v.intr.
`avoir la quantité, la qualité, la force
etc. nécessaire pour (qch.); suffire'}{souffit
\emph{3.p.sg. ind.prés.}} au
cirurgien ce q\emph{ue} dit est.
\pend
\pstart
La fourme\wdx{*forme}{f.
`apparence extérieure
donnant à un objet ou à un être sa
spécificité'}{fourme} du
nés\wdx{nés}{m.
`partie saillante du visage,
située dans son axe entre le front et la lèvre
supérieure, et qui abrite la partie antérieure des
fosses nasales; nez'}{}
contient parties qui sont
charneuses\wdx{*charnel}{adj. `qui est
essentiellement constitué de
chair'}{charneuses \emph{f.pl.}}
et ossues\wdx{ossu}{adj.
`qui est de la nature des
os; osseux'}{ossues \emph{f.pl.}} et
cartillagineuses\wdx{*cartilaginos}{adj. terme d'anat. `qui est formé de cartilage;
cartilagineux'}{cartillagineuses \emph{f.pl.}}.
La p\emph{ar}tie charneuse\wdx{*charnel}{adj. `qui est
essentiellement constitué de
chair'}{charneuse \emph{f.sg.}} ha le cuir et
deux muscules qui sont en la extremité\wdx{extremité}{f.
`partie extrème qui termine une chose'}{}
du
nés\wdx{nés}{m.
`partie saillante du visage,
située dans son axe entre le front et la lèvre
supérieure, et qui abrite la partie antérieure des
fosses nasales; nez'}{}.
La partie ossue\wdx{ossu}{adj.
`qui est de la nature des
os; osseux'}{ossue \emph{f.sg.}} ha deux os qui sont
en fourme\wdx{*forme}{f. `apparence extérieure
donnant à un objet ou à un être sa
spécificité'}{fourme} d'ung
triangle\wdx{trïangle}{m. `figure
géométrique, polygone à trois côtés;
triangle'}{triangle}, desqueulx le
angle\wdx{angle}{m. `saillant ou rentrant formé par deux lignes
ou deux surfaces qui se rencontrent; angle'}{} est
sur le nés\wdx{nés}{m.
`partie saillante du visage,
située dans son axe entre le front et la lèvre
supérieure, et qui abrite la partie antérieure des
fosses nasales; nez'}{}
et les
basses\wdx{bas}{adj. `qui se
trouve à une faible hauteur; bas'}{} se
joingnent\wdx{joindre}{v.pron. `se mettre ensemble de
manière à se toucher ou tenir ensemble'}{joingnent
\emph{3.p.pl. ind.prés.}} d'une partie parmy la
longueur\wdx{*longor}{f.
`dimension d'une
chose dans le sens de sa plus grande
étendue; longueur'}{longueur} du
nés\wdx{nés}{m.
`partie saillante du visage,
située dans son axe entre le front et la lèvre
supérieure, et qui abrite la partie antérieure des
fosses nasales; nez'}{},
et d'autre part selon la
longueur\wdx{*longor}{f.
`dimension d'une
chose dans le sens de sa plus grande
étendue; longueur'}{longueur}
des joues\wdx{*joë}{f. `partie latérale de la
face s'étendant entre le nez et l'oreille, du dessous
de l'\oe il au menton'}{joue}. Item, la
partie
cartillagineuse\wdx{*cartilaginos}{adj. terme d'anat. `qui est formé de cartilage;
cartilagineux'}{cartillagineuse \emph{f.sg.}}
est double\wdx{*doble}{adj.
`qui est
répété deux fois, qui vaut deux fois (la
chose désignée) ou qui existe deux fois'}{double}. L'une
va par dehors\wdx{*defors}{adv. `à
l'extérieur'}{dehors}
qui fait les extremités\wdx{extremité}{f.
`partie extrème qui termine une chose'}{}
du nés\wdx{nés}{m.
`partie saillante du visage,
située dans son axe entre le front et la lèvre
supérieure, et qui abrite la partie antérieure des
fosses nasales; nez'}{}. L'autre
va par
%
[23v\hoch{o}b]
dedans\wdx{dedans}{adv. `à
l'intérieur'}{} qui divise\wdx{deviser}{v.tr.
`séparer (qch.) en
plusieurs parties; diviser'}{divise \emph{3.p.sg.
ind.prés.}}
et separe\wdx{separer}{v.tr. `mettre à part les unes
des autres des choses, des personnes
réunies; séparer'}{} les
nazilles\wdx{*nasilles}{f.pl.
`les deux cavités
nasales'}{nazilles}.
Item, les nazilles\wdx{*nasilles}{f.pl.
`les deux cavités
nasales'}{nazilles},
ce so\emph{n}t deux
chanaulx\wdx{chanel}{m.
terme d'anat. `conduit naturel dans le
corps par lequel s'écoule un liquide ou
une matière organique'}{chanaulx
\emph{pl.}}
qui montent\wdx{monter}{v.tr.indir. `se déplacer dans
un mouvement de bas en
haut'}{} jusq\emph{ue}s aux\wdx{jusques a}{loc.prép.
qui marque
le terme final, la limite que l'on ne dépasse
pas}{}
os du
collatoire\wdx{collectoire}{m. terme d'anat.
`partie du
nez qui sert de réceptacle; fosse du nez (?)'}{collatoire}, ou lieu
ouquel
se appliquent\wdx{*apliquier}{v.pron.
`se
placer sur (qch.) de manière à y
adhérer'}{appliquent \emph{3.p.pl. ind.prés.}}
les addiccions mamillaires\wdx{mamillaire}{adj.
terme d'anat.
`qui a la forme d'un mamelon; mamillaire'}{} du
cervel,
esquelles est la vertu\wdx{vertu}{f. `principe
qui, dans une chose, est considéré comme la cause
des effets qu'elle produit; faculté, pouvoir'}{}
odorative\wdx{odoratif}{adj. `qui a rapport à
l'odorat'}{}, et
descendent\wdx{descendre}{v.tr.indir. `aller du haut
en bas'}{}
\text{jusques}\fnb{\emph{s} über der Zeile ersetzt
expungiertes \emph{nt}.}/ au\wdx{jusques a}{loc.prép.
qui marque
le terme final, la limite que l'on ne dépasse
pas}{}
palais\wdx{*palé}{m.
terme d'anat.
`cloison qui forme
la partie supérieure de la cavité buccale et la sépare
des fosses nasales; palais'}{palais} aprés la
uvle\wdx{*uvule}{f. `saillie médiane,
charnue, allongée, du bord postérieur du voile du
palais, qui contribue à la fermeture de la partie
nasale du pharynx lors de la déglutition; luette'}{uvle};
par lesquelz chanaulx\wdx{chanel}{m.
terme d'anat. `conduit naturel dans le
corps par lequel s'écoule un liquide ou
une matière organique'}{chanaulx
\emph{pl.}} on actrait\wdx{*atraire}{v.tr. 1\hoch{o}
`amener (qn, qch.) vers soi ou quelque
part'}{actrait \emph{3.p.sg. ind.prés.}}
la evaporacion\wdx{evaporacïon}{f.
`transformation d'un liquide en vapeur
par sa surface libre; évaporation'}{evaporacion}
et les fumees\wdx{fumee}{f. `mélange plus ou moins
dense et de couleur variable, de produits gazeux et de
très fines particules solides, qui se dégage de corps
en combustion ou portés à haute température; fumée'}{}
ou dit lieu, et p\emph{ar} la on inspire\wdx{inspirer}{v.tr.
`aspirer l'air dans les poumons; inspirer'}{}
et respire\wdx{respirer}{v.tr. `aspirer (qch.) par
les voies respiratoires'}{} l'air qui va au
poulmon\wdx{*poumon}{m. terme d'anat.
`chacun des deux viscères logés
symétriquement dans la cage thoracique qui sont les
organes de la respiration, aussi l'ensemble
des deux; poumon'}{poulmon}
en son
temps\wdx{*tens}{m. `milieu indéfini où paraissent se dérouler irréversiblement
les existences dans leur changement, les événements et les
phénomènes dans leur succession; temps'}{temps}, et par la sont
purgees\wdx{purgier}{v.tr. 2\hoch{o} terme de méd.
`expulser (qch.) de l'organisme' (des
substances, des impuretés nuisibles à la
santé)}{purgé
\emph{p.p.}}
les sup\emph{er}fluités\wdx{superfluité}{f. terme de méd.
`sécrétion abondante du corps'}{} du
cervel.
\pend
\pstart
Les oreilles\wdx{oreille}{f. 1\hoch{o} `l'un des
deux organes constituant l'appareil
auditif, aussi sa partie visible; oreille'}{} sont
cartillagineuses\wdx{*cartilaginos}{adj. terme d'anat. `qui est formé de cartilage;
cartilagineux'}{cartillagineuses \emph{f.pl.}}
et
affractueuses\wdx{affractueux}{adj. `qui est tordu' (?)}{affracteuses \emph{f.pl.}}, et sont
assises\wdx{asseoir}{v.tr. 2\hoch{o} `placer, poser
(qch.)'}{assis \emph{p.p.}}
sur l'os petroux\wdx{*os petros}{m. terme d'anat. `partie massive du temporal
en forme de pyramide quadrangulaire; rocher'}{os petroux}
et sont ordonnees\wdx{*ordener}{v.tr. 2\hoch{o}
`établir (qn, qch.) pour une foncion'}{ordonné
\emph{p.p.}}
pour oïr\wdx{oïr}{v.tr. empl.abs.
`percevoir par le sens de l'ouïe; ouïr'}{}.
Auxquelles viennent
-- par troux\wdx{*tro}{m. `ouverture au travers d'un
corps ou qui y pénètre à une certaine
profondeur; trou'}{troux \emph{pl.}}
tortus\wdx{tortu}{adj.
`qui présente des courbes
irrégulières; tortu'}{} du dit os
-- aucuns porres\wdx{*pore}{f. `interstice qui
sépare les parties d'un corps'}{porre} ou nerfz du
.viij.\hoch{e} pareil\wdx{pareil}{m.
`réunion de deux choses,
de deux êtres semblables
qui vont ensemble; paire'}{} dez nerfz du
cervel, esqueulx est
la vertu\wdx{vertu}{f. `principe
qui, dans une chose, est considéré comme la cause
des effets qu'elle produit; faculté, pouvoir'}{}
\text{auditive}\fnb{\emph{Nota} und
Zeichnung am Foliorand.}/%
\wdx{auditif}{adj. `qui a rapport à la faculté de
ouïr'}{}. Et dessoubz\wdx{*desoz}{prép.
qui marque la position en bas par rapport
à ce qui est en haut `sous'}{dessoubz}
les oreilles\wdx{oreille}{f. 1\hoch{o} `l'un des deux
organes constituant l'appareil
auditif, aussi sa partie visible; oreille'}{} sont
chars
glandelleuses\wdx{*glandulos}{adj. `qui contient des
glandes'}{glandelleuse
\emph{f.sg.}} qui sont les emu\emph{n}ctoires\wdx{*emomptoire}{m.
terme d'anat.
`organe
qui élimine les substances inutiles formées au
cours des processus de désassimilation (l'anus,
l'uretère, etc.)'}{emunctoire}
du cervel
et dalés\wdx{*dalez}{prép. `à côté de'}{dalés} ces
lieux
la passent\wdx{passer}{v.tr.indir. `aller d'un
lieu à un autre; passer'}{}
aucunes vaines, lesquelles, si co\emph{m}me dit
Laffrant\adx{Lanfrant}{}{Laffrant},
portent\wdx{porter}{v.tr.
`déplacer (qch.)
d'un lieu à un autre en le menant avec soi;
transporter'}{} une partie de
matere sp\emph{er}matique\wdx{matiere spermatique}{f.
terme de méd.
`semence, cellule ou groupe de cellules
dont se forme un organisme'}{matere
spermatique}
aux coillons\wdx{coillon}{m. 1\hoch{o} `gonade mâle suspendue dans le scrotum, qui
produit les spermatozoïdes; testicule'}{}, et se les
dictes vaines estoient
coupees, l'o\emph{m}me ne pourroit
engendrer\wdx{engendrer}{v.intr. `donner la vie à
un enfant'}{}. Mais Galien\adx{Galien}{}{}
tient\wdx{tenir}{v.tr. 2\hoch{o}
`considérer (qch.)'}{}
le contraire\wdx{contraire}{m. `ce qui est
opposé logiquement; contraire'}{}, si co\emph{m}me
Avicene\adx{Avicene}{}{} le
recite\wdx{reciter}{v.tr. `faire connaître (les
paroles, etc., de qn); rapporter'}{} ou
livre de
flebothomie\wdx{*flebotomie}{f.
`évacuation provoquée d'une certaine
quantité de sang; saignée'}{flebothomie}.
\pend
\pstart
Item, les temples\wdx{temple}{m.
terme d'anat.
`région
latérale de la tête, entre le coin de l'\oe il et le haut
de l'oreille'}{}, les
joues\wdx{*joë}{f. `partie latérale de la
face s'étendant entre le nez et l'oreille, du dessous
de l'\oe il au menton'}{joue} et les
nazilles\wdx{*nasilles}{f.pl.
`les deux cavités
nasales'}{nazilles},
ce sont parties
lateralles\wdx{lateral}{adj. `qui appartient au
côté, qui est situé sur le côté de qch.; latéral'}{}
de la face\wdx{face}{f. `partie antérieure de la
tête; visage'}{} et contiennent char
musculeuse\wdx{musculeux}{adj. terme d'anat. `qui est de la nature
des muscles'}{}
%
[24r\hoch{o}a]
avec vaines et arteres, avec parties
ossues\wdx{ossu}{adj.
`qui est de la nature des
os; osseux'}{ossues \emph{f.pl.}}.
Item, en ycelles parties sont pluseurs\wdx{*plusor}{adj. `un certain
nombre'}{pluseurs \emph{pl.}}
muscules.
Premier\wdx{premier}{adv.
`en
premier lieu, d'abord; premièrement'}{} il sont .vij.
muscules qui moiennent\wdx{*moiener}{v.tr. `servir
d'intermédiaire de qch.'}{moiennent \emph{3.p.pl.
ind.prés.}} les joues\wdx{*joë}{f. `partie latérale
de la
face s'étendant entre le nez et l'oreille, du dessous
de l'\oe il au menton'}{joue} et les
levres\wdx{levre}{m.
`chacune des deux parties charnues qui bordent la
bouche; lèvre'}{} q\emph{ue} sont dessus\wdx{*desus}{adv.
`au côté supérieur'}{dessus}.
Et selon Avicene\adx{Avicene}{}{}, les dis muscules viennent de la
fourchine\wdx{fourchine}{f.
terme d'anat. `os long, en
forme d'un S allongé, formant la partie antérieure de
la ceinture scapulaire; clavicule'}{}
ou fourcelle\wdx{*forcele}{f. 1\hoch{o} terme d'anat.
`os long, en
forme d'un S allongé, formant la partie antérieure de
la ceinture scapulaire; clavicule'}{fourcelle}
\text{et}\fnb{Über der Zeile nachgetragen.}/
des parties basses\wdx{bas}{adj. `qui se
trouve à une faible hauteur; bas'}{basse
\emph{f.sg.}}. Item,
en aprés \text{sont}\fnb{Ms. \emph{sont sont}.}/ .xij.
aultres, selon
Haliabas\adx{Haliabas}{}{}, \text{qui
mouvent la}\fnb{\emph{Nota} in Foliomitte.}/
mandibule\wdx{mandibule}{f. `chacun
des deux arcs osseux de la bouche dans lesquels
sont plantées les dents, qui sont articulés et qui
servent à mâcher; mâchoire'}{} de
dessoubz\wdx{*desoz}{adv. `à la face
inférieure'}{\textbf{de dessoubz} \emph{loc.adj.
`qui est situé au côté inférieur'}}. Et les aucuns la
oevrent
qui viennent des \text{lieux de
clefs}\fnb{Ms. \emph{clers} (cp.
GuiChaul\textsc{jl} 31,18 \emph{a loco
clavium}).}/\wdx{clef}{f. 2\hoch{o} terme d'anat. `os
long, en
forme d'un S allongé, formant la partie antérieure de
la ceinture scapulaire; clavicule'}{clefs}
devers\wdx{devers}{prép.
`du côté de'}{} la
partie des
oreilles\wdx{oreille}{f. 1\hoch{o} `l'un des deux organes
constituant l'appareil
auditif, aussi sa partie visible; oreille'}{}. Et aucuns la
cloient\wdx{clore}{v.tr. `boucher ce qui est ouvert;
fermer'}{cloient \emph{3.p.pl. ind.prés.}} qui
descendent\wdx{descendre}{v.tr.indir. `aller du haut
en bas'}{} de dessus\wdx{*desus}{adv. `au côté
supérieur'}{dessus}
et passent\wdx{passer}{v.tr.indir. `aller d'un
lieu à un autre; passer'}{}
par dessoubz\wdx{*desoz}{adv. `à la face
inférieure'}{dessoubz} l'ance\wdx{*anse}{f. terme
d'anat.
`structure
anatomique qui a la forme de la partie
recourbée et saillante d'un ustensile qui
permet de le saisir, de le porter' (par analogie de forme)}{ance}
des os des temples\wdx{temple}{m.
terme d'anat.
`région
latérale de la tête, entre le coin de l'\oe il et le haut
de l'oreille'}{}, et les appelle on
timporales\wdx{*temporal}{adj. terme d'anat. `qui
appartient aux tempes'}{timporal}, et
sont moult nobles\wdx{noble}{adj. `qui joue un rôle
important dans le corps'}{} et moult
sencitif\wdx{*sensitif}{adj.
1\hoch{o} `qui appartient à la
sensibilité'}{sencitif \emph{m.sg.}}, et quant ilz
sont blecez\wdx{*blecier}{v.tr. 1\hoch{o} `causer une
blessure' (à une partie du corps)}{blecez \emph{p.p.
m.pl.}}, c'est chose
\text{moult}\fnb{\emph{Nota} und Kürzel des
Schreibers am linken Foliorand.}/ perilleuse\wdx{*perillos}{adj.
`qui constitue un danger, présente du
danger; dangereux'}{perilleuse \emph{f.sg.}}. Et pour ce,
nature\wdx{nature}{f. 2\hoch{o}
`principe actif qui
anime, organise l'ensemble des choses existantes selon
un certain ordre'}{},
pour yceulx
garder\wdx{garder}{v.tr. 1\hoch{o} `préserver qch.
(d'un mal, d'un danger, etc.); protéger'}{},
ha sagemant\wdx{*sagement}{adv. `d'une
manière avisée, judicieuse; sagement'}{sagemant}
ordonné\wdx{*ordener}{v.tr.
1\hoch{o}
`disposer, mettre
dans un certain ordre; ordonner'}{ordonné
\emph{p.p.}}
la dicte
ance\wdx{*anse}{f. terme d'anat. `structure anatomique qui a la forme
de la partie
recourbée et saillante d'un ustensile qui
permet de le saisir, de le porter' (par analogie de forme)}{ance} es os du temple\wdx{temple}{m.
terme d'anat.
`région
latérale de la tête, entre le coin de l'\oe il et le haut
de l'oreille'}{}.
Item, y sont aultres muscules qui sont fais pour
mordre\wdx{mordre}{v.tr. empl.abs. `saisir et
serrer avec les dents de manière à blesser, à
entamer, à retenir; mordre'}{} et
mastiguer\wdx{*mastiquer}{v.tr. empl.abs.
`broyer, écraser avec les dents par le
mouvement des mâchoires, avant d'avaler;
mâcher'}{mastiguer}
ou machier\wdx{*maschier}{v.tr.
`broyer, écraser avec les dents par le
mouvement des mâchoires, avant d'avaler;
mâcher'}{machier \emph{inf.}}, et yceulx viennent
des po\emph{m}mes\wdx{*pome des maxilles}{f.
terme d'anat.
`partie plus
ou
moins saillante de la joue'}{pomme des maxilles} des
maxilles. Et a tous ces muscules ci
vie\emph{n}nent nerfz de la tierce pere\wdx{paire}{f.
`réunion de deux choses,
de deux êtres semblables
qui vont ensemble; paire'}{pere}
des nerfz du cervel.
Item, avec eulx y a
plus\emph{iors} vaines
et arteres, maiemant\wdx{*maismement}{adv. `plus que tout
autre chose; surtout'}{maiemant}
entour les
temples\wdx{temple}{m. terme d'anat. `région
latérale de la tête, entre le coin de l'\oe il et le haut
de l'oreille'}{} et entour les
angles\wdx{angle}{m. `saillant ou rentrant formé par
deux lignes ou deux surfaces qui se
rencontrent; angle'}{} des yeulx et entour
chescune
leffre\wdx{levre}{m.
`chacune des deux parties charnues qui bordent la
bouche; lèvre'}{leffre}.
\pend
\pstart
Item, les parties ossues\wdx{ossu}{adj.
`qui est de la nature des
os; osseux'}{ossues \emph{f.pl.}}
de icelles parties sont
pluseurs. Premier\wdx{premier}{adv.
`en
premier lieu, d'abord; premièrement'}{} les os des joues\wdx{*joë}{f. `partie latérale de la
face s'étendant entre le nez et l'oreille, du dessous
de l'\oe il au menton'}{joue},
ja soit ce qu\wdx{ja soit ce que}{loc.conj.
`bien que assurément'}{}'il n'en
%
[24r\hoch{o}b]
\text{appart}\fnb{Ms. \emph{appare}.}/ que deux qui
sont joings\wdx{joindre}{v.tr.
`mettre des choses
ensemble, de façon qu'elles se touchent ou tiennent
ensemble; joindre'}{joings \emph{p.p. m.pl.}}
desoubz\wdx{*desoz}{prép.
qui marque la position en bas par rapport
à ce qui est en haut `sous'}{desoubz}
le nés\wdx{nés}{m.
`partie saillante du visage,
située dans son axe entre le front et la lèvre
supérieure, et qui abrite la partie antérieure des
fosses nasales; nez'}{}.
Toutesvoies, ilz so\emph{n}t .ix., si co\emph{m}me le dit
Galien\adx{Galien}{}{}. Item,
\text{aussi}\fnb{\emph{Nota} und Kürzel des Schreibers
am Foliorand.}/
il sont deux os du temple\wdx{temple}{m.
terme d'anat.
`région
latérale de la tête, entre le coin de l'\oe il et le haut
de l'oreille'}{} qui font une p\emph{ar}tie
orbite\wdx{orbite}{adj.
`en forme de sphère; sphérique'}{}, c'est
a dire\wdx{dire}{v.tr.
`lire à haute voix;
réciter'}{\textbf{c'est a dire} \emph{loc.conj. qui
annonce une explication ou une précision}}
ronde\wdx{*rëont}{adj. `qui a la forme
circulaire'}{ronde \emph{f.sg.}}, et font
l'eminence\wdx{eminence}{f.
terme d'anat. `saillie à la surface d'une structure
anatomique'}{}
de la po\emph{m}me de
la joue\wdx{*pome de la joë}{f. terme d'anat.
`partie plus ou
moins saillante de la joue'}{pomme de la joue}
et font une
addicion\wdx{addicion}{f. `éminence à la surface d'une structure
osseuse ou cartilagineuse'}{}
vers la addicion\wdx{addicion}{f. `éminence à la surface d'une
structure osseuse ou cartilagineuse'}{} de l'os
petroux\wdx{*os petros}{m. terme d'anat. `partie massive du temporal
en forme de pyramide quadrangulaire; rocher'}{os petroux}
en ordonna\emph{n}t\wdx{*ordener}{v.tr. 1\hoch{o}
`disposer, mettre
dans un certain ordre; ordonner'}{ordonnant
\emph{p.prés.}}
la dicte ance\wdx{*anse}{f. terme d'anat.
`structure anatomique qui a la forme
de la partie
recourbée et saillante d'un ustensile qui
permet de le saisir, de le porter' (par analogie de forme)}{ance},
desoubz laquelle
sont gardés et passent\wdx{passer}{v.tr.indir. `aller d'un
lieu à un autre; passer'}{}
les muscules du
temple\wdx{temple}{m. terme d'anat. `région
latérale de la tête, entre le coin de l'\oe il et le haut
de l'oreille'}{}. Item,
es joues\wdx{*joë}{f. `partie latérale de la
face s'étendant entre le nez et l'oreille, du dessous
de l'\oe il au menton'}{joue} sont les os
de la mandibule\wdx{mandibule}{f. `chacun
des deux arcs osseux de la bouche dans lesquels
sont plantées les dents, qui sont articulés et qui
servent à mâcher; mâchoire'}{} de
dessoubz\wdx{*desoz}{adv.
`à la face
inférieure'}{\textbf{de dessoubz} \emph{loc.adj.
`qui est situé au côté inférieur'}},
desqueulx dit Galien\adx{Galien}{}{} en le .xj.\hoch{e}
livre, ou penultime\wdx{*penultieme}{adj.
`avant-dernier'}{penultime}
chappitre: \emph{inferioris vero gene \emph{et}
c\emph{etera}}. C'est a dire\wdx{dire}{v.tr.
`lire à haute voix;
réciter'}{\textbf{c'est a dire} \emph{loc.conj. qui
annonce une explication ou une précision}},
l'os de la joue\wdx{*joë}{f. `partie latérale de la
face s'étendant entre le nez et l'oreille, du dessous
de l'\oe il au menton'}{joue} de
dessoubz\wdx{*desoz}{adv. `à la face
inférieure'}{\textbf{de dessoubz} \emph{loc.adj.
`qui est situé au côté inférieur'}},
il ha en lui une seule\wdx{*sol}{adj. `qui n'est pas
avec d'autres semblables; seul'}{seul}
division\wdx{*devisïon}{f.
`action de
diviser (qch.) en parties, le résultat de
cette action'}{division}, no\emph{n} pas du tout
manifeste\wdx{manifest}{adj. `dont l'existence ou la nature
est évidente; manifeste'}{}
selon la
extremité\wdx{extremité}{f.
`partie extrème qui termine une chose'}{}
de la barbe\wdx{barbe}{f. `poil du
menton, des joues et de la lèvre supérieure;
barbe'}{}, laquelle est faicte
-- si co\emph{m}me no\emph{us} dient
--
\text{pour
cause}\fnb{Über der Zeile nachgetragen.}/\wdx{cause}{f. `ce qui produit
un effet (considéré par rapport à cet
effet)'}{}
de
drinato\emph{rum}\wdx{drinatorum}{lt. `?'}{}. Et es
extremités\wdx{extremité}{f.
`partie extrème qui termine une chose'}{}
il ha une inegalité\wdx{inegalité}{f.
`défaut d'uniformité, de regularité (d'une
surface)'}{} qui est mise\wdx{metre}{v.tr. 1\hoch{o}
`placer
(qch.) dans une position déterminée'}{}
entour le
muscule du \text{te\emph{m}ple}\fnb{Voranstehend gestrichenes
\emph{ch}.}/ et le \text{thenon}\fnb{Ms.
\emph{chenon}.}/\wdx{*tendant}{m. terme d'anat. `structure conjonctive fibreuse par laquelle
un muscle s'insère sur un os'}{thenon \emph{(Ms.}
chenon \emph{)}}
et les explantacions\wdx{explantacion}{f. `ce qui
présente un arrachement' (?)}{}.
\pend
\pstart
Et
en aprés il nous co\emph{n}vie\emph{n}t\wdx{*covenir}{v.tr.indir.
`être
convenable pour (qn)'}{convient \emph{3.p.sg. ind.prés.}}
venir aulx parties de la bouche\wdx{*boche}{f.
`cavité située à la partie
inférieure du visage de l'homme, bordée
par les lèvres; bouche'}{bouche} q\emph{u}e
sont
.v., c'est
assavoir\wdx{assavoir}{v.tr.}{\textbf{c'est
assavoir} \emph{loc.
`c'est-à-dire'}}
les levres\wdx{levre}{m.
`chacune des deux parties charnues qui bordent la
bouche; lèvre'}{}, desquelles dit est
ci devant, les
dens\wdx{dent}{m. et f.
`organe de la bouche, de
couleur blanchâtre, dur, implanté sur le
maxillaire'}{dens \emph{pl.}},
la lengue\wdx{langue}{f. `organe charnu, musculeux, allongé et mobile, placé dans la bouche;
langue'}{lengue},
le palatre\wdx{palatre}{m.
`cloison qui forme
la partie supérieure de la cavité buccale et la sépare
des fosses nasales; palais'}{} et la
goule\wdx{*gole}{f. `parties antérieures et latérales du cou;
gorge'}{goule}. Les
dens\wdx{dent}{m. et f.
`organe de la bouche, de
couleur blanchâtre, dur, implanté sur le
maxillaire'}{dens \emph{pl.}}
sont de la nature\wdx{nature}{f.
1\hoch{o} `ensemble des
caractères, des propriétés qui définissent un être,
une chose concrète ou abstraite'}{}
des os, ja soit ce qu'ilz aient
sentemant\wdx{*sentement}{m.
`faculté d'éprouver
les impressions que font les objets matériels, i.e.
goût, odorat, ouïe, toucher, vue'}{sentemant},
si co\emph{m}me on dit selon Galien\adx{Galien}{}{} ou
.xvj.\hoch{e}
livre de \flq Utilitate particularu\emph{m}\frq ; toutesvoies, ilz
ont sentemant\wdx{*sentement}{m.
`faculté d'éprouver
les impressions que font les objets matériels, i.e.
goût, odorat, ouïe, toucher, vue'}{sentemant}
pour cause
de aucuns nerfz qui
descendent\wdx{descendre}{v.tr.indir. `aller du haut
en bas'}{} de la tierce pere\wdx{paire}{f.
`réunion de deux choses,
de deux êtres semblables
qui vont ensemble; paire'}{pere}
dez nerfz du cervel,
%
[24v\hoch{o}a]
qui viennent a la racine\wdx{racine}{f. 1\hoch{o}
terme d'anat.
`portion d'un organe
servant à son implantation dans un autre organe'}{}
des dens\wdx{dent}{m. et f.
`organe de la bouche, de
couleur blanchâtre, dur, implanté sur le
maxillaire'}{dens \emph{pl.}}. Et
co\emph{m}munemant\wdx{*comunement}{adv. `en
général'}{communemant} on ha
.xxxij.
de\emph{n}s\wdx{dent}{m. et f.
`organe de la bouche, de
couleur blanchâtre, dur, implanté sur le
maxillaire'}{dens \emph{pl.}},
en chescune mandibule\wdx{mandibule}{f. `chacun
des deux arcs osseux de la bouche dans lesquels
sont plantées les dents, qui sont articulés et qui
servent à mâcher; mâchoire'}{}
.xvj., ja soit ce que
\text{aucuns}\fnb{Nachfolgend gestrichener
Buchstabenansatz.}/ n'en aient que .xxviij.
C'est
assavoir\wdx{assavoir}{v.tr.}{\textbf{c'est
assavoir} \emph{loc.
`c'est-à-dire'}}
les
duodenales\wdx{duodenale}{subst.
`chacun des deux
dents incisives intérieures'}{} qui sont premiers, et
deux quadruples\wdx{quadruple}{subst.
terme d'anat.
`dent
incisive
extérieure qui fait suite aux deux dents incisives
intérieures'}{} qui
\text{font}\fnb{Ms. \emph{sont}.}/
quatre, et deux canins\wdx{canin}{m. terme d'anat.
`dent pointue entre les prémolaires et les
incisives; canine'}{}
et .viij. mollaires\wdx{*molaire}{f. `dent de la
partie postérieure de la mâchoire, dont la fonction
est de broyer; molaire'}{mollaire} et
deux caissales\wdx{caissale}{f. terme d'anat. `dent
de sagesse' (?)}{},
et ont leur racines\wdx{racine}{f. 1\hoch{o} terme
d'anat.
`portion d'un organe
servant à son implantation dans un autre organe'}{}
fichees\wdx{fichier}{v.tr. `faire pénétrer et
fixer par la pointe'}{fiché
\emph{p.p.}} es mandibules\wdx{mandibule}{f. `chacun
des deux arcs osseux de la bouche dans lesquels
sont plantées les dents, qui sont articulés et qui
servent à mâcher; mâchoire'}{}.
Aucuns ont une
racine\wdx{racine}{f. 1\hoch{o} terme d'anat.
`portion d'un organe
servant à son implantation dans un autre organe'}{},
aucuns deux, aucuns trois et aucuns quatre,
et leur aide\wdx{aide}{f. 2\hoch{o}
`caractère de ce qui est
utile, qui satisfait un besoin; utilité'}{} est bien cogneue\wdx{*conoistre}{v.tr.
1\hoch{o} `avoir une idée de (qch.);
connaître'}{cogneue \emph{p.p. f.sg.}}.
\pend
\pstart
La
le\emph{n}gue\wdx{langue}{f. `organe charnu,
musculeux, allongé et mobile, placé dans la bouche;
langue'}{lengue}
est une partie charneuse\wdx{*charnel}{adj. `qui est
essentiellement constitué de
chair'}{charneuse \emph{f.sg.}},
\text{rude et spo\emph{n}gieuse}\fnb{Am Zeilenrand
nachgetragen.}/\wdx{rude}{adj.
`qui est âpre au toucher'}{}\wdx{*spongios}{adj.
`dont la structure ressemble à celle de l'éponge;
spongieux'}{spongieuse
\emph{f.sg.}}, faite de pluseurs nerfz, de
liguemans, de vaines et de arteres. Et est ordo\emph{n}nee\wdx{*ordener}{v.tr. 2\hoch{o}
`établir (qn, qch.) pour une foncion'}{ordonné
\emph{p.p.}}
principaulmant\wdx{*principaument}{adv. `d'une manière principale'}{principaulmant}
pour gouster\wdx{*goster}{v.tr. `percevoir, apprécier
par le sens du goût; goûter'}{gouster \emph{inf.}}
et pour parler\wdx{parler}{v.intr. `articuler les
sons d'une langue naturelle; parler' }{} et pour
governer\wdx{governer}{v.tr.
`diriger
(la conduite de qn, de qch.), exercer une influence
(sur qn, sur qch.)'}{} la
viande\wdx{viande}{f. `aliment dont on se nourrit'}{} en
la bouche\wdx{*boche}{f. `cavité située à la partie
inférieure du visage de l'homme, bordée
par les lèvres; bouche'}{bouche}. A la
le\emph{n}gue\wdx{langue}{f. `organe charnu, musculeux, allongé et mobile, placé dans la bouche;
langue'}{lengue}
vien\-nent nerfz
gustatifz\wdx{gustatif}{adj. `qui a rapport à
l'organe du goût'}{} et
motifz\wdx{motif}{adj. `qui fait mouvoir'}{motifz
\emph{m.pl.}} de
la quarte et .vj.\hoch{e} pere\wdx{paire}{f.
`réunion de deux choses,
de deux êtres semblables
qui vont ensemble; paire'}{pere}
des dis nerfz, et si ha
.ix.
muscules qui naisse\emph{n}t de l'addiccion du
sagitale\wdx{sagitale}{m. terme d'anat. `articulation
entre les os pariétaux du crâne; suture
sagitalle'}{} et de l'os que on appelle
landifourme\wdx{landifourme}{adj. `qui a la forme
d'un lambda grec'}{}. Et dessoubz\wdx{*desoz}{prép.
qui marque la position en bas par rapport
à ce qui est en haut `sous'}{dessoubz}
la lengue\wdx{langue}{f. `organe charnu, musculeux, allongé et mobile, placé dans la bouche;
langue'}{lengue}
sont chars
gla\emph{n}dellouses\wdx{*glandulos}{adj. `qui contient
des glandes'}{glandellouse \emph{f.sg.}},
ou sont deux oriffices\wdx{orifice}{m.
`ouverture faisant communiquer un conduit, un organe
avec une structure voisine ou avec l'extérieur;
orifice'}{oriffice} par ou la
\text{salive}\fnb{Voranstehend gestrichenes
\emph{s}.}/\wdx{salive}{f. `liquide clair produit
par les glandes de la bouche;
salive'}{} ist, qui sont
co\emph{m}me \text{coutes}\fnb{Ms. \emph{toutes}.}/\wdx{coute}{f. `pièce
d'une matière souple, cousue et remplie d'un
rembourrage, servant à supporter quelque partie du
corps; coussin'}{}
et humectatoires\wdx{humectatoire}{f. `ce qui sert à
humecter'}{} de ycelle.
Et derrier\wdx{derrier}{prép. `en arrière de'}{} la
lengue\wdx{langue}{f. `organe charnu, musculeux, allongé et mobile, placé dans la bouche;
langue'}{lengue},
vers le palais\wdx{*palé}{m. terme d'anat.
`cloison qui forme
la partie supérieure de la cavité buccale et la sépare
des fosses nasales; palais'}{palais}, sont les
joues\wdx{*joë}{f. `partie latérale de la
face s'étendant entre le nez et l'oreille, du dessous
de l'\oe il au menton'}{joue} et les
amigdalles\wdx{*amigdale}{f. terme d'anat. `chacune des deux organes
situés sur la paroie latérale du larynx,
entre les piliers
du voile du palais; amygdale'}{amigdalle} et la
\text{uvle}\fnb{Voranstehend gestrichenes
\emph{g}.}/\wdx{*uvule}{f.
`saillie médiane,
charnue, allongée, du bord postérieur du voile du
palais, qui contribue à la fermeture de la partie
nasale du pharynx lors de la déglutition; luette'}{uve}
qui pent\wdx{*pendiier}{v.intr. `être fixé par le
haut, la partie inférieure restant libre;
pendre'}{pent \emph{3.p.sg. ind.prés.}} pour
appareiller\wdx{appareiller}{v.tr. `mettre (qch.) en
état de remplir sa fonction'}{} l'air a son
aide\wdx{aide}{f. 2\hoch{o}
`caractère de ce qui est
utile, qui satisfait un besoin; utilité'}{}.
Le palais\wdx{*palé}{m. terme d'anat.
`cloison qui forme
la partie supérieure de la cavité buccale et la sépare
des fosses nasales; palais'}{palais}
on appelle toute la
haulte\wdx{hault}{adj.
`qui est d'une certaine dimension
dans le sens vertical; haut'}{}
p\emph{ar}tie de la bouche\wdx{*boche}{f. `cavité située à
la partie inférieure du visage de l'homme, bordée
par les lèvres; bouche'}{bouche}, et est
couvert\wdx{*covrir}{v.tr. `garnir qch.
en disposant qch. dessus'}{couvert \emph{p.p.}},
luy et toutes ces parties, d'ung panicle,
%
[24v\hoch{o}b]
lequel naist du panicle de dedans\wdx{dedans}{prép. `à
l'intérieur de'}{}
l'estomac\wdx{estomac}{m.
terme d'anat.
`organe de l'appareil
digestif qui reçoit les aliments; estomac'}{}
et du mery\wdx{meri}{m.
terme d'anat.
`canal musculo-membraneux qui va
du pharynx à l'estomac auquel il conduit
les aliments; \oe sophage'}{mery}. Et vous
souffise\wdx{*sofire}{v.intr.
`avoir la quantité, la qualité, la force
etc. nécessaire pour (qch.);
suffire'}{\textbf{*sofire a qn}
\emph{v.impers. + subj.prés.} vous souffise}
des
parties de la face\wdx{face}{f. `partie antérieure de la
tête; visage'}{}. Les dictes parties peullent\wdx{*pooir}{v.tr. +
inf. `avoir la possibilité de (faire qch.);
pouvoir'}{peullent \emph{3.p.pl. ind.prés.}}
souffrir\wdx{*sofrir}{v.tr. `supporter
qch. de pénible ou de désagréable;
souffrir'}{souffrir \emph{inf.}}
maintes\wdx{maint}{adj. `plusieurs; maint'}{}
maladies\wdx{maladie}{f.
 `altération organique ou
fonctionnelle considérée dans son évolution, et comme
une entité définissable; maladie'}{}
\text{et}\fnb{Über der Zeile
nachgetragen.}/
diversses\wdx{divers}{adj. `qui
présente une différence par rapport à une autre
chose ou une autre personne; différent'}{diversses
\emph{f.pl.}}. Et en curer\wdx{curer}{v.tr.
`soumettre à un traitement médical'}{} et
pronostiquer\wdx{pronostiquer}{v.tr. `émettre un
pronostic au sujet de l'évolution d'une maladie, de sa
gravité; pronostiquer'}{} peullent\wdx{*pooir}{v.tr. +
inf. `avoir la possibilité de (faire qch.);
pouvoir'}{peullent \emph{3.p.pl. ind.prés.}}
moult valoir
les choses devant d\emph{i}tes.
\pend
%
% \memorybreak
%
\pstartueber
Le tiers chappitre: De l'anathomie du
col\wdx{col}{m. 1\hoch{o} `partie du corps de
l'homme et de certains vertébrés qui unit la
tête au tronc'}{} et de ces
parties.
\pendueber
%
% \memorybreak
%
\pstart
CHESCUN peult bien cognoistre\wdx{*conoistre}{v.tr.
1\hoch{o}
`avoir une idée de (qch.); connaître'}{cognoistre
\emph{inf.}} quel chose
est le col\wdx{col}{m. 1\hoch{o} `partie du corps de
l'homme et de certains vertébrés qui unit la
tête au tronc'}{}, et qu'il y a vaines,
arteres qui passent\wdx{passer}{v.tr.indir. `aller d'un
lieu à un autre; passer'}{}
p\emph{ar}my, et nerfz qui montent\wdx{monter}{v.tr.indir.
`se déplacer dans un mouvement de bas en
haut'}{} et
descendent\wdx{descendre}{v.tr.indir. `aller du haut
en bas'}{} et vient
de
luy, si co\emph{m}me
il est demonstré\wdx{demonstrer}{v.tr.
`faire
voir, mettre devant les yeux; montrer'}{}
cleremant\wdx{*clerement}{adv. `d'une
manière explicite;
clairement'}{cleremant} ou
.viij.\hoch{e} livre de \flq Utilitate
partic\emph{u}la\emph{rum}\frq . Item, ou col ha deux parties: l'une
qui co\emph{n}stitue\wdx{constituer}{v.tr. 1\hoch{o} `former
l'essence de qch.'}{}
p\emph{ro}premant\wdx{*proprement}{adv.
`d'une manière précise'}{propremant}
le col,
et les aultres qui
sont contenues ou col et passent p\emph{ar}my. Les parties
qui contiennent et
constituent\wdx{constituer}{v.tr. 2\hoch{o}
`concourir, avec d'autres éléments, à former un
tout; constituer'}{} le col, c'est le cuir,
les chars, les muscules, les liguemans et les os.
Mais les parties qui y sont co\emph{n}tenues,
c'est
la traché\wdx{trachee}{f. terme d'anat.
`portion du conduit aérifère comprise entre
l'extrémité inférieure du larynx et l'origine des
bronches; trachée'}{traché}
que on appelle ysophagus\wdx{*isophagus}{mlt. terme
d'anat. `canal musculo-membraneux qui va
du pharynx à l'estomac auquel il conduit
les aliments; \oe sophage'}{ysophagus} ou
mery\wdx{meri}{m.
terme d'anat.
`canal musculo-membraneux qui va
du pharynx à l'estomac auquel il conduit
les aliments; \oe sophage'}{mery},
le epiglote\wdx{epiglote}{m. terme d'anat.
`lame cartilagineuse en forme de triangle qui fait
saillie dans la glotte et ferme le larynx (au moment
de la déglutination); épiglotte'}{} ou le
gorgeron\wdx{gorgeron}{m. `partie du cou considérée comme étant en
relation avec la
gorge; gosier'}{}
ou la goule\wdx{*gole}{f. `parties antérieures et
latérales
du cou; gorge'}{goule}, les ners\wdx{nerf}{m. terme
d'anat.
`structure blanchâtre en forme de fil qui relie soit
un muscle à un os, soit un centre nerveux
(cerveau, moelle) à un organe ou une structure organique;
tendon ou nerf'}{ners \emph{pl.}}, les
vaines, les arteres et une
porcion\wdx{porcïon}{f. `partie d'un tout
homogène qui n'est pas nombrable; portion'}{porcion}
de la nuque\wdx{nuche}{f. terme d'anat.
`substance moelleuse de l'intérieur
de l'épine dorsale; moelle épinière'}{nuque}
ou de
la medule dorssal\wdx{*medulle dorsal}{f. terme
d'anat. `substance moelleuse de l'intérieur de
l'épine dorsale; moelle épinière'}{medule dorssal}. Desqueulx
choses
nous en dirons ordonneemant\wdx{*ordeneement}{adv.
`d'une manière en bon ordre'}{ordonneemant}
l'anathomie
et co\emph{m}mencerons\wdx{*comencier}{v.tr.indir.
`faire d'abord (qch.)'}{commencerons
\emph{1.p.pl. ind.futur simple}} a la
trachee\wdx{trachee}{f. terme d'anat.
`portion du conduit aérifère comprise entre
l'extrémité inférieure du larynx et l'origine des
bronches; trachée'}{}
co\emph{m}me au plus principal\wdx{principal}{adj. `qui est le plus
important; principal'}{}.
\pend
\pstart
[25r\hoch{o}a]
Puis que l'anathomie sera faite \text{et
que}\fnb{Ms. \emph{et et que}.}/
la goule\wdx{*gole}{f. `parties antérieures et
latérales du cou; gorge'}{goule} ou le col s\emph{er}a
divisé ou tranché\wdx{*trenchier}{v.tr.
`séparer (une chose
en parties, deux choses unies) d'une manière nette, au
moyen d'un instrument dur et fin; trancher'}{tranché
\emph{p.p.}} de long\wdx{lonc}{adv.}{\textbf{de lonc}
\emph{loc.adv. `dans le sens de la longueur'} de
long}
par devant\wdx{devant}{adv. 1\hoch{o} `au côté du
visage,
à
la face'}{}, la trachee artere\wdx{trachee artere}{f.
terme d'anat.
`portion du conduit aérifère comprise entre
l'extrémité inférieure du larynx et l'origine des
bronches; trachée'}{}
aparra\wdx{*aparoir}{v.intr. `se montrer aux yeux; se
manifester'}{aparra \emph{3.p.sg.
ind.fut.}} premier\wdx{premier}{adv.
`en
premier lieu, d'abord; premièrement'}{}:
laquelle est la voie\wdx{voie}{f. 2\hoch{o}
terme d'anat. `conduit
naturel dans le corps par lequel s'écoule un liquide
ou une matière organique'}{} et le co\emph{n}duit\wdx{conduit}{m. `canal
ou tuyau qui sert à l'ecoulement ou au
transport d'une matière (un liquide, l'air, un
gaz, etc.)'}{} de l'air qui va au
polmon\wdx{*poumon}{m. terme d'anat.
`chacun des deux viscères logés
symétriquement dans la cage thoracique qui sont les
organes de la respiration, aussi l'ensemble
des deux; poumon'}{polmon}, et
procede\wdx{proceder}{v.tr.indir. `avoir son origine
dans; provenir'}{procede \emph{3.p.sg. ind.prés.}}
du col et va ou gorgeron\wdx{gorgeron}{m. `partie du cou considérée comme
étant en relation avec la gorge; gosier'}{} ou a
la goule\wdx{*gole}{f. `parties antérieures et
latérales du
cou; gorge'}{goule} ou a l'espiglote\wdx{epiglote}{m.
terme d'anat.
`lame cartilagineuse en forme de triangle qui fait
saillie dans la glotte et ferme le larynx (au moment
de la déglutination); épiglotte'}{espiglote},
et est composee\wdx{composer}{v.tr.
`former par l'assemblage, par la combinaison de
parties'}{composé \emph{p.p.}}
de pluseurs anneaulx\wdx{*anel}{m. `ce qui a la forme d'une
bague ou d'un cercle qui sert à
attacher qch.'}{anneaulx
\emph{pl.}}
cartillagineux\wdx{*cartilaginos}{adj. terme d'anat. `qui est formé de cartilage;
cartilagineux'}{cartillagineux \emph{m.pl.}}
qui ne \text{sont}\fnb{Ms. \emph{sot}.}/
pas perfais\wdx{parfaire}{v.tr.
`achever, de manière à conduire à
la perfection; parfaire'}{\textbf{parfait}
\emph{p.p. comme adj. `qui est au plus haut, dans
l'échelle des valeurs; parfait} perfais \emph{m.pl.}}
ne
acomplis\wdx{acomplir}{v.tr. `rendre (qch.) complet'}{}, et
co\emph{n}joins\wdx{conjoindre}{v.tr.
`mettre des choses
ensemble de façon qu'elles se touchent ou tiennent
ensemble; joindre'}{conjoins \emph{p.p. m.pl.}}
et ordonnés\wdx{*ordener}{v.tr. 1\hoch{o}
`disposer, mettre
dans un certain ordre; ordonner'}{ordonné
\emph{p.p.}}
l'um
aprés l'autre et
\text{loiés}\fnb{Ms.
\emph{loier}.}/\wdx{*liier}{v.tr.
`entourer plusieurs choses
avec un lien pour qu'elles tiennent ensemble'}{loié
\emph{p.p.}} d'um panicle fort\wdx{fort}{adj.
1\hoch{o}
`qui résiste; fort (de
choses)'}{} et legier\wdx{legier}{adj. 1\hoch{o}
`qui a peu de poids; léger'}{}.
\pend
\pstart
Item, aprés la dicte trachee\wdx{trachee}{f.
terme d'anat.
`portion du conduit aérifère comprise entre
l'extrémité inférieure du larynx et l'origine des
bronches; trachée'}{}
sur les spondilles\wdx{spondille}{f. terme d'anat.
`chacun des os qui
forment la colonne vertébrale; vertèbre'}{} est
le meri\wdx{meri}{m. terme d'anat.
`canal musculo-membraneux qui va
du pharynx à l'estomac auquel il conduit
les aliments; \oe sophage'}{} ou le
ysoplag\emph{u}s\wdx{*isophagus}{mlt.
terme
d'anat. `canal musculo-membraneux qui va
du pharynx à l'estomac auquel il conduit
les aliments; \oe sophage'}{ysoplagus}
qui est la \text{voie}\fnb{\emph{o} über der Zeile nachgetragen.}/
et le conduit\wdx{conduit}{m. `canal
ou tuyau qui sert à l'ecoulement ou au
transport d'une matière (un liquide, l'air, un
gaz, etc.)'}{} de la
viande\wdx{viande}{f.
`aliment dont on se nourrit'}{}, et
procede\wdx{proceder}{v.tr.indir.
`avoir son origine
dans; provenir'}{procede \emph{3.p.sg. ind.prés.}}
et vient du gorgeron\wdx{gorgeron}{m. `partie du cou
considérée comme étant
en relation avec la gorge; gosier'}{} en
penetrant\wdx{penetrer}{v.tr.
`passer à travers
de (qch.)'}{} le
diafragme\wdx{diafragme}{m. terme d'anat. `muscle
large et mince qui sépare le thorax de l'abdomen;
diaphragme'}{},
et vient jusques ou\wdx{jusques a}{loc.prép. qui
marque le terme final, la limite que l'on ne dépasse
pas}{jusques ou \emph{+ s.m.}}
ventre\wdx{ventre}{m. 1\hoch{o}
`partie antérieure du tronc
formant une cavité qui contient l'estomac et les intestins'}{} ou
jusq\emph{ue}s a\wdx{jusques a}{loc.prép.
qui marque
le terme final, la limite que l'on ne dépasse
pas}{}
l'estomac\wdx{estomac}{m. terme d'anat.
`organe de l'appareil
digestif qui reçoit les aliments; estomac'}{},
et est composee de deux
tunique\wdx{tunique}{f. terme d'anat.
`membrane qui enveloppe
certains organes ou qui constitue la paroi d'un
organ ou d'un vaisseau'}{}
ou sont pluseurs villis\wdx{villis}{m.pl.
terme d'anat.
`production organique longue et fine comme des
fils'}{}, et
par dedans\wdx{dedans}{adv. `à
l'intérieur'}{} ilz se
contienuent\wdx{continuer}{v.pron.
`s'étendre sans interruption
dans l'espace'}{contienuent \emph{3.p.pl. ind.prés.}} avec
la pel\wdx{pel}{f. 1\hoch{o} `enveloppe extérieure du
corps de l'homme; peau'}{} de
la bouche\wdx{*boche}{f. `cavité située à la partie
inférieure du visage de l'homme, bordée
par les lèvres; bouche'}{bouche},
et par dehors\wdx{*defors}{adv. `à
l'extérieur'}{dehors}
elle est charneuse\wdx{*charnel}{adj. `qui est
essentiellement constitué de
chair'}{charneuse \emph{f.sg.}}
et se continue\wdx{continuer}{v.pron.
`s'étendre sans interruption
dans l'espace'}{} avec les
piauselle\wdx{*piaucele}{f.
`couche de tissu qui enveloppe un organe, qui couvre
un orifice d'un organe ou qui tapisse une cavité dans
le corps'}{piauselle} du
ventre\wdx{ventre}{m. 1\hoch{o} `partie antérieure du
tronc formant une cavité
qui contient l'estomac et les intestins'}{}. Et sur ces deux
voies\wdx{voie}{f.
2\hoch{o} terme d'anat. `conduit
naturel dans le corps par lequel s'écoule un liquide
ou une matière organique'}{} ci, vers la partie de
la bouche\wdx{*boche}{f. `cavité située à la partie
inférieure du visage de l'homme, bordée
par les lèvres; bouche'}{bouche} est
assise\wdx{asseoir}{v.tr. 2\hoch{o} `placer, poser
(qch.)'}{assis \emph{p.p.}}
la
goule\wdx{*gole}{f.
`parties
antérieures et latérales du cou; gorge'}{goule} ou le
gorgeron\wdx{gorgeron}{m. `partie du cou considérée comme
étant en relation avec la
gorge; gosier'}{} ou le
epiglote\wdx{epiglote}{m.
terme d'anat.
`lame cartilagineuse en forme de triangle qui fait
saillie dans la glotte et ferme le larynx (au moment
de la déglutination); épiglotte'}{} que je
repute\wdx{reputer}{v.tr. `tenir qch. pour;
réputer'}{} pour tout ung
-- quant
est de prese\emph{n}t\wdx{present}{m.}{\textbf{de present}
\emph{loc.adv. `au moment où l'on parle'}}
--, et est une partie
cartillagineuse\wdx{*cartilaginos}{adj. terme d'anat. `qui est formé de cartilage;
cartilagineux'}{cartillagineuse \emph{f.sg.}},
creé\wdx{*crïer}{v.tr.
`donner l'être, la vie, l'existence à'}{creé
\emph{p.p.}} et faicte pour estre le
instrument\wdx{instrument}{m. 2\hoch{o}
`partie du corps remplissant une fonction
particulière; organe'}{}
de vois\wdx{*voiz}{f. `ensemble des sons
produits par le larynx; voix'}{vois}. Et est la
clef\wdx{clef}{f. 1\hoch{o} `ce qui donne
access' (par analogie de fonction)}{} et le co\emph{n}duit\wdx{conduit}{m.
`canal ou tuyau qui sert à l'ecoulement ou au
transport d'une matière (un liquide, l'air, un
gaz, etc.)'}{}
-- ou temps\wdx{*tens}{m. `milieu indéfini où paraissent se dérouler irréversiblement
les existences dans leur changement, les événements et les
phénomènes dans leur succession; temps'}{temps}
que on
doit aucune
chose transglutir\wdx{*transglotir}{v.tr. `faire
descendre par le gosier; avaler'}{transglutir}
-- avec une
aultre addicion\wdx{addicion}{f. `éminence à la surface d'une
structure osseuse ou cartilagineuse'}{} qui est
en fourme\wdx{*forme}{f. `apparence extérieure
donnant à un objet ou à un être sa
spécificité'}{fourme} de langue\wdx{langue}{f.
`organe
charnu, musculeux, allongé et mobile, placé dans la bouche;
langue'}{}, qui est en l'une
de ces aultres p\emph{ar}ties;
%
[25r\hoch{o}b]
laquelle est composee de trois
cartillages\wdx{*cartilage}{m. terme d'anat. `variété de tissu conjonctif,
translucide, résistant mais élastique, ne contenant ni vaisseaux
ni nerfs, qui recouvre les surfaces osseuses des articulations et qui constitue la charpente
de certaines organes et le squelette de certains vertébrés inférieurs;
cartilage'}{cartillage},
\text{entour laquelle}\fnb{Ms. \emph{entour
laquelles}, \emph{s}
expungiert.}/
sont plantés\wdx{planter}{v.tr. `fixer (qch.)'}{}
.xx. muscules qui
moiennent\wdx{*moiener}{v.tr. `servir
d'intermédiaire de qch.'}{moiennent \emph{3.p.pl.
ind.prés.}} toute la
goule\wdx{*gole}{f. `parties antérieures et latérales du cou; gorge'}{goule} ou
l'espiglote et aussi
chescune de ces parties en
montant\wdx{monter}{v.tr.indir.
`se déplacer dans un mouvement de bas en
haut'}{},
en descendant\wdx{descendre}{v.tr.indir. `aller du
haut en bas'}{}
et en faisant aultres mouvemans, si co\emph{m}me il est
cleremant\wdx{*clerement}{adv.
`d'une
manière explicite; clairement'}{cleremant}
demonstré\wdx{demonstrer}{v.tr.
`faire
voir, mettre devant les yeux; montrer'}{}
ou livre qui parle de la voix\wdx{*voiz}{f.
`ensemble des sons
produits par le larynx; voix'}{voix} et
des mouvemans liquides\wdx{liquide}{adj.
`qui semble se faire
aisément, sans effort'}{}. Et
aprés ce, tu considereras\wdx{considerer}{v.tr.
`regarder (qch.) attentivement;
considérer'}{considereras \emph{2.p.sg. ind.futur
simple}} les doubles\wdx{*doble}{adj.
`qui est
répété deux fois, qui vaut deux fois (la
chose désignée) ou qui existe deux fois'}{double}
nerfz qui descendent a
l'estomac\wdx{estomac}{m.
terme d'anat.
`organe de l'appareil
digestif qui reçoit les aliments; estomac'}{}, aux
entrailles\wdx{entrailles}{f.pl.
terme d'anat.
 `organes enfermés
dans l'abdomen de l'homme ou des
animaux; intestins'}{}, pour cause du sens\wdx{sens}{m. 1\hoch{o}
`faculté d'éprouver
les impressions que font les objets matériels, i.e.
goût, odorat, ouïe, toucher, vue'}{}
et qui se reverssent\wdx{*reverser}{v.pron. `se
tourner de manière que l'une des extrémités ou
l'une des faces vienne à la place qu'occupait
précédemment l'extrémité ou la face opposée'}{reverssent
\emph{3.p.pl. ind.prés.}} du
bas\wdx{bas}{m. `partie inférieure (de qch.);
bas'}{}
en hault\wdx{hault}{adv.
`en un endroit qui est d'une certaine
dimension dans le sens vertical; haut'}{\textbf{en
hault} \emph{loc.adv. `vers la partie haute'}} \text{pres
de}\fnb{\emph{de}
über der Zeile nachgetragen.}/ l'espiglote pour
cause de la voix\wdx{*voiz}{f.
`ensemble des sons
produits par le larynx; voix'}{voix}. Et aussi tu y
dois considerer\wdx{considerer}{v.tr.
`regarder (qch.) attentivement;
considérer'}{} les grans
voines
et arteres qui se ramefient\wdx{*ramifier}{v.pron.
`se diviser en plusieurs ramifications qui partent
d'un axe ou d'un centre de qch. (en parlant d'une
chose concrète)'}{ramefient \emph{3.p.pl. ind.prés.}}
pres de la fourchete\wdx{fourchete}{f.
terme d'anat. `os long, en
forme d'un S allongé, formant la partie antérieure de
la ceinture scapulaire; clavicule'}{} et
montent\wdx{monter}{v.tr.indir. `se déplacer dans un
mouvement de bas en
haut'}{} aux parties de dessus\wdx{*desus}{adv.
`au côté
supérieur'}{\textbf{de dessus} \emph{loc.adj. `qui
est situé au côté supérieur'}}
par
les
costés\wdx{costé}{m. `partie qui est à droite ou à gauche
(d'un corps); côté'}{} du col,
lequelles
on appelle guidegi\wdx{*guidege}{subst. terme
d'anat.
`un des vaisseaux qui conduisent le sang vers la tête
et qui sont situés dans la partie latérale du
cou'}{guidegi \emph{pl.}} ou
appopletice\wdx{*apoplectique}{adj. terme de méd.
`qui a rapport à l'apoplexie'}{appopletice
\emph{f.pl.}} ou
subbethales\wdx{subbethal}{adj. terme d'anat. `qui a rapport
à un sommeil pathologique' (?)}{},
et est moult perilleuse\wdx{*perillos}{adj.
`qui constitue un danger, présente du
danger; dangereux'}{perilleuse \emph{f.sg.}}
leur incision\wdx{incision}{f.
`action de fendre, de couper avec un instrument tranchant, son
résultat (surtout en médecine)'}{}.
\pend
\pstart
Item, aprés ce, y nous
co\emph{n}vient\wdx{*covenir}{v.tr.indir.
`être convenable pour (qn)'}{convient \emph{3.p.sg. ind.prés.}} veoir la
general\wdx{general}{adj. `qui est valable pour
toute une classe d'êtres ou de choses (par ce qui
est valable à une sous-classe)'}{}
anathomie des spondilles\wdx{spondille}{f.
terme d'anat.
`chacun des os qui
forment la colonne vertébrale; vertèbre'}{}
de tout le dos\wdx{dos}{m.
`partie du corps de l'homme
qui s'étend des épaules jusqu'aux reins, de chaque
côté de la colonne vertebrale; dos'}{}.
Spondilles\wdx{spondille}{f. terme d'anat.
`chacun des os qui
forment la colonne vertébrale; vertèbre'}{},
c'est ung os qui constitue\wdx{constituer}{v.tr.
1\hoch{o} `former
l'essence de qch.'}{} et
fait le dos\wdx{dos}{m.
`partie du corps de l'homme
qui s'étend des épaules jusqu'aux reins, de chaque
côté de la colonne vertebrale; dos'}{}, qui
\text{est}\fnb{Ms. \emph{es}.}/ traués\wdx{*tröer}{v.tr. `percer (qch.) par un
trou; trouer'}{traué \emph{p.p.}} ou milieu par quoy
passe la nuque\wdx{nuche}{f. terme d'anat.
`substance moelleuse de l'intérieur
de l'épine dorsale; moelle épinière'}{nuque},
et aussi es costés\wdx{costé}{m. `partie qui est à droite ou à
gauche (d'un corps); côté'}{} par ou passent et issent les
nerfz,
et ha pluseurs
additema\emph{n}s\wdx{*additement}{m. `éminence à la surface d'une structure osseuse ou cartilagineuse'}{additemans \emph{pl.}}
qui montent\wdx{monter}{v.tr.indir. `se déplacer dans
un mouvement de bas en
haut'}{}
et qui descendent et qui font
l'espine du dos\wdx{espine du dos}{f.
terme d'anat.
`colonne
vertébrale'}{},
au moins\wdx{moins}{adv.}{\textbf{au moins}
\emph{loc.adv. qui sert
à marquer une restriction}}
\text{ceulx}\fnb{Nachfolgend expungiertes \emph{qui}.}/
du milieu. Le dos
est ainsi que une nef\wdx{nef}{f. `construction
flottante de forme allongée destinée aux transports
sur mer; navire'}{} qui
n'est pas couverte\wdx{*covrir}{v.tr. `garnir qch.
en disposant qch. dessus'}{couvert \emph{p.p.}}, et
co\emph{m}mence\wdx{*comencier}{v.intr. `entrer
dans son commencement'}{commence
\emph{3.p.sg.
ind.prés.}}
au chief jusq\emph{ue}s
%
[25v\hoch{o}a]
au\wdx{jusques a}{loc.prép.
qui marque
le terme final, la limite que l'on ne dépasse
pas}{}
cul\wdx{cul}{m. `partie postérieure chez
l'homme; cul'}{}, et est fait et ordonné\wdx{*ordener}{v.tr. 1\hoch{o}
`disposer, mettre
dans un certain ordre; ordonner'}{ordonné
\emph{p.p.}}
de pluseurs
diversse\wdx{divers}{adj.
`qui présente une différence par rapport à une autre
chose ou une autre personne; différent'}{diversse
\emph{f.sg.}} spondilles\wdx{spondille}{f.
terme d'anat.
`chacun des os qui
forment la colonne vertébrale; vertèbre'}{}
successivemant\wdx{*successivement}{adv. `selon un
ordre de succession, par éléments successifs;
successivement'}{successivemant} pour
deffendre\wdx{*defendre}{v.tr. `protéger (qn, qch.)
contre (qn, qch.)'}{deffendre \emph{inf.}}
la nuque\wdx{nuche}{f. terme d'anat.
`substance moelleuse de l'intérieur
de l'épine dorsale; moelle épinière'}{nuque}.
Item, Ga\-lien\adx{Galien}{}{} dit ou .xij. et ou .xiij.\hoch{e}
de
\flq Utilitate particula\emph{rum}\frq , que au dos\wdx{dos}{m.
`partie du corps de l'homme
qui s'étend des épaules jusqu'aux reins, de chaque
côté de la colonne vertebrale; dos'}{}
ha quatre parties
principaulx\wdx{principal}{adj. `qui est le plus
important; principal'}{principaulx
\emph{f.pl.}}, c'est assavoir le col,
l'espaule\wdx{espaule}{f. `partie supérieure du bras à l'endroit
où il s'attache au thorax, pouvant désigner aussi l'omoplate'}{},
les reins\wdx{rein}{m. 2\hoch{o} au plur.
`la partie inférieure du dos au niveau des
vertèbres lombaires'}{}
et ce que aucuns appellent sacru\emph{m}
os\wdx{os sacrum}{m. terme d'anat. `os formé par la
réunion des cinq
vertèbres sacrées, a la partie inférieure de la
colonne vertébrale; sacrum'}{sacrum os} et aucuns
le appelles l'os
ample\wdx{os ample}{m. terme d'anat. `os formé par la réunion des cinq
vertèbres sacrées, à la partie inférieure de la
colonne
vertébrale; sacrum'}{}. Donc selon le col il y a
.vij., et selon l'espaule\wdx{espaule}{f. `partie supérieure du
bras à l'endroit où il s'attache au thorax, pouvant désigner aussi
l'omoplate'}{} .xij. et
selon les reins\wdx{rein}{m. 2\hoch{o} au plur.
`la partie inférieure du dos au niveau des
vertèbres lombaires'}{}
.v., donc ce sont .xxiiij.
vraies\wdx{*verai}{adj. `qui présente un caractère de vérité; vrai'}{vraies \emph{f.pl.}}
spo\emph{n}dilles\wdx{spondille}{f. terme d'anat.
`chacun des os qui
forment la colonne vertébrale; vertèbre'}{}. Et
selon l'os sacru\emph{m}\wdx{os sacrum}{m. terme d'anat. `os formé par la réunion des cinq
vertèbres sacrées, a la partie inférieure de la
colonne vertébrale; sacrum'}{} il
en y a quatre, et trois selon
l'os de la queue\wdx{os de la queue}{m. `extrémité
inférieure de la colonne vertébrale, articulée avec le
sacrum et formé de trois petit os; coccyx'}{} qui ne sont
\text{pas}\fnb{Über der Zeile nachgetragen.}/
vraies\wdx{*verai}{adj. `qui présente un caractère de vérité; vrai'}{vraies \emph{f.pl.}}
spondilles\wdx{spondille}{f. terme d'anat.
`chacun des os qui
forment la colonne vertébrale; vertèbre'}{}, mais elles sont
semblables\wdx{semblable}{adj.
`qui ressemble;
semblable'}{}
et ainsi que vicaires\wdx{vicaire}{m.
`ce qui exerce en
second les fonctions de qch. autre'}{},
car
les trois premiers sont moult grosses\wdx{gros}{adj.
1\hoch{o}
dans l'ordre physique, quantifiable `qui, dans
son genre, dépasse le volume ordinaire; gros (du corps
humain et de ses parties)'}{}
et n'ont nulz
addictemans\wdx{*additement}{m. `éminence à la surface d'une structure osseuse ou cartilagineuse'}{addictemans
\emph{pl.}}
ne nulz troux\wdx{*tro}{m. `ouverture au travers d'un
corps ou qui y pénètre à une certaine
profondeur; trou'}{troux \emph{pl.}} es
costés\wdx{costé}{m.
`partie qui est à droite ou
à gauche (d'un corps); côté'}{}, mais devant\wdx{devant}{adv. 1\hoch{o} `au
côté du visage, à la
face'}{}
sont
les troux\wdx{*tro}{m. `ouverture au travers d'un
corps ou qui y pénètre à une certaine
profondeur; trou'}{troux \emph{pl.}}, et sont moult
cartillagineuses\wdx{*cartilaginos}{adj. terme d'anat. `qui est formé de cartilage;
cartilagineux'}{cartillagineuses \emph{f.pl.}},
au moins\wdx{moins}{adv.}{\textbf{au moins}
\emph{loc.adv. qui sert
à marquer une restriction}}
les
darnieres\wdx{*derrenier}{adj.
`qui vient
après tous les autres, après lequel il n'y a pas
d'autre' (temporel ou spatial)}{darnier},
et se
agrellissent\wdx{*agraislir}{v.pron.
`devenir plus mince'}{agrellissent
\emph{3.p.pl. ind.prés.}}
a maniere\wdx{maniere}{f. 2\hoch{o} `forme
particulière
que revêt l'accomplissement d'une action, le
déroulement d'un fait, l'être ou
l'existence'}{\textbf{a\,/\,en maniere de}
\emph{loc.prép. `comme'}}
d'une queue\wdx{*cöe}{f. `appendice
qui prolonge la colonne vertébrale de l'homme
et de nombreux mammifères'}{queue}.
Et ansi\wdx{ainsi}{adv. `de cette façon'}{ansi} sur
le tout y sont
.xxx.
spondilles\wdx{spondille}{f. terme d'anat.
`chacun des os qui
forment la colonne vertébrale; vertèbre'}{},
et se par chescune spo\emph{n}dille passoit une
pere\wdx{paire}{f.
`réunion de deux choses,
de deux êtres semblables
qui vont ensemble; paire'}{pere}
de nerfz qui venissent de la nuque, ainsi y
seroient .xxx. pareilz\wdx{pareil}{m.
`réunion de deux choses,
de deux êtres semblables
qui vont ensemble; paire'}{pareilz \emph{pl.}} de
nerfz vena\emph{n}s de la nuque et
ung nerf sans
co\emph{m}paignon\wdx{*compagnon}{m.
`ce qui existe en relation étroite avec
une autre chose'}{compaignon}
qui naist de la queue\wdx{*cöe}{f. `appendice qui
prolonge la colonne vertébrale de l'homme et
de nombreux mammifères'}{queue} de la
nuque. Et se du cervel en naissent .vij.
perez\wdx{paire}{f.
`réunion de deux choses,
de deux êtres semblables
qui vont ensemble; paire'}{perez \emph{pl.}}, ainsi
s\emph{er}a la so\emph{m}me\wdx{*some}{f. `quantité formée de
quantités additionnées; somme'}{somme} des ners
.xxxviij., si co\emph{m}me dit est
devant des nerfz et ou chappitre ou il parle du chief
et de la nature\wdx{nature}{f.
1\hoch{o} `ensemble des
caractères, des propriétés qui définissent un être,
une chose concrète ou abstraite'}{}
%
[25v\hoch{o}b]
de la nuque. Item, avec ce, es
costés\wdx{costé}{m. `partie qui est à droite ou à gauche
(d'un corps); côté'}{} des
spondilles\wdx{spondille}{f. terme d'anat.
`chacun des os qui
forment la colonne vertébrale; vertèbre'}{}
du dos\wdx{dos}{m.
`partie du corps de l'homme
qui s'étend des épaules jusqu'aux reins, de chaque
côté de la colonne vertebrale; dos'}{}
devant dit,
de lonc\wdx{lonc}{adv.}{\textbf{de lonc}
\emph{loc.adv. `dans le sens de la longueur'}}
sont
aucunes chars musculeuses\wdx{musculeux}{adj. terme d'anat. `qui est de la nature
des muscles'}{}
qui
gisent\wdx{gesir}{v.intr. `être
étendu, être
placé, se trouver (en parlant de choses)'}{gisent \emph{3.p.pl. ind.prés.}} la, qui sont
la, couté\wdx{costé}{m.
`partie qui est à
droite ou à gauche (d'un corps); côté'}{couté}
aux nerfz, que on appelle
vulgaulmant\wdx{*vulgairement}{adv.
`dans la langue commune'}{vulgaulmant}
longes\wdx{lonc}{adj.
1\hoch{o}
`qui a une étendue supérieure à la moyenne
dans le sens de la longueur (dans l'espace)'}{}. Et avec ce, y ha
ung panicle
ainsi que sur le cra\emph{n}ne et sur les aultres os qui
liguent\wdx{liguer}{v.tr.
`entourer plusieurs choses avec un lien pour
qu'elles tiennent ensemble'}{liguent \emph{3.p.pl.
ind.prés.}}
toutes les spondilles. Doncques ou col sont
.vij. spondilles et en leur costés\wdx{costé}{m. `partie qui est à
droite ou à gauche (d'un corps); côté'}{}, par les
troux\wdx{*tro}{m. `ouverture au travers d'un
corps ou qui y pénètre à une certaine
profondeur; trou'}{troux \emph{pl.}} qui y sont,
issent
.vij.
paire\wdx{paire}{f.
`réunion de deux choses,
de deux êtres semblables
qui vont ensemble; paire'}{}
de nerfz de la
porcion\wdx{porcïon}{f.
`partie d'un tout
homogène qui n'est pas nombrable; portion'}{porcion}
de la nuque qui passe\emph{n}t p\emph{ar} la, lesqueulx
pourtent\wdx{porter}{v.tr.
`déplacer (qch.)
d'un lieu à un autre en le menant avec soi;
transporter'}{pourtent \emph{3.p.pl. ind.prés.}}
sentema\emph{n}t\wdx{*sentement}{m.
`faculté d'éprouver
les impressions que font les objets matériels, i.e.
goût, odorat, ouïe, toucher, vue'}{sentemant}
et mouvemant es espaules\wdx{espaule}{f. `partie
supérieure du bras à l'endroit où il s'attache au thorax,
pouvant désigner aussi l'omoplate'}{}
et aux bras\wdx{bras}{m. `membre
supérieur de l'homme comprenant le segment soit entre l'épaule
et le coude, soit entre le coude et la main, soit les deux segments
ensemble avec la main'}{} et a
aucunes parties du chief et du col. Item, les chars
du col sont triples\wdx{triple}{adj.  `qui
équivaut à trois, se présente comme trois;
triple'}{}.
C'est assavoir longues\wdx{lonc}{adj. 1\hoch{o}
`qui a une étendue supérieure à la moyenne
dans le sens de la longueur (dans l'espace)'}{longues
\emph{f.pl.}}
ou longales\wdx{longal}{adj. `qui a une forme
allongée; oblongue'}{}, que
on appelle p\emph{ro}premant\wdx{*proprement}{adv. `d'une
manière précise'}{propremant}
hasterel\wdx{*haterel}{adj. `qui a rapport à
la partie du corps entre la tête et le tronc dont la
partie
postérieure peut comprendre l'occiput'}{hasterel},
qui
sont pres des spondilles, si co\emph{m}me dit est. Et se y a
chars musculeuses\wdx{musculeux}{adj. terme d'anat. `qui est de la nature
des muscles'}{}, de quoy sont fais les tenans qui
mouvent le chief et le col, et so\emph{n}t .xx., se dit
Galien\adx{Galien}{}{}. Et se y sont les chars qui
remplissent\wdx{remplir}{v.tr.
`rendre (un
espace disponible) plein de (qch.)'}{}
les lieux vuys\wdx{*vuit}{adj. `qui ne contient
rien'}{vuys \emph{m.pl.}}.
Item, les liguemans co\emph{m}mu\emph{n}s\wdx{*comun}{adj. `qui appartient
à plusieurs personnes ou
choses'}{commun}
qui lient\wdx{*liier}{v.tr. `entourer plusieurs choses avec un lien pour
qu'elles tiennent ensemble'}{lient \emph{3.p.pl.
ind.prés.}}
le chief avec les espaules\wdx{espaule}{f. `partie supérieure
du bras à l'endroit où il s'attache au thorax, pouvant
désigner aussi l'omoplate'}{} sont
pluseurs, car en
la partie de devant\wdx{devant}{adv. 1\hoch{o} `au
côté du visage, à la face'}{\textbf{de devant}
\emph{loc.adj. `qui est situé au côté du visage, de la
face'}} en ha deux gros\wdx{gros}{adj.
1\hoch{o}
dans l'ordre physique, quantifiable `qui, dans
son genre, dépasse le volume ordinaire; gros (du corps
humain et de ses parties)'}{}
qui descendent de
dessus\wdx{*desus}{prép. qui marque la position en
haut par rapport à ce qui est en bas `sur'}{dessus}
les oreilles a la forcelle\wdx{*forcele}{f. 1\hoch{o}
terme d'anat.
`os long, en
forme d'un S allongé, formant la partie antérieure de
la ceinture scapulaire; clavicule'}{forcelle}.
Mais en la p\emph{ar}tie
de darrier\wdx{derrier}{adv. `du côté
opposé au visage, à la face'}{\textbf{de
derrier} \emph{loc.adj.
`qui est situé au côté opposé au visage, à la
face'} de darrier} sont aultres plus grans qui
lient\wdx{*liier}{v.tr. `entourer plusieurs choses avec un lien pour
qu'elles tiennent ensemble'}{lient \emph{3.p.pl.
ind.prés.}} les
spondilles du dos\wdx{dos}{m.
`partie du corps de l'homme
qui s'étend des épaules jusqu'aux reins, de chaque
côté de la colonne vertebrale; dos'}{}. Mais es costés\wdx{costé}{m. `partie qui est à
droite ou à gauche (d'un corps); côté'}{} du col
\text{on a}\fnb{\emph{on} überschreibt
Buchstaben, \emph{a} ist
über der
Zeile nachgetragen.}/ d'autres plus
%
[26r\hoch{o}a]
grans qui descendent aux espaules\wdx{espaule}{f. `partie
supérieure du bras à l'endroit où il s'attache au thorax,
pouvant désigner aussi l'omoplate'}{},
an\wdx{en}{prép. marquant en général la position
à l'intérieur de limites spatiales, temporelles ou
notionelles `en'}{an} telle
maniere\wdx{maniere}{f. 2\hoch{o} `forme particulière
que revêt l'accomplissement d'une action, le
déroulement d'un fait, l'être ou
l'existence'}{\textbf{en telle maniere que}
\emph{loc.conj. `de sorte que'}} que
muscules, thenons\wdx{*tendant}{m. terme d'anat. `structure conjonctive fibreuse par laquelle
un muscle s'insère sur un os'}{thenons \emph{pl.}} et
liguemans sont entour le col, et font la une
corale\wdx{corale}{f.
terme d'anat.
`disposition circulaire que
forment les muscles et les liguements au côtés du
col (?)'}{} et ploie\emph{n}t\wdx{ploier}{v.tr. `courber
une chose flexible; plier'}{} et
eslievent\wdx{eslever}{v.tr. `mettre ou porter (qch.)
plus haut'}{eslievent \emph{3.p.pl. ind.prés.}} et
virent\wdx{virer}{v.tr.
`faire mouvoir autour d'un axe; tourner'}{}
le col et le chief. Car sans ceulx ci, on ne peult
point\wdx{point}{m. `endroit fixé et déterminé (où qch. à
lieu)'}{\textbf{ne{\dots} point} \emph{adv. de la négation `ne{\dots}
pas'}}
fere\wdx{faire}{v.tr.
`réaliser
ou effectuer (qch.)'}{fere}
articulacion\wdx{*articulation}{f.
terme d'anat.
`partie du corps formée par la jointure entre
deux ou plusieurs os; articulation'}{articulacion},
selon Galie\emph{n}\adx{Galien}{}{} ou livre devant dit. Et pour ce
apparent\wdx{*aparoir}{v.intr. `se montrer aux yeux; se
manifester'}{apparent \emph{3.p.pl. ind.prés.}}
les .vj. ou les .vij. choses q\emph{ue} on enquiert\wdx{enquerre}{v.tr.
`chercher à savoir
(qch.) en examinant ou en interrogeant'}{enquiert
\emph{3.p.sg. ind.prés.}}
en
chescum me\emph{m}bre.
\pend
\pstart
Or nous
co\emph{n}vient\wdx{*covenir}{v.tr.indir. `être convenable
pour
(qn)'}{convient \emph{3.p.sg. ind.prés.}} veoir les
maladies\wdx{maladie}{f.
 `altération organique ou
fonctionnelle considérée dans son évolution, et comme
une entité définissable; maladie'}{}.
Le col peult avoir pluseurs maladies\wdx{maladie}{f.
 `altération organique ou
fonctionnelle considérée dans son évolution, et comme
une entité définissable; maladie'}{}
en lui ou en ces
parties, si co\emph{m}me plaies\wdx{plaie}{f.
`ouverture
dans les chairs, les tissus, due à une cause
externe (traumatisme, intervention chirurgicale) et
présentant une solution de continuité des téguments;
plaie'}{},
dislocacions\wdx{dislocacïon}{f.
`déplacement violent d'une partie du
corps (d'un membre, d'un os, d'une
articulation)'}{dislocacion},
apostumacions\wdx{apostumacion}{f. `production de pus; suppuration'}{},
et toutes ces choses cy y sont perilleuses\wdx{*perillos}{adj.
`qui constitue un danger, présente du
danger; dangereux'}{perilleuses \emph{f.pl.}}. Et
appert\wdx{*aparoir}{v.intr. `se montrer aux yeux; se
manifester'}{appert \emph{3.p.sg. ind.prés.}} que les incisions\wdx{incision}{f. `action de fendre,
de couper avec un instrument tranchant, son
résultat (surtout en médecine)'}{} que on y veult
fere\wdx{faire}{v.tr.
`réaliser
ou effectuer (qch.)'}{fere}, que on
les doit fere du lonc, car ainsi vont ces parties.
Et y a p\emph{ro}pre\wdx{propre}{adj. 1\hoch{o} `qui
appartient exclusivement à (qn, qch.)'}{} maniere
de\wdx{maniere}{f. 2\hoch{o} `forme particulière
que revêt l'accomplissement d'une action, le
déroulement d'un fait, l'être ou
l'existence'}{}
loier\wdx{*liier}{v.tr.
`entourer plusieurs choses
avec un lien pour qu'elles tiennent
ensemble'}{loier \emph{inf.}}, qui sera dicte ci
aprés.
\pend
%
% \memorybreak
%
\pstartueber
Le quart chappitre: De l'anathomie des
espaules\wdx{espaule}{f. `partie supérieure du bras à l'endroit
où il s'attache au thorax, pouvant désigner aussi l'omoplate'}{},
des bras\wdx{bras}{m. `membre supérieur de l'homme
comprenant
le segment soit entre l'épaule et le coude, soit entre le coude
et la main, soit les deux segments ensemble avec la
main'}{} et des
mains\wdx{main}{f. `partie du corps humain située à l'extrémité
du bras'}{}.
\pendueber
%
% \memorybreak
%
\pstart
APR\'{E}S LE col, en descendant,
s'ensuit\wdx{*ensivre}{v.pron. 1\hoch{o}
`être placé ou considéré après dans un
ordre donné'}{ensuit \emph{3.p.sg.
ind.prés.}}
le four\wdx{*for}{m. 2\hoch{o}
`partie du corps humain qui s'étend
des épaules à
l'abdomen et qui contient le c\oe ur et les
poumons; thorax'}{four} ou le
pis\wdx{*piz}{m.
terme d'anat.
`partie du corps humain qui s'étend
des épaules à
l'abdomen et qui contient le c\oe ur et les
poumons; thorax'}{pis}. Mais pour ce que sur les
parties par dehors\wdx{*defors}{adv. `à
l'extérieur'}{dehors} sont
plantees\wdx{planter}{v.tr.
`fixer (qch.)'}{}
les espaules et
les bras\wdx{bras}{m. `membre supérieur de l'homme
comprenant le
segment soit entre l'épaule et le coude, soit entre le coude et
la main, soit les deux segments ensemble avec la main'}{},
pour ce il nous
co\emph{n}vient\wdx{*covenir}{v.tr.indir. `être convenable
pour
(qn)'}{convient \emph{3.p.sg. ind.prés.}} premier\wdx{premier}{adv.
`en
premier lieu, d'abord; premièrement'}{}
dire\wdx{dire}{v.tr.indir. \textbf{\emph{dire de}} `parler de'}{}
de yceulx.
Homoplates\wdx{homoplate}{f. terme d'anat. `os plat
triangulaire
qui forme la partie postérieure de l'épaule, (par
extension) l'épaule dans sa totalité'}{}
et spatules\wdx{spatule}{f. terme d'anat. `os plat
triangulaire
qui forme la partie postérieure de l'épaule,
(par
extension) l'épaule dans sa totalité'}{} ou espaules, c'est
tout ung, quant est de present\wdx{present}{m.}{\textbf{de present}
\emph{loc.adv. `au moment où l'on parle'}}.
Et appert\wdx{*aparoir}{v.intr. `se montrer aux yeux; se
manifester'}{appert \emph{3.p.sg. ind.prés.}} la, quelles
choses
%
[26r\hoch{o}b]
se sont et en quel lieu elles
sielle\emph{n}t\wdx{*sëeler}{v.intr. `former un
tout'}{siellent
\emph{3.p.pl. ind.prés.}} et quelle est leur
colligance\wdx{colligance}{f. 1\hoch{o}
`force qui maintient réunis les éléments d'un système matériel;
liaison'}{}; et so\emph{n}t
fais pour embrasser\wdx{*embracier}{v.tr. `être
autour de (qch.) de manière à enfermer;
entourer'}{embrasser \emph{inf.}} et
pour deffe\emph{n}dre\wdx{*defendre}{v.tr. `protéger (qn,
qch.) contre (qn, qch.)'}{deffendre \emph{inf.}} les
aultres me\emph{m}bres,
si co\emph{m}me il est dit ou p\emph{re}mier livre partout de
\flq Utilitate partic\emph{u}la\emph{rum}\frq . Car le
createur\wdx{*crïator}{m.
`celui qui crée, qui tire
(qch.) du néant; créateur'}{createur}
\text{garny}\fnb{Im Ms. \emph{garnir} zu \emph{garny}
korrigiert.}/%
\wdx{garnir}{v.tr. `pourvoir qn ou qch. (de
qch.)'}{garny \emph{3.p.sg. p.simple.}} l'o\emph{m}me de
mains\wdx{main}{f. `partie du corps humain située à l'extrémité
du bras'}{} et de raison\wdx{raison}{f.
1\hoch{o} `ce qui permet d'agir conformément à
des principes'}{};
et Galien\adx{Galien}{}{} venoit
[aprés]
Aristote\adx{Aristote}{}{}
qui avoit no\emph{m}mé\wdx{*nomer}{v.tr. `indiquer (une
personne, une chose) en disant ou en écrivant son
nom'}{nommé \emph{p.p.}}
premier\wdx{premier}{adv.
`en
premier lieu, d'abord; premièrement'}{}
les mains\wdx{main}{f. `partie du corps humain située à l'extrémité
du bras'}{} ains
que\wdx{*ainz}{conj. `plutôt, de
préférence'}{\textbf{ainz que} \emph{loc.conj.
`de la même façon que'} ains que} les organes
et
\text{raison}\fnb{Ms. \emph{Raison}.}/\wdx{raison}{f.
1\hoch{o} `ce qui permet d'agir conformément à
des principes'}{}
devant les
ars\wdx{art}{m. et f. `ensemble de moyens, de
procédés réglés qui tendent à une certaine fin;
art'}{ars \emph{pl.}}.
Vecy les parties qui les composent, c'est assavoir
le cuir, la char, les vaines, les arteres, les
nerfz,
les muscules, les cordes, les liguemans, les
panicles, les
cartillages\wdx{*cartilage}{m. terme d'anat. `variété de tissu conjonctif,
translucide, résistant mais élastique, ne contenant ni vaisseaux
ni nerfs, qui recouvre les surfaces osseuses des articulations et qui constitue la charpente
de certaines organes et le squelette de certains vertébrés inférieurs;
cartilage'}{cartillage}
et les os, desqueulx il
nous fault dire\wdx{dire}{v.tr.indir. \textbf{\emph{dire de}} `parler de'}{}
cy l'um
aprés l'autre. Premier\wdx{premier}{adv.
`en
premier lieu, d'abord; premièrement'}{}, de
l'espaule et du cuir et de la char dit est cy devant.
Les muscules \text{et les cordes}\fnb{\emph{et} über
der Zeile
nachgetragen.}/ qui mouvent le bras\wdx{bras}{m.
`membre
supérieur de l'homme comprenant le segment soit entre l'épaule et
le coude, soit entre le coude et la main, soit les deux segments
ensemble avec la main'}{}, qui
descendent du col et du pis\wdx{*piz}{m.
terme d'anat.
`partie du
corps humain qui
s'étend des épaules à l'abdomen et qui contient le c\oe ur et les
poumons; thorax'}{pis},
\text{passent}\fnb{Ms. \emph{passet}.}/ par l'espaule
et compreignent\wdx{comprendre}{v.tr. 2\hoch{o} `être
autour de (qch.) de manière à enfermer ou embrasser partiellement ou
complètement; entourer'}{compreignent
\emph{3.p.pl. ind.prés.}} et
e\emph{n}velloupent\wdx{*envoleper}{v.tr.
`entourer
(qch.) d'une chose souple qui couvre de tous
côtés; envelopper'}{envelloupent \emph{3.p.pl. ind.prés.}}
la joincture de l'os
et se \text{plante\emph{n}t}\fnb{Im Ms. \emph{s} am Wortende
korrigiert in \emph{t}.}/\wdx{planter}{v.pron.
`s'appliquer (sur quelque chose)'}{}
en l'os de l'adjuctoire\wdx{*ajutoire}{m. terme d'anat.
`os supérieur du bras, (par extension) la partie
supérieure du bras'}{adjuctoire}. Les nerfz
viennent de la nuque du col. Et les vaines et les
arteres, qui ont pluseurs rames\wdx{rame}{m. et f.
terme d'anat. `subdivision d'un vaisseau
sanguin'}{}, viennent
de bas\wdx{bas}{m. `partie inférieure (de qch.);
bas'}{}, si co\emph{m}me dit est. Et pour ce que eulx
ne se monstrent\wdx{*mostrer}{v.pron.
`devenir visible'}{monstrent
\emph{3.p.pl. ind.prés.}} pas moult en l'espaule,
nous en volons abregier\wdx{abregier}{v.tr. `diminuer
la durée de qch.'}{} n\emph{ost}re parolle\wdx{parole}{f.
`ensemble de mots
qui expriment une idée'}{parolle}.
\pend
\pstart
Des os tu
dois savoir
que il
sont
deux. Le premier, c'est l'os
de l'espaule devers\wdx{devers}{prép.
`du côté de'}{} le dos\wdx{dos}{m.
`partie du corps de l'homme
qui s'étend des épaules jusqu'aux reins, de chaque
côté de la colonne vertebrale; dos'}{}. Le
second, c'est l'os fourchu\wdx{*forchu}{adj. `qui
se divise en forme de fourche; fourchu'}{fourchu}
%
[26v\hoch{o}a]
qui est devers\wdx{devers}{prép.
`du côté de'}{} la partie du pis\wdx{*piz}{m.
terme d'anat.
`partie du corps humain
qui s'étend des épaules à l'abdomen et qui contient le c\oe ur et les
poumons; thorax'}{pis} ou
de la
poitrine\wdx{poitrine}{f.
terme d'anat.
`partie du corps humain qui s'étend des
épaules à l'abdomen et qui contient le c\oe ur et les poumons;
thorax'}{}. L'os d'espaule se\emph{m}ble\wdx{sembler}{v.tr.
`avoir des traits communs avec; ressembler'}{}
une pale\wdx{*pele}{f. `outil composé d'une
plaque mince de métal ou de bois ajustée à un
manche; pelle'}{pale}, car il est
leés\wdx{*liier}{v.tr. `entourer plusieurs choses avec un lien pour
qu'elles tiennent ensemble'}{leé \emph{p.p.}}
et
tenves\wdx{tenve}{adj.
`qui a peu d'épaisseur; mince'}{} vers la
partie du dos, avec une eminence\wdx{eminence}{f.
terme
d'anat. `saillie à la surface d'une structure
anatomique'}{}
tenve\wdx{tenve}{adj.
`qui a peu d'épaisseur; mince'}{} qui va par le milieu.
Et vers la partie de la joincture il est ung
petit\wdx{petit}{adj. 3\hoch{o}
dans
l'ordre qualitatif, non quantifiable `qui est d'un
degré inférieur à la moyenne en ce qui concerne la
qualité, l'intensité,
l'importance'}{\textbf{un petit} \emph{adv.
`un peu'}} lonc\wdx{lonc}{adj.
1\hoch{o}
`qui a une étendue supérieure à la moyenne
dans le sens de la longueur (dans l'espace)'}{} et rond\wdx{*rëont}{adj.
`qui a la forme circulaire'}{rond \emph{m.sg.}}
a maniere\wdx{maniere}{f. 2\hoch{o} `forme
particulière
que revêt l'accomplissement d'une action, le
déroulement d'un fait, l'être ou
l'existence'}{\textbf{a\,/\,en maniere de}
\emph{loc.prép. `comme'}}
d'ung manche\wdx{manche}{m. `partie d'un outil, d'un
instrument par laquelle on le tient; manche'}{},
avec trois additamens\wdx{*additement}{m. `éminence à la surface d'une structure osseuse ou cartilagineuse'}{additamens
\emph{pl.}} qui sont en la fin. Le premier, c'est la
foce\wdx{fosse}{f. terme d'anat. `concavité d'assez
grandes dimensions, le plus souvent osseuse, mais pouvant se trouver
aussi dans d'autres structures anatomiques'}{foce}
qui est
au milieu, qui reçoit\wdx{recevoir}{v.tr.
`faire entrer (qch.)'}{reçoit \emph{3.p.sg. ind.prés.}}
le chief de
ulne\wdx{ulne}{subst. terme d'anat. `os
supérieur du bras, (par extension) la partie
supérieure du bras'}{} ou de
l'adjuctoire\wdx{*ajutoire}{m. terme d'anat.
`os supérieur du bras, (par extension) la partie
supérieure du bras'}{adjuctoire}
du bras\wdx{bras}{m. `membre
supérieur de l'homme comprenant le segment soit entre l'épaule et
le coude, soit entre le coude et la main, soit les deux segments
ensemble avec la main'}{}.  Le
second est
\text{par}\fnb{Nachfolgend gestrichener
Buchstabenansatz.}/
dessus\wdx{*desus}{adv.
`au côté
supérieur'}{dessus}
courbes\wdx{*corp}{adj.
`qui change
de direction sans former d'angles; courbe'}{courbe}
et agu\wdx{agu}{adj. `ce qui termine en pointe'}{}
en maniere\wdx{maniere}{f. 2\hoch{o} `forme
particulière
que revêt l'accomplissement d'une action, le
déroulement d'un fait, l'être ou
l'existence'}{\textbf{a\,/\,en maniere de}
\emph{loc.prép. `comme'}}
d'ung
bec\wdx{bec}{m. `bouche cornée et saillante des
oiseaux, formée de deux mandibules qui recouvrent
respectivement les maxillaires supérieur et
inférieur, démunis de dents; bec'}{} de
corbeau\wdx{*corbel}{m. `oiseau (de la
famille des passereaux et des corvidés) à
plumage noir, à cri strident, qui se
nourrit de fruits, d'animaux et de la chair des
cadavres; corbeau'}{corbeau}. Le tiers est vers la partie
salvage\wdx{sauvage}{adj. `qui est situé en dehors'}{salvage} qui
est de dehors, et est en maniere\wdx{maniere}{f.
2\hoch{o}
`forme particulière
que revêt l'accomplissement d'une action, le
déroulement d'un fait, l'être ou
l'existence'}{\textbf{a\,/\,en maniere de}
\emph{loc.prép. `comme'}}
d'une
encre\wdx{*ancre}{f. `fort croc de métal, de bois ou
de pierre, qui immobilise le navire en se fixant sur
le fond; ancre'}{encre}. L'os qui est
fourchu\wdx{*forchu}{adj.
`qui
se divise en forme de fourche; fourchu'}{fourchu},
il est rond\wdx{*rëont}{adj.
`qui a la forme circulaire'}{rond \emph{m.sg.}}
et se
ferme\wdx{fermer}{v.pron. `être attaché (à qch.)'}{}
en une
concavité\wdx{concavité}{f. `espace vide à
l'intérieur d'un corps solide; concavité'}{} en la
partie de dessus\wdx{*desus}{adv.
`au côté
supérieur'}{\textbf{de dessus} \emph{loc.adj. `qui
est situé au côté supérieur'}}
de l'os de la
\text{poitryne}\fnb{Im Ms. \emph{e} korrigiert zu
\emph{y}.}/\wdx{poitrine}{f.
terme d'anat.
`partie du corps humain qui s'étend
des épaules à l'abdomen et qui contient le c\oe ur et les
poumons; thorax'}{poitryne}
et ha deux branches\wdx{branche}{f. 2\hoch{o}
`ramification ou division d'un organe, d'un
appareil, etc., qui part d'un axe ou d'un
centre' (par analogie de forme)}{}: l'une
tent\wdx{tendre}{v.tr.indir. `être porté ou
dirigé vers un lieu'}{} a une espaule et l'autre
a l'autre espaule; et lie et ferme\wdx{fermer}{v.tr.
`faire tenir (à une chose) au
moyen d'une attache, d'un lien; attacher'}{}
les deux addiccions
rostrales\wdx{rostral}{adj. `qui a la forme d'un
bec'}{} qui sont en maniere de\wdx{maniere}{f.
2\hoch{o}
`forme particulière
que revêt l'accomplissement d'une action, le
déroulement d'un fait, l'être ou
l'existence'}{\textbf{a\,/\,en maniere de}
\emph{loc.prép. `comme'}}
bec\wdx{bec}{m.
`bouche cornée et saillante des
oiseaux, formée de deux mandibules qui recouvrent
respectivement les maxillaires supérieur et
inférieur, démunis de dents; bec'}{},
afin que\wdx{afin que}{loc.conj. qui marque l'intention,
le but `pour que'}{}
ce la, qui \text{ha}\fnb{Voranstehend
expungiertes
\emph{est}.}/ la fosse\wdx{fosse}{f. terme d'anat.
`concavité d'assez
grandes dimensions, le plus souvent osseuse, mais pouvant se trouver
aussi
dans d'autres structures anatomiques'}{}
du mileu\wdx{milieu}{m. `partie d'une chose
qui est à égale distance des extrémités de cette
chose'}{mileu},
tiegne\wdx{tenir}{v.tr. 1\hoch{o} `faire
rester (qch.) en place'}{tiegne
\emph{3.p.sg. subj.prés.}}
plus fermema\emph{n}t\wdx{*fermement}{adv. `d'une manière
ferme; fermement'}{fermemant}
le chief de l'adjuctoire\wdx{*ajutoire}{m. terme d'anat.
`os supérieur du bras, (par extension) la partie
supérieure du bras'}{adjuctoire}
\text{en la
joincture}\fnb{Über der Zeile nachgetragen.}/. Et
saches que ces
addictemans sont de la substance de l'os de l'espaule.
Et ne sont pas autre os
separés\wdx{separer}{v.tr. `mettre à part les unes
des autres des choses, des personnes
réunies; séparer'}{},
si co\emph{m}me le dient Lanfrant\adx{Lanfrant}{}{} et
Henry\adx{Henri de Mondeville}{}{Henry}. Et que ce
soit vraie\wdx{*verai}{adj. `qui présente un
caractère de vérité; vrai'}{vraie \emph{f.sg.}}
chose, experie\emph{n}ce\wdx{*esperïence}{f.
`le fait
d'éprouver qch., considéré comme un
élargissement ou un enrichissement de la
connaissance; expérience'}{experience} le
demonstre\wdx{demonstrer}{v.tr.
`faire
voir, mettre devant les yeux; montrer'}{}, et aussi le dit
Galien\adx{Galien}{}{} ou .xiij. livre de
\flq Utilitate
p\emph{arti}cula\emph{rum}\frq , en le .xj.\hoch{e} ou .xij.\hoch{e}
chappitre,
%
[26v\hoch{o}b]
ou il dit: \emph{ipsas
homoplatas} et \emph{c}etera. Et
avec ce, il ha quatre grans
cartillaiges\wdx{*cartilage}{m. terme d'anat. `variété de tissu conjonctif,
translucide, résistant mais élastique, ne contenant ni vaisseaux
ni nerfs, qui recouvre les surfaces osseuses des articulations et qui constitue la charpente
de certaines organes et le squelette de certains vertébrés inférieurs;
cartilage'}{cartillage}
qui issent du
chief de l'espaule, et vont a
l'adjuctoire\wdx{*ajutoire}{m. terme d'anat.
`os supérieur du bras, (par extension) la partie
supérieure du bras'}{adjuctoire}
et
l'estreingnent\wdx{estreindre}{v.tr.
`serrer fortement'}{estreingnent
\emph{3.p.pl. ind.prés.}} tout au tour; et naisse\emph{n}t
des
grans tenans, qui \text{sont}\fnb{Ms. \emph{sons}.}/
nés\wdx{nés}{m.
`partie saillante du visage,
située dans son axe entre le front et la lèvre
supérieure, et qui abrite la partie antérieure des
fosses nasales; nez'}{}
des tres grans muscules qui viennent du
pis\wdx{*piz}{m.
terme d'anat.
`partie du corps humain qui s'étend
des épaules à
l'abdomen et qui contient le c\oe ur et les
poumons; thorax'}{pis} et de
l'espaule, et sont en l'os de
l'adjuctoire\wdx{*ajutoire}{m. terme d'anat.
`os supérieur du bras, (par extension) la partie
supérieure du bras'}{adjuctoire}
et le font mouvoir. Et les aucuns vont par
dessus\wdx{*desus}{adv.
`au côté
supérieur'}{dessus}
et
aucuns par dessoubz\wdx{*desoz}{adv. `à la face
inférieure'}{dessoubz}
et aucuns tout
au tour. Item, la
partie dessoubz\wdx{*desoz}{prép.
qui marque la position en bas par rapport
à ce qui est en haut `sous'}{dessoubz}
la dicte joincture, dessoubz\wdx{*desoz}{prép. qui
marque la position en bas par rapport à ce qui est en
haut `sous'}{dessoubz} la
aisselle\wdx{*aissele}{f. terme d'anat. `dépression
située entre l'extrémité
supérieure du bras et la paroi latérale du
thorax'}{aisselle}, elle est faicte de char
glandellouse\wdx{*glandulos}{adj. `qui contient des
glandes'}{glandellouse \emph{f.sg.}},
ou est le emuctoire\wdx{*emomptoire}{m.
terme d'anat.
`organe
qui élimine les substances inutiles formées au
cours des processus de désassimilation (l'anus,
l'uretère, etc.)'}{emuctoire}
du cuer\wdx{cuer}{m. terme d'anat. `viscère de forme de cône
renversé, situé entre les poumons, qui est l'organe central de la
distribution du sang dans le corps'}{}.
\pend
\pstart
En aprés il nous
co\emph{n}vient\wdx{*covenir}{v.tr.indir. `être convenable
pour
(qn)'}{convient \emph{3.p.sg. ind.prés.}}
dire\wdx{dire}{v.tr.indir. \textbf{\emph{dire de}} `parler de'}{}
du bras\wdx{bras}{m. `membre supérieur de l'homme
comprenant
le segment soit entre l'épaule et le coude, soit entre le coude et
la main, soit les deux segments ensemble avec la main'}{}
que on appelle la
grant mein\wdx{grant main}{f. terme d'anat.
`membre supérieur de l'homme qui s'attache au tronc,
compris
entre l'épaule et les doigts; bras'}{grant mein}, qui
est divisé en trois parties. Et Galien\adx{Galien}{}{}
-- ou second livre ou second chapitre de \flq
Utilitate partic\emph{u}la\emph{rum}\frq\
-- le divise\wdx{deviser}{v.tr. `séparer (qch.) en
plusieurs parties; diviser'}{divise \emph{3.p.sg.
ind.prés.}}
einsi\wdx{ainsi}{adv. `de cette façon'}{einsi}: la
premiere partie, il le appelle
ulne\wdx{ulne}{subst. terme d'anat. `os supérieur
du bras, (par extension) la partie
supérieure du bras'}{}, la
seconde le petit bras\wdx{petit bras}{m. terme d'anat. `segment du
membre supérieur de l'homme compris entre le coude et
la main; avant-bras'}{} et la
tierce la petite main\wdx{petite main}{f.
terme d'anat.
`membre situé à l'extrémité du bras; main'}{}; et
ha cuir, char, vaines \emph{et} c\emph{etera}, si co\emph{m}me dit est
devant. De la char et du cuir dit est assés. Des
arteres
et des vaines manifestes\wdx{manifest}{adj. `dont l'existence ou la nature
est évidente; manifeste'}{}
qui sont au bras, tu en
dois
savoir, car, depuis que\wdx{depuis}{prép. `à partir de
(en parlant de l'espace)'}{\textbf{depuis que}
\emph{loc.conj. `après que'}} elles sont
ramiffiees\wdx{*ramifier}{v.tr.
`diviser en plusieurs ramifications qui partent
d'un axe ou d'un centre de qch. (en parlant d'une
chose concrète)'}{ramiffiees \emph{p.p. f.pl.}}
et divisees\wdx{deviser}{v.tr. `séparer (qch.) en
plusieurs parties; diviser'}{divisé \emph{p.p.}}
de leur
\text{origine}\fnb{\emph{gi}
über der Zeile nachgetragen.}/\wdx{origine}{f.
`endroit d'où
quelque chose provient'}{},
elles viennent
dessoubz\wdx{*desoz}{prép. qui marque la position en
bas par rapport à ce qui est en haut
`sous'}{dessoubz} les aisselles\wdx{*aissele}{f.
terme d'anat.
`dépression située
entre l'extrémité supérieure du bras et la paroi latérale du
thorax'}{aisselle} et la,
elles se divisent\wdx{deviser}{v.pron. `se séparer en
parties; se diviser'}{divisent \emph{3.p.pl.
ind.prés.}} en deux parties, desquelles l'une
va par dehors\wdx{*defors}{prép. `à
l'extérieur de'}{dehors} le bras et l'autre par
dedens. Celle qui va par dehors,
tantost\wdx{tantost}{adv. `dans un temps prochain, un
proche avenir; tantôt'}{} elle se ramefie\wdx{*ramifier}{v.pron.
`se diviser en plusieurs ramifications qui partent
d'un axe ou d'un centre de qch. (en parlant d'une
chose concrète)'}{ramefie
\emph{3.p.sg. ind.prés.}}, et l'une
des
%
[27r\hoch{o}a]
parties monte\wdx{monter}{v.tr.indir. `se déplacer
dans un mouvement de bas en
haut'}{}
amont\wdx{amont}{adv. `vers le haut'}{},
darrier\wdx{derrier}{prép. `en arrière de'}{darrier}
l'espaule et va
au chief. Et l'autre, en descendant, se
ramefie\wdx{*ramifier}{v.pron.
`se diviser en plusieurs ramifications qui partent
d'un axe ou d'un centre de qch. (en parlant d'une
chose concrète)'}{ramefie
\emph{3.p.sg. ind.prés.}} en
deux parties, desquelles parties
\text{l'une}\fnb{Ms. \emph{lune lune}.}/ se
divise\wdx{deviser}{v.pron. `se séparer en
parties; se diviser'}{divise \emph{3.p.sg.
ind.prés.}} par dehors\wdx{*defors}{prép. `à
l'extérieur de'}{dehors} le bras et
\text{l'appell'en}\fnb{Nachfolgend gestrichenes
\emph{f}.}/\wdx{on}{pron.pers. indéfini 3\hoch{e} personne}{en}
la fune\wdx{fune du bras}{f. terme d'anat.
`vaisseau sanguin
principal de la
partie supérieure du bras'}{} ou
la corde du bras\wdx{corde du bras}{f. terme
d'anat. `vaisseau sanguin principal de la
partie supérieure du bras'}{}, et
ne se demonstre\wdx{demonstrer}{v.pron. `devenir
visible'}{}
point\wdx{point}{m. `endroit fixé et déterminé (où qch. à
lieu)'}{\textbf{ne{\dots} point} \emph{adv. de la négation `ne{\dots}
pas'}}
en la curvature\wdx{curvature}{f. `forme de ce
qui est rond; courbure'}{} du bras ne ailleurs. Et
l'autre rame\wdx{rame}{m. et f. terme
d'anat. `subdivision d'un vaisseau sanguin'}{}, il
descend par dessus\wdx{*desus}{prép. qui marque la
position en haut par rapport à ce qui est en bas
`sur'}{dessus}
le bras et appert\wdx{*aparoir}{v.intr. `se montrer aux yeux; se
manifester'}{appert \emph{3.p.sg. ind.prés.}} en la
curvature\wdx{curvature}{f. `forme de ce qui est rond; courbure'}{}
du coude\wdx{*cote}{m. `partie du
membre supérieur correspondant à l'articulation du bras et
l'avant-bras'}{coude}
et l'appell'en\wdx{on}{pron.pers. indéfini 3\hoch{e} personne}{en}
cephalica\wdx{cephalica}{lt. terme d'anat.
`vaisseau sanguin superficielle du bras'}{}. Et de
ce lieu
la, elle descend en la mein\wdx{main}{f. `partie du corps humain
située à l'extrémité du bras'}{mein} et se
manifeste\wdx{manifester}{v.pron.
`se révéler
clairement dans son existence ou sa nature'}{}
entre le poulce\wdx{*pouz}{m. `le plus gros et le
plus fort des doigts de la main et du pied'}{poulce}
et le second doy\wdx{doi}{m. 1\hoch{o} `chacun des cinq
prolongements qui terminent la main'}{doy}.
Et la appell'en\wdx{on}{pron.pers. indéfini 3\hoch{e} personne}{en}
cephalica
occullaire\wdx{cephalica occullaire}{f. terme
d'anat. `suite de la veine céphalique sur le dos
de la main qui est visible entre le pouce et le
second doigt'}{}. Mais l'autre p\emph{ar}tie de la
voine -- qui estoit
divisee\wdx{deviser}{v.tr. `séparer (qch.) en
plusieurs parties; diviser'}{divisé \emph{p.p.}} en
la partie de dessoubz les
aisselles\wdx{*aissele}{f.
terme d'anat.
`dépression située entre
l'extrémité
supérieure du bras et la paroi latérale du thorax'}{aisselle}, qui
tent\wdx{tendre}{v.tr.indir. `être porté ou
dirigé vers un lieu'}{}
a la partie de dedans\wdx{dedans}{adv. `à l'intérieur'}{\textbf{de dedans} \emph{loc.adj. `qui est
situé à l'intérieur'}} --, en descendant, elle se
demonstre\wdx{demonstrer}{v.pron.
`devenir
visible'}{} en la
curvature\wdx{curvature}{f. `forme de ce qui est rond; courbure'}{}
du coude\wdx{*cote}{m. `partie du
membre supérieur correspondant à l'articulation du bras et
l'avant-bras'}{coude}
et l'appell'en\wdx{on}{pron.pers. indéfini 3\hoch{e} personne}{en}
bazilique\wdx{*veine basilique}{f. terme
d'anat.
`veine superficielle du bras qui est située à sa
face interne'}{voine bazilique}.
Et de ce lieu la, elle descend a la main\wdx{main}{f. `partie du
corps humain située à l'extrémité du bras'}{} et se
demonstre\wdx{demonstrer}{v.pron.
`devenir
visible'}{} entre
le petit doy\wdx{petit doi}{m. 1\hoch{o} terme d'anat. `le plus petit
des prolongements qui terminent la main'}{petit doy}
et l'autre
prochain\wdx{prochain}{adj. `qui est proche
(dans l'espace)'}{}, et la
appell'en salvatelle\wdx{*veine salvatelle}{f. terme
d'anat. `vaisseau sanguin superficiel du bras
descendant
à la main et qui se manifeste entre le quatrième et le
petit doigt'}{}. Et de ces
deux vaines cy devant dictes, quant elles sont en la
curvature\wdx{curvature}{f. `forme de ce qui est rond; courbure'}{}
du coude\wdx{*cote}{m.
`partie du membre supérieur
correspondant à l'articulation du bras et l'avant-bras'}{coude}, elles
font ung
rame\wdx{rame}{m. et f. terme d'anat. `subdivision
d'un vaisseau sanguin'}{}
co\emph{m}mun\wdx{*comun}{adj. `qui appartient
à plusieurs personnes ou
choses'}{commun}
qui appert entre elles deux, que on appelle
medienne\wdx{*veine mediane}{f. terme
d'anat. `vaisseau sanguin du bras qui est située au
pli du coude
entre la veine céphalique et la veine basilique'}{medienne} ou
corporalle\wdx{*veine corporale}{f.
terme d'anat. `veine du bras qui est située au pli du coude
entre la veine céphalique et la veine basilique'}{corporalle}.
Donc au bras ha quatre
ou cinq grosses\wdx{gros}{adj.
1\hoch{o}
dans l'ordre physique, quantifiable `qui, dans
son genre, dépasse le volume ordinaire; gros (du corps
humain et de ses parties)'}{}
vaines et autant de arteres, et est la
incision\wdx{incision}{f. `action de fendre,
de couper avec un instrument tranchant, son
résultat (surtout en médecine)'}{}
perilleuse\wdx{*perillos}{adj.
`qui constitue un danger, présente du
danger; dangereux'}{perilleuse \emph{f.sg.}}
pour le grant\wdx{grant}{adj. 3\hoch{o}
dans
l'ordre qualitatif, non quantifiable `qui est d'un
degré supérieur à la moyenne en ce qui concerne la
qualité, l'intensité, l'importance'}{}
flux\wdx{*flus}{m. `action de couler (dit
d'un liquide)'}{flux} de sang\wdx{*sanc}{m.
terme de méd.
`liquide visqueux, de couleur
rouge, qui est porté par les vaisseaux dans tout
l'organisme où il joue des rôles multiples (l'une
des quatre humeurs de l'humorisme)'}{sang}
que
elles ont. Et se y a pluseurs aultres
rames\wdx{rame}{m. et f. terme d'anat. `subdivision
d'un vaisseau sanguin'}{} qui sont
%
[27r\hoch{o}b]
petis, de quoy le cirurgien ne doit pas fere
grant\wdx{grant}{adj. 3\hoch{o}
dans
l'ordre qualitatif, non quantifiable `qui est d'un
degré supérieur à la moyenne en ce qui concerne la
qualité, l'intensité, l'importance'}{grant
\emph{f.sg.}}
cure\wdx{cure}{f. `préoccupation qui inquiète;
soin'}{}.
\pend
\pstart
Des nerfz on dit \emph{qu}e ilz descendent de
la nuq\emph{ue} par les spondilles, et viennent a ch\emph{acu}m bras
quatre nerfz notables\wdx{notable}{adj. `qui est
digne d'être remarqué; notable'}{}. L'ung va par
derrier\wdx{derrier}{adv.
`du côté opposé au visage, à la face'}{},
l'autre par deva\emph{n}t\wdx{devant}{adv. 1\hoch{o} `au
côté du visage, à la face'}{},
l'autre par dessus\wdx{*desus}{adv. `au côté
supérieur'}{dessus}, l'autre par desoubz;
lesqueulx se divisent\wdx{deviser}{v.pron. `se séparer en
parties; se diviser'}{divisent \emph{3.p.pl.
ind.prés.}} et passent par le
p\emph{ar}fond\wdx{parfont}{m.
`ce qui est profond; profond'}{parfond} du corps, ou
ilz
se meslent\wdx{mesler}{v.pron. `être mis ensemble de
manière à former un tout; se mêler'}{} avec
les muscules et les cordes et avec les liguemans, et
portent\wdx{porter}{v.tr.
`déplacer (qch.)
d'un lieu à un autre en le menant avec soi;
transporter'}{}
sentemant\wdx{*sentement}{m.
`faculté d'éprouver
les impressions que font les objets matériels, i.e.
goût, odorat, ouïe, toucher, vue'}{sentemant}
et mouvemant a tous les bras. Les
muscules sont fais de nerfz, de chars, de panicles.
Et sont quatre
principalx\wdx{principal}{adj. `qui est le plus
important; principal'}{principalx
\emph{m.pl.}} et grans qui mouvent
le petit bras\wdx{petit bras}{m.
terme d'anat. `segment du
membre supérieur de l'homme compris entre le coude et
la main;
avant-bras'}{}, et quatre au petit bras\wdx{petit bras}{m.
terme d'anat. `segment du
membre supérieur de l'homme compris entre le coude et
la main; avant-bras'}{}
qui
mouvent
la main petite\wdx{petite main}{f.
terme d'anat.
`membre situé à l'extrémité du bras; main'}{main petite}, et
.v. en la petite main\wdx{petite main}{f.
terme d'anat.
`membre situé à l'extrémité du bras; main'}{} qui
mouvent les dois\wdx{doi}{m. 1\hoch{o}
`chacun des cinq prolongements qui terminent la main'}{},
desqueulx les cordes nerveuses\wdx{*nervos}{adj. terme d'anat. `qui a le caractère des nerfs
ou des tendons'}{nerveuses \emph{f.pl.}}
se manifestent\wdx{manifester}{v.pron.
`se révéler
clairement dans son existence ou sa nature'}{}, si co\emph{m}me dit
est devant. Et se
denuent\wdx{denüer}{v.pron. `se dépouiller de'}{}
de char a trois dois\wdx{doi}{m. 3\hoch{o}
`mesure approximative, équivalent à un travers
de doigt'}{} pres de la
joi\emph{n}ture\wdx{jointure}{f.
`partie du corps formée par la jointure entre
deux ou plusieurs os; articulation'}{}, et se ilz
estoient plains\wdx{plain}{adj. `qui est plat, égal;
plain'}{},
ce s\emph{er}oit moult grant\wdx{grant}{adj. 3\hoch{o}
dans
l'ordre qualitatif, non quantifiable `qui est d'un
degré supérieur à la moyenne en ce qui concerne la
qualité, l'intensité, l'importance'}{}
peril\wdx{peril}{m. `état, situation où l'on
court de grands risques; péril'}{}. Item, au bras
ha pluseurs liguemans qui descendent des os et passent
par les joinctures et tiennent\wdx{tenir}{v.tr.
1\hoch{o} `faire rester (qch.) en place'}{} les
joinctures
liee
avec
unes cordes qui se eslargissent\wdx{eslargir}{v.pron. `devenir plus large'}{eslargissent
\emph{3.p.pl. ind.prés.}};
et qui les coperoit\wdx{coper}{v.tr.
`diviser (qch.) avec
un instrument tranchant; couper'}{}, ce s\emph{er}oit
chouse\wdx{chose}{f. `toute réalité
concrète ou abstraite qu'on désigne d'une
manière déterminé'}{chouse}
moult domageuse\wdx{*damajos}{adj. `qui
cause du dommage; dommageable'}{domageuse
\emph{f.sg.}}.
\pend
\pstart
Finalment\wdx{finalment}{adv. `à la
fin'}{} des os il nous
en co\emph{n}vient\wdx{*covenir}{v.tr.indir. `être
convenable
pour (qn)'}{convient \emph{3.p.sg. ind.prés.}}
dire\wdx{dire}{v.tr.indir. \textbf{\emph{dire de}} `parler de'}{}
selon la dicte division\wdx{*devisïon}{f.
`action de
diviser (qch.) en parties, le résultat de
cette action'}{division}
de la grant mein\wdx{grant
main}{f. terme d'anat. `membre supérieur de
l'homme qui s'attache au tronc, compris entre
l'épaule et les doigts; bras'}{grant mein}. Donc en
la premiere p\emph{ar}tie du
%
[27v\hoch{o}a]
\text{grans bras}\fnb{Ms. \emph{gras bras}.}/\wdx{grant bras}{m.
terme d'anat. `membre
supérieur de l'homme qui s'attache au tronc, compris
entre l'épaule et les doigts; bras'}{grans bras \emph{sg.}},
que on appelle
ulne\wdx{ulne}{subst. terme d'anat. `os supérieur
du bras, (par extension)
la partie supérieure du bras'}{} ou
adjuctoire\wdx{*ajutoire}{m. terme d'anat.
`os supérieur du bras, (par extension) la partie
supérieure du bras'}{adjuctoire},
ha ung seul\wdx{*sol}{adj. `qui n'est pas
avec d'autres semblables; seul'}{seul}
os qui est
remplis\wdx{remplir}{v.tr.
`rendre (un
espace disponible) plein de (qch.)'}{} de
medulle\wdx{medulle}{f. terme d'anat.
`substance moelleuse de l'intérieur d'une
structure osseuse'}{},
et est rond\wdx{*rëont}{adj.
`qui a la forme circulaire'}{rond \emph{m.sg.}}
d'une partie et d'autre.
Et la partie de
\text{dessubz}\fnb{Ms. \emph{dessoubz}, \emph{o}
gestrichen.}/\wdx{*desus}{adv. `au
côté supérieur'}{\textbf{de dessus}
\emph{loc.adj. `qui est situé au côté supérieur'}
de dessubz}
ronde qui est
toute
seule\wdx{*sol}{adj. `qui n'est pas
avec d'autres semblables; seul'}{seul}, elle
entre\wdx{entrer}{v.tr.indir.
`s'emboîter
(dans qch.) (de choses)'}{} dedans\wdx{dedans}{prép. `à
l'intérieur de'}{} la boite\wdx{boite}{f. terme
d'anat.
`concavité d'un
os dans laquelle s'emboîte un autre os'}{} ou la fosse\wdx{fosse}{f.
terme d'anat.
`concavité d'assez
grandes dimensions, le plus souvent osseuse, mais pouvant se trouver aussi
dans d'autres structures anatomiques'}{}
de
l'espaule et constitue\wdx{constituer}{v.tr.
1\hoch{o} `former
l'essence de qch.'}{}
et \text{fait la}\fnb{Im
Anschluß gestrichener Buchstabenansatz.}/ joincture de
l'espaulle\wdx{espaule}{f. `partie
supérieure du bras à l'endroit où il s'attache au thorax, pouvant
désigner aussi l'omoplate'}{espaulle}. Mais
la rondesse\wdx{*rëondece}{f. `état de ce qui est rond; rondeur'}{rondesse}
de dessoubz\wdx{*desoz}{adv. `à la face
inférieure'}{\textbf{de dessoubz} \emph{loc.adj.
`qui est situé au côté inférieur'}}
est double\wdx{*doble}{adj.
`qui est
répété deux fois, qui vaut deux fois (la
chose désignée) ou qui existe deux fois'}{double}, et ou milieu ha
ung
degré\wdx{degré}{m. 3\hoch{o} terme d'anat. `structure
articulaire en forme de poulie, s'élévant entre deux
glènes; trochlée'}{}, et est
ainsi que une poulie\wdx{*polie}{f. `petite roue
qui porte sur sa jante une corde, une courroie, et
sert à soulever des fardeaux, à transmettre un
mouvement; poulie'}{poulie} double\wdx{*doble}{adj.
`qui est
répété deux fois, qui vaut deux fois (la
chose désignée) ou qui existe deux fois'}{double}, par ou
passent les cordes quant on
trait\wdx{traire}{v.tr.
`faire venir
dans une certaine direction (qn, qch.)'}{}
l'aigue\wdx{aigue}{f.
`liquide incolore, inodore et transparent; eau'}{}
hors du puis\wdx{*puiz}{m. `cavité
circulaire, profonde et étroite, pratiquée
dans le sol pour atteindre une nappe d'eau
souterraine; puits'}{puis}. Et en la partie de
dedans\wdx{dedans}{adv. `à l'intérieur'}{\textbf{de dedans} \emph{loc.adj. `qui est
situé à l'intérieur'}} est une petite\wdx{petit}{adj.
2\hoch{o} dans l'ordre
physique, quantifiable `qui est d'une extension
au-dessous de la moyenne; petit (du corps humain et
de ses parties)'}{} eminence\wdx{eminence}{f.
terme
d'anat. `saillie à la surface d'une structure
anatomique'}{} ou
bossete\wdx{*bocete}{f. terme
d'anat. `petite saillie à la surface d'une structure
anatomique'}{bossete},
et par derrier\wdx{derrier}{adv. `du
côté opposé au visage, à la face'}{}
ha une concavité\wdx{concavité}{f.
`espace vide à
l'intérieur d'un corps solide; concavité'}{},
en laquelle entre\wdx{entrer}{v.tr.indir.
`s'emboîter (dans qch.) (de choses)'}{}
le chief ou le
additemant\wdx{*additement}{m. `éminence à la surface d'une structure osseuse ou cartilagineuse'}{additamens \emph{pl.}}
bectal\wdx{bectal}{adj.
`qui a la forme d'un bec'}{}
du grant
focille\wdx{grant focile}{m. terme d'anat. `le plus
gros des deux os de l'avant-bras ou de la jambe;
cubitus, tibia'}{grant focille}
qu\emph{an}t on eslieve\wdx{eslever}{v.tr. `mettre ou porter (qch.)
plus haut'}{eslieve \emph{3.p.sg. ind.prés.}}
le bras, en telle
maniere\wdx{maniere}{f. 2\hoch{o} `forme particulière
que revêt l'accomplissement d'une action, le
déroulement d'un fait, l'être ou
l'existence'}{\textbf{en telle maniere que}
\emph{loc.conj. `de sorte que'}} que les
rondesses\wdx{*rëondece}{f.
`état de ce qui est rond; rondeur'}{rondesse} devant d\emph{i}tes
entrent\wdx{entrer}{v.tr.indir.
`s'emboîter (dans qch.) (de choses)'}{}
dedans\wdx{dedans}{prép. `à
l'intérieur de'}{} les concavités\wdx{concavité}{f.
`espace vide à
l'intérieur d'un corps solide; concavité'}{} des
focilles\wdx{focile}{m. terme d'anat. `chacun des deux os de
l'avant-bras ou de la jambe; cubitus, radius, tibia, péroné'}{focille},
et se revolvent\wdx{revolver}{v.pron. `se mettre
en sens inverse ou dans une certaine direction; se
tourner'}{} et tourne\emph{n}t\wdx{tourner}{v.pron.
`se mouvoir circulairement ou décrire une ligne
courbe'}{} quant
on estent\wdx{estendre}{v.tr. `déployer (un membre,
une partie du corps) dans sa longueur'}{} le bras
et quant on le ploie\wdx{ploier}{v.tr.
`courber
une chose flexible; plier'}{} aussi, et
font
\text{la}\fnb{Ms. \emph{la la}.}/
joincture
du coude\wdx{*cote}{m. `partie du membre supérieur
correspondant
à l'articulation du bras et l'avant-bras'}{coude}. En
laquelle joincture
co\emph{m}mence\wdx{*comencier}{v.intr. `entrer
dans son commencement'}{commence
\emph{3.p.sg.
ind.prés.}}
le petit
bras\wdx{petit bras}{m.
terme d'anat. `segment du
membre supérieur de l'homme compris entre le coude et
la main; avant-bras'}{}
qui est la seconde partie devant dicte, ouquel
bras sont les deux os que on appelle
focille\wdx{focile}{m. terme d'anat. `chacun des deux os de
l'avant-bras ou de la jambe; cubitus, radius, tibia, péroné'}{focille},
c'est assavoir le gra\emph{n}t
focille\wdx{grant focile}{m. terme d'anat. `le plus
gros des deux os de l'avant-bras ou de la jambe;
cubitus, tibia'}{grant focille} qui est
dessoubz\wdx{*desoz}{adv.
`à la face
inférieure'}{dessoubz}, et
est plus lonc de l'autre -- pour cause de l'addiccion
bectue\wdx{*becu}{adj.
`qui a la forme d'un
bec'}{bectue \emph{f.sg.}} devant dicte --, et s'en
va vers le petit doy\wdx{petit doi}{m. 1\hoch{o} terme d'anat. `le plus petit
des prolongements qui terminent la main'}{petit doy}
en faisa\emph{n}t une eminence\wdx{eminence}{f.
terme
d'anat. `saillie à la surface d'une structure
anatomique'}{} bossue\wdx{*boçu}{adj. `qui
présente une ou plusieurs saillies arrondies; bossu'}{bossu}
en maniere d'une
cheville\wdx{cheville}{f. `saillie
des os de l'articulation du pied, formée en dedans
par le tibia, en dehors par le péroné, aussi la partie située
entre le pied et la jambe; cheville'}{}. Mais le petit\wdx{petit}{adj.
2\hoch{o} dans l'ordre
physique, quantifiable `qui est d'une extension
au-dessous de la moyenne; petit (du corps humain
et de ses parties)'}{}
os,
%
[27v\hoch{o}b]
il est par dessus\wdx{*desus}{adv. `au côté
supérieur'}{dessus}
et va de la
plicadure\wdx{plicature}{f.
`formation de pli dans le
corps humain'}{plicadure} du
coude\wdx{*cote}{m. `partie du membre supérieur
correspondant à
l'articulation du bras et l'avant-bras'}{coude}
jusques a\wdx{jusques a}{loc.prép.
qui marque
le terme final, la limite que l'on ne dépasse
pas}{}
la mein vers
le polz\wdx{*pouz}{m. `le plus gros et
le plus fort des doigts de la main et du
pied'}{polz}. Et en ch\emph{acu}m
chief d'iceulx sont fosses\wdx{fosse}{f.
terme d'anat.
`concavité d'assez
grandes dimensions, le plus souvent osseuse, mais pouvant se trouver aussi
dans d'autres structures anatomiques'}{}, car vers le
coude\wdx{*cote}{m. `partie du membre supérieur
correspondant
à l'articulation du bras et l'avant-bras'}{coude}
elles reçoivent\wdx{recevoir}{v.tr. `faire entrer
(qch.)'}{reçoivent \emph{3.p.pl. ind.prés.}}
les rondesses\wdx{*rëondece}{f. `état
de ce qui est rond; rondeur'}{rondesse}
graduales\wdx{*graduel}{adj. terme d'anat. `qui a
rapport
à la structure osseuse appelée \flq degré\frq\
(i.e. trochlée)'}{graduales \emph{f.pl.}} de
adjuctoire\wdx{*ajutoire}{m. terme d'anat.
`os supérieur du bras, (par extension) la partie
supérieure du bras'}{adjuctoire}
avec le
additame\emph{n}t\wdx{*additement}{m. `éminence à la surface d'une structure osseuse ou cartilagineuse'}{additament}
bectal\wdx{bectal}{adj. `qui a la forme d'un
bec'}{} devant dit, et vers la
main\wdx{main}{f.
`partie du corps humain située à l'extrémité du bras'}{},
elles reçoivent\wdx{recevoir}{v.tr.
`faire entrer
(qch.)'}{reçoivent \emph{3.p.pl. ind.prés.}} les
rondesses\wdx{*rëondece}{f.
`état de ce qui est rond; rondeur'}{rondesse}
des os des meins. Et sont tous deux gros\wdx{gros}{adj.
1\hoch{o}
dans l'ordre physique, quantifiable `qui, dans
son genre, dépasse le volume ordinaire; gros (du corps
humain et de ses parties)'}{}
et se
joingnent\wdx{joindre}{v.pron.
`se mettre ensemble de
manière à se toucher ou tenir ensemble'}{joingnent
\emph{3.p.pl. ind.prés.}}
ensemble\wdx{ensemble}{adv. `l'un avec l'autre'}{}
en la joincture, et au
milieu ilz sont plus grailles\wdx{*graisle}{adj. `qui a des
formes relativement étroites pour leur longueur et qui donne
une impression de finesse, gracilité, minceur'}{graille} et
plus dista\emph{n}t\wdx{distant}{adj.
`qui est à une certaine distance; éloigné'}{},
afin qu'ilz\wdx{afin que}{loc.conj. qui marque l'intention,
 le but `pour que'}{}
contiegne\emph{n}t\wdx{contenir}{v.tr.
`comprendre en soi,
dans sa capacité, son étendue, sa substance;
contenir'}{contiegnent \emph{3.p.pl. ind.prés.}} les
nerfz et les
muscules.
\pend
\pstart
Et ou ces deux focilles\wdx{focile}{m. terme d'anat.
`chacun des deux os de l'avant-bras ou de la jambe; cubitus, radius,
tibia, péroné'}{focille}
se terminent\wdx{terminer}{v.pron. `prendre fin;
se terminer'}{}
et se co\emph{n}tignent\wdx{contigner}{v.pron.
`entrer en contact'}{contignent
\emph{3.p.pl. ind.prés.}} avec \text{les os de
la}\fnb{Nachfolgend gestrichenes
\emph{la}.}/
mein, la est la joincture \text{de}\fnb{Im
Ms. \emph{da} korrigiert in \emph{de}.}/
la mein et la co\emph{m}mence\wdx{*comencier}{v.intr. `entrer
dans son commencement'}{commence
\emph{3.p.sg.
ind.prés.}}
la mein, et y a trois
acies\wdx{acie}{f. `catégorie de choses considérée d'après sa
structure, son organisation ou sa place dans une série'
(dit des os)}{},
c'est a dire trois
ordre\wdx{ordre}{m.
2\hoch{o} `catégorie de choses considérée d'après sa
structure, son organisation ou sa place dans une série'}{}
d'os, et la les os de bas\wdx{bas}{m. `partie
inférieure (de qch.); bas'}{} se
conjoingnent\wdx{conjoindre}{v.pron.
`se mettre ensemble de manière à se
toucher ou tenir ensemble'}{conjoingnent
\emph{3.p.pl. ind.prés.}} par
\text{leur}\fnb{Nachfolgend gestrichener
Buchstabenansatz.}/ rondesse\wdx{*rëondece}{f. `état de ce qui est
rond; rondeur'}{rondesse} es fosses\wdx{fosse}{f. terme d'anat.
`concavité d'assez
grandes dimensions, le plus souvent osseuse, mais pouvant se trouver aussi
dans d'autres structures anatomiques'}{}
des
os qui sont hault\wdx{hault}{adv.
`en un endroit qui est d'une certaine dimension
dans le sens vertical; haut'}{}. En la premiere
acie\wdx{acie}{f. `catégorie de choses considérée d'après sa
structure, son organisation ou sa place dans une série'
(dit des os)}{} sont trois
os, car l'addiccion du focille\wdx{focile}{m. terme d'anat. `chacun
des deux os de l'avant-bras ou de la jambe; cubitus, radius, tibia,
péroné'}{focille} qui
est p\emph{ar} dessus\wdx{*desus}{adv. `au côté
supérieur'}{dessus} est
ainsi que tenant\wdx{tenir}{v.tr.
1\hoch{o} `faire rester (qch.) en place'}{tenant
\emph{p.prés.}} le lieu d'ung os. En
la seconde acie\wdx{acie}{f. `catégorie de choses considérée
d'après sa structure, son organisation ou sa place dans une série'
(dit des os)}{}
sont quatre os, et en celui qui est
hault\wdx{hault}{adv.
`en un endroit qui est d'une certaine dimension
dans le sens vertical; haut'}{}
ha une petite\wdx{petit}{adj.
2\hoch{o} dans l'ordre
physique, quantifiable `qui est d'une extension
au-dessous de la moyenne; petit (du corps humain
et de ses parties)'}{}
boite\wdx{boite}{f. terme d'anat. `concavité d'un
os dans laquelle s'emboîte un autre os'}{} en laquelle se
ferme\wdx{fermer}{v.pron.
`être attaché (à qch.)'}{} le premier
os du polz\wdx{*pouz}{m. `le plus gros et le plus
fort des doigts de
la main et du pied'}{polz}, et les os de ces deux
accies\wdx{acie}{f. `catégorie de choses considérée d'après sa
structure, son organisation ou sa place dans une série'
(dit des os)}{accie}
sont cours. En la tierce accie sont quatre os qui
sont plus
loncs\wdx{lonc}{adj.
1\hoch{o}
`qui a une étendue supérieure à la moyenne
dans le sens de la longueur (dans l'espace)'}{loncs \emph{m.pl.}} que les
aultres.
La premiere partie des deux accies, on l'appelle la
rachete\wdx{rachete}{f. terme d'anat. `ensemble de
petits os qui forme une partie de la main, situé entre
les os de l'avant-bras et le métacarpe, et celle du
pied, situé entre les os de la jambe et le métatarse;
carpe, tarse'}{}
ou carpus\wdx{carpus}{lt.
terme d'anat. `partie articulaire constituant le
poignet et qui est formée de huit os courts disposés
en deux rangées composées chacune de quatre os;
carpe'}{}. L'autre partie, on la
appelle le pine\wdx{*peigne}{m.
terme d'anat.
`ensemble des os (métacarpiens, métatarsiens)
qui constituent la partie du squelette de la main
entre le carpe et les premiers phalanges des doigts,
et celle du squelette du pied entre le tarse et les
premiers phalanges des orteils; métacarpe,
métatarse'}{pine}
ou methacarpus\wdx{*metacarpus}{mlt.
terme d'anat. `partie du squelette de la main
formée de cinq os, comprise entre le carpe et les
phalanges; métacarpe'}{methacarpus}. Et en aprés
vie\emph{n}nent les dois\wdx{doi}{m. 1\hoch{o}
`chacun des cinq prolongements qui terminent la main'}{},
et en chescun doy\wdx{doi}{m. 1\hoch{o}
`chacun des cinq prolongements qui terminent la
main'}{doy}
so\emph{n}t
%
[28r\hoch{o}a]
trois os et y a .v. dois\wdx{doi}{m. 1\hoch{o}
`chacun des cinq prolongements qui terminent la
main'}{}. Donc il y a au
dois\wdx{doi}{m. 1\hoch{o}
`chacun des cinq prolongements qui terminent la
main'}{}
.xv. os et .xj. en la main\wdx{main}{f. `partie du corps
humain située à l'extrémité du bras'}{} et deux au bras et
ung en l'adjuctoire, et ainsi, en tout le bras, sont
.xxix. os. Et ainsi appart le nombre\wdx{nombre}{m.
`mot servant à caractériser une pluralité de
choses ou de personnes; nombre'}{}
des parties des
dis membres.
\pend
\pstart
Donc il nous co\emph{n}vie\emph{n}t veoir des
maladies\wdx{maladie}{f.
 `altération organique ou
fonctionnelle considérée dans son évolution, et comme
une entité définissable; maladie'}{}
qui peullent advenir\wdx{avenir}{v.tr.indir. `venir
ou
être sur le point d'être; arriver'}{advenir
\emph{inf.}} en yceulx me\emph{m}bres,
c'est assavoir empostumes\wdx{empostume}{m. ou f.
terme de méd. `tumeur
accompagné de suppuration'}{}, plaies\wdx{plaie}{f.
`ouverture
dans les chairs, les tissus, due à une cause
externe (traumatisme, intervention chirurgicale) et
présentant une solution de continuité des téguments;
plaie'}{},
dislocacions\wdx{dislocacïon}{f.
`déplacement violent d'une partie du
corps (d'un membre, d'un os, d'une
articulation)'}{dislocacion},
fractures\wdx{fracture}{f. `lésion osseuse formée par
une solution de continuité avec ou sans déplacement
des fragments'}{},
p\emph{ar}alizie\wdx{*paralisie}{f. terme de méd.
`déficience ou perte de
la fonction motrice d'une partie du corps; paralysie'}{paralizie}.
Et pour leur anathomie vous
poués voir\wdx{*vëoir}{v.tr. `percevoir (qch.) par le sens de la vue'}{voir
\emph{inf.}} que les
incisions\wdx{incision}{f. `action de fendre,
de couper avec un instrument tranchant, son
résultat (surtout en médecine)'}{} doivent estre faictes du
long\wdx{lonc}{adv.}{\textbf{du lonc}
\emph{loc.adv. `dans le sens de la longueur'} du
long}, car ainsi vont les muscules. Et se poués veoir
que la joincture du coude, c'est la plus
difficille\wdx{*dificile}{adj.
`qui
n'est pas facile; difficile'}{difficille} a
desliever\wdx{desliever}{v.tr. `provoquer la
luxation de (un os, une articulation); disloquer'}{}
et a remetre\wdx{remetre}{v.tr. `faire
passer (qch.) de nouveau dans son ancien
état, son ancienne place'}{} a point\wdx{point}{m. `endroit fixé et déterminé (où qch. à
lieu)'}{}.
Et celle de
l'espaule, c'est la plus aisee\wdx{aisé}{adj. `qui se fait avec aise;
facile'}{}, et celle de la
main\wdx{main}{f. `partie du corps humain située à l'extrémité
du bras'}{}, c'est la moienne\wdx{moien}{adj. `qui est au milieu'}{}.
Et se poués aussi veoir
queulx parties so\emph{n}t plus tost\wdx{tost}{adv. `en
un temps bref; rapidement'}{}
desloees\wdx{*desliier}{v.tr.
`provoquer la luxation de (un os, une articulation);
disloquer'}{desloé \emph{p.p.}}. Et poués
co\emph{n}siderer\wdx{considerer}{v.tr.
`regarder (qch.) attentivement;
considérer'}{}
que, en paralizie\wdx{*paralisie}{f. terme de méd.
`déficience ou perte de
la fonction motrice d'une partie du corps; paralysie'}{paralizie} de
ces me\emph{m}bres
ci, les remedes\wdx{remede}{m. et f. `ce qui est
employé au traitement d'une maladie; remède'}{}
doivent estre
appliqués\wdx{*apliquier}{v.tr. `mettre une chose sur
une autre de manière qu'elle la recouvre et y adhère;
appliquer'}{appliqué
\emph{p.p.}} vers
les spondilles du col, car de la viennent les nerfz.
\pend
%
% \memorybreak
%
\pstartueber
Le .v.\hoch{e} chappitre parle de l'anathomie du pis\wdx{*piz}{m.
terme d'anat.
`partie
du corps humain qui s'étend des épaules à l'abdomen et qui contient
le c\oe ur et les poumons; thorax'}{pis} et
de
ces parties.
\pendueber
%
% \memorybreak
%
\pstart
LES PIS\wdx{*piz}{m. terme d'anat. `partie
du corps humain qui s'étend des épaules à
l'abdomen et qui contient le c\oe ur et les
poumons; thorax'}{pis} ou la
poitrine\wdx{poitrine}{f.
terme d'anat.
`partie du corps humain qui s'étend des
épaules à l'abdomen et qui contient le c\oe ur et les poumons;
thorax'}{},
c'est l'arche\wdx{arche}{f. terme d'anat. `partie
supériere
du tronc limitée en dessous par le diaphragme, en
arrière et lateralement par les arcs dorsaux et dont
l'intérieur constitue la cavité où sont logés le c\oe ur et les
poumons'}{} des
membres spirituelz\wdx{*esperituel}{adj. terme de méd.
`qui contient les esprits vitaux, qui a rapport à la
transformation ou à la diffusion des esprits
vitaux'}{spirituelz
\emph{m.pl.}}, et pour ce
y a aucunes parties qui co\emph{n}tiennent et aucunes qui
%
[28r\hoch{o}b]
sont contenues.  Les parties
q\emph{ue}
contiennent sont quatre, c'est assavoir le cuir, la
char musculeuse\wdx{musculeux}{adj. terme d'anat. `qui est de la nature
des muscles'}{},
les mamelles\wdx{*mamele}{f.
`organe glanduleux
qui sécrète le lait (chez les mammifères);
mamelle'}{mamelle} et les
os. Les p\emph{ar}ties
qui sont dedans\wdx{dedans}{prép. `à
l'intérieur de'}{} la dicte arche\wdx{arche}{f.
terme d'anat.
`partie supériere du
tronc limitée en dessous par le diaphragme, en
arrière et lateralement par les arcs dorsaux et dont
l'intérieur constitue la cavité où sont logés le c\oe ur et les
poumons'}{} sont
.viij., \text{c'est assavoir}\fnb{Ms. \emph{cessavoir}.}/
le cuer\wdx{cuer}{m. terme d'anat. `viscère de forme de cône
renversé, situé entre les poumons, qui est l'organe central de la
distribution du sang dans le corps'}{}, le poulmon\wdx{*poumon}{m.
terme d'anat.
`chacun des deux viscères logés
symétriquement dans la cage thoracique qui sont les
organes de la respiration, aussi l'ensemble
des deux; poumon'}{poulmon},
les panicles, les liguemans, les ners, les vaines,
les arteres, le mery\wdx{meri}{m.
terme d'anat.
`canal musculo-membraneux qui va
du pharynx à l'estomac auquel il conduit
les aliments; \oe sophage'}{mery}
ou le
ysophagus\wdx{*isophagus}{mlt.
terme
d'anat. `canal musculo-membraneux qui va
du pharynx à l'estomac auquel il conduit
les aliments; \oe sophage'}{ysophagus}. De la
char et du cuir il appart assez\wdx{assez}{adv.
`en suffisance;
assez'}{}.
Que c'est des mamelles\wdx{*mamele}{f.
`organe glanduleux
qui sécrète le lait (chez les mammifères);
mamelle'}{mamelle}
qui
sont sur la char, saches que elles sont faites de char
blanche\wdx{blanc}{adj. `qui est de
la couleur de la neige; blanc'}{},
glandelleuse\wdx{*glandulos}{adj. `qui contient des
glandes'}{glandelleuse \emph{f.sg.}},
spongieuse\wdx{*spongios}{adj. `dont
la structure ressemble à celle de l'éponge;
spongieux'}{spongieuse
\emph{f.sg.}} et de
vaines et d'arteres et de nerfz. Et pour ce elles ont
colligance\wdx{colligance}{f. 1\hoch{o}
`force qui maintient réunis les éléments d'un système matériel;
liaison'}{} au cuer, au
foye\wdx{foie}{m. terme d'anat. `organe
situé dans la partie supérieure droite de
l'abdomen et qui sécrète la bile;
foie'}{foye}
et au cervel
et aux me\emph{m}bres
generatifz\wdx{membres generatis}{m.pl. `organes
servant à la reproduction'}{membres generatifz}.
Des muscules -- en
parlant
\text{brié\-mant}\fnb{Im Ms. \emph{vraiemant}
korrigiert in \emph{briemant}.}/\wdx{*briément}{adv.
`en peu de temps ou en peu de mots; en
bref'}{briémant} selon
Avicene\adx{Avicene}{}{} --, ou pis\wdx{*piz}{m.
terme d'anat.
`partie du corps
humain qui s'étend des épaules à l'abdomen et qui contient le c\oe ur
et les poumons; thorax'}{pis} en ha .lxxx. ou
.lxxxx. muscules, et
aucuns sont co\emph{m}muns\wdx{*comun}{adj. `qui appartient
à plusieurs personnes ou
choses'}{commun}
au col, aucuns es espaules et
aucuns au diafragme\wdx{diafragme}{m.
terme d'anat. `muscle large et
mince qui sépare le thorax de l'abdomen;
diaphragme'}{}. Et aucuns sont
p\emph{ro}pres\wdx{propre}{adj. 1\hoch{o} `qui
appartient exclusivement à (qn, qch.)'}{} au pis,
aucuns aux costés\wdx{costé}{m. `partie
qui est à droite ou à gauche (d'un corps); côté'}{},
aucuns au dos.
\pend
\pstart
Les os du pis\wdx{*piz}{m. terme d'anat.
`partie du corps humain qui s'étend
des épaules à
l'abdomen et qui contient le c\oe ur et les
poumons; thorax'}{} sont
triples\wdx{triple}{adj.
`qui
équivaut à trois, se présente comme trois;
triple'}{}. En la p\emph{ar}tie de
devant\wdx{devant}{adv. 1\hoch{o} `au côté du
visage, à la face'}{\textbf{de devant}
\emph{loc.adj. `qui est situé au côté du visage, de la
face'}} en ha .vij.
\text{que}\fnb{Im Ms. \emph{de quoy} korrigiert in
\emph{que}.}/ on appelles
les os du pis\wdx{*piz}{m. terme d'anat.
`partie du corps humain qui s'étend
des épaules à
l'abdomen et qui contient le c\oe ur et les
poumons; thorax'}{}
ou
du coral\wdx{coral}{m.
terme d'anat.
`partie du corps humain qui s'étend
des épaules à
l'abdomen et qui contient le c\oe ur et les
poumons; thorax'}{}, et sont moult
cartillagineux\wdx{*cartilaginos}{adj. terme d'anat. `qui est formé de cartilage;
cartilagineux'}{cartillagineux \emph{m.pl.}}
par
dessus\wdx{*desus}{adv. `au côté
supérieur'}{dessus}.
Desqueulx, le premier vers la goule\wdx{*gole}{f.
`parties antérieures et latérales du cou;
gorge'}{goule} est receu
en la boite\wdx{boite}{f. terme d'anat. `concavité
d'un os dans laquelle s'emboîte un autre os'}{} du
pié de la fourchete\wdx{pié de la fourchete}{m. terme
d'anat.
`segment supérieur du sternum qui s'articule avec
les deux clavicules et la première paire des côtes;
manubrium sternal'}{}
 devant
dicte. Et par \text{desous}\fnb{Ms. \emph{deSous}.}/,
en la
fourcelles\wdx{*forcele}{f.
2\hoch{o} terme d'anat.
`fourche du sternum'}{fourcelle}, en
l'orifice\wdx{orifice}{m.
`ouverture faisant communiquer un conduit, un organe
avec une structure voisine ou avec l'extérieur;
orifice'}{}
de l'estomac\wdx{estomac}{m. terme d'anat.
`organe de l'appareil
digestif qui reçoit les aliments; estomac'}{},
ha ung
addittament\wdx{*additement}{m. `éminence à la surface d'une structure osseuse ou cartilagineuse'}{addittament}
cartillagineux\wdx{*cartilaginos}{adj. terme d'anat. `qui est formé de cartilage;
cartilagineux'}{cartillagineux \emph{m.sg.}}
que l'en appelle
%
[28v\hoch{o}a]
ensifourme\wdx{*ensiforme}{adj. `qui
est en forme d'épée; ensiforme'}{ensifourme}.
En la partie de dariere\wdx{derrier}{adv. `du
côté opposé au visage, à la face'}{\textbf{de
derrier} \emph{loc.adj. `qui est situé au
côté opposé au visage, à la face'} de dariere}, vers
le dos, sont .xij. spondilles par lesquelles passe la
nuque, de laquelle naissent .xij. pareilz\wdx{pareil}{m.
`réunion de deux choses,
de deux êtres semblables
qui vont ensemble; paire'}{pareilz \emph{pl.}}
de nerfz qui
portent\wdx{porter}{v.tr.
`déplacer (qch.)
d'un lieu à un autre en le menant avec soi;
transporter'}{}
sentemant\wdx{*sentement}{m.
`faculté d'éprouver
les impressions que font les objets matériels, i.e.
goût, odorat, ouïe, toucher, vue'}{sentemant}
et mouvemant aux dis muscules.
Mais vers chescun costé\wdx{costé}{m. `partie qui est à droite ou
à gauche (d'un corps); côté'}{} sont .xij.
costes\wdx{coste}{f. terme d'anat. `os plat
et courbe du thorax qui s'articule sur la colonne vertébrale et le
sternum; côte'}{}, c'est
assavoir .vij. vraies\wdx{*verai}{adj. `qui présente
un caractère de vérité; vrai'}{vraies \emph{f.pl.}}
et .v. faulces\wdx{*faus}{adj. `qui n'est pas ce qu'on le nomme;
faux'}{faulces \emph{f.pl.}} ou
mendouses\wdx{*mentos}{adj. `qui n'est pas ce qu'on le nomme; faux'}{mendouse \emph{f.sg.}},
car elles
ne sont pas entieres\wdx{entier}{adj. `dans tout
son étendue; entier'}{} ainsi q\emph{ue}
les aultres .vij., et ch\emph{ac}um en peult veoir la fourme\wdx{*forme}{f.
`apparence extérieure
donnant à un objet ou à un être sa
spécificité'}{fourme}.
Et vous souffise\wdx{*sofire}{v.intr.
`avoir la quantité, la qualité, la force
etc. nécessaire pour (qch.);
suffire'}{\textbf{*sofire a qn}
\emph{v.impers. + subj.prés.} vous souffise}
des parties contenans ce que dit en
est.
\pend
\pstart
Mais se tu veulx bien fere l'anathomie des
p\emph{ar}ties qui so\emph{n}t co\emph{n}tenues ou pis ou dedans\wdx{dedans}{prép. `à
l'intérieur de'}{} la
poitrine\wdx{poitrine}{f.
terme d'anat.
`partie du corps humain qui s'étend des
épaules à l'abdomen et qui contient le c\oe ur et les
poumons; thorax'}{},
il co\emph{n}vient q\emph{ue} tu copes\wdx{coper}{v.tr.
`diviser (qch.) avec
un instrument tranchant; couper'}{}
et tra\emph{n}chez\wdx{*trenchier}{v.tr.
`séparer (une chose
en parties, deux choses unies) d'une manière nette, au
moyen d'un instrument dur et fin; trancher'}{tranchez
\emph{2.p.sg. ind.prés.}} le pis vers les costés et
que
tu ostes\wdx{oster}{v.tr. `enlever (qch.) de la place
qu'il occupait; ôter'}{} la partie de
devant\wdx{devant}{adv. 1\hoch{o}
`au côté du visage, à la face'}{\textbf{de devant}
\emph{loc.adj. `qui est situé au côté du visage, de la
face'}}
--
cautemant\wdx{*cautement}{adv. `avec
prudence'}{cautemant}
pour le mediastin\wdx{mediastin}{m. terme d'anat.
`région du thorax qui sépare la face interne des
poumons; médiastin'}{}
--,
et ta\emph{n}tost\wdx{tantost}{adv.
`dans un temps prochain, un
proche avenir; tantôt'}{} tu verras les choses
de dedans\wdx{dedans}{adv. `à l'intérieur'}{\textbf{de dedans} \emph{loc.adj. `qui est
situé à l'intérieur'}}.
\text{Desqueulx}\fnb{Nachfolgend gestrichenes
\emph{et}.}/, le premier et le principal\wdx{principal}{adj. `qui est le plus
important; principal'}{},
c'est le cuer qui est
co\emph{m}mencemant\wdx{*comencement}{m. `première partie de qch.,
celle que d'autres doivent suivre et qu'aucune ne précède
(dans le temps ou dans l'espace)'}{commencemant} de
vie. Et pour ce, co\emph{m}me
\text{roy}\fnb{Ms. \emph{Roy}.}/\wdx{*roi}{m.
`chef souverain d'un État ayant le titre de
royaume; roi'}{roy} il est
assis\wdx{asseoir}{v.tr. 2\hoch{o} `placer, poser
(qch.)'}{assis \emph{p.p.}}
droit\wdx{droit}{adv. `d'une manière exacte'}{}
au milieu du pis, sans aller ne sa ne la quant est du
centre\wdx{centre}{m. `le milieu d'un espace'}{}
-- selon le dit\wdx{dit}{m. `expression
verbale de la pensée'}{}
de Galien\adx{Galien}{}{} ou livre \flq De utilitate
p\emph{ar}ticula\emph{rum}\frq\ --, ja soit ce q\emph{ue} la partie
baisse\wdx{bas}{adj. `qui se
trouve à une faible hauteur; bas'}{baisse
\emph{f.sg.}}
du cuer se samble encliner\wdx{encliner}{v.pron. `se
pencher (de choses)'}{} a lla
senextre\wdx{senestre}{adj. `qui est du
côté gauche; gauche'}{senextre} partie pour le
lieu du foie\wdx{foie}{m. terme d'anat. `organe
situé dans la partie supérieure droite de
l'abdomen et qui sécrète la bile;
foie'}{}.
Et quant a la haulte\wdx{hault}{adj.
`qui est d'une certaine dimension
dans le sens vertical; haut'}{}
partie, il s'e\emph{n}cline\wdx{encliner}{v.pron.
`se
pencher (de choses)'}{} a
la dextre\wdx{*destre}{f. `le côté droit;
droite'}{dextre} pour donner\wdx{*doner}{v.tr. `mettre (qch.) à la disposition de qn,
de qch.'}{donner \emph{inf.}}
lieu
aux arteres. Donc la fourme du cuer est -- a maniere
de\wdx{maniere}{f. 2\hoch{o} `forme particulière
que revêt l'accomplissement d'une action, le
déroulement d'un fait, l'être ou
l'existence'}{\textbf{a\,/\,en maniere de}
\emph{loc.prép. `comme'}} po\emph{m}me de
pin\wdx{*pome de pin}{f. `fruit produit par le
pin'}{pomme de pin}
--
%
[28v\hoch{o}b]
enversee\wdx{enverser}{v.tr. `mettre de façon que la
partie supérieure devienne inférieure; renverser'}{},
pour ce que l'agueté\wdx{agüeté}{f. `état
de ce qui est aigu'}{} du [cuer] va
vers
les basses\wdx{bas}{adj. `qui se
trouve à une faible hauteur; bas'}{basse
\emph{f.sg.}} parties du corps. Et le
leet\wdx{*lé}{m.
`la plus petite dimension de qch. (en opposition à sa
longueur); largeur'}{leet} du
cuer,
qui est la racine\wdx{racine}{f. 1\hoch{o}
terme d'anat.
`portion d'un organe
servant à son implantation dans un autre organe'}{},
va \text{vers}\fnb{\emph{r} über der Zeile nachgetragen.}/ les parties
de dessus\wdx{*desus}{adv. `au côté
supérieur'}{\textbf{de dessus} \emph{loc.adj. `qui
est situé au côté supérieur'}}.
La substance du cuer est dure\wdx{dur}{adj. `qui résiste à la pression, qui ne se
laisse pas déformer facilement'}{}
ainsi que
lacertouse\wdx{*lacertos}{adj. terme d'anat. `qui est de la
nature des muscles'}{lacertouse
\emph{f.sg.}},
et ha en lui deux ventrissons\wdx{ventrisson}{m.
terme d'anat. `cavité particulière à certains
organes; ventricule' (dans le cerveau, dans le c\oe
ur)}{},
le dextre\wdx{*destre}{adj. `qui est du côté droit;
droit'}{dextre} et le
senextre\wdx{senestre}{adj. `qui est du
côté gauche; gauche'}{senextre},
et au milieu une fosse\wdx{fosse}{f. terme d'anat.
`concavité d'assez
grandes dimensions, le plus souvent osseuse, mais pouvant se trouver aussi
dans d'autres structures anatomiques'}{}
-- se dit
Galien\adx{Galien}{}{} --, esquelx se
digere\wdx{digerer}{v.pron. terme de méd. `se
convertir en sucs par les mécanismes de la
digestion'}{} le gros\wdx{gros}{adj. 4\hoch{o} `qui
est consistant, filant (en parlant du sang)'}{}
sang\wdx{*sanc}{m. terme de méd.
`liquide visqueux, de couleur
rouge, qui est porté par les vaisseaux dans tout
l'organisme où il joue des rôles multiples (l'une
des quatre humeurs de l'humorisme)'}{sang}
nourrissant\wdx{nourrissant}{p.prés. comme adj.
`qui a une valeur nutritive, qui nourrit'}{}
q\emph{ue} vient du foye\wdx{foie}{m. terme d'anat. `organe
situé dans la partie supérieure droite de
l'abdomen et qui sécrète la bile;
foie'}{foye}.
Et est subtil\wdx{subtil}{adj.
2\hoch{o} `qui est fluide, qui coule aisément
(en parlant d'un liquide)'}{}
et est espirituel\wdx{*esperituel}{adj. terme de méd.
`qui contient les esprits vitaux, qui a rapport à la
transformation ou à la diffusion des esprits
vitaux'}{espirituel \emph{m.sg.}} qui
est envoié\wdx{envoiier}{v.tr. `faire aller qn ou
qch.
(quelque part)'
(le sujet n'étant pas personnel)}{envoié \emph{p.p.}} par les
arteres par tout le corps, maiemant\wdx{*maismement}{adv. `plus que tout
autre chose; surtout'}{maiemant}
aux
aultres me\emph{m}bres principaulx, au cervel, auquel, en
digerant\wdx{digerer}{v.pron. terme de méd.
`se convertir en sucs
par les mécanismes de la
digestion'}{digerant \emph{p.prés.}},
il pre\emph{n}t\wdx{prendre}{v.tr. 2\hoch{o} `commencer à avoir (un mode d'être)'}{prendre autre
nature} autre nature et est sang\wdx{*sanc}{m.
terme de méd.
`liquide visqueux, de couleur
rouge, qui est porté par les vaisseaux dans tout
l'organisme où il joue des rôles multiples (l'une
des quatre humeurs de l'humorisme)'}{sang}
animal\wdx{animal}{adj.
terme de méd. `qui
contient les esprits affinés dans le cerveau qui
contrôlent l'activité mentale et nerveuse, qui a
rapport
à la
transformation ou à la diffusion de ces esprits'}{}.
Et
\text{au foye est}\fnb{Ms. \emph{au foye et
est}.}/ fait
sang
naturel\wdx{naturel}{adj. 1\hoch{o} terme de méd.
`qui contient les esprits transformés dans le foie
qui contrôlent l'alimentation, le grandissement et la
génération de l'homme, qui a rapport
à la
transformation ou à la diffusion de ces esprits'}{}
et aux coillons\wdx{coillon}{m. 1\hoch{o} `gonade mâle suspendue dans le scrotum, qui
produit les spermatozoïdes; testicule'}{}, et lors
il est
generatifz\wdx{generatif}{adj.
`qui produit (qch.)'}{}. Et
se
l'envoie\wdx{envoiier}{v.tr. `faire aller qn ou qch.
(quelque part)'
(le sujet n'étant pas personnel)}{envoie
\emph{3.p.sg. ind.prés.}} a tous les aultres
me\emph{m}bres pour eulx appareiller\wdx{appareiller}{v.tr.
`mettre (qch.) en
état de remplir sa fonction'}{}
et viviffier\wdx{vivifier}{v.tr.
`donner la vie,
l'énérgie vitale à, entretenir la vie, l'énérgie
vitale de'}{viviffier \emph{inf.}}. Car c'est
l'instrument\wdx{instrument}{m. 2\hoch{o}
`partie du corps remplissant une fonction
particulière; organe'}{} de toutes les
vertus\wdx{vertu}{f.
`principe
qui, dans une chose, est considéré comme la cause
des effets qu'elle produit; faculté, pouvoir'}{} du
corps et c'est le loien\wdx{*liien}{m. `ce
qui établit entre deux ou plusieurs personnes
ou choses des relations (d'ordre social,
moral, affectif, etc.)'}{loien}
complet\wdx{complet}{adj. `auquel ne manque aucun
des éléments qui doivent le constituer; complet'}{}
de l'ame. Et pour
ce au cuer sont deux
oriffices\wdx{orifice}{m.
`ouverture faisant communiquer un conduit, un organe
avec une structure voisine ou avec l'extérieur;
orifice'}{oriffice}: par
dextre\wdx{*destre}{f. `le côté droit; droite'}{dextre}
entre\wdx{entrer}{v.intr. `passer du dehors en
dedans'}{} une rame\wdx{rame}{m. et f. terme d'anat.
`subdivision d'un vaisseau sanguin'}{} de la vaine
ascendant\wdx{ascendant}{adj. terme
d'anat.
`qui monte (surtout en parlant d'un vaisseau
sanguin)'}{} qui
porte\wdx{porter}{v.tr.
`déplacer (qch.)
d'un lieu à un autre en le menant avec soi;
transporter'}{}
le sang du foie\wdx{foie}{m. terme d'anat. `organe
situé dans la partie supérieure droite de
l'abdomen et qui sécrète la bile;
foie'}{}
en hault\wdx{hault}{adv.
`en un endroit qui est d'une certaine
dimension dans le sens vertical; haut'}{\textbf{en
hault} \emph{loc.adv. `vers la partie haute'}}; et
de lui ist hors une partie que on
appelle la vaine arterial\wdx{*veine arteriale}{f. terme d'anat.
`vaisseau sanguin
qui s'étend du ventricule droit du c\oe ur au
poumon'}{vaine arterial}
\text{et va}\fnb{\emph{et} über der Zeile
nachgetragen.}/ au poulmon\wdx{*poumon}{m.
terme d'anat.
`chacun des deux viscères logés
symétriquement dans la cage thoracique qui sont les
organes de la respiration, aussi l'ensemble
des deux; poumon'}{poulmon}
pour luy \text{norrir}\fnb{Ms. \emph{norri}.}/\wdx{norrir}{v.tr.
`entretenir, faire vivre en
donnant à manger ou en procurant les aliments nécessaires à la
subsistance; nourrir'}{}, et le residu\wdx{residu}{m.
`ce qui reste; résidu'}{} se ramefie en
montant\wdx{monter}{v.tr.indir.
`se déplacer dans un mouvement de bas en
haut'}{}
amont\wdx{amont}{adv. `vers le haut'}{}
par pluseurs rames\wdx{rame}{m. et f.
terme d'anat. `subdivision d'un vaisseau
sanguin'}{}, et monte\wdx{monter}{v.tr.indir. `se
déplacer dans un mouvement de bas en
haut'}{} sur
le chief,
\text{si}\fnb{Über der Zeile nachgetragen.}/ co\emph{m}me dit
est. Et de l'oriffice\wdx{orifice}{m.
`ouverture faisant communiquer un conduit, un organe
avec une structure voisine ou avec l'extérieur;
orifice'}{oriffice}
senextre\wdx{senestre}{adj. `qui est du
côté gauche; gauche'}{senextre} ist
la vaine pulsatille\wdx{pulsatil}{adj. terme
d'anat. `qui
a, qui présente de pulsations (en parlant d'un
vaisseau sanguin)'}{pulsatille \emph{f.sg.}}, de
laquelle une partie va
%
[29r\hoch{o}a]
au poulmon\wdx{*poumon}{m. terme d'anat.
`chacun des deux viscères logés
symétriquement dans la cage thoracique qui sont les
organes de la respiration, aussi l'ensemble
des deux; poumon'}{poulmon}
que on appelle la
venale\wdx{venale}{f. terme d'anat.
`vaisseau sanguin qui s'étend de l'oreillette
gauche du c\oe ur au poumon'}{},
et porte\wdx{porter}{v.tr.
`déplacer (qch.)
d'un lieu à un autre en le menant avec soi;
transporter'}{}
vapours\wdx{*vapor}{f.
terme de physiol.
`liquide du corps humain à l'état gazeux'}{vapour} au
poulmon\wdx{*poumon}{m. terme d'anat.
`chacun des deux viscères logés
symétriquement dans la cage thoracique qui sont les
organes de la respiration, aussi l'ensemble
des deux; poumon'}{poulmon}
et
introduit\wdx{introduire}{v.tr. `amener à un
endroit'}{} l'air dedans\wdx{dedans}{adv. `à
l'intérieur'}{} pour
\text{reffroidir}\fnb{\emph{dir} über der Zeile
ersetzt gestrichenes
\emph{ff}.}/\wdx{*refroidir}{v.tr. `rendre plus
froid'}{reffroidir \emph{inf.}} le cuer.
Et l'autre partie se ramefie\wdx{*ramifier}{v.pron.
`se diviser en plusieurs ramifications qui partent
d'un axe ou d'un centre de qch. (en parlant d'une
chose concrète)'}{ramefie
\emph{3.p.sg. ind.prés.}} et hault
et bas\wdx{bas}{adv. `à faible hauteur; bas'}{}, si
co\emph{m}me dit est des
aultres vaines. Et sur ces oriffices\wdx{orifice}{m.
`ouverture faisant communiquer un conduit, un organe
avec une structure voisine ou avec l'extérieur;
orifice'}{oriffice}
sont trois
piaucelles\wdx{*piaucele}{f.
`couche de tissu qui enveloppe un organe, qui couvre
un orifice d'un organe ou qui tapisse une cavité dans
le corps'}{piaucelle},
ouvrans\wdx{ovrir}{v.tr.
`faire que ce qui était fermé ne le soit plus;
ouvrir'}{ouvrans \emph{p.prés.}}
et cloiantz\wdx{clore}{v.tr.
`boucher ce qui est ouvert;
fermer'}{cloiantz \emph{p.prés.}}
l'e\emph{n}tree\wdx{entree}{f. `ce qui donne accès dans un
lieu; entrée'}{} du sang
et de
l'esperit\wdx{esperit}{m.
terme de méd.
`ensemble de corpuscules subtils qui assurent
toutes les fonctions de la vie dans
l'organisme humain'}{} en te\emph{m}ps\wdx{*tens}{m. `milieu indéfini où paraissent se dérouler irréversiblement
les existences dans leur changement, les événements et les
phénomènes dans leur succession; temps'}{temps}
co\emph{n}venables\wdx{*covenable}{adj.
`qui est approprié'}{convenable}; et
pres de la sont
deux oreilles\wdx{oreille}{f. 2\hoch{o}
terme d'anat. `chacune des deux
cavités supérieures du c\oe ur; oreillette'}{}, par
lesquelles l'air entre\wdx{entrer}{v.intr. `passer
du dehors en dedans'}{} et ist que
le poulmon\wdx{*poumon}{m. terme d'anat.
`chacun des deux viscères logés
symétriquement dans la cage thoracique qui sont les
organes de la respiration, aussi l'ensemble
des deux; poumon'}{poulmon}
appareille\wdx{appareiller}{v.tr.
`mettre (qch.) en
état de remplir sa fonction'}{} au cuer. Et
se y treuve\wdx{*trover}{v.tr. `rencontrer qn ou qch. qu'on cherche;
trouver'}{treuve \emph{3.p.sg. ind.prés.}}
on ung os cartillagineux\wdx{*cartilaginos}{adj. terme d'anat. `qui est formé de cartilage;
cartilagineux'}{cartillagineux \emph{m.sg.}}
pour lui
mieulx fermer\wdx{fermer}{v.tr.
`faire tenir (à une chose) au
moyen d'une attache, d'un lien; attacher'}{}
et roborer\wdx{roborer}{v.tr. `rendre (qch.)
fort, solide; fortifier' (des choses concrètes)}{}.
Item,
le cuer est couvert d'une chapete\wdx{chapete}{f.
terme d'anat.
`membrane qui enveloppe un organe'}{} forte\wdx{fort}{adj. 1\hoch{o}
`qui résiste; fort (de
choses)'}{forte \emph{f.sg.}}
de
panicles
que Galie\emph{n}\adx{Galien}{}{} \text{appelle}\fnb{Am
Zeilenrand nachgetragen.}/
p\emph{er}icardiu\emph{m}\wdx{pericardium}{mlt.
terme d'anat.
`membrane qui enveloppe le c\oe ur; péricarde'}{}, a laquelle
descendent
nerfz ainsi que aux aultres
antrailles\wdx{entrailles}{f.pl.
terme d'anat.
`organes enfermés dans
l'abdomen de l'homme ou des
animaux; intestins'}{antrailles}
de dedans\wdx{dedans}{adv. `à l'intérieur'}{\textbf{de dedans} \emph{loc.adj. `qui est
situé à l'intérieur'}}. Et le cuer est loié avec le poulmon\wdx{*poumon}{m.
terme d'anat.
`chacun des deux viscères logés
symétriquement dans la cage thoracique qui sont les
organes de la respiration, aussi l'ensemble
des deux; poumon'}{poulmon}
et
soustenus\wdx{*sostenir}{v.tr. `tenir (qch.)
par-dessous en servant de support ou
d'appui; soutenir'}{} et fermez par
mediastinu\emph{m}\wdx{mediastinum}{lt.
terme d'anat.
`région du thorax qui sépare la face interne des
poumons; médiastin'}{}.
Et pour ce appart il que il ha
colligance\wdx{colligance}{f. 1\hoch{o}
`force qui maintient réunis les éléments d'un système matériel;
liaison'}{}
 a tous les
me\emph{m}bres. Et si appert
que il est de si grant\wdx{grant}{adj. 3\hoch{o}
dans
l'ordre qualitatif, non quantifiable `qui est d'un
degré supérieur à la moyenne en ce qui concerne la
qualité, l'intensité, l'importance'}{grant
\emph{f.sg.}} digneté\wdx{*dignité}{f. `fonction ou
charge qui donne un rang éminent à
qn ou qch.'}{digneté} que
il ne soustient\wdx{*sostenir}{v.tr. `tenir (qch.)
par-dessous en servant de support ou
d'appui; soutenir'}{} pas
longuemant\wdx{*longement}{adv. `pendant un long
temps'}{longuemant}
passions\wdx{passïon}{f. `souffrance physique'}{}.
\pend
\pstart
Et sur le cuer,
pour le reffroidir\wdx{*refroidir}{v.tr. `rendre plus
froid'}{reffroidir \emph{inf.}},
vollete\wdx{*voleter}{v.intr. `s'agiter d'un
mouvement
semblable à celui des ailes'}{vollete \emph{3.p.sg.
ind.prés.}} le poulmon qui est de substance molle\wdx{mol}{adj.
`qui cède facilement à la pression, au toucher; mou'}{molle \emph{f.sg.}},
rare\wdx{*rer}{adj. `qui n'est pas
dense'}{rar}
ou clere\wdx{cler}{adj. `qui n'est pas dense'}{} et
spongieuse\wdx{*spongios}{adj. `dont
la structure ressemble à celle de l'éponge;
spongieux'}{spongieuse
\emph{f.sg.}}
et
blanche\wdx{blanc}{adj. `qui est de
la couleur de la neige; blanc'}{}. Dedans
lequel poulmon sont trois
manieres\wdx{maniere}{f. 1\hoch{o} `nature propre à
plusieurs personnes ou choses, qui permet de les
considérer comme appartenant à une catégorie
distincte'}{}
de rainseaulx\wdx{*raincel}{m.
terme d'anat. `subdivision d'un vaisseau
sanguin'}{rainseaulx \emph{pl.}}
ou de rames\wdx{rame}{m. et f. terme
d'anat. `subdivision
d'un vaisseau sanguin'}{}, c'est assavoir la
rame\wdx{rame}{m. et f. terme d'anat. `subdivision
d'un vaisseau sanguin'}{} de la vaine
arteriale\wdx{*veine arteriale}{f. terme d'anat.
`vaisseau sanguin
qui s'étend du ventricule droit du c\oe ur au
poumon'}{vaine arteriale} qui naist, si co\emph{m}me dit
\text{est}\fnb{Über der Zeile nachgetragen.}/, du
dextre\wdx{*destre}{adj. `qui est du côté droit;
droit'}{dextre} ventrisson\wdx{ventrisson}{m. terme d'anat.
`cavité particulière à certains organes; ventricule'
(dans le cerveau, dans le c\oe ur)}{}
%
[29r\hoch{o}b]
du cuer, et ung rame\wdx{rame}{m. et f.
terme d'anat. `subdivision d'un vaisseau sanguin'}{}
de la
vaine arteriale\wdx{*veine arteriale}{f. terme d'anat.
`vaisseau sanguin
qui s'étend du ventricule droit du c\oe ur au
poumon'}{vaine
arteriale} qui vient du
\text{senextre}\fnb{Zweites \emph{e} über der Zeile nachgetragen.}/%
\wdx{senestre}{m. et f. `le côté
gauche'}{senextre}, et, avec ce, aultres
rames\wdx{rame}{m. et f. terme d'anat. `subdivision
d'un vaisseau sanguin'}{} de l'artere
trachee\wdx{trachee artere}{f.
terme d'anat.
`portion du conduit aérifère comprise entre
l'extrémité inférieure du larynx et l'origine des
bronches; trachée'}{artere trachee}
qui portent\wdx{porter}{v.tr.
`déplacer (qch.)
d'un lieu à un autre en le menant avec soi;
transporter'}{}
au poulmon l'air pour
le cuer, lesquelx vaisseaux\wdx{vaissel}{m.
2\hoch{o} terme d'anat. `organe
ou canal du corps qui contient ou dans lequel circule
un liquide organique'}{vaisseaux
\emph{pl.}} se espandent\wdx{espandre}{v.pron.
`s'étendre'}{} par tout
le poulmon. Le poulmon ha .v. lobes\wdx{lobe}{m.
terme d'anat. `partie arrondie et saillante de
divers organes; lobe'}{} ou
.v. pe\emph{n}ne\wdx{*pene}{f.
terme d'anat. `partie arrondie et saillante de divers
organes; lobe'}{penne}: deux
en la partie senextre\wdx{senestre}{adj. `qui est du
côté gauche; gauche'}{senextre},
et trois en la dextre\wdx{*destre}{adj. `qui est
du côté droit; droit'}{dextre}.
\pend
\pstart
Et
darrier\wdx{derrier}{prép. `en arrière de'}{darrier}
le poulmon, vers la
.v.\hoch{e}
spondille, passe le mery\wdx{meri}{m.
terme d'anat.
`canal musculo-membraneux qui va
du pharynx à l'estomac auquel il conduit
les aliments; \oe sophage'}{mery}
ou le
ysophagus\wdx{*isophagus}{mlt.
terme
d'anat. `canal musculo-membraneux qui va
du pharynx à l'estomac auquel il conduit
les aliments; \oe sophage'}{ysophagus},
desqueulx dit
est si devant, et passe par la la vaine
concave\wdx{*veine concave}{f. terme
d'anat. `vaisseau sanguin qui ramène le sang
nutritif du foie à tout le corps'}{vaine
concave}
ascendent\wdx{ascendant}{adj. terme
d'anat.
`qui monte (surtout en parlant d'un
vaisseau sanguin)'}{ascendent},
de laquelle nous dirons cy aprés, et passent
toutes deux par le diafragme\wdx{diafragme}{m.
terme d'anat. `muscle large et
mince qui sépare le thorax de l'abdomen;
diaphragme'}{}. Et aussi y passe par
la la vaine aborchi\wdx{*veine aborthi}{f.
terme d'anat. `vaisseau sanguin qui naisse au
c\oe ur'}{vaine
aborchi} qui monte\wdx{monter}{v.tr.indir. `se
déplacer dans un mouvement de bas en
haut'}{}
du cuer
amont\wdx{amont}{adv. `vers le haut'}{},
et toutes ces choses cy, avec la
trachee\wdx{trachee}{f.
terme d'anat.
`portion du conduit aérifère comprise entre
l'extrémité inférieure du larynx et l'origine des
bronches; trachée'}{}, elles font
ung tron\wdx{*tronc}{m. terme d'anat. `partie
principale d'une structure organique
prise isolément sans ses
ramifications' (dit surtout
des vaisseaux sanguins)}{tron}
jusques a la goule\wdx{*gole}{f.
`parties antérieures
et latérales du cou; gorge'}{goule} qui est rempli de
panicles et de fors liguemans et
de char
glandelleuse\wdx{*glandulos}{adj. `qui contient des
glandes'}{glandelleuse \emph{f.sg.}}.
En aprés au pis sont trois
panicles: le premier panicle qui coevre par dedans\wdx{dedans}{adv. `à
l'intérieur'}{}
toutes les costes\wdx{coste}{f. terme d'anat. `os plat et courbe du
thorax qui
s'articule sur la colonne vertébrale et le sternum; côte'}{} est
appellé
pleura\wdx{pleura}{gr. terme d'anat. `membrane séreuse située à l'intérieur
de la cavité thoracique et qui tapisse les parois
internes de la cavité
thoracique et couvre la surface des
poumons; plèvre'}{}.
Le second est
appellé mediastinu\emph{m}\wdx{mediastinum}{lt.
terme d'anat.
`région du thorax qui sépare la face interne des
poumons; médiastin'}{}, qui
divise tout le
four\wdx{*for}{m. 2\hoch{o}
`partie du corps humain qui s'étend
des épaules à
l'abdomen et qui contient le c\oe ur et les
poumons; thorax'}{four} ou
l'arche\wdx{arche}{f. terme d'anat.
`partie supériere du
tronc limitée en dessous par le diaphragme, en
arrière et lateralement par les arcs dorsaux et dont
l'intérieur constitue la cavité où sont logés le c\oe ur et les
poumons'}{} en p\emph{ar}tie dextre\wdx{*destre}{adj. `qui
est du côté droit; droit'}{dextre}
et en
senextre\wdx{senestre}{adj. `qui est du
côté gauche; gauche'}{senextre}. Le tiers,
c'est le diafragme\wdx{diafragme}{m.
terme d'anat. `muscle large et
mince qui sépare le thorax de l'abdomen;
diaphragme'}{}
qui divise
et separe\wdx{separer}{v.tr. `mettre à part les unes
des autres des choses, des personnes
réunies; séparer'}{} tous les
me\emph{m}bres
espirituaulx\wdx{*esperituel}{adj. terme de méd.
`qui contient les esprits vitaux, qui a rapport à la
transformation ou à la diffusion des esprits
vitaux'}{espirituaulx
\emph{m.pl.}} des me\emph{m}bres
nutritis\wdx{nutritif}{adj. terme de méd. `qui a
rapport
aux esprits qui contrôlent
l'alimentation, le grandissement et la génération de
l'homme'}{nutritis \emph{m.pl.}}.
Et est composé de pleura\wdx{pleura}{gr. terme
d'anat. `membrane séreuse située à l'intérieur
de la cavité thoracique et qui tapisse les parois
internes de la cavité
thoracique et couvre la surface des
poumons; plèvre'}{} et de
ciphac\wdx{sifac}{m. terme d'anat. `membrane séreuse qui
recouvre les organes contenus dans
la cavité abdominale et pelvienne, à l'exception de
l'ovaire; péritoine viscéral'}{ciphac} et d'ung
%
[29v\hoch{o}a]
panicle cordeux\wdx{cordeux}{adj. `qui est fait de
tendons'}{} au milieu -- qui est nez des
nerfz qui des spondilles sont
envoiés\wdx{envoiier}{v.tr. `faire
aller qn ou qch. (quelque part)' (le sujet n'étant
pas personnel)}{envoié \emph{p.p.}} -- et de p\emph{ar}ties
charneuses\wdx{*charnel}{adj. `qui est
essentiellement constitué de
chair'}{charneuses \emph{f.pl.}}, maiemant\wdx{*maismement}{adv. `plus que tout
autre chose; surtout'}{maiemant}
de pres
des costes\wdx{coste}{f. terme d'anat.
`os plat et
courbe du thorax qui s'articule sur la colonne vertébrale et le
sternum; côte'}{}, pour il
appert que c'est ung muscule duquel la
op\emph{er}acion\wdx{operacïon}{f.
`action d'un pouvoir, d'une fonction,
d'un organe qui produit un effet selon sa
nature; opération'}{operacion} -- c'est pour
elever\wdx{eslever}{v.tr.
`mettre ou porter (qch.)
plus haut'}{elever \emph{inf.}} et
attraire\wdx{*atraire}{v.tr.
1\hoch{o}
`amener (qn, qch.) vers soi ou quelque
part'}{attraire \emph{inf.}} l'air a son
aide\wdx{aide}{f. 2\hoch{o}
`caractère de ce qui est
utile, qui satisfait un besoin; utilité'}{}
-- vault
a expellir\wdx{expellir}{v.tr. terme de méd. `faire
évacuer (qch.) de l'organisme; expulser'}{} les sup\emph{er}fluités\wdx{superfluité}{f.
terme de méd.
`sécrétion abondante du corps'}{}.
\pend
%
% \memorybreak
%
\pstartueber
Le .vj.\hoch{e} chappitre parlle du
ventre\wdx{ventre}{m. 1\hoch{o} `partie antérieure du
tronc formant une
cavité qui contient l'estomac et les intestins'}{} et
de la partie d'icelluy.
\pendueber
%
% \memorybreak
%
\pstart
LE VENTRE\wdx{ventre}{m. 1\hoch{o} `partie antérieure
du tronc formant une
cavité qui contient l'estomac et les intestins'}{}, quant est de
present\wdx{present}{m.}{\textbf{de present}
\emph{loc.adv. `au moment où l'on parle'}}, on
le prent pour deux parties: premier,
par une partie q\emph{ue} en arabique\wdx{arabique}{adj.
`qui appartient, est
relatif à l'Arabie et ses habitants'}{}
translacion\wdx{translacïon}{f.
`traduction d'une langue dans une
autre; traduction'}{translacion} on
\text{apele}\fnb{Ms. \emph{aple}.}/\wdx{apeler}{v.tr. `donner un nom à qn, qch.; appeler'}{}
\flq stomac\emph{us}\frq\wdx{*stomachus}{lt.
`organe de l'appareil
digestif qui reçoit les aliments;
estomac'}{stomacus}: estomac\wdx{estomac}{m.
terme d'anat.
`organe de l'appareil
digestif qui reçoit les aliments; estomac'}{}.
Et \flq stomacus\frq\wdx{*stomachus}{lt.
`organe de l'appareil
digestif qui reçoit les aliments; estomac'}{stomacus}
en greke\wdx{grec}{adj. `qui appartient, est
relatif à la Grèce et ses habitants'}{grek}
translacion\wdx{translacïon}{f.
`traduction d'une langue dans une
autre; traduction'}{translacion}
est apellé
\flq mery\frq\wdx{meri}{m.
terme d'anat.
`canal musculo-membraneux qui va
du pharynx à l'estomac auquel il conduit
les aliments; \oe sophage'}{mery}
ou \flq ysophagus\frq\wdx{*isophagus}{mlt.
terme
d'anat. `canal musculo-membraneux qui va
du pharynx à l'estomac auquel il conduit
les aliments; \oe sophage'}{ysophagus},
car
\flq stomacus\frq\wdx{*stomachus}{lt.
`organe de l'appareil
digestif qui reçoit les aliments; estomac'}{stomacus}
en arabic\wdx{*arabique}{m. `la langue
parlée par les arabes'}{arabic} est appellé
\flq venter\frq\wdx{venter}{lt.
`partie antérieure du tronc formant une
cavité qui contient l'estomac et les intestins'}{}:
ventre\wdx{ventre}{m. 1\hoch{o} `partie antérieure du
tronc formant une
cavité qui contient l'estomac et les intestins'}{}. Secondemant
\text{le ventre}\fnb{Ersetzt voranstehendes gestrichenes
\emph{venter}.}/
est appellé, et \text{plus}\fnb{\emph{l} über der Zeile nachgetragen.}/
pour
toute la region des me\emph{m}bres nutritis. Et einsi nous le
prenons cy et en vollons
parler\wdx{parler}{\textbf{\emph{parler de}} v.tr.indir.
`s'entretenir de; parler de'}{}. Et pour ce, du
ventre\wdx{ventre}{m. 1\hoch{o} `partie antérieure du
tronc formant une
cavité qui contient l'estomac et les intestins'}{} nous voulons
enquerre\wdx{enquerre}{v.tr. `chercher à savoir
(qch.) en examinant ou en interrogeant'}{enquerre \emph{inf.}} les
.ix. choses devant dictes qui sont enquises\wdx{enquerre}{v.tr. `chercher à savoir
(qch.) en examinant ou en interrogeant'}{enquises
\emph{p.p. f.pl.}}
 des autres
me\emph{m}bres, selon Mestre\wdx{*maistre}{m. 3\hoch{o}
appellation devant le prénom en parlant à ou d'une
personne}{Mestre}
Mondin\adx{Mestre
Mundin}{}{Mestre Mondin}. Premier,
de sa posicion\wdx{posicïon}{f. `lieu où quelque chose est placée,
située'}{posicion}
et de son siege\wdx{siege}{m.
`lieu où qch.
réside'}{} en
general\wdx{general}{adj. `qui est valable pour
toute une classe d'êtres ou de choses (par ce qui
est valable à une sous-classe)'}{\textbf{en general}
\emph{loc.adv. `d'un point de vue général'}}
et en tout
il semble que\wdx{sembler}{v.tr.
`avoir des traits communs avec; ressembler'}{\textbf{il
semble que} \emph{v.impers. `il paraît que'}}
il soit dessoubz\wdx{*desoz}{prép. qui
marque la position en bas par rapport à ce qui est en
haut `sous'}{dessoubz} la region des membres
esp\emph{ir}ituelz\wdx{*esperituel}{adj. terme de méd.
`qui contient les esprits vitaux, qui a rapport à la
transformation ou à la diffusion des esprits
vitaux'}{espirituelz
\emph{m.pl.}}. Mais
%
[29v\hoch{o}b]
de son siege\wdx{siege}{m.
`lieu où qch.
réside'}{}
parti\mbox{cu}lier\wdx{*particuler}{adj. `qui
appartient en propre (à qn, à qch., ou à une catégorie
de personnes, de choses); particulier'}{particulier},
y semble\wdx{sembler}{v.tr.
`avoir des traits communs avec; ressembler'}{\textbf{semble que}
\emph{v.impers. `il paraît que'}}
que son
oriffice\wdx{orifice}{m.
`ouverture faisant communiquer un conduit, un organe
avec une structure voisine ou avec l'extérieur;
orifice'}{oriffice}
-- que les anciens\wdx{anciens}{m.pl. `ceux qui ont vécu dans des temps
fort éloignés de nous'}{}
appellerent precordial\wdx{precordial}{adj.
terme d'anat.
`qui a
rapport à la région thoracique située au-devant du
c\oe ur, qui a son siège dans cette région;
précordial'}{}
-- soit vers la fourcelle\wdx{*forcele}{f. 2\hoch{o}
terme d'anat.
`fourche du sternum'}{fourcelle}. Et la partie de l'estomac\wdx{estomac}{m.
terme d'anat.
`organe de l'appareil
digestif qui reçoit les aliments; estomac'}{}
dure\wdx{durer}{v.intr. `s'étendre (dans l'espace)'}{}
depuis\wdx{depuis}{prép. `à partir de
(en parlant de l'espace)'}{} la,
\text{jusques}\fnb{Nachfolgend am
Zeilenbeginn gestrichenes
\emph{ju}.}/ a trois
dois\wdx{doi}{m. 3\hoch{o}
`mesure approximative, équivalent à un travers de
doigt'}{}
pres du nombril\wdx{nombril}{m. `cicatrice
arrondie formant une petite cavité ou une saillie,
placée sur la ligne médiane du ventre des mammifères,
à l'endroit où le cordon ombilical a été sectionné;
nombril'}{}. La partie du nombril\wdx{nombril}{m. `cicatrice
arrondie formant une petite cavité ou une saillie,
placée sur la ligne médiane du ventre des mammifères,
à l'endroit où le cordon ombilical a été sectionné;
nombril'}{}
dure\wdx{durer}{v.intr. `s'étendre (dans l'espace)'}{}
depuis\wdx{depuis}{prép. `à partir de
(en parlant de l'espace)'}{}
le no\emph{m}bril\wdx{nombril}{m. `cicatrice
arrondie formant une petite cavité ou une saillie,
placée sur la ligne médiane du ventre des mammifères,
à l'endroit où le cordon ombilical a été sectionné;
nombril'}{}
en aval\wdx{aval}{adv.}{\textbf{en aval}
\emph{loc.adv. `vers le bas'}}, et au costé ou
au las\wdx{*lez}{m.
`partie qui est à droite
ou à gauche (d'un corps); côté'}{las}, dessoubz
les costes\wdx{coste}{f. terme d'anat.
`os plat
et courbe du thorax qui s'articule sur la colonne vertébrale et le
sternum; côte'}{}, so\emph{n}t
les
ypocondies\wdx{*ipocondie}{f.
terme d'anat.
`chacune des deux parties latérales de la partie
abdominale, situées sous les fausses côtes;
hypocondre'}{ypocondie}, et les
entrailles\wdx{entrailles}{f.pl.
terme d'anat.
`organes enfermés dans
l'abdomen de l'homme ou des
animaux; intestins'}{}
sont sur les hanches\wdx{hanche}{f. `chacune des deux parties
du corps formant saillie au-dessous des flancs,
entre la fesse en arrière et le pli de l'aine
en avant; hanche'}{}. Le no\emph{m}bre\wdx{nombre}{m.
`mot servant à caractériser une pluralité de
choses ou de personnes; nombre'}{}
des parties du ventre\wdx{ventre}{m. 1\hoch{o}
`partie antérieure du tronc
formant une cavité qui contient l'estomac et les intestins'}{}
ne font
anathomie,
ne peullent bien estre veues, se on ne oevre le ventre\wdx{ventre}{m.
1\hoch{o} `partie antérieure du tronc
formant une cavité qui contient l'estomac et les intestins'}{}
de
long\wdx{lonc}{adv.}{\textbf{de lonc}
\emph{loc.adv.
 `dans le sens de la longueur'} de long} ou de
travers\wdx{travers}{adv.}{\textbf{de travers}
\emph{loc.adv. `dans une direction transversale'}},
si co\emph{m}me dit est. Et quant
il s\emph{er}a ainsi ouvert\wdx{ovrir}{v.tr. `faire que ce
qui était fermé ne
le soit plus; ouvrir'}{ouvert \emph{p.p.}}, on doit
considerer\wdx{considerer}{v.tr.
`regarder (qch.) attentivement;
considérer'}{}
les parties
contenans et les parties contenues.
\pend
\pstart
Les parties
co\emph{n}tenans sont en la partie de devant\wdx{devant}{adv.
1\hoch{o}
`au côté du
visage, à la face'}{\textbf{de devant}
\emph{loc.adj. `qui est situé au côté du visage, de la
face'}}, c'est assavoir
myrac\wdx{mirac}{m.
terme d'anat. `membrane séreuse
qui tapisse les parois
intérieures de la cavité abdominale et pelvienne;
péritoine pariétal'}{myrac} et
siphac\wdx{sifac}{m. terme d'anat. `membrane séreuse qui
recouvre les organes contenus dans
la cavité abdominale et pelvienne, à l'exception de
l'ovaire; péritoine viscéral'}{siphac}.
Et en la partie de \text{darrier}\fnb{Ms.
\emph{darrie}.}/\wdx{derrier}{adv.
`du côté opposé au
visage, à la face'}{\textbf{de derrier}
\emph{loc.adj.
`qui est situé au côté opposé au visage, à la
face'} de darrier} sont les
.v. spondilles des reins\wdx{rein}{m. 2\hoch{o} au
plur.
`la partie inférieure du dos au niveau des
vertèbres lombaires'}{}
et la char qui est
sus\wdx{sus}{adv. de lieu `à la face supérieure;
dessus'}{}. Mirac\wdx{mirac}{m.
terme d'anat. `membrane séreuse
qui tapisse les parois
intérieures de la cavité abdominale et pelvienne;
péritoine pariétal'}{} est composé de
\text{.iiij.}\fnb{\emph{.iiij.} über der
Zeile ersetzt expungiertes \emph{trois}.}/
parties, c'est assavoir de cuir, de
graisse\wdx{*craisse}{f. `substance onctueuse, de fusion facile,
répartie en diverses parties du corps de l'homme et des mammifères;
graisse'}{graisse}, de
pani\-cle charnoux\wdx{*charnel}{adj. `qui
est essentiellement constitué de chair'}{charnoux
\emph{m.sg.}} et de muscules, desquelz viennent
les cordes. Ciphac\wdx{sifac}{m. terme d'anat. `membrane séreuse qui
recouvre les organes contenus dans
la cavité abdominale et pelvienne, à l'exception de
l'ovaire; péritoine viscéral'}{ciphac}
n'est aultre
chose que ung panicule\wdx{*pannicle}{m. terme d'anat.
`couche de tissu musculaire ou cellulaire qui recouvre
une structure organique du corps humain (un
organe, un os, une articulation, un
muscle, etc.)'}{panicule}
par dedans\wdx{dedans}{adv. `à
l'intérieur'}{\textbf{par dedans} \emph{loc.adj.
`qui est situé à l'intérieur'}} qui se
adhert\wdx{adherer}{v.pron. `s'attacher à'}{adhert
\emph{3.p.sg. ind.prés.}} au dit mirac\wdx{mirac}{m.
terme d'anat. `membrane séreuse
qui tapisse les parois
intérieures de la cavité abdominale et pelvienne;
péritoine pariétal'}{}.
Et ainsi
appert la difference\wdx{*diference}{f. `caractère ou
ensemble de caractères
qui distingue une chose d'une autre, un être
d'un autre'}{difference} entre mirac\wdx{mirac}{m.
terme d'anat. `membrane séreuse
qui tapisse les parois
intérieures de la cavité abdominale et pelvienne;
péritoine pariétal'}{}
et
ciphac\wdx{sifac}{m. terme d'anat. `membrane séreuse qui
recouvre les organes contenus dans
la cavité abdominale et pelvienne, à l'exception de
l'ovaire; péritoine viscéral'}{ciphac}. Les
parties \text{de dedans}\fnb{Ms. \emph{de dedas}.}/ le
ventre\wdx{ventre}{m. 1\hoch{o}
`partie antérieure du tronc
formant une cavité qui contient l'estomac et les intestins'}{}
sont
.vij.:
premier y est zirbus\wdx{zirbus}{m.
terme d'anat.
`repli du péritoine qui relie entre eux
les organes abdominaux; épiploon'}{}, et aprés
les intestines\wdx{intestines}{f.pl. terme d'anat.
`conduit digestif qui s'étend depuis l'estomac à
l'anus, comprenant le duodénum, le jéjunum, l'iléon,
le c\ae cum, le côlon et le
rectum, aussi chacune de ses parties; intestin'}{},
et aprés l'estomac, le foye,
%
[30r\hoch{o}a]
l'esplein\wdx{esplein}{m. terme d'anat.
`organe
lymphoïde situé sous la partie gauche du
diaphragme; rate'}{}, le mesentere\wdx{mesentere}{m.
terme d'anat. `repli du péritoine qui relie le
jéjunum et l'iléon à la paroi abdominale postérieure;
mésentère'}{} et les reins\wdx{rein}{m. 1\hoch{o}
terme d'anat. `chacun des deux organes
sécréteurs
glandulaires situés symétriquement dans les fosses
lombaires et qui élaborent l'urine'}{}, car
de la vescie\wdx{vessie}{f. terme d'anat. `réservoir
musculo-membraneux dans lequel s'accumule
l'urine qui arrive des reins par les
uretères; vessie'}{vescie} et
de la matrice\wdx{matrice}{f. `organe situé dans la
cavité pelvienne destiné à contenir l'\oe uf
fécondé jusqu'à son complet développement; utérus'}{}
s\emph{er}a dit en l'anathomie
des hanches\wdx{hanche}{f. `chacune des deux parties
du corps formant saillie au-dessous des flancs,
entre la fesse en arrière et le pli de l'aine
en avant; hanche'}{}. Et nous fault
dire\wdx{dire}{v.tr.indir. \textbf{\emph{dire de}} `parler de'}{}
d'ung
chescum, l'um
aprés l'autre. Premier du cuir, de la
graisse\wdx{*craisse}{f. `substance onctueuse, de fusion facile,
répartie en diverses parties du corps de l'homme et des mammifères;
graisse'}{graisse} et du
panicle charneux\wdx{*charnel}{adj. `qui est
essentiellement constitué de
chair'}{charneux \emph{m.sg.}}, chescum
scest\wdx{savoir}{v.tr.
`avoir présent à l'esprit
(un objet de pensée qu'on identifie et qu'on tient
pour réel); savoir'}{scest \emph{3.p.sg. ind.prés.}}
que c'est. Les muscules sont
creé\wdx{*crïer}{v.tr.
`donner l'être, la vie, l'existence à'}{creé
\emph{p.p.}} ou ventre pour
enforcir\wdx{enforcir}{v.tr.
`rendre (qn, qch.) plus fort'}{} et
conforter\wdx{conforter}{v.tr. `donner des forces
physiques à (qn, qch.)'}{} le ventre, et avec ce
il aide\emph{n}t\wdx{aidier}{v.tr.
`appuyer (qn ou qch.) en apportant son
aide'}{aident \emph{3.p.pl. ind.prés.}}
les aultres me\emph{m}mbres\wdx{membre}{m.
terme d'anat. `chacune des parties du
corps humain ou animal remplissant une fonction déterminée'}{memmbre} a
expellir\wdx{expellir}{v.tr. terme de méd. `faire
évacuer (qch.) de l'organisme; expulser'}{}
leur superfluités\wdx{superfluité}{f. terme de méd.
`sécrétion abondante du corps'}{},
et sont .viij. muscules, selon Galien\adx{Galien}{}{}
ou quart livre
de \flq Utilitate\frq\ et ou .vj.\hoch{e} de
\flq Terapeutique\frq\wdx{*therapeutique}{f.
terme de méd.
`partie de la médecine qui étudie et
met en application les moyens propres à guerir et à
soulager les malades'}{terapeutique}, c'est
assavoir deux vont de long, qui
procedent\wdx{proceder}{v.tr.indir.
`avoir son origine
dans; provenir'}{procedent \emph{3.p.pl. ind.prés.}}
du clipee\wdx{clipee}{m. `ce dont la forme ou la
fontion ressemble à celle d'un bouclier'}{} de
l'estomac\wdx{estomac}{m. terme d'anat.
`organe de l'appareil
digestif qui reçoit les aliments; estomac'}{}
jusques aux
os du pis\wdx{*piz}{m. terme d'anat.
`partie du corps humain qui s'étend
des épaules à
l'abdomen et qui contient le c\oe ur et les
poumons; thorax'}{}. Et deux
q\emph{ue} vont de travers\wdx{travers}{prép.}{\textbf{de travers}
\emph{loc.prép. `dans une direction transversale de'}}
ou du
large\wdx{large}{adv.}{\textbf{du large}
\emph{loc.adv. `à travers'}} du ventre et se
intercequent\wdx{*intersequer}{v.pron.
`se mettre en travers l'un sur l'autre; se
croiser'}{intercequent \emph{3.p.pl. ind.prés.}}
ou entrechangent\wdx{entrechangier}{v.intr.
`devenir échangé'}{entrechangent \emph{3.p.pl.
ind.prés.}} parmy le ventre a
angles \text{drois}\fnb{\emph{i} über der Zeile nachgetragen.}/%
\wdx{angle droit}{m. `saillant ou rentrant formé par deux lignes
ou deux surfaces qui se rencontrent et dont l'un
est perpendiculaire
à l'autre'}{}
et viennent du
\text{dos}\fnb{Ms. \emph{dors}.}/\wdx{dos}{m.
`partie du corps de l'homme
qui s'étend des épaules jusqu'aux reins, de chaque
côté de la colonne vertebrale; dos'}{}
par
sur le ventre. Et les aultres quatre muscules sont
traverssans\wdx{*traversant}{adj. `qui
traverse une chose en la coupant
perpendiculairement à sa plus grande
dimension; transversal'}{traverssans
\emph{m.pl.}}, desqueulx les deux naissent des costes
du las\wdx{*lez}{m.
`partie qui est à droite
ou à gauche
 (d'un corps); côté'}{las} dextre\wdx{*destre}{adj. `qui
est du côté droit; droit'}{dextre} et vont a la
senestre\wdx{senestre}{m. et f.
`le côté
gauche'}{} des
os
des hanches\wdx{hanche}{f. `chacune des deux parties
du corps formant saillie au-dessous des flancs,
entre la fesse en arrière et le pli de l'aine
en avant; hanche'}{}
et du penil\wdx{penil}{m. terme d'anat. `eminence large
et arrondie, située au-devant du pubis'}{}. Et
les aultres deux
naisse\emph{n}t des costes\wdx{coste}{f. terme d'anat. `os plat et courbe
du thorax qui
s'articule sur la colonne vertébrale et le sternum; côte'}{}
senextres\wdx{senestre}{adj. `qui est du
côté gauche; gauche'}{senextre} et
vont
a dextre\wdx{*destre}{f. `le côté droit;
droite'}{\textbf{a destre}
\emph{loc.adv. `sur le côté droit'} a dextre}
des os devant ditz, et se
croisent\wdx{*croisier}{v.pron. `être ou se mettre en
travers l'un sur l'autre'}{croisent \emph{3.p.pl.
ind.prés.}} parmy le ventre
selon la fourme d'une lectre\wdx{*letre}{f. `signe
graphique qui représente une des 26 parties qui
constituent la langue écrite'}{lectre} qui se
appelle
\flq x\frq . Donc,
quant ces .viij. muscules seront
eslevés\wdx{eslever}{v.tr.
`mettre ou porter (qch.)
plus haut'}{eslevé \emph{p.p.}} et
tranchés\wdx{*trenchier}{v.tr.
`séparer (une chose
en parties, deux choses unies) d'une manière nette, au
moyen d'un instrument dur et fin; trancher'}{tranché
\emph{p.p.}}, lors
tu pourras voir\wdx{*vëoir}{v.tr. `percevoir (qch.)
par le sens de la vue'}{voir \emph{inf.}}
siphac\wdx{sifac}{m. terme d'anat. `membrane séreuse qui
recouvre les organes contenus dans
la cavité abdominale et pelvienne, à l'exception de
l'ovaire; péritoine viscéral'}{siphac} que
Galien\adx{Galien}{}{}
appelle p\emph{er}itoneu\emph{m}\wdx{peritoneum}{lt.
terme d'anat. `membrane séreuse
qui tapisse les parois
intérieures de la cavité abdominale et pelvienne et
qui recouvre les organes contenus dans
la cavité abdominale et pelvienne, à l'exception de
l'ovaire'; péritoine'}{},
qui est dit de \flq peri\frq\ en grec\wdx{grec}{m.
`la langue parlée par les grecs'}{} qui
signiffie\wdx{*senefiier}{v.tr.
`avoir le
sens de; signifier'}{signiffie \emph{3.p.sg.
ind.prés.}}
\flq circu\emph{m}\frq\ en latin\wdx{latin}{m. `la
langue qui appartient à la culture latine'}{}:
%
[30r\hoch{o}b]
\emph{tendo tendis quasi
cir\emph{con}tendens}, c'est a dire que le dit
ciphac\wdx{sifac}{m. terme d'anat. `membrane séreuse qui
recouvre les organes contenus dans
la cavité abdominale et pelvienne, à l'exception de
l'ovaire; péritoine viscéral'}{ciphac}
est tendu\wdx{tendre}{v.tr. `tirer sur une chose
en la
rendant droite; tendre'}{tendu \emph{p.p.}} de sa et
de la, tout environ\wdx{environ}{adv. 1\hoch{o} `dans
l'espace environnant; alentour'}{}.
Et est ciphac\wdx{sifac}{m. terme d'anat. `membrane séreuse qui
recouvre les organes contenus dans
la cavité abdominale et pelvienne, à l'exception de
l'ovaire; péritoine viscéral'}{siphac}
ung panicle nerveux\wdx{*nervos}{adj. terme d'anat. `qui a le caractère des nerfs
ou des tendons'}{nerveux \emph{m.sg.}},
subtil\wdx{subtil}{adj. 1\hoch{o}
`qui a peu
d'épaisseur; mince (d'un tissu organique, textile,
etc.)'}{} et dur\wdx{dur}{adj. `qui résiste
à la pression, qui ne se
laisse pas déformer facilement'}{},
afin que il
deffende que les muscules
ne compriment\wdx{comprimer}{v.tr. `exercer une
pression sur (qch.) en diminuer le volume;
comprimer'}{} les me\emph{m}bres
\text{na\-turelz}\fnb{Ms.
\emph{natuelz}.}/\wdx{naturel}{adj.
1\hoch{o} terme de méd.
`qui contient les esprits
transformés dans le foie qui
contrôlent l'alimentation, le grandissement et la
génération de l'homme, qui a rapport
à la
transformation ou à la diffusion de ces
esprits'}{naturelz \emph{m.pl.}}
 et qui se
puissent dillacter\wdx{dilater}{v.pron. `augmenter de volume; s'étendre'}{dillacter} et
estendre\wdx{estendre}{v.pron.
`augmenter en
longueur ou en largeur'}{} a la nature des
autres
me\emph{m}bres, et affin qu\wdx{afin que}{loc.conj. qui marque l'intention,
 le but `pour que'}{affin
que}'ilz ne ro\emph{m}pent\wdx{rompre}{v.intr.
`se séparer en deux ou
plusieurs parties par un effort brusque; rompre'}{}
legieremant\wdx{*legierement}{adv. `sans effort;
facilement'}{legieremant} et
que les choses de dedans ne issent hors -- si co\emph{m}me il
advient
a ceulx qui sont crevés\wdx{crever}{v.tr.
`faire éclater (qch.) par excès de
tension; crever'}{crevé \emph{p.p.}} --,
et afin qu'il liast\wdx{*liier}{v.tr. `entourer plusieurs choses avec un lien pour
qu'elles tiennent ensemble'}{liast
\emph{3.p.sg. subj.prés.}}
les intestines\wdx{intestines}{f.pl. terme d'anat.
`conduit digestif qui s'étend depuis l'estomac à
l'anus, comprenant le duodénum, le jéjunum, l'iléon,
le c\ae cum, le côlon et le
rectum, aussi chacune de ses parties; intestin'}{}
\text{du dos}\fnb{Ms. \emph{du du
dos}.}/ et qu'il aide\wdx{aidier}{v.tr.
`appuyer (qn ou qch.) en apportant son aide'}{aide
\emph{3.p.sg. ind.prés.}} les me\emph{m}bres
ordo\emph{n}nés\wdx{*ordener}{v.tr.
2\hoch{o}
`établir (qn, qch.) pour une foncion'}{ordonné
\emph{p.p.}}
pour expellir\wdx{expellir}{v.tr. terme de méd. `faire
évacuer (qch.) de l'organisme; expulser'}{}
ce qui appertient\wdx{*apartenir}{v.tr.indir. `être
convenable pour; convenir'}{appertient
\emph{3.p.sg. ind.prés.}} a
expellir\wdx{expellir}{v.tr. terme de méd. `faire
évacuer (qch.) de l'organisme; expulser'}{}. Et ainsi appart la
disposicion\wdx{disposicïon}{f.
1\hoch{o} `action de mettre dans un certain ordre, le
résultat de cette action'}{disposicion} des parties
qui sont
contenues dedens le ventre. Et pour ce appert ce q\emph{ue}
Galien\adx{Galien}{}{} disoit ou .vj.\hoch{e} de
\flq Terapeutique\frq\wdx{*therapeutique}{f.
terme de méd.
`partie de la médecine qui étudie et
met en application les moyens propres à guerir et à
soulager les malades'}{terapeutique},
que le plaies\wdx{plaie}{f.
`ouverture
dans les chairs, les tissus, due à une cause
externe (traumatisme, intervention chirurgicale) et
présentant une solution de continuité des téguments;
plaie'}{}
et les incisions\wdx{incision}{f. `action de fendre,
de couper avec un instrument tranchant, son
résultat (surtout en médecine)'}{}
qui so\emph{n}t fetes ou ventre sont plus
perilleuses\wdx{*perillos}{adj.
`qui constitue un danger, présente du
danger; dangereux'}{perilleuses \emph{f.pl.}}
et plus difficilles\wdx{*dificile}{adj.
`qui
n'est pas facile; difficile'}{difficille}
a
saigner\wdx{saner}{v.tr. `délivrer d'un mal physique;
guérir'}{saigner \emph{inf.}} que celles qui sont
faites es costés ne es parties
voisines\wdx{voisin}{adj.
`qui
est à côté; voisin'}{},
car elles sont plus
tractables\wdx{*traitable}{adj. `qui peut être
traité médicalement'}{tractable} que les aultres.
Et se
app\emph{er}t aussi q\emph{ue} les
plaies penetrans du
\text{ve\emph{n}tre}\fnb{Nachfolgend
gestrichenes
\emph{q}.}/\wdx{plaie}{f.
`ouverture
dans les chairs, les tissus, due à une cause
externe (traumatisme, intervention chirurgicale) et
présentant une solution de continuité des téguments;
plaie'}{}\wdx{penetrer}{v.intr. `entrer profondément
en passant à travers ce qui fait obstacle'}{}\wdx{*aparoir}{v.pron.
`se montrer aux yeux; se
manifester'}{appert \emph{3.p.sg. ind.prés.}},
se
on ne keut\wdx{*cosdre}{v.tr. `assembler au
moyen d'un fil passé par une aiguille;
coudre'}{keut
\emph{3.p.sg.
ind.prés.}} ciphac\wdx{sifac}{m. terme d'anat. `membrane séreuse qui
recouvre les organes contenus dans
la cavité abdominale et pelvienne, à l'exception de
l'ovaire; péritoine viscéral'}{siphac}
avec mirac\wdx{mirac}{m.
terme d'anat. `membrane séreuse
qui tapisse les parois
intérieures de la cavité abdominale et pelvienne;
péritoine pariétal'}{},
on ne fera pas bo\emph{n}ne
scarnacion\wdx{*carnation}{f. `action de produire de
la chair, de se faire chair'}{scarnacion}.
\pend
\pstart
Aprés il
nous co\emph{n}vient veoir
des parties qui sont contenues dedans le ventre. Et
premier appert
zirb\emph{us}\wdx{zirbus}{m.
terme d'anat.
`repli du péritoine qui relie entre eux
les organes abdominaux; épiploon'}{}
que Galien\adx{Galien}{}{} appelle
omentu\emph{m}\wdx{omentum}{lt.
terme d'anat.
`repli du péritoine qui relie entre eux
les organes abdominaux; épiploon'}{} ou
epiplou\emph{n}\wdx{*epiploon}{m.
terme d'anat.
`repli du péritoine qui relie entre eux
les organes abdominaux; épiploon'}{epiploun}, et est
dit de
\flq epi\frq\ en grec\wdx{grec}{m. `la langue parlée par
les grecs'}{} qui signiffie\wdx{*senefiier}{v.tr.
`avoir le
sens de; signifier'}{signiffie \emph{3.p.sg.
ind.prés.}}
\flq supra\frq\ en latin\wdx{latin}{m. `la langue
qui appartient à la culture latine'}{}, et
\flq plou\emph{n}\frq\ qui signifie\wdx{*senefiier}{v.tr.
`avoir le
sens de; signifier'}{signifie \emph{3.p.sg.
ind.prés.}}
\flq eminere quasi
eminens\frq , c'est
%
[30v\hoch{o}a]
a dire que le dit zirbus\wdx{zirbus}{m.
terme d'anat.
`repli du péritoine qui relie entre eux
les organes abdominaux; épiploon'}{}
appert sur tous
les aultres me\emph{m}bres
dedans\wdx{dedans}{adv. `à
l'intérieur'}{}. Et est zirbus\wdx{zirbus}{m.
terme d'anat.
`repli du péritoine qui relie entre eux
les organes abdominaux; épiploon'}{}
ung panicle qui envelloupe\wdx{*envoleper}{v.tr.
`entourer
(qch.) d'une chose souple qui couvre de tous
côtés; envelopper'}{envelloupe \emph{3.p.sg. ind.prés.}}
et coevre l'estomac\wdx{estomac}{m. terme d'anat.
`organe de l'appareil
digestif qui reçoit les aliments; estomac'}{}
et les
intestins\wdx{intestin}{m. terme d'anat.
`conduit digestif qui s'étend depuis l'estomac à
l'anus, comprenant le duodénum, le jéjunum, l'iléon,
le c\ae cum, le côlon et le
rectum, aussi chacune de ses parties; intestin'}{} de deux
tuniques\wdx{tunique}{f.
terme d'anat.
`membrane qui enveloppe
certains organes ou qui constitue la paroi d'un
organ ou d'un vaisseau'}{}
espesses\wdx{espés}{adj.
`qui est gros considéré
dans son épaisseur; épais'}{espesse f.} et
subtilles\wdx{subtil}{adj.
1\hoch{o}
`qui a peu
d'épaisseur; mince (d'un tissu organique, textile,
etc.)'}{}
qui sont mises\wdx{metre}{v.tr. 1\hoch{o} `placer
(qch.)
dans une position déterminée'}{} l'une sur l'autre. Et
est fait
de pluseurs arteres et vaines et de moult de graisse
\text{ordo\emph{n}né}\fnb{Über der Zeile nachgetragen.}/\wdx{*ordener}{v.tr. 2\hoch{o}
`établir (qn, qch.) pour une foncion'}{ordonné
\emph{p.p.}}
pour
eschauffer\wdx{*eschaufer}{v.tr. `rendre
chaud'}{eschauffer \emph{inf.}} les dis
\text{me\emph{m}bres}\fnb{Ist über der Zeile nachgetragen.}/
 -- si co\emph{m}me il
est escript ou quart livre, ou .xiiij. chappitre du
livre qui se intitule \flq De utilitate
particularu\emph{m}\frq\
--,
duquel la naissance\wdx{naissance}{f. terme d'anat.
`endroit où commence
qch. (en parlant des membres, organes ou structures organiques
du corps)'}{}
est du dos, si co\emph{m}me le dit
peritoneu\emph{m}\wdx{peritoneum}{lt.
terme d'anat. `membrane séreuse
qui tapisse les parois
intérieures de la cavité abdominale et pelvienne et
qui recouvre les organes contenus dans
la cavité abdominale et pelvienne, à l'exception de
l'ovaire'; péritoine'}{}. Pour quoy il
\text{appert}\fnb{Ms. \emph{apper}.}/
que, quant aucune petite\wdx{petit}{adj.
1\hoch{o} dans l'ordre
physique, quantifiable `qui est d'une extension
au-dessous de la moyenne; petit (des choses)'}{}
partie de lui ist
\text{hors}\fnb{Voranstehend gestrichenes
\emph{horsi}.}/ par les plaies\wdx{plaie}{f.
`ouverture
dans les chairs, les tissus, due à une cause
externe (traumatisme, intervention chirurgicale) et
présentant une solution de continuité des téguments;
plaie'}{}
du
ventre, tantost\wdx{tantost}{adv.
`dans un temps prochain, un
proche avenir; tantôt'}{}
legieremant\wdx{*legierement}{adv.
`sans effort; facilement'}{legieremant}
elle se altere\wdx{alterer}{v.pron. `se
transformer'}{}
pour cause de la graisse, et que on la doit
lier\wdx{*liier}{v.tr. `entourer plusieurs choses avec un lien pour
qu'elles tiennent ensemble'}{lier \emph{inf.}} et
no\emph{n} pas coper\wdx{coper}{v.tr.
`diviser (qch.) avec
un instrument tranchant; couper'}{}, et se doubte\wdx{*doter}{v.tr.
`envisager (qn, qch.) comme dangereux et en avoir
peur; craindre'}{doubte \emph{3.p.sg. ind.prés.}} on
emorogie\wdx{*hemorrhagie}{f. terme de méd.
`écoulement de
sang hors d'un vaisseau'}{emorogie}, selon
Galien\adx{Galien}{}{} ou lieu
devant allegué\wdx{*aleguer}{v.tr. `rapporter un
passage, un texte (écrit d'une autorité);
citer'}{allegué \emph{p.p.}}
ou livre de
\flq Terapeutique\frq\wdx{*therapeutique}{f.
terme de méd.
`partie de la médecine qui étudie et
met en application les moyens propres à guerir et à
soulager les malades'}{terapeutique}.
\pend
\pstart
Item,
aprés, pour ce que les intestins\wdx{intestin}{m.
terme d'anat.
`conduit digestif qui s'étend depuis l'estomac à
l'anus, comprenant le duodénum, le jéjunum, l'iléon,
le c\ae cum, le côlon et le
rectum, aussi chacune de ses parties; intestin'}{}
\text{e\emph{m}pechent}\fnb{Voranstehend gestrichenes
\emph{empt}.}/\wdx{*empëechier}{v.tr.
`faire en sorte de ne se produire pas ou
de ne se faire pas; empêcher'}{empechent
\emph{3.p.pl. ind.prés.}}
a veoir la anathomie \text{des autres}\fnb{Ms.
\emph{des au autres}.}/
me\emph{m}bres, y nous
en co\emph{n}vient \text{dire}\fnb{\emph{Nota} am Foliorand.}/%
\wdx{dire}{v.tr.indir. \textbf{\emph{dire de}} `parler de'}{}: donc les
intestins\wdx{intestin}{m. terme d'anat.
`conduit digestif qui s'étend depuis l'estomac à
l'anus, comprenant le duodénum, le jéjunum, l'iléon,
le c\ae cum, le côlon et le
rectum, aussi chacune de ses parties; intestin'}{}, ce sont
vaisseaulx\wdx{vaissel}{m. 2\hoch{o} terme d'anat.
`organe
ou canal du corps qui contient ou dans lequel circule
un liquide organique'}{vaisseaulx \emph{pl.}}
qui sont fais de deux
tuniques\wdx{tunique}{f.
terme d'anat.
`membrane qui enveloppe
certains organes ou qui constitue la paroi d'un
organ ou d'un vaisseau'}{}
pour
p\emph{ar}faire\wdx{parfaire}{v.tr.
`achever, de manière à conduire à
la perfection; parfaire'}{}
la
premiere digestion\wdx{digestion}{f. terme de méd.
`conversion des substances dans le corps
en sucs nécessaires pour l'assimilation et
la désassimilation (des aliments en chyle, le chyle en sang,
etc.)'}{}
et pour rendre le chile\wdx{chile}{m.
terme de méd. `suc
qui est assimilé des aliments par la digestion
et qui est
transporté au foie pour y servir à la formation du
sang'}{} au foye, pour l'aide\wdx{aide}{f. 1\hoch{o}
`action d'aider quelqu'un, concours que l'on prête,
soutien moral ou secours matériel que l'on
apporte; aide'}{} des vaines
miseraques\wdx{*veine mesaraïque}{f.
terme d'anat. `vaisseau sanguin qui est un
prolongement
ramifié de la \flq porte\frq\ et qui
transporte des sucs de l'intestin au foie'}{vaine
miseraque} et pour expellir\wdx{expellir}{v.tr. terme de méd. `faire
évacuer (qch.) de l'organisme; expulser'}{} la sup\emph{er}fluité\wdx{superfluité}{f. terme
de méd.
`sécrétion abondante du corps'}{}
fetale\wdx{fetal}{adj. `qui dégage une odeur
déplaisante; puant'}{}
et puente\wdx{*püant}{adj. `qui dégage une odeur
déplaisante; puant'}{puente
\emph{f.sg.}}. Donc ilz sont
.vj.
intestins, trois grailles\wdx{*intestin
graisle}{m.
terme d'anat.
`partie du tube digestif entre
l'estomac et le c\ae cum, comprenant le duodénum,
le jéjunum et l'iléon, aussi chacune de ses
parties'}{intestin graille} et
trois gros\wdx{intestin gros}{m.
terme d'anat.
`partie du tube
digestif entre
l'iléon et l'anus, comprenant le c\ae cum, le côlon
et le rectum, aussi chacune de ses
parties'}{}, ja ssoit ce que
ilz sont co\emph{n}tinuelz\wdx{continuel}{adj.
`qui n'est pas
interrompu dans l'espace'}{}, toutesvoies
ont ilz diverses\wdx{divers}{adj.
`qui
présente une différence par rapport à une autre
chose ou une autre personne; différent'}{diverses
\emph{f.pl.}} fourmes et divers\wdx{divers}{adj.
`qui
présente une différence par rapport à une autre
chose ou une autre personne; différent'}{divers
\emph{m.pl.}} offices\wdx{*ofice}{m. `fonction
qu'une chose doit remplir dans l'ensemble dont elle
fait partie'}{office} par quoy
%
[30v\hoch{o}b]
ilz sont distingués\wdx{distinguer}{v.tr. `permettre
de reconnaître (une personne ou une chose d'une
autre), en parlant d'une différence constitutive,
d'un trait caractéristique; distinguer'}{distingué
\emph{p.p.}}, desqueulx
la cathologue\wdx{*catalogue}{m.
`liste indicative des pièces qui composent une
collection; catalogue'}{cathologue} et le
no\emph{m}bre\wdx{nombre}{m.
`mot servant à caractériser une pluralité de
choses ou de personnes; nombre'}{}
et le nom\wdx{nom}{m.
`mot
servant à désigner les êtres, les choses
qui appartiennent à une même catégorie
logique'}{}
est mis\wdx{metre}{v.tr. 2\hoch{o} `présenter ou
exposer (un fait, une idée, un ensemble de faits ou
d'idées)'}{} de Galien\adx{Galien}{}{} ou quart
livre, ou .xxvj.\hoch{e} chapp\emph{itre} du livre qui se
intitule \flq De
utilitate particula\emph{rum}\frq . Le premier qui vient
aprés le ventre est appellé fissis\wdx{fissis}{subst.
terme d'anat.
`partie initiale de l'intestin grêle
accolée à la paroi abdominale postérieure et
qui s'étend au jéjunum; duodénum'}{}
ou
portnier\wdx{portenier}{m.
terme d'anat.
`orifice faisant
communiquer l'estomac avec le duodénum, pouvant
désigner aussi, par métonymie, le duodénum'}{portnier} ou
duodene\wdx{*duodenum}{m. terme d'anat.
`partie initiale de l'intestin grêle
accolée à la paroi abdominale postérieure et
qui s'étend au jéjunum; duodénum'}{duodene}.
Le second est appellé jejunu\emph{m}\wdx{jejunum}{m.
terme d'anat. `partie du tube digestif qui fait
suite au duodénum; jéjunum'}{}.
Le tiers est appellé subtil\wdx{subtil}{m. terme
d'anat. `partie du tube digestif entre le jéjunum et
le c\ae cum; iléon'}{}. Le quart orbe\wdx{orbe}{m.
terme d'anat. `partie du
tube digestif entre l'iléon et le côlon; c\ae cum'}{},
le q\emph{ui}nt colon\wdx{colon}{m. terme d'anat.
`portion moyenne du gros intestin comprise entre
le c\ae cum et le rectum; côlon'}{}, le
.vj.\hoch{e} rectu\emph{m}\wdx{rectum}{m. terme d'anat.
`portion terminale du gros intestin, faisant
partie au côlon et s'étendant jusqu'à l'anus;
rectum'}{} ouquel, vers la fin, sont muscules qui
gouvernent\wdx{governer}{v.tr. `diriger (la conduite
de qn,
de qch.), exercer une influence (sur qn, sur qch.)'}{gouverner}
les sup\emph{er}fluités\wdx{superfluité}{f. terme de méd.
`sécrétion abondante du corps'}{}. Et
afin que on voie\wdx{*vëoir}{v.tr. `percevoir (qch.)
par le sens de la vue'}{voie \emph{3.p.sg.
subj.prés.}} mieux la anathomie, il nous co\emph{n}vient
co\emph{m}mencer\wdx{*comencier}{v.tr.indir.
`faire d'abord (qch.)'}{commence
\emph{3.p.sg. ind.prés.}}
au
darnier\wdx{*derrenier}{adj.
`qui vient
après tous les autres, après lequel il n'y a pas
d'autre' (temporel ou spatial)}{darnier}
intestin\wdx{intestin}{m. terme d'anat.
`conduit digestif qui s'étend depuis l'estomac à
l'anus, comprenant le duodénum, le jéjunum, l'iléon,
le c\ae cum, le côlon et le
rectum, aussi chacune de ses parties; intestin'}{}
que on appelle
rectu\emph{m}\wdx{rectum}{m.
terme d'anat.
`portion terminale du gros intestin, faisant
partie au côlon et s'étendant jusqu'à l'anus;
rectum'}{}
ou longaon\wdx{longäon}{m. terme d'anat.
`portion terminale du gros intestin qui
s'étend du côlon jusqu'à l'anus; rectum'}{longaon}. Et
afin que les feces\wdx{feces}{f.pl. terme de méd.
`excréments solides de l'homme, formés des résidus de
la digestion; fèces'}{} ou les
sup\emph{er}fluité\wdx{superfluité}{f.
terme de méd.
`sécrétion abondante du corps'}{}
ne vo\emph{us} empechent\wdx{*empëechier}{v.tr.
`faire en sorte de ne se produire pas ou
de ne se faire pas; empêcher'}{empechent
\emph{3.p.pl. ind.prés.}}, on le
liera\wdx{*liier}{v.tr. `entourer plusieurs choses avec un lien pour
qu'elles tiennent ensemble'}{liera \emph{3.p.sg. fut.}}
vers la partie de dessus\wdx{*desus}{adv. `au côté
supérieur'}{\textbf{de dessus} \emph{loc.adj. `qui
est situé au côté supérieur'}}
en deux lieux et le copera\wdx{coper}{v.tr.
`diviser (qch.) avec
un instrument tranchant; couper'}{}
on au milieu de la ligadure\wdx{ligadure}{f.
`action de réunir, à fixer, à serrer avec un lien
quelconque'}{}, et lairés\wdx{laiier}{v.tr. `ne pas s'occuper
de'}{lairés \emph{2.p.pl. ind.prés.}}
la basse\wdx{bas}{adj. `qui se
trouve à une faible hauteur; bas'}{basse
\emph{f.sg.}} partie
et allés avant, en decharnant\wdx{*descharner}{v.tr.
`ôter la chair de'}{decharnant \emph{p.prés.}}
jusques pres des entrailles\wdx{entrailles}{f.pl.
terme d'anat.
`organes enfermés dans
l'abdomen de l'homme ou des
animaux; intestins'}{};
et la co\emph{m}mence\wdx{*comencier}{v.intr. `entrer
dans son commencement'}{commence
\emph{3.p.sg.
ind.prés.}}
le intestin\wdx{intestin}{m. terme d'anat.
`conduit digestif qui s'étend depuis l'estomac à
l'anus, comprenant le duodénum, le jéjunum, l'iléon,
le c\ae cum, le côlon et le
rectum, aussi chacune de ses parties; intestin'}{}
colon\wdx{colon}{m.
terme d'anat.
`portion moyenne du gros intestin comprise entre
le c\ae cum et le rectum; côlon'}{} qui
est gros\wdx{intestin gros}{m. terme d'anat.
`partie du tube
digestif entre
l'iléon et l'anus, comprenant le c\ae cum, le côlon
et le rectum, aussi chacune de ses
parties'}{} et
ha une celle\wdx{*cele}{f. terme d'anat. `partie
creuse
dans une structure organique; cavité'}{celle} en
laquelle
les sup\emph{er}fluités\wdx{superfluité}{f. terme de méd.
`sécrétion abondante du corps'}{} preingne\emph{n}t
leur fourme\wdx{prendre}{v.tr. 2\hoch{o} `commencer à avoir (un mode d'être)'}{prendre sa
fourme}, et ha bien de long deux
brassés\wdx{*braciee}{f. `longueur de bras'}{brassé}
ou deux toizes\wdx{*toise}{f. `mesure de longueur
valant six pieds, aussi la longueur de six pieds;
toise'}{toize}, et se encline\wdx{encliner}{v.pron.
`se
pencher (de choses)'}{} moult vers les
reins\wdx{rein}{mn 2\hoch{o} au plur.
`la partie inférieure du dos au niveau des
vertèbres lombaires'}{} senextres\wdx{senestre}{adj.
`qui est du côté gauche; gauche'}{senextre}.
Et en montant\wdx{monter}{v.tr.indir. `se déplacer
dans un mouvement de bas en
haut'}{} vers l'esplein\wdx{esplein}{m.
terme d'anat.
`organe
lymphoïde situé sous la partie gauche du
diaphragme; rate'}{}, il se
revolve\wdx{revolver}{v.pron.
`se mettre
en sens inverse ou dans une certaine direction; se
tourner'}{}
vers la
partie de devant\wdx{devant}{adv. 1\hoch{o}
`au côté du
visage, à la face'}{\textbf{de devant}
\emph{loc.adj. `qui est situé au côté du visage, de la
face'}}, a la dextre\wdx{*destre}{f. `le côté droit; droite'}{dextre} de
l'estomac\wdx{estomac}{m. terme d'anat.
`organe de l'appareil
digestif qui reçoit les aliments; estomac'}{}, dessoubz la
tierce pe\emph{n}ne\wdx{*pene}{f.
terme d'anat. `partie arrondie et saillante de divers
organes; lobe'}{penne} du foye, ou il
ressoit une porcion\wdx{porcïon}{f.
`partie d'un tout
homogène qui n'est pas nombrable; portion'}{porcion} de
cole\wdx{cole}{f. terme de méd.
`bile jaune (l'une des quatre
humeurs de l'humorisme')}{} qui le
esmeut\wdx{esmovoir}{v.pron. `pousser (qch.,
qn) à agir'}{esmeut
\emph{3.p.sg. ind.prés.}}
a espellir\wdx{expellir}{v.tr. terme de méd. `faire
évacuer (qch.) de l'organisme; expulser'}{espellir
\emph{inf.}} les sup\emph{er}fluités\wdx{superfluité}{f.
terme de méd.
`sécrétion abondante du corps'}{}.
Et en revolvant\wdx{revolver}{v.pron.
`se mettre
en sens inverse ou dans une certaine direction; se
tourner'}{revolvant \emph{p.prés.}} il
descend aux reins\wdx{rein}{m. 2\hoch{o} au
plur.
`la partie inférieure du dos au niveau des
vertèbres lombaires'}{} dextres\wdx{*destre}{adj.
`qui est du côté droit;
droit'}{dextre}, au terme\wdx{terme}{m. `limite
fixée (dans l'espace)'}{}
de la hanche\wdx{hanche}{f. `chacune des deux parties
du corps formant saillie au-dessous des flancs,
entre la fesse en arrière et le pli de l'aine
en avant; hanche'}{}.
Et la co\emph{m}mence\wdx{*comencier}{v.intr. `entrer
dans son commencement'}{commence
\emph{3.p.sg.
ind.prés.}}
l'autre intestin\wdx{intestin}{m. terme d'anat.
`conduit digestif qui s'étend depuis l'estomac à
l'anus, comprenant le duodénum, le jéjunum, l'iléon,
le c\ae cum, le côlon et le
rectum, aussi chacune de ses parties; intestin'}{}
que on
%
[31r\hoch{o}a]
appelle orbe\wdx{orbe}{m.
terme d'anat. `partie du
tube digestif entre l'iléon et le côlon; c\ae cum'}{} ou
secus\wdx{secus}{lt. terme d'anat. `partie du
tube digestif entre l'iléon et le côlon; c\ae cum'}{}
ou monoculus\wdx{monoculus}{mlt.
terme d'anat. `partie du
tube digestif entre l'iléon et le côlon; c\ae cum'}{}, car
il samble que il n'aye que ung oriffice\wdx{orifice}{m.
`ouverture faisant communiquer un conduit, un organe
avec une structure voisine ou avec l'extérieur;
orifice'}{oriffice}, ja ssoit ce
qu'il
en aye deux, moult procheins\wdx{prochain}{adj.
`qui est proche
(dans l'espace)'}{procheins \emph{m.pl.}}
l'ung de l'autre; par
l'ung oriffice\wdx{orifice}{m.
`ouverture faisant communiquer un conduit, un organe
avec une structure voisine ou avec l'extérieur;
orifice'}{oriffice}
entre\wdx{entrer}{v.intr. `passer du
dehors en dedans'}{} la
matere\wdx{matiere}{f. 1\hoch{o} `substance
qui constitue les corps, qui est objet
d'intuition dans l'espace et qui possède une
masse mécanique'}{matere}
et par l'autre elle ist. Et le appelle on aussi
sac\wdx{sac}{m. terme d'anat. `cavité ou enveloppe
dans le corps en forme de poche'}{} en
maniere de l'estomac\wdx{estomac}{m. terme d'anat.
`organe de l'appareil
digestif qui reçoit les aliments; estomac'}{},
car c'est ung aultre estomac, et est
court\wdx{*cort}{adj.
`qui a une étendue inférieure à la moyenne
dans le sens de la longueur (dans l'espace)'}{court \emph{m.sg.}}
d'une bo\emph{n}ne paulme\wdx{*paume}{f. `mesure
d'environ un travers de main; palme'}{paulme} de
long.
Et po\emph{ur} la prochaineté\wdx{prochaineté}{f.
`situation d'une chose qui est à peu de distance
d'une autre, de plusieurs choses qui sont proches;
proximité'}{} qu'il ha
aux inguines\wdx{inguine}{m. `partie latérale et inférieure
du bas-ventre; aine'}{} et
pour ce aussi qu'il n'est pas bien lié\wdx{*liier}{v.tr. `entourer plusieurs choses avec un lien pour
qu'elles tiennent ensemble'}{lié \emph{p.p.}}
es
crepatures\wdx{crepature}{f. `action de crever,
le résultat de cette action'}{},
il descend
legieremant\wdx{*legierement}{adv.
`sans effort; facilement'}{legieremant}
dedans l'ossee\wdx{ossee}{m. `enveloppe des
testicules; scrotum'}{},
selon Avicene\adx{Avicene}{}{}. Et de lui naist
ylion\wdx{*ileon}{terme d'anat. `la dernière et la
plus
longue partie de l'intestin grêle; iléon'}{ylion}
qui est ung intestin\wdx{intestin}{m. terme d'anat.
`conduit digestif qui s'étend depuis l'estomac à
l'anus, comprenant le duodénum, le jéjunum, l'iléon,
le c\ae cum, le côlon et le
rectum, aussi chacune de ses parties; intestin'}{}
long\wdx{lonc}{adj. 1\hoch{o}
`qui a une étendue supérieure à la moyenne
dans le sens de la longueur (dans l'espace)'}{long
\emph{m.sg.}} et
graille\wdx{*intestin graisle}{m.
terme d'anat.
`partie du tube digestif entre
l'estomac et le c\ae cum, comprenant le duodénum,
le jéjunum et l'iléon, aussi chacune de ses
parties'}{intestin graille}, qui ha bien ou
.vij. ou
.viij. brassees\wdx{*braciee}{f. `longueur de
bras'}{brasee} et se tourne\wdx{tourner}{v.pron.
`se mouvoir circulairement ou décrire une ligne
courbe'}{}
et revolve\wdx{revolver}{v.pron.
`se mettre
en sens inverse ou dans une certaine direction; se
tourner'}{} moult entour
les entrailles\wdx{entrailles}{f.pl.
terme d'anat.
`organes enfermés dans
l'abdomen de l'homme ou des
animaux; intestins'}{} et entour le
dos. Aprés tu
trouveras\wdx{*trover}{v.tr. `rencontrer qn ou qch. qu'on cherche;
trouver'}{trouveras \emph{2.p.sg. ind.futur}}
le intestin\wdx{intestin}{m. terme d'anat.
`conduit digestif qui s'étend depuis l'estomac à
l'anus, comprenant le duodénum, le jéjunum, l'iléon,
le c\ae cum, le côlon et le
rectum, aussi chacune de ses parties; intestin'}{}
jejune\wdx{jejune}{adj.
terme d'anat. `qui est vide (dit d'une partie du tube digestif qui fait
suite au duodénum, i.e. du jéjunum'}{} qui est
\text{vuyz}\fnb{Zusätzlich am Rand
nachgetragenes
\emph{vuiz}.}/\wdx{*vuit}{adj. `qui ne contient
rien'}{vuyz \emph{m.sg.}}
et est fait de grant\wdx{grant}{adj. 3\hoch{o}
dans
l'ordre qualitatif, non quantifiable `qui est d'un
degré supérieur à la moyenne en ce qui concerne la
qualité, l'intensité, l'importance'}{grant
\emph{f.sg.}} multitude\wdx{multitude}{f.
`grande quantité
(d'êtres, d'objets); multitude'}{}
de voines miseraques\wdx{*veine mesaraïque}{f.
terme d'anat. `vaisseau sanguin qui est un
prolongement
ramifié de la \flq porte\frq\ et qui
transporte des sucs de l'intestin au foie'}{voine
miseraque} et par grant\wdx{grant}{adj. 3\hoch{o}
dans
l'ordre qualitatif, non quantifiable `qui est d'un
degré supérieur à la moyenne en ce qui concerne la
qualité, l'intensité, l'importance'}{grant
\emph{f.sg.}} porcion\wdx{porcïon}{f.
`partie d'un tout
homogène qui n'est pas nombrable; portion'}{porcion}
de cole\wdx{cole}{f. terme de
méd.
`bile jaune (l'une des quatre
humeurs de l'humorisme')}{}, qui
sont envoiés\wdx{envoiier}{v.tr. `faire aller qn ou
qch.
(quelque part)' (le sujet n'étant
pas personnel)}{envoié \emph{p.p.}} entre
lui et le pourtenaire\wdx{portenaire}{m. et f. terme d'anat.
`orifice faisant
communiquer l'estomac avec le duodénum, pouvant
désigner aussi, par métonymie, le duodénum'}{pourtenaire}
ou le portier\wdx{portier}{m. terme d'anat. `orifice
faisant
communiquer l'estomac avec le duodénum, pouvant
désigner aussi, par métonymie, le duodénum'}{},
auquel se
\text{contienue}\fnb{Im Ms. \emph{contrenue} in
\emph{contienue} korrigiert.}/\wdx{continuer}{v.pron.
`s'étendre sans interruption
dans l'espace'}{contienue \emph{3.p.sg. ind.prés.}}
le duodene\wdx{*duodenum}{m.
terme d'anat.
`partie initiale de l'intestin grêle
accolée à la paroi abdominale postérieure et
qui s'étend au jéjunum; duodénum'}{duodene}, car
sa longuesse\wdx{*longuece}{f.
`dimension d'une
chose dans le sens de sa plus grande étendue;
longueur'}{longuesse} co\emph{n}tient
.xij. dois\wdx{doi}{m. 3\hoch{o}
`mesure approximative, équivalent à un travers de
doigt'}{}.
Et le appelle on
portenier\wdx{portenier}{m.
terme d'anat.
`orifice faisant
communiquer l'estomac avec le duodénum, pouvant
désigner aussi, par métonymie, le duodénum'}{} pour
cause de son office\wdx{*ofice}{m. `fonction
qu'une chose doit remplir dans l'ensemble dont elle
fait partie'}{office}, car c'est la
basse porte\wdx{porte}{f. 1\hoch{o}
`ouverture dans un
organe qui permet le passage d'un liquide, etc.'}{} de
l'estomac, ainsi que mery\wdx{meri}{m.
terme d'anat.
`canal musculo-membraneux qui va
du pharynx à l'estomac auquel il conduit
les aliments; \oe sophage'}{mery}
est la
haulte porte\wdx{porte}{f. 1\hoch{o} `ouverture dans
un
organe qui permet le passage d'un liquide, etc.'}{}.
Par lesqueulx choses vous
poués veoir es passions\wdx{passïon}{f. `souffrance physique'}{}
la
invencion\wdx{invencion}{f. terme de méd. `action
de rechercher et de trouver (une application
médicale)'}{} ou la
injeccion\wdx{*injection}{f. `action
d'introduire un liquide dans un conduit,
une cavité organique ou un tissu'}{injeccion}
de clisteres\wdx{clistere}{m. et f. `injection
d'un liquide dans le gros intestin par l'anus, au
moyen d'un appareil; lavement'}{} et les
lieux ou l'en doit
appliquer\wdx{*apliquier}{v.tr.
`mettre une chose sur
une autre de manière qu'elle la recouvre et y adhère;
appliquer'}{appliquer
\emph{inf.}}
les remedes\wdx{remede}{m. et f. `ce qui est
employé au traitement d'une maladie; remède'}{}. Car en
colique\wdx{colique}{f. `douleur sous forme
d'accès violent, ressentie au niveau des viscères
abdominaux; colique'}{}, on
les doit appliquer\wdx{*apliquier}{v.tr.
`mettre une chose sur
une autre de manière qu'elle la recouvre et y adhère;
appliquer'}{appliquer
\emph{inf.}} a la partie de devant\wdx{devant}{adv.
1\hoch{o}
`au côté du
visage, à la face'}{\textbf{de devant}
\emph{loc.adj. `qui est situé au côté du visage, de la
face'}}
et en la p\emph{ar}tie senextre\wdx{senestre}{adj. `qui est du
côté gauche; gauche'}{senextre} et
dextre\wdx{*destre}{adj. `qui est du côté droit; droit'}{dextre},
et en
yliaque\wdx{*iliaque}{f. `obstruction
intestinale'}{yliaque}
%
[31r\hoch{o}b]
entour les costés, et poués veoir q\emph{ue}
plaies\wdx{plaie}{f.
`ouverture
dans les chairs, les tissus, due à une cause
externe (traumatisme, intervention chirurgicale) et
présentant une solution de continuité des téguments;
plaie'}{}
des grailles intestines\wdx{*intestines
graisles}{f.pl.
terme d'anat.
`partie du tube digestif entre
l'estomac et le c\ae cum, comprenant le duodénum,
le jéjunum et l'iléon, aussi chacune de ses
parties'}{grailles intestines}
ne so\emph{n}t pas curees, car
ilz
sont de nature et de matere\wdx{matiere}{f. 1\hoch{o} `substance
qui constitue les corps, qui est objet
d'intuition dans l'espace et qui possède une
masse mécanique'}{matere}
de panicles. Mais les
gros intestins\wdx{intestin gros}{m. terme d'anat.
`partie du tube
digestif entre
l'iléon et l'anus, comprenant le c\ae cum, le côlon
et le rectum, aussi chacune de ses
parties'}{gros intestin}
\text{sont}\lemma{intestins sont}\fnb{\emph{sont} über
der Zeile nachgetragen.}/ aucune
foys sanés\wdx{saner}{v.tr.
 `délivrer d'un mal physique;
guérir'}{}, car ilz
sont charnus\wdx{charnu}{adj. `qui est
essentiellement constitué de
chair'}{charnus \emph{m.pl.}}. Et afin
q\emph{ue} tu voies\wdx{*vëoir}{v.tr. `percevoir (qch.) par
le sens de la vue'}{voies \emph{2.p.sg. subj.prés.}}
mieux les aultres choses, c'est bon de
loier\wdx{*liier}{v.tr.
`entourer plusieurs choses
avec un lien pour qu'elles tiennent
ensemble'}{loier \emph{inf.}} vers
la
portenaire\wdx{portenaire}{m. et f. terme d'anat.
`orifice faisant
communiquer l'estomac avec le duodénum, pouvant
désigner aussi, par métonymie, le duodénum'}{}
et tranche\wdx{*trenchier}{v.tr.
`séparer (une chose
en parties, deux choses unies) d'une manière nette, au
moyen d'un instrument dur et fin; trancher'}{tranche
\emph{3.p.sg. ind.prés.}} si co\emph{m}me tu as fait devant
et en tray\wdx{traire}{v.tr.
`faire venir
dans une certaine direction (qn, qch.)'}{tray
\emph{3.p.sg.
ind.prés.}} les intestins\wdx{intestin}{m.
terme d'anat.
`conduit digestif qui s'étend depuis l'estomac à
l'anus, comprenant le duodénum, le jéjunum, l'iléon,
le c\ae cum, le côlon et le
rectum, aussi chacune de ses parties; intestin'}{}.
\pend
\pstart
Et, se tu \text{voes}\fnb{Im Ms. \emph{vues} in
\emph{voes} korrigiert.}/, si regarde premier
messenteriu\emph{m}\wdx{*mesenterium}{mlt.
terme d'anat. `repli du péritoine qui relie le
jéjunum et l'iléon à la paroi abdominale postérieure;
mésentère'}{messenterium} --
c'est a dire la
testure\wdx{*tisture}{f.
 terme d'anat.
`disposition, entrelacement des fibres qui
composent un tissu organique'}{testure} ou
tissure\wdx{tissure}{f. terme d'anat.
`disposition, entrelacement des fibres qui
composent un tissu organique'}{}
des vaines mis\emph{er}aques\wdx{*veine mesaraïque}{f.
terme d'anat. `vaisseau sanguin qui est un
prolongement
ramifié de la \flq porte\frq\ et qui
transporte des sucs de l'intestin au foie'}{vaine
miseraque}
imvunerables\wdx{*invulnerable}{adj. `qui ne peut
être blessé; invulnérable'}{imvunerable} qui sont
rameffiees\wdx{*ramifier}{v.tr.
`diviser en plusieurs ramifications qui partent
d'un axe ou d'un centre de qch. (en parlant d'une
chose concrète)'}{rameffiees \emph{p.p. f.pl.}} de la
vaine que on appelle la
porte du foie\wdx{porte}{f. 2\hoch{o}
terme d'anat. `vaisseau sanguin qui va du foie aux
intestins'}{}, qui va
aux intestins\wdx{intestin}{m. terme d'anat.
`conduit digestif qui s'étend depuis l'estomac à
l'anus, comprenant le duodénum, le jéjunum, l'iléon,
le c\ae cum, le côlon et le
rectum, aussi chacune de ses parties; intestin'}{},
qui est couverte et
\text{garnie}\fnb{Nachfolgend gestrichenes
\emph{q}.}/\wdx{garnir}{v.tr. `pourvoir qn ou qch.
(de qch.)'}{}
de pani\-cles et de liguemans qui conjoingne\emph{n}t\wdx{conjoindre}{v.tr.
`mettre des choses
ensemble de façon qu'elles se touchent ou tiennent
ensemble; joindre'}{conjoingnent
\emph{3.p.pl. ind.prés.}}
les
intestins avec le dos, plaines\wdx{*plein}{adj.
`qui contient toute
la quantité possible; plein'}{plain}
de graisse et de
char
glandelleuse\wdx{*glandulos}{adj. `qui contient des
glandes'}{glandelleuse \emph{f.sg.}} -- que on
appelle
vulgauma\emph{n}t\wdx{*vulgairement}{adv.
`dans la langue commune'}{vulgaumant}
rodel\wdx{rodel}{subst. terme d'anat. `repli du
péritoine qui relie les intestins à la paroi
abdominale postérieure; mésentère'}{}, et
le verras quant il s\emph{er}a separé\wdx{separer}{v.tr. `mettre à part les unes
des autres des choses, des personnes
réunies; séparer'}{}
des intestins. Et quant
il s\emph{er}a hors, tu regarderas la anathomie de l'estomac.
\pend
\pstart
L'estomac ou le ventre, c'est l'organe\wdx{*orgene}{m.
terme d'anat.
`partie du corps
remplissant une fonction particulière;
organe'}{organe}
ou
l'instrument\wdx{instrument}{m. 2\hoch{o}
`partie du corps remplissant une fonction
particulière; organe'}{}
de la premiere digestion\wdx{digestion}{f. terme de
méd.
`conversion des substances dans le corps
en sucs nécessaires pour l'assimilation et
la désassimilation (des aliments en chyle, le chyle en sang,
etc.)'}{} et
engendre\wdx{engendrer}{v.tr. au fig. `faire naître, faire
exister; produire'}{}
le chile\wdx{chile}{m.
terme de méd. `suc
qui est assimilé des aliments par la digestion
et qui est
transporté au foie pour y servir à la formation du
sang'}{}. Car, ainsi que les
mis\emph{er}aques\wdx{*veine mesaraïque}{f.
terme d'anat. `vaisseau sanguin qui est un
prolongement
ramifié de la \flq porte\frq\ et qui
transporte des sucs de l'intestin au
foie'}{miseraque}
sont p\emph{re}paratoires\wdx{preparatoire}{adj. `qui
prépare; préparatoire'}{} de la
digestion\wdx{digestion}{f. terme de méd.
`conversion des substances dans le corps
en sucs nécessaires pour l'assimilation et
la désassimilation (des aliments en chyle, le chyle en sang,
etc.)'}{}
du foye, ainsi est la bouche\wdx{*boche}{f. `cavité
située à la partie
inférieure du visage de l'homme, bordée
par les lèvres; bouche'}{bouche} de l'estomac. Et
pour ce dit Avicene\adx{Avicene}{}{}: ce que on
mache\wdx{*maschier}{v.tr.
`broyer, écraser avec les dents par le
mouvement des mâchoires, avant d'avaler;
mâcher'}{mache \emph{3.p.sg. ind.prés.}}, il ha
aucune digestion\wdx{digestion}{f. terme de méd.
`conversion des substances dans le corps
en sucs nécessaires pour l'assimilation et
la désassimilation (des aliments en chyle, le chyle en sang,
etc.)'}{},
auquel sert\wdx{servir}{v.tr.indir. `aider en
étant utile ou utilisé' (de choses)}{}
le mery\wdx{meri}{m. terme d'anat.
`canal musculo-membraneux qui va
du pharynx à l'estomac auquel il conduit
les aliments; \oe sophage'}{mery}
ou le
ysophagus\wdx{*isophagus}{mlt.
terme
d'anat. `canal musculo-membraneux qui va
du pharynx à l'estomac auquel il conduit
les aliments; \oe sophage'}{ysophagus}
en la partie
de dessus\wdx{*desus}{adv. `au côté
supérieur'}{\textbf{de dessus} \emph{loc.adj. `qui
est situé au côté supérieur'}}
a mener\wdx{mener}{v.tr. `conduire (qn., qch.)
quelque part'}{} les viandes\wdx{viande}{f.
`aliment dont on se nourrit'}{} a lui,
et les intestins aussi avec les
mis\emph{er}aques\wdx{*veine mesaraïque}{f.
terme d'anat. `vaisseau sanguin qui est un
prolongement
ramifié de la \flq porte\frq\ et qui
transporte des sucs de l'intestin au
foie'}{miseraque}, pour
expellir\wdx{expellir}{v.tr. terme de méd. `faire
évacuer (qch.) de l'organisme; expulser'}{} les
%
[31v\hoch{o}a] choses nocives\wdx{nocif}{adj. `qui
est nuisible; nocif'}{} et pour
distribuer\wdx{distribuer}{v.tr. `diviser (qch.)
entre plusieurs personnes, lieux, etc., en donnant
une part à chacun; distribuer'}{}
les choses prouffitables\wdx{*porfitable}{adj.
`qui est
avantageux, utile; profitable'}{prouffitable}
et
digestes\wdx{digest}{adj. `qui est
digéré'}{} et
\text{chilozees}\fnb{Im Ms. \emph{c} zu \emph{l}
korrigiert.}/\wdx{*chiloser}{v. terme de
méd. `transformer en chyle'}{chilozé
\emph{p.p.}}, car
l'estomac est ainsi q\emph{ue} ung celier\wdx{celier}{m.
`lieu aménagé pour y conserver du vin, des
provisions, etc.; cellier'}{} co\emph{m}mun\wdx{*comun}{adj.
`qui appartient
à plusieurs personnes ou
choses'}{commun}
a toutes pars, qui est
assis\wdx{asseoir}{v.tr. 2\hoch{o} `placer, poser
(qch.)'}{assis \emph{p.p.}}
au
\text{millieu}\fnb{Ms. \emph{millie}.}/\wdx{milieu}{m. `partie
d'une chose
qui est à égale distance des extrémités de cette
chose'}{millieu \emph{(Ms.}
millie\emph{)}} de la beste\wdx{beste}{f. `être
vivant non végétal et non humain; animal'}{}, selon
Galien\adx{Galien}{}{} au quart livre et au
p\emph{re}mier chappitre du livre \flq De utilitate
partic\emph{u}la\emph{rum}\frq .
Et ja ssoit ce qu'il soit mis\wdx{metre}{v.tr.
1\hoch{o}
`placer (qch.) dans une position déterminée'}{} ou
milieu, dessoubz le
pis, toutesvoies la p\emph{ar}tie de dessus\wdx{*desus}{adv.
`au côté
supérieur'}{\textbf{de dessus} \emph{loc.adj. `qui
est situé au côté supérieur'}}
du cuer
s'encline\wdx{encliner}{v.pron.
`se
pencher (de choses)'}{} ung
petit\wdx{petit}{adj. 3\hoch{o}
dans
l'ordre qualitatif, non quantifiable `qui est d'un
degré inférieur à la moyenne en ce qui concerne la
qualité, l'intensité,
l'importance'}{\textbf{un petit} \emph{adv.
`un peu'}}
au senextre\wdx{senestre}{adj. `qui est du
côté gauche; gauche'}{senextre}
las\wdx{*lez}{m.
`partie qui est à droite
ou à gauche (d'un corps); côté'}{las} vers la
.xij.\hoch{e} spondille, ou
termine\wdx{terminer}{v.intr. `prendre fin; se
terminer'}{} le diaframe\wdx{diafragme}{m.
terme d'anat. `muscle large et
mince qui sépare le thorax de l'abdomen;
diaphragme'}{diaframe}, et la p\emph{ar}tie
basse se trait\wdx{traire}{v.pron. `se diriger
quelque part'}{} a dextre\wdx{*destre}{f. `le côté
droit;
droite'}{\textbf{a destre} \emph{loc.adv. `sur le
côté droit'} a dextre}. La
op\emph{er}acion\wdx{operacïon}{f.
`action d'un pouvoir, d'une fonction,
d'un organe qui produit un effet selon sa
nature; opération'}{operacion} de l'estomac,
c'est digerer\wdx{digerer}{v.tr. empl.abs. terme de
méd.
`convertir des
substances dans le corps en sucs par les mécanismes de
la digestion (des aliments en chyle, le chyle
en sang, etc.)'}{} par
challeur\wdx{*chalor}{f.
terme de méd.
`dans l'humorisme, qualité qui gouverne essentiellement
l'équilibre du sang et de la bile'}{challeur}
de la p\emph{ro}pre\wdx{propre}{adj. 1\hoch{o} `qui
appartient exclusivement à (qn, qch.)'}{}
carnousité\wdx{*charnosité}{f. `qualité
de ce qui est charnu'}{carnousité}
de son fons\wdx{*fonz (de l'estomac)}{m.
`partie de l'estomac opposée à
l'entrée de l'\oe sophage'}{fons} -- si co\emph{m}me le dit
Avicene\adx{Avicene}{}{} --
et par aultres challeurs\wdx{*chalor}{f.
terme de méd.
`dans l'humorisme, qualité qui gouverne essentiellement
l'équilibre du sang et de la bile'}{challeur}
aquises\wdx{aquerre}{v.tr. `arriver à obtenir
(qch.)'}{aquise \emph{p.p.}}
des parties voisines\wdx{voisin}{adj.
`qui
est à côté; voisin'}{}. Et a
dextre\wdx{*destre}{f. `le côté droit;
droite'}{\textbf{a destre} \emph{loc.adv. `sur le
côté
droit'} a dextre} de l'estomac est le foie qui le
eschauffe\wdx{*eschaufer}{v.tr. `rendre
chaud'}{eschauffe \emph{3.p.sg. ind.prés.}} par
ces lobes\wdx{lobe}{m.
terme d'anat. `partie arrondie et saillante de
divers organes; lobe'}{} ou ales\wdx{*ele}{f.
`chacune des deux
parties latérales (d'une chose) dont la
forme ressemble à celle de l'organe du vol chez les oiseaux' (par analogie de forme)}{ale}.
Et l'esplein\wdx{esplein}{m. terme d'anat.
`organe
lymphoïde situé sous la partie gauche du
diaphragme; rate'}{}
est a senextre\wdx{senestre}{m. et f.
`le côté
gauche'}{\textbf{a senextre} \emph{loc.adv. `sur
le côté gauche'}} de
travers\wdx{travers}{adv.}{\textbf{de travers}
\emph{loc.adv. `dans une direction transversale'}},
avec sa graisse et avec
ces vaines
et melancolie\wdx{melancolie}{f. terme de méd.
`bile noire (l'une des quatre humeurs de
l'humorisme)'}{} avec qui lui donne\wdx{*doner}{v.tr. `mettre (qch.) à la disposition de qn,
de qch.'}{donne \emph{3.p.sg. ind.prés.}}
appetit\wdx{apetit}{m. `désir de manger'}{appetit}.
Et par dessus\wdx{*desus}{adv.
`au côté
supérieur'}{dessus}
est le cuer, avec ces arteres qui le
vivefient\wdx{vivifier}{v.tr.
`donner la vie,
l'énérgie vitale à, entretenir la vie, l'énérgie
vitale de'}{vivefient \emph{3.p.pl. ind.prés.}}. Et
par dessus
est le cervel pour sentir. Et a la partie
du dos sont les voines
kily\wdx{*veine kilis}{f.
terme
d'anat. `vaisseau sanguin qui ramène le sang
nutritif du foie à tout le corps'}{voine kily} et
aborchy\wdx{*veine aborthi}{f.
terme d'anat. `vaisseau sanguin qui naisse au
c\oe ur'}{voine aborchy}
des\-cendans\wdx{*descendant}{adj. terme d'anat. `qui
va du haut en
bas
(surtout en parlant d'un vaisseau
sanguin)'}{descendans \emph{f.pl.}} et pluseurs
liguemans,
de quoy les spo\emph{n}dilles des reins\wdx{rein}{m.
2\hoch{o} au plur.
`la partie inférieure du dos au niveau des
vertèbres lombaires'}{}
sont liés avec le dit estomac. Et ainsi appart son
accion\wdx{accion}{f. `tout ce que l'on fait (dit aussi de choses)'}{},
sa posicion\wdx{posicïon}{f. `lieu où quelque chose est placée,
située'}{posicion}
et sa colligance\wdx{colligance}{f. 1\hoch{o}
`force qui maintient réunis les éléments d'un système matériel;
liaison'}{}.
Et est composé de deux tuniques.
L'une charneuse\wdx{*charnel}{adj. `qui est
essentiellement constitué de
chair'}{charneuse \emph{f.sg.}} qui est par
dessoubz\wdx{*desoz}{adv.
`à la face
inférieure'}{dessoubz}; par dehors et
par dedans elle est
nerveuse\wdx{*nervos}{adj. terme d'anat. `qui a le caractère des nerfs
ou des tendons'}{nerveuse \emph{f.sg.}} et
villeuse\wdx{villeux}{adj. terme d'anat. `qui porte
de petites saillies filiformes qui donnent un
aspect velu; villeux'}{villeuse
\emph{f.sg.}} de villis\wdx{villis}{m.pl.
terme d'anat.
`production organique longue et fine comme des
fils'}{} qui vo\emph{n}t
%
[31v\hoch{o}b]
de long\wdx{lonc}{adv.}{\textbf{de lonc}
\emph{loc.adv.  `dans le sens de la longueur'} de
long}
pour expellir\wdx{expellir}{v.tr. terme de méd. `faire
évacuer (qch.) de l'organisme; expulser'}{} la
superfluité. Sa
figure\wdx{figure}{f. `forme extérieure d'un corps'}{}
est ronde et oblige\wdx{oblige}{adj.
`qui change
de direction sans former d'angles; courbe' (?)}{}
en maniere d'une cohorde\wdx{*cöorde}{f. `fruit
d'une plante potagère; courge'}{cohorde}
courbe\wdx{*corp}{adj.
`qui change de
direction sans former d'angles; courbe'}{courbe},
en telle maniere\wdx{maniere}{f. 2\hoch{o} `forme
particulière
que revêt l'accomplissement d'une action, le
déroulement d'un fait, l'être ou
l'existence'}{\textbf{en telle maniere que}
\emph{loc.conj. `de sorte que'}}
que les diz oriffices\wdx{orifice}{m.
`ouverture faisant communiquer un conduit, un organe
avec une structure voisine ou avec l'extérieur;
orifice'}{oriffice}
sont plus hault que le corps de l'estomac, afin que ce
qui est dedans ne puisse issir
\text{sans}\fnb{Nachfolgend gestrichenes
\emph{provisio}.}/
provizion\wdx{*provisïon}{f. `faculté ou action de
prévoir; prévision'}{provizion}. Sa q\emph{uan}tité\wdx{*cantité}{f. `nombre
d'unités ou mesure qui sert à déterminer une
portion de matière ou une collection de choses
considérées comme
homogènes'}{quantité} est manifeste\wdx{manifest}{adj. `dont l'existence ou la nature
est évidente; manifeste'}{}, car
il contient co\emph{m}munemant\wdx{*comunement}{adv.
`en
général'}{communemant} deux
ou trois pichiers\wdx{pichier}{m. `vase destinée à
contenir de la boisson'}{} de vin\wdx{vin}{m.
`boisson alcoolisée provenant
de la fermentation des raisins; vin'}{}.
Et peult avoir maintes\wdx{maint}{adj. `plusieurs; maint'}{}
maladies, et li peult
on aider\wdx{aidier}{v.tr.
`appuyer (qn ou qch.) en apportant son aide'}{aider
\emph{inf.}}
-- si co\emph{m}me
\text{il apert}\fnb{Ms. \emph{il pert}.}/
par la anathomie
-- en mettant\wdx{metre}{v.tr. 1\hoch{o} `placer
(qch.) dans une position déterminée'}{} les
remedes\wdx{remede}{m.
et f. `ce qui est
employé au traitement d'une maladie; remède'}{} par dessus,
vers
les .xij. spondilles et en la p\emph{ar}tie de devant,
depuis\wdx{depuis}{prép. `à partir de
(en parlant de l'espace)'}{}
la forcelle\wdx{*forcele}{f. 2\hoch{o} terme
d'anat. `fourche du sternum'}{forcelle} jusques pres
du no\emph{m}bril\wdx{nombril}{m. `cicatrice
arrondie formant une petite cavité ou une saillie,
placée sur la ligne médiane du ventre des mammifères,
à l'endroit où le cordon ombilical a été sectionné;
nombril'}{}.
\pend
\pstart
Aprés il nous
co\emph{n}vient dire\wdx{dire}{v.tr.indir. \textbf{\emph{dire de}} `parler de'}{}
du foye. Le foye, c'est le organe\wdx{*orgene}{m.
terme d'anat.
`partie du corps
remplissant une fonction particulière;
organe'}{organe},
c'est a dire l'instru\-ment\wdx{instrument}{m. 2\hoch{o}
`partie du corps remplissant une fonction
particulière; organe'}{}, de la
seconde
\text{digestion}\fnb{\emph{di} über der Zeile
nachgetragen; \emph{-gis-} in \emph{-ges-} im Ms.
korrigiert.}/\wdx{digestion}{f. terme de méd.
`conversion des substances dans le corps
en sucs nécessaires pour l'assimilation et
la désassimilation (des aliments en chyle, le chyle en sang,
etc.)'}{} q\emph{ue} engendre\wdx{engendrer}{v.tr.
au fig. `faire naître, faire exister; produire'}{}
le sang,
et est
assis\wdx{asseoir}{v.tr. 2\hoch{o} `placer, poser
(qch.)'}{assis \emph{p.p.}}
au dextre\wdx{*destre}{adj. `qui est du côté droit;
droit'}{dextre} cousté
dessoubz
les faulces\wdx{*faus}{adj. `qui n'est pas ce qu'on le nomme;
faux'}{faulces \emph{f.pl.}}
costes\wdx{coste}{f. terme d'anat. `os plat et
courbe du
thorax qui s'articule sur la colonne vertébrale et le sternum;
côte'}{},
et est en fourme
de la lune\wdx{lune}{f. `satellite de la terre,
recevant sa lumière du soleil; aussi son aspect vu
d'un point de la terre; lune'}{} qui est
bossu\wdx{*boçu}{adj.
`qui
présente une ou plusieurs saillies arrondies; bossu'}{bossu} vers les
costes\wdx{coste}{f. terme d'anat. `os plat et
courbe du thorax qui s'articule sur la colonne vertébrale et le
sternum; côte'}{}
et concave\wdx{concave}{adj. `qui présente une
surface en creux; concave'}{} vers l'estomac, et ha
.v. lobes\wdx{lobe}{m.
terme d'anat. `partie arrondie et saillante de
divers organes; lobe'}{} ou
.v. pe\emph{n}nes\wdx{*pene}{f.
terme d'anat. `partie arrondie et saillante de divers
organes; lobe'}{penne} ou .v.
dois\wdx{doi}{m.
1\hoch{o}
`chacun des cinq prolongements qui terminent la main'}{}
en maniere
d'une main, et co\emph{m}prent\wdx{comprendre}{v.tr. 2\hoch{o} `être
autour de (qch.) de manière à enfermer ou embrasser partiellement ou
complètement; entourer'}{comprent \emph{3.p.sg.
ind.prés.}}
et embrasse\wdx{*embracier}{v.tr. `être
autour de (qch.) de manière à enfermer;
entourer'}{embrasse \emph{3.p.sg. ind.prés.}}
l'estomac. Et ha le dit foye, ainsi
que les aultres entrailles\wdx{entrailles}{f.pl.
terme d'anat.
`organes enfermés dans
l'abdomen de l'homme ou des
animaux; intestins'}{},
ung panicle qui
le coevre, auquel, pour cause de sentemant\wdx{*sentement}{m.
`faculté d'éprouver
les impressions que font les objets matériels, i.e.
goût, odorat, ouïe, toucher, vue'}{sentemant}, vient ung
petit\wdx{petit}{adj.
2\hoch{o} dans l'ordre
physique, quantifiable `qui est d'une extension
au-dessous de la moyenne; petit (du corps humain et
de ses parties)'}{}
nerf.
Et est le foye et son panicle lié ou diaframe\wdx{diafragme}{m.
terme d'anat. `muscle large et
mince qui sépare le thorax de l'abdomen;
diaphragme'}{diaframe}
et, par
consequent\wdx{consequent}{adj.}{\textbf{par
consequent}
\emph{loc.adv. `comme suite logique'}}, avec les fors liguema\emph{n}s de
dehors\wdx{*defors}{adv. `à
l'extérieur'}{\textbf{de dehors} \emph{loc.adj.
`qui est situé à l'extérieur'}} et avec
le dos et l'estomac et les intestins, et ha
colligance\wdx{colligance}{f. 1\hoch{o}
`force qui maintient réunis les éléments d'un système matériel;
liaison'}{}
avec \text{eux}\fnb{Voranstehend gestrichenes
\emph{h}.}/, avec le cuer, avec les reins\wdx{rein}{m.
1\hoch{o} terme d'anat. `chacun des deux organes
sécréteurs
glandulaires situés symétriquement dans les fosses
lombaires et qui élaborent l'urine'}{}
et avec les
coillons\wdx{coillon}{m. 1\hoch{o} `gonade mâle suspendue dans le scrotum, qui
produit les spermatozoïdes; testicule'}{}
et avec tous me\emph{m}bres.
%
[32r\hoch{o}a]
La substance du foye est rouge\wdx{*roge}{adj. `qui
est de la couleur du sang; rouge'}{rouge} et
charneuse\wdx{*charnel}{adj. `qui est
essentiellement constitué de
chair'}{charneuse \emph{f.sg.}},
ainsi que se c'estoit sang
coagulé\wdx{coaguler}{v.tr. `transformer en une
masse plus ou moins solide les particules
en suspension dans un liquide; coaguler'}{coagulé
\emph{p.p.}}, et tissue\wdx{tistre}{v.tr.
terme de méd.
`former par entrelacement (de fibres ou
structures organiques)'}{tissu \emph{p.p.}} partout
de vaines et arteres -- si co\emph{m}me y s\emph{er}a dit --, ainsi
q\emph{ue} se
c'estoit sang tissu\wdx{tistre}{v.tr.
terme de méd.
`former par entrelacement (de fibres ou
structures organiques)'}{tissu \emph{p.p.}}; et
est composé de pluseurs
choses, toutesvoies il ha une simple partie
charnue\wdx{charnu}{adj. `qui est
essentiellement constitué de
chair'}{charnue \emph{f.sg.}},
pour quoy il est co\emph{m}mencemant\wdx{*comencement}{m. `première
partie de qch., celle que d'autres doivent suivre et qu'aucune
ne précède (dans le temps ou dans l'espace)'}{commencemant}
de
sanguinacion\wdx{sanguinacion}{f. terme de med.
`production de sang'}{} et de
vaines. Car ainsi q\emph{ue} dit Galien\adx{Galien}{}{} ou second
livre, ou darrier\wdx{derrier}{adj. `qui vient après tous
les autres, après lequel il n'y a pas
d'autre'}{darrier} chappitre du \flq Livre des
vertus naturelles\frq\
et ou quart livre, ou qui\emph{n}t chapp\emph{itre} du livre
\flq De utilitate partic\emph{u}la\emph{rum}\frq , il dit que le
moust\wdx{*most}{m. `jus de raisin qui vient
d'être exprimé et n'a pas encore subi la
fermentation alcoolique; moût'}{moust} qui
est au vaissel\wdx{vaissel}{m. 1\hoch{o} `récipient
pour les liquides'}{}, qua\emph{n}t il\linebreak
vueil,
\text{il en ist}\fnb{Ms. \emph{il est ist},
\emph{est} expungiert, \emph{en en}
über der Zeile nachgetragen, davon ist das erste
\emph{en} expungiert.}/ trois
substances, c'est assavoir l'escume\wdx{escume}{f.
`mousse qui se forme à la surface d'un liquide qu'on
agite, qu'on chauffe ou qui fermente'}{},
le vin\wdx{vin}{m. `boisson alcoolisée provenant
de la fermentation des raisins; vin'}{}
et la lie\wdx{lie}{f. `dépôt qui se forme au fond
des récipients contenant des boissons fermentées;
lie'}{}.
Einsi
est il du chille\wdx{chile}{m. terme de méd. `suc
qui est assimilé des aliments par la digestion
et qui est
transporté au foie pour y servir à la formation du
sang'}{chille} qui est au foye, que par
decoccion\wdx{*decoction}{f. terme de méd.
`transformation que subissent les aliments dans un
organe pour être assimilés'}{decoccion} s'e\emph{n} font
trois substa\emph{n}ces,
c'est assavoir deux sup\emph{er}fluités et une substance
naturelle\wdx{naturel}{adj.
1\hoch{o} terme de méd.
`qui contient les esprits
transformés dans le foie qui
contrôlent l'alimentation, le grandissement et la
génération de l'homme, qui a rapport
à la
transformation ou à la diffusion de ces esprits'}{}
avec aquosité\wdx{aquosité}{f. `ce qui
est de la nature aqueuse, aussi la qualité de ce qui
est de la nature de l'eau'}{}, et celle substance
est co\emph{m}mune\wdx{*comun}{adj. `qui appartient
à plusieurs personnes ou
choses'}{commun}
aux autres humeurs\wdx{humeur}{f. 2\hoch{o} terme de méd.
`dans l'humorisme, un
des quatre liquides du corps qui gouvernent son
équilibre (le sang, le flegme, la bile,
l'atrabile)'}{}. Et le appell'en
massa
sa\emph{n}guinaria\wdx{massa sanguinaria}{lt. terme de
méd.
`substance qui est produite dans le foie et qui
contient les
substances alimentaires (dans l'humorisme)'}{}, et
contient quatre substa\emph{n}ces
naturelles\wdx{naturel}{adj.
1\hoch{o} terme de méd.
`qui contient les esprits
transformés dans le foie qui
contrôlent l'alimentation, le grandissement et la
génération de l'homme, qui a rapport
à la
transformation ou à la diffusion de ces esprits'}{}
et nutrimentelles\wdx{*nutrimental}{adj.
`qui a une valeur nutritive, qui
nourrit'}{nutrimentel}, si co\emph{m}me il est
demonstré\wdx{demonstrer}{v.tr.
`faire
voir, mettre devant les yeux; montrer'}{}
plainemant\wdx{*pleinement}{adv.
`d'une manière
pleine, totale; pleinement'}{plainemant} ou
second \flq Livre des ellemans\frq\wdx{*element}{m. `partie
constitutive d'une chose; élément'}{ellemans
\emph{pl.}}.
\pend
\pstart
Item, ces humeurs\wdx{humeur}{f. 1\hoch{o} terme de méd.
`substance liquide qui se trouve dans un organisme
humain ou animal'}{}, qui sont
engendrees\wdx{engendrer}{v.tr. au fig. `faire
naître, faire exister; produire'}{}
ou foye du chile\wdx{chile}{m.
terme de méd. `suc
qui est assimilé des aliments par la digestion
et qui est
transporté au foie pour y servir à la formation du
sang'}{}, sont doubles\wdx{*doble}{adj.
`qui est
répété deux fois, qui vaut deux fois (la
chose désignée) ou qui existe deux fois'}{double}: aucunes
sont
naturelles\wdx{naturel}{adj.
1\hoch{o} terme de méd.
`qui contient les esprits
transformés dans le foie qui
contrôlent l'alimentation, le grandissement et la
génération de l'homme, qui a rapport
à la
transformation ou à la diffusion de ces esprits'}{}
qui viennent de droite\wdx{droit}{adj. 2\hoch{o}
`qui suit un
raisonnement correct'}{}
nutriccion\wdx{*nutricion}{f.
`processus d'assimilation et de désassimilation dans
un organisme vivant; nutrition'}{nutriccion}
naturelle\wdx{naturel}{adj.
1\hoch{o} terme de méd.
`qui contient les esprits
transformés dans le foie qui
contrôlent l'alimentation, le grandissement et la
génération de l'homme, qui a rapport
à la
transformation ou à la diffusion de ces esprits'}{},
les aultres so\emph{n}t non naturelles.
Les naturelles\wdx{naturel}{adj.
1\hoch{o} terme de méd.
`qui contient les esprits
transformés dans le foie qui
contrôlent l'alimentation, le grandissement et la
génération de l'homme, qui a rapport
à la
transformation ou à la diffusion de ces esprits'}{}
avec le sang sont
envoyees\wdx{envoiier}{v.tr. `faire aller qn ou qch.
(quelque part)' (le sujet n'étant pas
personnel)}{envoyé \emph{p.p.}} par tout
le corps pour norryr\wdx{norrir}{v.tr. `entretenir,
faire vivre en
donnant à manger ou en procurant les aliments nécessaires à la
subsistance; nourrir'}{norryr \emph{inf.}} et
\text{engendrer}\fnb{Ms. \emph{engendre}.}/\wdx{engendrer}{v.tr.
au fig. `faire naître, faire exister; produire'}{}
le corps.
Les non naturelles\wdx{naturel}{adj.
1\hoch{o} terme de méd.
`qui contient les esprits
transformés dans le foie qui
contrôlent l'alimentation, le grandissement et la
génération de l'homme, qui a rapport
à la
transformation ou à la diffusion de ces esprits'}{}
sont
devisees: en sont envoyés \text{ens
certains}\lemma{e\emph{n}s certains lieux}\fnb{Ms. \emph{e}n\emph{s
lieux certains lieux}.}/
%
[32r\hoch{o}b] lieux pour aucuns
aides\wdx{aide}{f. 2\hoch{o}
`caractère de ce qui est
utile, qui satisfait un besoin; utilité'}{}, ou
elles \text{so\emph{n}t}\fnb{Ms. \emph{so}n\emph{t},
Zeilenumbruch und erneutes \emph{sont}.}/
expellees\wdx{expeller}{v.tr. terme de méd. `faire
évacuer (qch.) de l'organisme; expulser'}{expellé
\emph{p.p.}}
du
corps. Et saches que la cole\wdx{cole}{f. terme de méd.
`bile jaune (l'une des quatre
humeurs de l'humorisme')}{}
est envoyee ou cestim du
fiel\wdx{cestim du fiel}{m. terme d'anat.
`vésicule bilaire'}{},
et la
merancolie\wdx{melancolie}{f.
terme de méd.
`bile noire (l'une des quatre humeurs de
l'humorisme)'}{merancolie} a l'esplein\wdx{esplein}{m.
terme d'anat.
`organe
lymphoïde situé sous la partie gauche du
diaphragme; rate'}{},
et le fleume\wdx{fleume}{m.
terme de
méd. `lymphe (l'une des quatre humeurs de
l'humorisme)'}{} aux joinctures, et la
sup\emph{er}fluité aigouse\wdx{*ewos}{adj. `qui est de la
nature de l'eau; aqueux'}{aigouse
\emph{f.sg.}} aux reins\wdx{rein}{m. 1\hoch{o}
terme d'anat. `chacun des deux organes
sécréteurs
glandulaires situés symétriquement dans les fosses
lombaires et qui élaborent l'urine'}{}
et a la
vecie\wdx{vessie}{f. terme d'anat.
`réservoir
musculo-membraneux dans lequel s'accumule
l'urine qui arrive des reins par les
uretères; vessie'}{vecie}. Celles
qui sont expellees\wdx{expeller}{v.tr. terme de méd. `faire
évacuer (qch.) de l'organisme; expulser'}{expellé
\emph{p.p.}}
du corps, elles vont avec le sang
et se putreffient\wdx{*putrefïer}{v.pron.
`se pourrir; se putréfier'}{putreffient
\emph{3.p.pl. ind.prés.}}
aucunes fois\wdx{*foiz}{f. `cas où un fait se
produit, moment du temps où un
événement, conçu comme identique à d'autres
événements, se produit; fois'}{fois}
et font fievres\wdx{fievre}{f. `état
maladif caractérisé par l'augmentation de la chaleur
du corps'}{}. Et aucunes sont expellees au cuir et se
resolvent\wdx{resolver}{v.pron. `se résoudre'}{}
insenciblemant\wdx{*insensiblement}{adv. `sans
pouvoir être perçu par les sens'}{insenciblemant}
ou
senciblemant\wdx{*sensiblement}{adv.
`en pouvant être perçu par les sens'}{senciblemant},
par
suour\wdx{*süor}{f. `produit de la
transpiration qui sort par les pores de la
peau et condensé en gouttes ou en gouttelettes;
sueur'}{suour} ou par roigne\wdx{roigne}{f. terme de
méd.
`maladie cutanée contagieuse; gale'}{}
ou p\emph{ar} pustules\wdx{pustule}{f. terme de méd.
`petit soulèvement de l'épiderme ou du derme à
contenu purulent; pustule'}{} ou par
empostumacions\wdx{empostumacion}{f. terme de
méd.
`développement d'un tumeur accompagné de
suppuration'}{}.
Donc ilz sont quatre humours\wdx{humeur}{f. 2\hoch{o} terme de méd.
`dans l'humorisme, un
des quatre liquides du corps qui gouvernent son
équilibre (le sang, le flegme, la bile,
l'atrabile)'}{humour}
naturelles\wdx{naturel}{adj.
1\hoch{o} terme de méd.
`qui contient les esprits
transformés dans le foie qui
contrôlent l'alimentation, le grandissement et la
génération de l'homme, qui a rapport
à la
transformation ou à la diffusion de ces esprits'}{}
et quatre non naturelles\wdx{naturel}{adj.
1\hoch{o} terme de méd.
`qui contient les esprits
transformés dans le foie qui
contrôlent l'alimentation, le grandissement et la
génération de l'homme, qui a rapport
à la
transformation ou à la diffusion de ces esprits'}{}
et
la aquosité\wdx{aquosité}{f.
`ce qui
est de la nature aqueuse, aussi la qualité de ce qui
est de la nature de l'eau'}{}.
\pend
\pstart
Lesquelles
humours\wdx{humeur}{f. 2\hoch{o} terme de méd.
`dans l'humorisme, un
des quatre liquides du corps qui gouvernent son
équilibre (le sang, le flegme, la bile,
l'atrabile)'}{humour}
--
les anciens\wdx{anciens}{m.pl. `ceux qui ont vécu dans des temps
fort éloignés de nous'}{} les appelloient
sang\wdx{*sanc}{m. terme de méd.
`liquide visqueux, de couleur
rouge, qui est porté par les vaisseaux dans tout
l'organisme où il joue des rôles multiples (l'une
des quatre humeurs de l'humorisme)'}{sang},
cole\wdx{cole}{f. terme de méd.
`bile jaune (l'une des quatre
humeurs de l'humorisme')}{},
fleume\wdx{fleume}{mn
terme de
méd. `lymphe (l'une des quatre humeurs de
l'humorisme)'}{},
melancolie\wdx{melancolie}{f.
terme de méd.
`bile noire (l'une des quatre humeurs de
l'humorisme)'}{} --, lesquelles se
engendrent\wdx{engendrer}{v.pron.
au fig. `naître; se produire'}{} au foye
et se distribue\emph{n}t\wdx{distribuer}{v.pron. `se
répartir dans plusieurs endroits; se distribuer'}{}
par ceste maniere ci:
de la p\emph{ar}tie du foie ist une vaine que on appelle la
porte\wdx{porte}{f. 2\hoch{o}
terme d'anat. `vaisseau sanguin qui va du foie aux
intestins'}{},
qui se devise\wdx{deviser}{v.pron. `se séparer en
parties; se diviser'}{} et partit
en g\emph{ra}nt quantité\wdx{*cantité}{f.
`nombre d'unités ou mesure qui sert à déterminer une
portion de matière ou une collection de choses
considérées comme
homogènes'}{quantité}
de vaines mis\emph{er}aques\wdx{*veine mesaraïque}{f.
terme d'anat. `vaisseau sanguin qui est un
prolongement
ramifié de la \flq porte\frq\ et qui
transporte des sucs de l'intestin au foie'}{vaine
miseraque} qui sont plantees\wdx{planter}{v.tr.
`fixer (qch.)'}{}
en l'estomac et es intes\-tins \text{et attraient
la}\fnb{Über der Zeile
nachgetragen.}/\wdx{*atraire}{v.tr.
1\hoch{o}
`amener (qn, qch.) vers soi ou quelque
part'}{attraient \emph{3.p.pl. ind.prés.}}
succosité\wdx{succosité}{f. terme de méd. `qualité de
ce qui est humide'}{} du
\text{chile}\fnb{Vorstehend gestrichenes
\emph{foye}.}/\wdx{chile}{m.
terme de méd. `suc
qui est assimilé des aliments par la digestion
et qui est
transporté au foie pour y servir à la formation du
sang'}{} et la portent\wdx{porter}{v.tr.
`déplacer (qch.)
d'un lieu à un autre en le menant avec soi;
transporter'}{}
au foie.
Et la dicte porte\wdx{porte}{f. 2\hoch{o}
terme d'anat. `vaisseau sanguin qui va du foie aux
intestins'}{}, avec ces
\text{racines}\fnb{Über der Zeile nachgetragen,
ersetzt expungiertes \emph{parties}.}/\wdx{racine}{f. 1\hoch{o}
terme d'anat.
`portion d'un organe
servant à son implantation dans un autre organe'}{}, la
distribue\wdx{distribuer}{v.tr. `diviser (qch.)
entre plusieurs personnes, lieux, etc., en donnant
une part à chacun; distribuer'}{} et
espa\emph{n}t\wdx{espandre}{v.tr. `étendre qch.'}{} p\emph{ar} ces
racines\wdx{racine}{f. 1\hoch{o} terme d'anat.
`portion d'un organe
servant à son implantation dans un autre organe'}{}
par tout le
foie. Et de la bosse\wdx{*boce}{f. terme
d'anat. `saillie à la surface d'une structure
anatomique'}{bosse} du foie ist une
voine que on appelle concave\wdx{*veine concave}{f.
terme
d'anat. `vaisseau sanguin qui ramène le sang
nutritif du foie à tout le corps'}{voine concave}
ou kylis\wdx{*veine kilis}{f.
terme
d'anat. `vaisseau sanguin qui ramène le sang
nutritif du foie à tout le corps'}{kylis},
laquelle, avec ces racines\wdx{racine}{f. 1\hoch{o}
terme d'anat.
`portion d'un organe
servant à son implantation dans un autre organe'}{},
se
obvie\wdx{obvier}{v.tr.indir. `aller au-devant (de
qn, de qch.)'}{} ou
re\emph{n}contre\wdx{rencontrer}{v.tr. `entrer en contact
avec (qch.)' (dit de choses)}{}
a l'encontre\wdx{encontre}{m. et f. `fait de
rencontrer qn ou qch.'}{} des aultres et
attrait\wdx{*atraire}{v.tr.
1\hoch{o}
`amener (qn, qch.) vers soi ou quelque
part'}{attrait \emph{3.p.sg. ind.prés.}}
du tout le sang qui y est enge\emph{n}dré\wdx{engendrer}{v.tr.
au fig. `faire naître, faire
exister; produire'}{}. Et
celle
vaine
kilis\wdx{*veine kilis}{f.
terme
d'anat. `vaisseau sanguin qui ramène le sang
nutritif du foie à tout le corps'}{vaine kilis}, elle se
ramefie\wdx{*ramifier}{v.pron.
`se diviser en plusieurs ramifications qui partent
d'un axe ou d'un centre de qch. (en parlant d'une
chose concrète)'}{ramefie \emph{3.p.sg. ind.prés.}}
%
[32v\hoch{o}a]
en montant amont\wdx{amont}{adv. `vers le haut'}{} et en descenda\emph{n}t
bas\wdx{bas}{adv. `à faible hauteur; bas'}{}, et
distribue\wdx{distribuer}{v.tr. `diviser (qch.)
entre plusieurs personnes, lieux, etc., en donnant
une part à chacun; distribuer'}{}
et envoie\wdx{envoiier}{v.tr. `faire aller
qn ou qch.
(quelque part)' (le sujet n'étant pas
personnel)}{envoie \emph{3.p.sg. ind.prés.}} le sang
par tout le corps pour le
nourrir\wdx{norrir}{v.tr. `entretenir, faire vivre en
donnant
à manger ou en procurant les aliments nécessaires à la
subsistance; nourrir'}{nourrir \emph{inf.}},
et la se perfait\wdx{parfaire}{v.pron. `devenir
achevé'}{perfait \emph{3.p.sg. ind.prés.}}
la tierce digestion\wdx{digestion}{f. terme de méd.
`conversion des substances dans le corps
en sucs nécessaires pour l'assimilation et
la désassimilation (des aliments en chyle, le chyle en sang,
etc.)'}{} et la quarte. Item, du foye
\text{issent}\fnb{Ms. \emph{issent issent}.}/
p\emph{ro}pres\wdx{propre}{adj. 1\hoch{o} `qui
appartient exclusivement à (qn, qch.)'}{}
vaines et conduis\wdx{conduit}{m. `canal
ou tuyau qui sert à l'ecoulement ou au
transport d'une matière (un liquide, l'air, un
gaz, etc.)'}{conduis
\emph{pl.}}
qui \text{portant}\fnb{Zur Form, cf.
\flq Die Sprache\frq .}/\wdx{porter}{v.tr.
`déplacer (qch.)
d'un lieu à un autre en le menant avec soi;
transporter'}{portant \emph{3.p.pl. ind.prés.}}
la
sup\emph{er}fluité de la dicte digestion\wdx{digestion}{f.
terme de méd.
`conversion des substances dans le corps
en sucs nécessaires pour l'assimilation et
la désassimilation (des aliments en chyle, le chyle en sang,
etc.)'}{} en p\emph{ro}pres\wdx{propre}{adj. 1\hoch{o} `qui
appartient exclusivement à (qn, qch.)'}{}
lieux,
qui s\emph{er}ont dis cy aprés. Et par ce que dit est
appert la accion\wdx{accion}{f. `tout ce que l'on fait (dit aussi de choses)'}{},
posicion\wdx{posicïon}{f. `lieu où quelque chose est placée,
située'}{posicion},
substance et la
colligance\wdx{colligance}{f. 1\hoch{o}
`force qui maintient réunis les éléments d'un système matériel;
liaison'}{}
du foye et les autres choses qui sont
enquises\wdx{enquerre}{v.tr. `chercher à savoir
(qch.) en examinant ou en interrogeant'}{enquises
\emph{p.p. f.pl.}}
es autres
me\emph{m}bres. Le foie peult
souffrir\wdx{*sofrir}{v.tr.
`supporter
qch. de pénible ou de désagréable;
souffrir'}{souffrir \emph{inf.}}
maintes\wdx{maint}{adj.
`plusieurs; maint'}{} malladies\wdx{maladie}{f.
 `altération organique ou
fonctionnelle considérée dans son évolution, et comme
une entité définissable; maladie'}{malladie}
par la accion\wdx{accion}{f. `tout ce que l'on fait (dit aussi de choses)'}{}
sanguine\wdx{sanguin}{adj. 2\hoch{o}
terme de méd. `qui produit du sang'}{} qui est sa
p\emph{ro}pre\wdx{propre}{adj. 1\hoch{o} `qui
appartient exclusivement à (qn, qch.)'}{}
op\emph{er}acion\wdx{operacïon}{f.
`action d'un pouvoir, d'une fonction,
d'un organe qui produit un effet selon sa
nature; opération'}{operacion},
peut\wdx{*pooir}{v.tr. +
inf. `avoir la possibilité de (faire qch.);
pouvoir'}{peut \emph{3.p.sg. ind.prés.}}
estre
blessé\wdx{*blecier}{v.tr. 1\hoch{o} `causer une
blessure' (à une partie du corps)}{blessé
\emph{p.p.}}
et devient cacochimes\wdx{*cacochyme}{adj. `qui a
une constitution appauvrie'}{cacochime}, c'est a
dire ydropices\wdx{*idropique}{adj. `qui est atteint
d'hydropisie'}{ydropice}, car
ydropizie\wdx{*idropisie}{f.
`accumulation morbide de sérosité dans une cavité
naturelle du corps; hydropisie'}{ydropizie}, c'est
erreur\wdx{*error}{f. `acte de l'esprit qui tient
pour vrai ce qui est faux et inversement;
erreur'}{erreur}
en digestion\wdx{digestion}{f. terme de méd.
`conversion des substances dans le corps
en sucs nécessaires pour l'assimilation et
la désassimilation (des aliments en chyle, le chyle en sang,
etc.)'}{} du foye,
selon
Galien\adx{Galien}{}{} ou second \flq Livre des Vertus\frq\
et ou .vj.\hoch{e} livre
qui se intitule \flq De egritudine et
sinthomate\frq .
Et appert p\emph{ar} ce que dit est que les
medicacions\wdx{medicacïon}{f. 2\hoch{o}
`substance employée pour traiter une
affection ou une manifestation morbide;
medecine'}{medicacion}
du foye doivent estre
appliquees\wdx{*apliquier}{v.tr.
`mettre une chose sur
une autre de manière qu'elle la recouvre et y adhère;
appliquer'}{appliqué
\emph{p.p.}}
au dextre lees\wdx{*lez}{m.
`partie qui est à droite
ou à gauche (d'un corps); côté'}{lees} et au
dextre costé et q\emph{ue}, pour cause de sa substance, elles
doivent avoir aucune
stipticité\wdx{stipticité}{f. `ce qui est
astringent, styptique'}{}.
\pend
\pstart
Aprés la anathomye\wdx{*anatomie}{f.
1\hoch{o} `structure
et composition du corps
humain et animal, et, en parlant dans un sens abstrait, science
de cette structure'}{anathomye}
du foie il nous co\emph{n}vient dire\wdx{dire}{v.tr.indir. \textbf{\emph{dire de}} `parler de'}{}
des parties ou sont
envoiees\wdx{envoiier}{v.tr. `faire aller qn ou qch.
(quelque part)' (le sujet n'étant pas
personnel)}{envoié \emph{p.p.}} les sup\emph{er}fluités
qui sont engendrees\wdx{engendrer}{v.tr. au
fig. `faire naître, faire exister; produire'}{}
au foye,
si co\emph{m}me dit est. Et premier du cistis
fellis\wdx{cistis fellis}{lt. `vésicule bilaire'}{}.
Cistis fellis\wdx{cistis
fellis}{lt. `vésicule bilaire'}{}, c'est une
bource\wdx{*borse}{f. 1\hoch{o} `structure organique
en forme de poche ou de sac arrondie'}{bource}
ou une vessie\wdx{vessie}{f. terme d'anat.
`réservoir
musculo-membraneux dans lequel s'accumule
l'urine qui arrive des reins par les
uretères; vessie'}{}, faicte ainsi que
panicle,
qui est mise\wdx{metre}{v.tr. 1\hoch{o} `placer
(qch.)
dans une position déterminée'}{} en la co\emph{n}cavité\wdx{concavité}{f.
`espace vide à
l'intérieur d'un corps solide; concavité'}{}
du
foy\wdx{foie}{m. terme d'anat. `organe
situé dans la partie supérieure droite de
l'abdomen et qui sécrète la bile;
foie'}{foy},
pres du moien\wdx{moien}{adj.
`qui est au milieu'}{}
lobe\wdx{lobe}{m.
terme d'anat. `partie arrondie et saillante de
divers organes; lobe'}{}
ou de la moienne\wdx{moien}{adj.
`qui est au milieu'}{}
pa\emph{n}ne\wdx{*pene}{f.
terme d'anat. `partie arrondie et saillante de divers
organes; lobe'}{panne}
 ou du moyen doy\wdx{doi}{m. 4\hoch{o}
`ce qui a la forme
d'un doigt' (par analogie de forme)}{doy}, po\emph{ur}
%
[32v\hoch{o}b]
recevoir\wdx{recevoir}{v.tr.
`faire entrer (qch.)'}{\emph{inf.}}
la sup\emph{er}fluité
colerique\wdx{colerique}{adj. terme de méd. `qui a
rapport à la bile jaune'}{}
\text{ordonnee}\fnb{Über der
Zeile nachgetragen.}/\wdx{*ordené}{adj.
`qui est préparé'}{ordonné} qui se
engendre\wdx{engendrer}{v.pron.
au fig. `naître; se produire'}{} au dit foye.
Laquelle bource\wdx{*borse}{f. 1\hoch{o}
`structure organique
en forme de poche ou de sac arrondie'}{bource} ha deux
oriffices\wdx{orifice}{m.
`ouverture faisant communiquer un conduit, un organe
avec une structure voisine ou avec l'extérieur;
orifice'}{oriffice}
ou deux coulz\wdx{col}{m. 2\hoch{o}
terme d'anat. `partie rétrécie
(d'une cavité organique)' (par analogie de forme)}{coulz \emph{m.pl.}} ou
tuelz\wdx{tuel}{m. terme d'anat. `conduit dans le
corps par lequel s'écoule un liquide'}{tuelz
\emph{pl.}}, unyz\wdx{unir}{v.tr. `mettre avec
ou mettre ensemble de manière à former un tout;
unir'}{unyz \emph{p.p. m.pl.}} et
joings\wdx{joindre}{v.tr.
`mettre des choses
ensemble, de façon qu'elles se touchent ou tiennent
ensemble; joindre'}{joings \emph{p.p. m.pl.}} assés
pres l'ung de l'autre -- selon Mestre\wdx{*maistre}{m. 3\hoch{o} appellation
devant le prénom en parlant à ou d'une
personne}{Mestre}
Mu\emph{n}din\adx{Mestre Mundin}{}{} --, et
l'u\emph{n}g s'en va vers le milieu du foie pour recevoir\wdx{recevoir}{v.tr.
`faire entrer (qch.)'}{\emph{inf.}}
la
cole. L'autre va au fons de l'estomac\wdx{*fonz
(de l'estomac)}{m. `partie de l'estomac opposée à
l'entrée de l'\oe sophage'}{fons de l'estomac} et aux
intestins
pour mo\emph{n}diffier\wdx{*mondefiier}{v.tr. `rendre
pur; purifier'}{mondiffier \emph{inf.}}
et expellir la cole de la
dicte bource\wdx{*borse}{f. 1\hoch{o}
`structure organique
en forme de poche ou de sac arrondie'}{bource} pour
les utilités\wdx{utilité}{f.
`caractère
de ce qui est utile; utilité'}{}
devant dictes.
Et par ce appert le siege\wdx{siege}{m.
`lieu où qch.
réside'}{}
et le lieu et
l'accion\wdx{accion}{f. `tout ce que l'on fait (dit aussi de choses)'}{}, la
substance et la fourme, les p\emph{ar}ties et les
colligances\wdx{colligance}{f. 1\hoch{o}
`force qui maintient réunis les éléments d'un système matériel;
liaison'}{}.
La quantité\wdx{*cantité}{f. `nombre
d'unités ou mesure qui sert à déterminer une
portion de matière ou une collection de choses
considérées comme
homogènes'}{quantité}, on
la peult bien veoir, que elle tieng
plain\wdx{plain}{adj. `qui est plat, égal; plain'}{}
voirre\wdx{voirre}{m. `substance transparente qui
ressemble à la glace, au verre'}{}
\text{environ}\fnb{Ms. \emph{on
environ}.}/\wdx{environ}{adv. 1\hoch{o} `dans
l'espace environnant; alentour'}{}.
Et aussi vous poués considerer\wdx{considerer}{v.tr.
`regarder (qch.) attentivement;
considérer'}{}
des
maladies, que elle peult estre
opilee\wdx{opiler}{v.tr. terme de méd. `obstruer
(un conduit naturel dans l'organisme)'}{opilé
\emph{p.p.}} et estoupee\wdx{*estoper}{v.tr.
`boucher
(qch.)'}{estoupé \emph{p.p.}}
en son col p\emph{ro}pre\wdx{propre}{adj. 1\hoch{o} `qui
appartient exclusivement à (qn, qch.)'}{}
et au co\emph{m}mun\wdx{*comun}{adj. `qui appartient
à plusieurs personnes ou
choses'}{commun}.
Et quant le col co\emph{m}mun
est estoupé\wdx{*estoper}{v.tr. `boucher
(qch.)'}{estoupé \emph{p.p.}}, lors la cole n'est
pas actraite\wdx{*atraire}{v.tr. 2\hoch{o} `amasser
(qch.) en quantité'}{actraite
\emph{p.p. f.sg.}} ne aussi
expellee, ains\wdx{*ainz}{conj. `plutôt, de
préférence'}{ains} elle demeure avec le
sang et fait l'orine\wdx{orine}{f. `liquide organique
clair et ambré qui se forme dans le rein,
qui séjourne dans la vessie et qui est evacué par
l'urètre'}{}
citrine\wdx{citrin}{adj. `qui est de la couleur du
citron; citrin'}{} et
tout le corps, quant on n'y mect p\emph{ro}pres\wdx{propre}{adj. 1\hoch{o} `qui
appartient exclusivement à (qn, qch.)'}{}
aides\wdx{aide}{f. 1\hoch{o}
`action d'aider quelqu'un, concours que l'on prête,
soutien moral ou secours matériel que l'on
apporte; aide'}{} pour
les me\emph{m}bres ou la dicte colle\wdx{cole}{f. terme de méd.
`bile jaune (l'une des quatre
humeurs de l'humorisme')}{colle}
\text{va}\fnb{Voranstehend expungiertes
\emph{est}.}/, et
se engendre\wdx{engendrer}{v.pron.
au fig. `naître; se produire'}{} mauvais
accidens,
selon Galien\adx{Galien}{}{} ou .vj.\hoch{e} du
livre \flq De egritudine et
sintho\emph{ma}te\frq\
et ou q\emph{ui}nt livre de
\flq Interioribus\frq .
\pend
\pstart
L'esplein\wdx{esplein}{m. terme d'anat.
`organe
lymphoïde situé sous la partie gauche du
diaphragme; rate'}{}, c'est le
receptacle\wdx{receptacle}{m. `contenant qui
reçoit son contenu de diverses provenances;
réceptacle'}{} de
la sup\emph{er}fluité\wdx{superfluité}{f. terme de méd.
`sécrétion abondante du corps'}{} de merancolie
qui se enge\emph{n}dre\wdx{engendrer}{v.pron.
au fig. `naître; se produire'}{}
au
foie, et enbrasse\wdx{*embracier}{v.tr. `être autour
de (qch.) de manière à enfermer; entourer'}{enbrasse
\emph{3.p.sg. ind.prés.}} de
travers\wdx{travers}{prép.}{\textbf{de travers}
\emph{loc.prép. `dans une direction transversale de'}}
l'estomac en la partie
senextre, et est sa substance
rare\wdx{*rer}{adj. `qui n'est pas dense'}{rar}
et clere\wdx{cler}{adj. `qui n'est pas dense'}{} et
spongieuse\wdx{*spongios}{adj. `dont
la structure ressemble à celle de l'éponge;
spongieux'}{spongieuse
\emph{f.sg.}}
et plus noire\wdx{noir}{adj. `qui est d'une
couleur foncée'}{} que le foie,
et ha figure\wdx{figure}{f. `forme extérieure d'un corps'}{}
oblongue\wdx{oblong}{adj.
`qui est plus long que large; oblong'}{oblongue
\emph{f.sg.}} et
estroicte\wdx{*estroit}{adj. `qui a peu de largeur;
étroit'}{estroicte \emph{f.sg.}},
et est lié par son panicle avec les costes\wdx{coste}{f. terme d'anat.
`os plat et courbe du thorax qui
s'articule sur la colonne vertébrale et le sternum; côte'}{}
vers la
%
[33r\hoch{o}a]
bosse\wdx{*boce}{f.
terme
d'anat. `saillie à la surface d'une structure
anatomique'}{bosse}, c'est a
entendre\wdx{entendre}{v.tr. `saisir par l'intelligence'}{} que la
bosse\wdx{*boce}{f.
terme
d'anat. `saillie à la surface d'une structure
anatomique'}{bosse} de
l'esplein\wdx{esplein}{m. terme d'anat.
`organe
lymphoïde situé sous la partie gauche du
diaphragme; rate'}{}
est vers les costes\wdx{coste}{f. terme d'anat. `os plat
et courbe du thorax qui s'articule
sur la colonne vertébrale et le sternum; côte'}{}
lié
du dit panicle, et
la concavité est lié\wdx{concavité}{f.
`espace vide à
l'intérieur d'un corps solide; concavité'}{}
a
l'estomac et au zirbum\wdx{zirbus}{m.
terme d'anat.
`repli du péritoine qui relie entre eux
les organes abdominaux; épiploon'}{zirbum}.
Item, l'esplein ha deux conduis\wdx{conduit}{m. `canal
ou tuyau qui sert à l'ecoulement ou au
transport d'une matière (un liquide, l'air, un
gaz, etc.)'}{conduis
\emph{pl.}}: par l'ung
elle trait\wdx{traire}{v.tr.
`faire venir
dans une certaine direction (qn, qch.)'}{} la
dicte sup\emph{er}fluité du foye, par l'autre
conduit\wdx{conduit}{m. `canal
ou tuyau qui sert à l'ecoulement ou au
transport d'une matière (un liquide, l'air, un
gaz, etc.)'}{}
elle
l'envoie\wdx{envoiier}{v.tr. `faire aller qn ou qch.
(quelque part)' (le sujet n'étant pas
personnel)}{envoie \emph{3.p.sg. ind.prés.}} a
l'estomac pour les aides\wdx{aide}{f. 2\hoch{o}
`caractère de ce qui est
utile, qui satisfait un besoin; utilité'}{} devant dictes.
L'esplein peult souffrir\wdx{*sofrir}{v.tr.
`supporter
qch. de pénible ou de désagréable;
souffrir'}{souffrir \emph{inf.}}
maintes\wdx{maint}{adj. `plusieurs; maint'}{}
maladies, c'est
assavoir duresse\wdx{*durece}{f. `action
de se durcir (pathologiquement),
le resultat de cette action (dit d'un organe ou d'une
partie du corps)'}{duresse} et
opilacion\wdx{opilacïon}{f. terme de méd.
`obstruction des conduits naturels (dans
l'organisme)'}{opilacion}
pour la dicte matere\wdx{matiere}{f. 1\hoch{o} `substance
qui constitue les corps, qui est objet
d'intuition dans l'espace et qui possède une
masse mécanique'}{matere}.
Et se p\emph{ar} ce elle ne se povoit
mu\emph{n}diffier\wdx{*mondefiier}{v.pron. `se
rendre pur; se purifier'}{mundiffier \emph{inf.}}
ne purgier\wdx{purgier}{v.pron. `se débarasser
d'impuretés
nuisibles à la santé (en parlant de l'organisme)'}{}, le corps
devie\emph{n}ndroit denué\wdx{denüer}{v.tr. `priver de'}{}
et descoloré\wdx{descoloré}{adj. `qui a perdu sa
couleur'}{}. Et se l'autre oriffice estoit
estoupés\wdx{*estoper}{v.tr. `boucher
(qch.)'}{estoupé \emph{p.p.}}
et que elle ne le peult envoier a l'estomac, le
apetit\wdx{apetit}{m. `désir de manger'}{}
en s\emph{er}oit blessé\wdx{*blecier}{v.tr. 2\hoch{o} au fig
`causer une impression désagréable, pénible'}{blessé
\emph{p.p.}}, si
co\emph{m}me il est escript es livres devant dis. Item, les
solucions de continuité\wdx{solucion
de continuité}{f. terme de méd. `séparation des
tissus qui sont normalement continus' (p.ex.
plaie, fracture)}{} ne
y sont pas si perilleuses\wdx{*perillos}{adj.
`qui constitue un danger, présente du
danger; dangereux'}{perilleuses \emph{f.pl.}}
en l'esplein co\emph{m}me au foye.
Et l'esplein soustient\wdx{*sostenir}{v.tr. `tenir (qch.)
par-dessous en servant de support ou
d'appui; soutenir'}{}
mieux plus fortes\wdx{fort}{adj. 2\hoch{o} `qui agit
avec force, qui est capable de grands effets; efficace'}{fortes
\emph{f.pl.}} medicines\wdx{*medecine}{f.
`substance employée pour traiter une affection ou
une manifestation morbide'}{medicine}
que ne fait le foye. On le purge\wdx{purgier}{v.tr.
2\hoch{o} terme de méd.
`expulser (qch.) de l'organisme' (des
substances, des impuretés nuisibles à la
santé)}{purge \emph{3.p.sg. ind.prés.}}
p\emph{ro}premant\wdx{*proprement}{adv. `d'une
manière précise'}{propremant}
par le ventre et le medicin\wdx{*medecin}{m. 2\hoch{o}
`substance employée pour traiter une affection ou
une manifestation morbide'}{medicin}
en vers le costé
senestre\wdx{senestre}{adj. `qui est du
côté gauche; gauche'}{}, si
co\emph{m}me Galien\adx{Galien}{}{} le dit au .xiij.\hoch{e} de
\flq Terape\emph{u}tiq\emph{ue}\frq\wdx{*therapeutique}{f.
terme de méd.
`partie de la médecine qui étudie et
met en application les moyens propres à guerir et à
soulager les malades'}{terapeutique}.
\pend
\pstart
Les
\text{reins}\fnb{Ms. \emph{Reins}.}/\wdx{rein}{m.
1\hoch{o} terme d'anat. `chacun des deux organes
sécréteurs
glandulaires situés symétriquement dans les fosses
lombaires et qui élaborent l'urine'}{}
ou les
rognons\wdx{*reignon}{m.
terme d'anat.
`chacun des deux
organes sécréteurs glandulaires situés
symétriquement dans les fosses lombaires et
qui élaborent l'urine; rein'}{rognon} sont
petites\wdx{petit}{adj. 1\hoch{o} dans l'ordre
physique, quantifiable `qui est d'une extension
au-dessous de la moyenne; petit (des choses)'}{}
parties
ordonnees\wdx{*ordener}{v.tr. 2\hoch{o}
`établir (qn, qch.) pour une foncion'}{ordonné
\emph{p.p.}}
pour
mu\emph{n}difier\wdx{*mondefiier}{v.tr. `rendre
pur; purifier'}{mundifier \emph{inf.}} les sup\emph{er}\-fluités
aigouses\wdx{*ewos}{adj. `qui est de la nature
de l'eau; aqueux'}{aigouse \emph{f.sg.}}
ou aq\emph{ua}tiques\wdx{aquatique}{adj. `qui est de la
nature de l'eau'}{} qui viennent du sang. Et so\emph{n}t
deux rongnons\wdx{*reignon}{m. terme d'anat.
`chacun des deux
organes sécréteurs glandulaires situés
symétriquement dans les fosses lombaires et
qui élaborent l'urine; rein'}{rongnon}, l'ung a
dextre\wdx{*destre}{f. `le côté droit;
droite'}{\textbf{a destre}
\emph{loc.adv. `sur le côté droit'} a dextre},
l'autre a senextre\wdx{senestre}{m. et f.
`le côté
gauche'}{\textbf{a senextre} \emph{loc.adv. `sur le
côté gauche'}}. Le premier est pres du foye et est plus hault que
l'autre; leur substance est
charneuse et dure\wdx{dur}{adj. `qui résiste à la pression, qui ne se
laisse pas déformer facilement'}{}
et ont
\text{fourme}\fnb{Nachfolgend expungiertes
\emph{dure}.}/ ronde, ainsi que ung
euf\wdx{*uef}{m. `corps calcaire arrondie
que produisent les femelles des oiseaux et qui
contient le germe de l'embryon et les substances
destinés à le nourrir pendant l'incubation; \oe uf'}{euf}
compressé\wdx{compresser}{v.tr. `exercer une
pression sur (qch.) en changeant la forme;
compresser'}{compressé \emph{p.p.}}, et ont
co\emph{n}cavités\wdx{concavité}{f.
`espace vide à
l'intérieur d'un corps solide; concavité'}{}
dedans eulx, ou est mis\wdx{metre}{v.tr. 1\hoch{o}
`placer (qch.) dans une position déterminée'}{} ce
qu'ilz
%
[33r\hoch{o}b]
actraient\wdx{*atraire}{v.tr.
2\hoch{o} `amasser
(qch.) en quantité'}{actraient
\emph{3.p.pl. ind.prés.}}. Item, en ch\emph{acu}m
d'eulx
sont deux colz: par l'ung est
actraicte\wdx{*atraire}{v.tr.
1\hoch{o}
`amener (qn, qch.) vers soi ou quelque
part'}{actraicte \emph{p.p. f.sg.}} la
aquousité\wdx{aquosité}{f.
`ce qui
est de la nature aqueuse, aussi la qualité de ce qui
est de la nature de l'eau'}{aquousité} de la vaine
kily\wdx{*veine kilis}{f.
terme
d'anat. `vaisseau sanguin qui ramène le sang
nutritif du foie à tout le corps'}{vaine kily}
et, par co\emph{n}sequent\wdx{consequent}{adj.}{\textbf{par
consequent}
\emph{loc.adv. `comme suite logique'}}, du foye.
Et par l'autre, ycelle
\mbox{aquosité}\wdx{aquosité}{f.
`ce qui
est de la nature aqueuse, aussi la qualité de ce qui
est de la nature de l'eau'}{} qui est
appellee urine\wdx{orine}{f. `liquide organique
clair et ambré qui se forme dans le rein,
qui séjourne dans la vessie et qui est evacué par
l'urètre'}{urine}, elle va
a lla
vecie\wdx{vessie}{f. terme d'anat.
`réservoir
musculo-membraneux dans lequel s'accumule
l'urine qui arrive des reins par les
uretères; vessie'}{vecie}. Item, aus ditz
rongno\emph{n}s\wdx{*reignon}{m. terme d'anat.
`chacun des deux
organes sécréteurs glandulaires situés
symétriquement dans les fosses lombaires et
qui élaborent l'urine; rein'}{rongnon} ou reins viennent
vaines, arteres, nerfz,
desquelx sont fais les panicles qui les coevrent, et
sont liés ou dos. Et les
rongnons\wdx{*reignon}{m. terme d'anat.
`chacun des deux
organes sécréteurs glandulaires situés
symétriquement dans les fosses lombaires et
qui élaborent l'urine; rein'}{rongnon}
ont graisse ainsi
que cien\wdx{*saim}{m. `substance onctueuse, de fusion facile,
répartie en diverses parties du corps de l'homme et des
mammifères'}{cien} tout entour. Et
derrir\wdx{derrier}{prép. `en arrière de'}{derrir}
\text{les}\lemma{derrir les}\fnb{Ms. \emph{derrir
les}, Zeilenumbruch und erneutes
\emph{les}.}/
rongnons\wdx{*reignon}{m. terme d'anat.
`chacun des deux
organes sécréteurs glandulaires situés
symétriquement dans les fosses lombaires et
qui élaborent l'urine; rein'}{rongnon},
pres les spondilles, passent la vaine
kilis\wdx{*veine kilis}{f.
terme
d'anat. `vaisseau sanguin qui ramène le sang
nutritif du foie à tout le corps'}{vaine kilis} et
la vaine aborchi\wdx{*veine aborthi}{f.
terme d'anat. `vaisseau sanguin qui naisse au
c\oe ur'}{vaine
aborchi} et vont aux
membres de dessoubz\wdx{*desoz}{adv. `à la face
inférieure'}{\textbf{de dessoubz} \emph{loc.adj.
`qui est situé au côté inférieur'}}.
Desqueulx vaines assés pres des
rongnons\wdx{*reignon}{m. terme d'anat.
`chacun des deux
organes sécréteurs glandulaires situés
symétriquement dans les fosses lombaires et
qui élaborent l'urine; rein'}{rongnon}
naissent les vaisseaux\wdx{vaissel}{m. 2\hoch{o} terme
d'anat.
`organe
ou canal du corps qui contient ou dans lequel circule
un liquide organique'}{vaisseaux \emph{pl.}}
sp\emph{er}matiques\wdx{spermatique}{adj. terme de méd.
`qui a rapport au sperme'}{}, desqueulx nous
dirons cy aprés. Les reins\wdx{rein}{m. 1\hoch{o}
terme d'anat. `chacun des deux organes
sécréteurs
glandulaires situés symétriquement dans les fosses
lombaires et qui élaborent l'urine'}{}
peullent souffrir\wdx{*sofrir}{v.tr.
`supporter
qch. de pénible ou de désagréable;
souffrir'}{souffrir \emph{inf.}}
et
avoir pluseurs maladies, maiema\emph{n}t\wdx{*maismement}{adv. `plus que tout
autre chose; surtout'}{maiemant}
opilacions\wdx{opilacïon}{f. terme de méd.
`obstruction des conduits naturels (dans
l'organisme)'}{} et
pierres\wdx{piere}{f. 2\hoch{o} terme de méd.
`amas de gravier qui se forme dans certains organes (le rein, la
vessie, etc.)' (par analogie de forme)}{pierre}, et la maniere de
curer\wdx{curer}{v.tr.
`soumettre à un traitement médical'}{}
est moult difficille\wdx{*dificile}{adj.
`qui
n'est pas facile; difficile'}{difficille},
tant de l'un co\emph{m}me de l'autre. Et quant tu auras
tout cela veu, tu pourras tout
oster\wdx{oster}{v.tr.
`enlever (qch.) de la place
qu'il occupait; ôter'}{} excepté\wdx{excepté}{prép.
`à la réserve
de'}{}
l'estomac -- ou cas q\emph{ue} tu vouldroies fere la
anathomie
des me\emph{m}bres de dessus\wdx{*desus}{adv. `au côté
supérieur'}{\textbf{de dessus} \emph{loc.adj. `qui
est situé au côté supérieur'}} --
\text{et}\fnb{Über der Zeile
nachgetragen.}/
excepté\wdx{excepté}{prép.
`à la réserve
de'}{}
les ditz rongnons\wdx{*reignon}{m. terme d'anat.
`chacun des deux
organes sécréteurs glandulaires situés
symétriquement dans les fosses lombaires et
qui élaborent l'urine; rein'}{rongnon},
pour veoir la anathomie des parties et des me\emph{m}bres de
dessoubz\wdx{*desoz}{adv. `à la face
inférieure'}{\textbf{de dessoubz} \emph{loc.adj.
`qui est situé au côté inférieur'}}.
Et adonc tu regarderas le nombre\wdx{nombre}{m.
`mot servant à caractériser une pluralité de
choses ou de personnes; nombre'}{}
et la
quantité des spondilles et tu y trouveras\wdx{*trover}{v.tr. `rencontrer qn ou qch. qu'on cherche;
trouver'}{trouveras \emph{2.p.sg. ind.futur}}
.v. plus
grosses\wdx{gros}{adj.
1\hoch{o}
dans l'ordre physique, quantifiable `qui, dans
son genre, dépasse le volume ordinaire; gros (du corps
humain et de ses parties)'}{}
des aultres, par lesquelles descendent .v.
paire\wdx{paire}{f.
`réunion de deux choses,
de deux êtres semblables
qui vont ensemble; paire'}{}
de nerfz qui
viennent de la nuque
\text{a}\fnb{\emph{Nota} am Foliorand.}/ tout le
ve\emph{n}tre et aux parties des cuisses\wdx{cuisse}{f. `partie
de la jambe qui s'articule à la hanche et s'étend
jusqu'au genou; cuisse'}{} et des
grans piés\wdx{grant pié}{m. terme d'anat. `membre inférieur en entier
de l'homme y compris le pied'}{}.
\pend
%
% \memorybreak
%
\pstartueber
[33v\hoch{o}a]
Le .vij. chappitre parle de
l'anathomie des
hanches\wdx{hanche}{f. `chacune des deux parties
du corps formant saillie au-dessous des flancs,
entre la fesse en arrière et le pli de l'aine
en avant; hanche'}{} et de ces parties.
\pendueber
%
% \memorybreak
%
\pstart
PAR LES hanches\wdx{hanche}{f. `chacune des deux parties
du corps formant saillie au-dessous des flancs,
entre la fesse en arrière et le pli de l'aine
en avant; hanche'}{}
on entent\wdx{entendre}{v.tr.
`saisir par l'intelligence'}{}
les basses parties du ventre en
co\emph{m}prenant\wdx{comprendre}{v.tr. 1\hoch{o}
`contenir (qch.) en soi comme partie de
l'ensemble'}{comprenant \emph{p.prés.}}
le vit\wdx{vit}{m. `organe de la copulation chez
l'homme'}{}, les
coillons\wdx{coillon}{m. 1\hoch{o} `gonade mâle suspendue dans le scrotum, qui
produit les spermatozoïdes; testicule'}{}, le
con\wdx{con}{m.
terme d'anat.
`ensemble des organes génitaux
externes de la femme; vulve'}{} et le
cul\wdx{cul}{m.
`partie postérieure chez
l'homme; cul'}{}
jusques aux cuisses\wdx{cuisse}{f. `partie
de la jambe qui s'articule à la hanche et s'étend
jusqu'au genou; cuisse'}{}. Desqueulx hanches\wdx{hanche}{f. `chacune des deux parties
du corps formant saillie au-dessous des flancs,
entre la fesse en arrière et le pli de l'aine
en avant; hanche'}{}
les p\emph{ar}ties
sont triples\wdx{triple}{adj.
`qui
équivaut à trois, se présente comme trois;
triple'}{},
c'est a dire devisees en trois parties: aucunes
contiennent, aucunes sont contenues et aucunes
issent hors. Les parties qui co\emph{n}tie\emph{n}nent, c'est
mirac\wdx{mirac}{m.
terme d'anat. `membrane séreuse
qui tapisse les parois
intérieures de la cavité abdominale et pelvienne;
péritoine pariétal'}{}
et ciphac et
zirbus\wdx{zirbus}{m.
terme d'anat.
`repli du péritoine qui relie entre eux
les organes abdominaux; épiploon'}{}
et les os. Les parties qui
so\emph{n}t contenues, c'est la vessie\wdx{vessie}{f.
terme d'anat.
`réservoir
musculo-membraneux dans lequel s'accumule
l'urine qui arrive des reins par les
uretères; vessie'}{}
et les
vaiceaulx\wdx{vaissel}{m. 2\hoch{o} terme d'anat.
`organe
ou canal du corps qui contient ou dans lequel circule
un liquide organique'}{vaiceaulx \emph{pl.}}
sp\emph{er}matiques\wdx{spermatique}{adj. terme de méd.
`qui a rapport au sperme'}{}
et la matrice\wdx{matrice}{f.
`organe situé dans la
cavité pelvienne destiné à contenir l'\oe uf
fécondé jusqu'à son complet développement; utérus'}{} es
fe\emph{m}mes\wdx{*feme}{f.
`être humain du sexe
féminin; femme'}{femme} et
longaon\wdx{longäon}{m.
terme d'anat.
`portion terminale du gros intestin qui
s'étend du côlon jusqu'à l'anus; rectum'}{longaon} ou le
droit intestin\wdx{intestin droit}{m. terme d'anat.
`partie terminale du gros intestin
entre le
côlon et l'anus; rectum'}{droit intestin}, les nerfz, les vaines, les arteres
qui descendent bas. Et les p\emph{ar}ties q\emph{ue} procedent\wdx{proceder}{v.tr.indir.
`avoir son origine
dans; provenir'}{procedent \emph{3.p.pl. ind.prés.}}
\text{et}\fnb{Nachfolgend expungiertes \emph{qui}.}/
vont par dehors, c'est assavoir le
dindime\wdx{*didime}{m. terme d'anat. `membrane qui
s'étend du péritoine au scrotum et qui enveloppe le
conduit séminal'}{dindime}, les
coillons\wdx{coillon}{m. 1\hoch{o} `gonade mâle suspendue dans le scrotum, qui
produit les spermatozoïdes; testicule'}{},
le vit\wdx{vit}{m. `organe de la
copulation chez l'homme'}{}, les
inguines\wdx{inguine}{m. `partie latérale et inférieure du
bas-ventre; aine'}{},
le peritoneum\wdx{peritoneum}{lt.
terme d'anat. `membrane séreuse
qui tapisse les parois
intérieures de la cavité abdominale et pelvienne et
qui recouvre les organes contenus dans
la cavité abdominale et pelvienne, à l'exception de
l'ovaire'; péritoine'}{}, les
nages\wdx{*nache}{f. `chacune des deux parties
charnues de la région du bassin, dans l'espèce
humaine et chez certains mammifères; fesse'}{nage} et
les muscules
qui descendent a lla cuisse\wdx{cuisse}{f. `partie
de la jambe qui s'articule à la hanche et s'étend
jusqu'au genou; cuisse'}{},
desqueulx no\emph{us} dirons ci aprés. Mais des parties qui
contiennent, quant est de mirac\wdx{mirac}{m.
terme d'anat. `membrane séreuse
qui tapisse les parois
intérieures de la cavité abdominale et pelvienne;
péritoine pariétal'}{}
et de ciphac et du
zirbu\emph{m}\wdx{zirbus}{m.
terme d'anat.
`repli du péritoine qui relie entre eux
les organes abdominaux; épiploon'}{zirbum},
dit est cy devant ou chapp\emph{itre} du ventre.
\pend
\pstart
Quant est des os, tu dois savoir q\emph{ue} es hanches\wdx{hanche}{f.
`chacune des deux parties
du corps formant saillie au-dessous des flancs,
entre la fesse en arrière et le pli de l'aine
en avant; hanche'}{}
sont
deux manieres\wdx{maniere}{f. 1\hoch{o} `nature
propre
à plusieurs personnes ou choses, qui permet de les
considérer comme appartenant à une catégorie
distincte'}{} d'os: premier,
vers la partie
du dos, en ha trois ou quatre spondilles de l'os
sacré\wdx{os sacré}{m. terme d'anat. `os formé par la réunion des cinq
vertèbres sacrées, a la partie inférieure de la
colonne vertébrale; sacrum'}{} et
deux ou trois cartillagineux qui
%
[33v\hoch{o}b]
sont de l'os de la queue\wdx{os de la queue}{m.
`extrémité
inférieure de la colonne vertébrale, articulée avec le
sacrum et formé de trois petit os; coccyx'}{}.
Desqueulx, le
premier de l'os sacré\wdx{os sacré}{m. terme d'anat. `os formé par la réunion des cinq
vertèbres sacrées, a la partie inférieure de la
colonne vertébrale; sacrum'}{} est
moult gros\wdx{gros}{adj.
1\hoch{o}
dans l'ordre physique, quantifiable `qui, dans
son genre, dépasse le volume ordinaire; gros (du corps
humain et de ses parties)'}{}
et les aultres,
qui viennent \text{aprés}\fnb{Nachfolgend
gestrichenes \emph{de}.}/, vont en
diminuant\wdx{diminüer}{v.intr. `devenir moins
grand ou moins considérable; diminuer'}{diminuant \emph{p.prés.}}, en descendant vers le
cul\wdx{cul}{m.
`partie postérieure chez
l'homme; cul'}{} et vers la fin
du
dos. Et les troux d'iceulx, par ou passent les nerfz,
sont
devant\wdx{devant}{adv. 1\hoch{o} `au
côté du visage, à la face'}{}
et non pas au costé, et aussi es autres os du
dos. Et devers\wdx{devers}{prép.
`du côté de'}{} les costés sont deux grans os, en
ch\emph{acu}m costé ung, et se
co\emph{n}join\mbox{gnent}\wdx{conjoindre}{v.pron.
`se mettre ensemble de manière à se
toucher ou tenir ensemble'}{conjoingnent
\emph{3.p.pl. ind.prés.}}
avec ce grant\wdx{grant}{adj. 2\hoch{o}
dans l'ordre
physique, quantifiable `qui est d'une extension
au-dessus de la moyenne; grand (du corps humain
et de ses parties)'}{grant \emph{f.sg.}}
spondille de l'os sacré\wdx{os sacré}{m. terme d'anat. `os formé par la réunion des cinq
vertèbres sacrées, a la partie inférieure de la
colonne vertébrale; sacrum'}{} par
derrier\wdx{derrier}{adv. `du
côté opposé au visage, à la face'}{}
et, par devant\wdx{devant}{adv. 1\hoch{o}
`au côté du visage, à la face'}{},
ou penil\wdx{penil}{m. terme d'anat. `eminence large
et arrondie, située au-devant du pubis'}{} ou au
pectiné\wdx{pectiné}{m. terme d'anat. `partie antérieure de
l'os iliaque; pubis'}{}; et
sont ces
deux os larges\wdx{large}{adj.
`qui a une
étendue supérieure à la moyenne dans le sens de la
largeur; large'}{} vers les
entrailles\wdx{entrailles}{f.pl.
terme d'anat.
`organes enfermés dans
l'abdomen de l'homme ou des
animaux; intestins'}{}, et \text{au milieu d'iceux}\fnb{Ms.
\emph{aux}{\dots}, \emph{x} expungiert;
\emph{d'iceux} am Zeilenrand
nachgetragen.}/, par dehors, sont
concavités\wdx{concavité}{f.
`espace vide à
l'intérieur d'un corps solide; concavité'}{}
appellees
boites\wdx{boite}{f. terme d'anat. `concavité d'un os
dans laquelle s'emboîte un autre os'}{}, ou sont
resseux\wdx{recevoir}{v.tr.
`faire entrer
(qch.)'}{resseux \emph{p.p. m.pl.}} les
tournans\wdx{tournant}{m. terme
d'anat. `extrémité arrondie d'un os'}{} des os
des cuisses, et aprés, tantost\wdx{tantost}{adv.
`dans un temps prochain, un
proche avenir; tantôt'}{}
bas vers le
cul\wdx{cul}{m.
`partie postérieure chez
l'homme; cul'}{}, ha
en ch\emph{acu}m ung g\emph{ra}nt\wdx{grant}{adj. 1\hoch{o}
dans l'ordre
physique, quantifiable `qui est d'une extension
au-dessus de la moyenne; grand (des choses)'}{}
trou, de quoy dit
Galien\adx{Galien}{}{} ou
.xvj.\hoch{e}
livre et ou .ix.\hoch{e} chappitre du livre qui se
intitule
\flq De utilitate particula\emph{rum}\frq\ ou il dit ainsi: ou
milieu
du chief de la cuisse\wdx{cuisse}{f. `partie
de la jambe qui s'articule à la hanche et s'étend
jusqu'au genou; cuisse'}{} et
de l'os du penil\wdx{penil}{m. terme d'anat. `eminence large et
arrondie, située au-devant du pubis'}{} fut
necessité\wdx{necessité}{f.
`caractère nécessaire
(d'une chose, d'une action); nécessité'}{}
de y estre ung grant\wdx{grant}{adj.
1\hoch{o} dans l'ordre
physique, quantifiable `qui est d'une extension
au-dessus de la moyenne; grand (des choses)'}{}
trou et ung che\emph{m}in\wdx{chemin}{m. `voie qui
permet d'aller d'un lieu à un autre; chemin'}{}, par
lequel descendissent les nerfz et les muscules, les
vaines et les arteres qui se
ferment\wdx{fermer}{v.pron.
`être attaché (à qch.)'}{}
de hault en bas\wdx{bas}{adv. `à faible hauteur;
bas'}{\textbf{en bas} \emph{loc.adv. `vers la terre;
vers le bas'}},
et vers le penil\wdx{penil}{m. terme d'anat. `eminence large et
arrondie, située au-devant du pubis'}{} ilz sont
estrois\wdx{*estroit}{adj.
`qui a peu de largeur; étroit'}{estrois \emph{m.pl.}}
en maniere d'une branche\wdx{branche}{f.
1\hoch{o} `ramification latérale de la tige
ligneuse de l'arbre; branche'}{} et se
joingnent\wdx{joindre}{v.pron.
`se mettre ensemble de
manière à se toucher ou tenir ensemble'}{joingnent
\emph{3.p.pl. ind.prés.}}
ensamble\wdx{ensemble}{adv. `l'un avec l'autre'}{ensamble} au
pe\emph{n}nil\wdx{penil}{m. terme d'anat. `eminence large et arrondie,
située au-devant du pubis'}{pennil}. Et ja soit ce q\emph{ue} se
soit ung seul\wdx{*sol}{adj. `qui n'est pas
avec d'autres semblables; seul'}{seul}
os
realmant\wdx{*rëelment}{adv. `d'une
manière réelle; réellement'}{realmant},
\text{toutes\-voies}\fnb{Über der Zeile nachgetragen.}/
il ha trois
noms\wdx{nom}{m.
`mot
servant à désigner les êtres, les choses
qui appartiennent à une même catégorie
logique'}{}. Et
aucuns dient que ce sont trois os. On l'appelle en
hault l'os des
yliés\wdx{*ilier}{m. `parties latérales et
inférieures du bas-ventre'}{ylié}, et au lieu bas
l'os du
pectiné\wdx{pectiné}{m. terme d'anat. `partie antérieure de
l'os iliaque; pubis'}{} ou du
penil\wdx{penil}{m. terme d'anat. `eminence large et arrondie,
située au-devant du pubis'}{} et au milieu
%
[34r\hoch{o}a]
l'os de la cuisse.
\pend
\pstart
Voicy les
\text{parties}\fnb{Ms. \emph{parti ties}.}/ qui y
sont \text{contenues}\fnb{Ms. \emph{contetnues}.}/,
c'est assavoir la vacie\wdx{vessie}{f. terme d'anat.
`réservoir
musculo-membraneux dans lequel s'accumule
l'urine qui arrive des reins par les
uretères; vessie'}{vacie} --
qui est ainsi que une pel\wdx{pel}{f. 2\hoch{o}
terme d'anat.
`couche de tissu qui enveloppe un organe ou qui
tapisse une cavité ou un conduit naturel;
membrane'}{} ou ung
sac\wdx{sac}{m. terme d'anat. `cavité ou enveloppe
dans le corps en forme de poche'}{}
-- qui ressoit la superfluité
de l'orine\wdx{orine}{f. `liquide organique
clair et ambré qui se forme dans le rein,
qui séjourne dans la vessie et qui est evacué par
l'urètre'}{} qui y
coule\wdx{*coler}{v.tr.indir. `se mouvoir
naturellement; couler (dit d'un liquide)'}{coule
\emph{3.p.sg. ind.prés.}} des reins\wdx{rein}{m.
1\hoch{o} terme d'anat. `chacun des deux organes
sécréteurs
glandulaires situés symétriquement dans les fosses
lombaires et qui élaborent l'urine'}{}
ou
des rongnons\wdx{*reignon}{m. terme d'anat.
`chacun des deux
organes sécréteurs glandulaires situés
symétriquement dans les fosses lombaires et
qui élaborent l'urine; rein'}{rongnon}.
Sa substa\emph{n}ce est ainsi que ung panicle
fort\wdx{fort}{adj. 1\hoch{o}
`qui résiste; fort (de
choses)'}{}
et est faicte de deux panicles, sa fourme est
ronde
et contient environ\wdx{environ}{adv. 2\hoch{o} `à
peu près'}{} ung pichié\wdx{pichier}{m.
`vase destinée à
contenir de la boisson'}{pichié}, et est assize\wdx{asseoir}{v.tr. 2\hoch{o} `placer, poser
(qch.)'}{assize \emph{p.p. f.sg.}}
droit\wdx{droit}{adv. `d'une manière exacte'}{} dessoubz
la
pectiné\wdx{pectiné}{m. terme d'anat.
`partie antérieure de l'os iliaque; pubis'}{}. Et
dedans ceste vecie\wdx{vessie}{f. terme d'anat.
`réservoir
musculo-membraneux dans lequel s'accumule
l'urine qui arrive des reins par les
uretères; vessie'}{vecie}
sont plantés\wdx{planter}{v.tr. `fixer (qch.)'}{}
deux
conduis\wdx{conduit}{m. `canal
ou tuyau qui sert à l'ecoulement ou au
transport d'une matière (un liquide, l'air, un
gaz, etc.)'}{conduis \emph{pl.}}
\text{lo\emph{n}gs}\fnb{Zeichnung
in Foliomitte; ähnelt in reduzierter Form der
kleeblattähnlichen Zeichnung auf
f\hoch{o}21v\hoch{o}a.}/ qui descendent
des reins\wdx{rein}{m. 1\hoch{o}
terme d'anat. `chacun des deux organes
sécréteurs
glandulaires situés symétriquement dans les fosses
lombaires et qui élaborent l'urine'}{},
que on appelle porry uritides\wdx{*porri
uritides}{lt. pl. terme d'anat. `les deux canaux qui
conduisent l'urine des reins à la vessie; les uretères'}{porry uritides},
et \text{entre\emph{n}t}\fnb{Ms.
\emph{entren}.}/\wdx{entrer}{v.intr. `passer du
dehors en dedans'}{} par les costés
dedans la dicte vecie\wdx{vessie}{f. terme d'anat.
`réservoir
musculo-membraneux dans lequel s'accumule
l'urine qui arrive des reins par les
uretères; vessie'}{vecie}
et
\text{tournoiant}\fnb{Zur Form,
cf. \flq Die Sprache\frq .}/\wdx{*tornoiier}{v.intr.
`se mouvoir circulairement ou décrire une ligne
courbe'}{tournoiant \emph{3.p.pl. ind.prés.}}
et portent dedens l'orine\wdx{orine}{f. `liquide
organique clair et ambré qui se forme dans le rein,
qui séjourne dans la vessie et qui est evacué par
l'urètre'}{}; et le
col
de l'urine\wdx{col}{m. 2\hoch{o} terme d'anat.
`partie rétrécie
(d'une cavité organique)' (par analogie de forme)}{}\wdx{orine}{f.
`liquide organique
clair et ambré qui se forme dans le rein,
qui séjourne dans la vessie et qui est evacué par
l'urètre'}{urine} est charneux et
y a muscules cloians\wdx{*cloiant}{p.prés. comme
adj. `qui a la qualité de pouvoir se
fermer'}{cloians
\emph{m.pl.}} et ouvrans\wdx{*ovrant}{p.prés. comme
adj.
`qui a la qualité pouvoir se ouvrir'}{ouvrans
\emph{m.pl.}} qui viennent
de la vecie\wdx{vessie}{f. terme d'anat.
`réservoir
musculo-membraneux dans lequel s'accumule
l'urine qui arrive des reins par les
uretères; vessie'}{vecie}
en fasant\wdx{faire}{v.tr.
`réaliser
ou effectuer (qch.)'}{fasant \emph{p.prés.}}
refleccion\wdx{*reflexion}{f. `action
de se replier sur soi-même'}{refleccion},
et passe parmy le peritoneu\emph{m}\wdx{peritoneum}{lt.
terme d'anat. `membrane séreuse
qui tapisse les parois
intérieures de la cavité abdominale et pelvienne et
qui recouvre les organes contenus dans
la cavité abdominale et pelvienne, à l'exception de
l'ovaire'; péritoine'}{}. Et es
ho\emph{m}mes\wdx{*ome}{m.
`être appartenant à l'espèce
animale la plus évoluée de la terre; être
humain'}{homme} jusques
a la verge\wdx{verge}{f. `organe de la copulation
chez l'homme; pénis'}{} va le col de la dicte
vecie\wdx{vessie}{f. terme d'anat.
`réservoir
musculo-membraneux dans lequel s'accumule
l'urine qui arrive des reins par les
uretères; vessie'}{vecie}, mais es fe\emph{m}mes\wdx{*feme}{f.
`être humain du sexe
féminin; femme'}{femme} il va
jusques a deux dois\wdx{doi}{m. 3\hoch{o} `mesure
approximative, équivalent à un travers de doigt'}{}
\text{enmy}\fnb{Voranstehend expungiertes \emph{pres}.}/%
\wdx{enmi}{prép.
`au milieu de'}{enmy}
le con,
sans point fere
de refleccion\wdx{*reflexion}{f.
`action
de se replier sur soi-même'}{refleccion}, et par ce
col\wdx{col}{m. 2\hoch{o} terme d'anat.
`partie rétrécie
(d'une cavité organique)' (par analogie de forme)}{}
l'urine\wdx{orine}{f. `liquide organique
clair et ambré qui se forme dans le rein,
qui séjourne dans la vessie et qui est evacué par
l'urètre'}{urine} est expellee hors
de la vecie. Et par ce
appert l'accion, la substance, la
posicion\wdx{posicïon}{f. `lieu où quelque chose est placée,
située'}{posicion}
\emph{et} c\emph{etera}
de la vecie. Item, il appert que legieremant\wdx{*legierement}{adv.
`sans effort; facilement'}{legieremant}
le col
de la vecie peult estre estoupés et si
peullent enge\emph{n}drer
pierres\wdx{piere}{f. 2\hoch{o} terme de méd.
`amas de gravier qui se forme dans certains organes (le rein, la
vessie, etc.)' (par analogie de forme)}{pierre} pour cause de
l'urine\wdx{orine}{f. `liquide organique
clair et ambré qui se forme dans le rein,
qui séjourne dans la vessie et qui est evacué par
l'urètre'}{urine},
qui est gravelleuse\wdx{*gravelos}{adj. `qui
contient du calcul, de la concrétion rénale'}{gravelleuse \emph{f.sg.}},
\text{que la vecie}\fnb{Nachfolgend
gestrichener Buchstabenansatz.}/
ressoit et soustient\wdx{*sostenir}{v.tr. `tenir (qch.)
par-dessous en servant de support ou
d'appui; soutenir'}{};
et selon la vecie
on peult [considerer] la maniere
de cyringuer\wdx{*seringuer}{v.tr.
`injecter (qch. à qn) à l'aide d'une
seringue'}{cyringuer}.
Et aussi il appert au col de la vessie et par
%
[34r\hoch{o}b]
dehors\wdx{*defors}{prép. `à
l'extérieur de'}{dehors} la
cousture\wdx{*costure}{f. terme d'anat. `ligne
apparente constituant la jonction entre deux organes, deux
parties d'un os, etc.'}{cousture} qui est
\text{pres du cul}\fnb{Über der Zeile nachgetragen.}/\wdx{cul}{m.
`partie postérieure chez
l'homme; cul'}{},
par qui on veult traire\wdx{traire}{v.tr.
`faire venir
dans une certaine direction (qn, qch.)'}{} la
piere\wdx{piere}{f. 2\hoch{o} terme de méd.
`amas de gravier qui se forme dans certains organes (le rein, la
vessie, etc.)' (par analogie de forme)}{}, si co\emph{m}me y
s\emph{er}a
dit tantost\wdx{tantost}{adv.
`dans un temps prochain, un
proche avenir; tantôt'}{}.
\pend
\pstart
Item, les vaisseaux
sp\emph{er}matiques\wdx{spermatique}{adj. terme de méd.
`qui a rapport au sperme'}{},
ce sont
vaines qui naissent de pres des rongnons\wdx{*reignon}{m.
terme d'anat.
`chacun des deux
organes sécréteurs glandulaires situés
symétriquement dans les fosses lombaires et
qui élaborent l'urine; rein'}{rongnon},
descendans\wdx{*descendant}{adj. terme d'anat. `qui
va du haut en bas
(surtout en parlant d'un vaisseau
sanguin)'}{descendans \emph{f.pl.}}
\text{de}\fnb{Zeichnung am Foliorand, cf. supra.}/
la vaine kily\wdx{*veine kilis}{f.
terme
d'anat. `vaisseau sanguin qui ramène le sang
nutritif du foie à tout le corps'}{vaine kily} et
aborthi\wdx{*veine
aborthi}{f.
terme d'anat. `vaisseau sanguin qui naisse au
c\oe ur'}{vaine aborthi} et portent le sang
aux coillons\wdx{coillon}{m. 1\hoch{o} `gonade mâle suspendue dans le scrotum, qui
produit les spermatozoïdes; testicule'}{},
tant de l'o\emph{m}me co\emph{m}me de la fame\wdx{*feme}{f.
`être humain du sexe
féminin; femme'}{fame}. Et la
se fait une autre digestion\wdx{digestion}{f. terme de
méd.
`conversion des substances dans le corps
en sucs nécessaires pour l'assimilation et
la désassimilation (des aliments en chyle, le chyle en sang,
etc.)'}{} et se
mue\wdx{müer}{v.pron. `se transformer'}{} en
sp\emph{er}me\wdx{esperme}{m. terme de méd.
`semence, cellule ou groupe de cellules
dont se forme un organisme'}{sperme}.
L'esp\emph{er}me\wdx{esperme}{m. terme de méd.
`semence, cellule ou groupe de cellules
dont se forme un organisme'}{}, c'est la semence\wdx{semence}{f.
`semence, cellule ou groupe de cellules dont se
forme un être vivant' (de l'espèce humaine et des
animaux)}{}
et le germo\emph{n}\wdx{germon}{m. `semence, cellule ou groupe de
cellules dont se forme un être vivant'}{}
de nature humaine\wdx{nature}{f. 1\hoch{o} `ensemble
des
caractères, des propriétés qui définissent un être,
une chose concrète ou
abstraite'}{\textbf{nature humaine}
\emph{`ensemble des
caractères, des propriétés qui définissent l'espèce
humaine'}}.
Item, les
ditz vaisseaulx, es ho\emph{m}mes\wdx{*ome}{m.
`être appartenant à l'espèce
animale la plus évoluée de la terre; être
humain'}{homme} ilz
viennent hors aux coillons\wdx{coillon}{m.
1\hoch{o}
`gonade mâle suspendue dans le scrotum, qui
produit les spermatozoïdes; testicule'}{} qui sont dehors. Mais es
fames\wdx{*feme}{f.
`être humain du sexe
féminin; femme'}{fame}, ilz demeurent dedans, car leur
coillons\wdx{coillon}{m. 2\hoch{o} terme d'anat. `glande génitale
femelle qui produit les ovules; ovaire'}{}
sont
dedans. Et appert par ce que dit est que, pour
cause
de la naissance\wdx{naissance}{f. terme
d'anat. `endroit où commence
qch. (en parlant des membres, organes ou structures organiques
du corps)'}{}
des dis vaisseaulx, l'esp\emph{er}me\wdx{esperme}{m.
terme de méd.
`semence, cellule ou groupe de cellules
dont se forme un organisme'}{}
seut\wdx{savoir}{v.tr.
`avoir présent à l'esprit
(un objet de pensée qu'on identifie et qu'on tient
pour réel); savoir'}{seut \emph{3.p.sg. ind.prés.}}
la nature du cuer, du foye et
des roignons\wdx{*reignon}{m.
terme d'anat.
`chacun des deux
organes sécréteurs glandulaires situés
symétriquement dans les fosses lombaires et
qui élaborent l'urine; rein'}{roignon}. Et par les nerfz
q\emph{ue}, pour cause
de dellectacion\wdx{*delectacïon}{f.
`plaisir que l'on savoure;
délectation'}{dellectation},
descendent du cervel et vont
aux coillons\wdx{coillon}{m. 1\hoch{o} `gonade mâle suspendue dans le scrotum, qui
produit les spermatozoïdes; testicule'}{},
le cervel ha
co\emph{m}municacion\wdx{*comunicacïon}{f.
`rapport entre des choses connexes;
connexion'}{communicacion} ad\wdx{a}{prép.
marquant des rapports de direction, de position `à'}{ad}
ce et,
par consequent\wdx{consequent}{adj.}{\textbf{par
consequent}
\emph{loc.adv. `comme suite logique'}}, tout le corps.
Donc selon ce que dit
est l'esp\emph{er}me\wdx{esperme}{m. terme de méd.
`semence, cellule ou groupe de cellules
dont se forme un organisme'}{}
vient de tout le corps par
decizion\wdx{*decision}{f. `action de
décider; décision'}{decizion},
no\emph{n} pas p\emph{ar} sa pesatume\wdx{*pesantume}{f. `caractère
de ce qui a un grand poids; pesanteur'}{pesatume}, mais
p\emph{ar} vigueur\wdx{*vigor}{f. `force qui a son plein
développement' (dit de la nature)}{vigueur}
de nature\wdx{nature}{f. 2\hoch{o}
`principe actif qui
anime, organise l'ensemble des choses existantes selon
un certain ordre'}{},
si co\emph{m}me le tient\wdx{tenir}{v.tr. 2\hoch{o}
`considérer (qch.)'}{}
le co\emph{n}silliateur\wdx{*conciliateur}{m. `personne qui
s'efforce de concilier les personnes entre elles'}{consilliateur}.
\pend
\pstart
En aprés,
pour cause des fe\emph{m}mes\wdx{*feme}{f.
`être humain du sexe
féminin; femme'}{femme},
il nous co\emph{n}vient dire\wdx{dire}{v.tr.indir. \textbf{\emph{dire de}}
`parler de'}{}
de la
matrice\wdx{matrice}{f.
`organe situé dans la
cavité pelvienne destiné à contenir l'\oe uf
fécondé jusqu'à son complet développement; utérus'}{}.
La matrice\wdx{matrice}{f.
`organe situé dans la
cavité pelvienne destiné à contenir l'\oe uf
fécondé jusqu'à son complet développement; utérus'}{}, c'est le
champ\wdx{champ}{m. `pièce de terre destinée à
être cultivée, non enclose de murs; champ' (comme
métaphore)}{} de
generacion\wdx{generacïon}{f. `action ou faculté
d'engendrer'}{generacion} humaine\wdx{*umain}{adj.
`qui appartient ou qui
est propre à l'homme; humain'}{humain}
et est le
instrument\wdx{instrument}{m. 2\hoch{o}
`partie du corps remplissant une fonction
particulière; organe'}{} qui
reçoit\wdx{recevoir}{v.tr.
`faire entrer (qch.)'}{reçoit \emph{3.p.sg. ind.prés.}}
la
d\emph{i}te semence\wdx{semence}{f.
`semence, cellule ou groupe de cellules dont se
forme un être vivant' (de l'espèce humaine et des
animaux)}{}; elle est assize\wdx{asseoir}{v.tr. 2\hoch{o} `placer, poser
(qch.)'}{assize \emph{p.p. f.sg.}}
entre
longaon\wdx{longäon}{m.
terme d'anat.
`portion terminale du gros intestin qui
s'étend du côlon jusqu'à l'anus; rectum'}{longaon}
et la vessie,
et sa substance est de panicles et est composee de
deux tuniques et est
%
[34v\hoch{o}a]
de fourme ronde et ha deux cornes\wdx{corne}{f.
`excroissance conique' (ici: sur l'utérus)}{}
ou deux bras en maniere de celles\wdx{*cele}{f. terme
d'anat.
`partie creuse
dans une structure organique; cavité'}{celle}
ou de cha\emph{m}bres\wdx{chambre}{f.
`partie creuse
dans une structure organique; cavité'}{}, et au
chief ha ung petit\wdx{petit}{adj. 2\hoch{o}
dans l'ordre
physique, quantifiable `qui est d'une extension
au-dessous de la moyenne; petit (du corps humain et
de ses parties)'}{}
coillon\wdx{coillon}{m. 2\hoch{o} terme d'anat. `glande génitale
femelle qui produit les ovules; ovaire'}{}
qui est planté\wdx{planter}{v.tr.
`fixer (qch.)'}{} en
la partie hault\wdx{hault}{adj.
`qui est d'une certaine dimension
dans le sens vertical; haut'}{}, et
ha ung chanel\wdx{chanel}{m.
terme d'anat. `conduit naturel dans le
corps par lequel s'écoule un liquide ou
une matière organique'}{} ou ung
conduit\wdx{conduit}{m. `canal
ou tuyau qui sert à l'ecoulement ou au
transport d'une matière (un liquide, l'air, un
gaz, etc.)'}{}
large\wdx{large}{adj.
`qui a une
étendue supérieure à la moyenne dans le sens de la
largeur; large'}{} en la partie de devant, et
est ainsi que ung vit\wdx{vit}{m. `organe de la
copulation chez l'homme'}{}
ra\emph{n}verssé\wdx{*renverser}{v.tr. `mettre à
l'envers'}{ranverssé
\emph{p.p.}},
si co\emph{m}me il est escript au .xiiij.\hoch{e} livre
de
\flq Utilitate\frq . Et par dessus, elle ha deux bras
ou deux branches\wdx{branche}{f. 2\hoch{o}
`ramification ou division d'un organe, d'un
appareil, etc., qui part d'un axe ou d'un
centre' (par analogie de forme)}{} ou sont
\text{celles ou}\fnb{Nachfolgend expungiertes und
gestrichenes \emph{bra\~{n}}.}/\wdx{*cele}{f. terme
d'anat.
`partie creuse
dans une structure organique; cavité'}{celle}
che\emph{m}bres\wdx{chambre}{f.
`partie creuse
dans une structure organique; cavité'}{chembre}, avec les
coillons\wdx{coillon}{m. 2\hoch{o} terme d'anat. `glande génitale
femelle qui produit les ovules; ovaire'}{},
ainsi que la bource\wdx{*borse}{f.
1\hoch{o} `structure organique
en forme de poche ou de sac arrondie'}{bource}
des coillons.
Et au milieu est ung ventre
co\emph{m}mun\wdx{ventre}{m.
2\hoch{o} terme d'anat. `partie creuse
dans le corps'}{}, ainsi q\emph{ue} sont les parties
pectinales\wdx{pectinal}{adj. terme d'anat. `qui
a rapport au \flq pectiné\frq\ (i.e. pubis)'}{},
et ha le col bas vuyt\wdx{*vuit}{adj. `qui ne
contient rien'}{vuyt \emph{m.pl.}} ainsi que ung
chenal\wdx{chanel}{m.
terme d'anat. `conduit naturel dans le
corps par lequel s'écoule un liquide ou
une matière organique'}{chenal} ou
einsi que\wdx{ainsi}{adv.
`de cette façon'}{\textbf{ainsi que}
\emph{loc.conj.
`de la même façon que'} einsi que}
ha le vit\wdx{vit}{m. `organe de la
copulation chez l'homme'}{}. Et ha aussi le con\wdx{con}{m.
terme d'anat.
`ensemble des organes génitaux
externes de la femme; vulve'}{}
ainsi que ung
chapel\wdx{chapel}{m. `ce dont la forme
ressemble à celle d'une coiffure' (par analogie de forme)}{}
ou une mitre\wdx{mitre}{f. `sortie de l'urètre' (par analogie de forme)}{}.
Et si y est la tentigine\wdx{tentigine}{f.
terme d'anat.
`petit
organe erectile de la vulve; clitoris'}{},
e\emph{n}si\wdx{ainsi}{adv. `de cette façon'}{ensi}
le prepuce\wdx{prepuce}{m. `repli tégumentaire qui
entoure le gland de la verge; prépuce'}{} du vit\wdx{vit}{m. `organe de la
copulation chez l'homme'}{}, c'est a dire la
pel\wdx{pel}{f. 3\hoch{o} terme d'anat.
`repli tégumentaire qui
entoure le gland de la verge; prépuce'}{}.
Et est longue, ainsi que ung vit\wdx{vit}{m. `organe de la
copulation chez l'homme'}{} q\emph{ue} ha
.viij. ou
.ix.
polz\wdx{*pouz}{m. `le plus gros et le plus fort des
doigts de la main et du pied'}{polz \emph{pl.}} de
long\wdx{lonc}{adv.}{\textbf{de lonc}
\emph{loc.adv.
 `dans le sens de la longueur'} de long}. Et ja soit
ce que la \text{matri\-ce}\fnb{\emph{r} über der Zeile
nachgetragen.}/\wdx{matrice}{f.
`organe situé dans la
cavité pelvienne destiné à contenir l'\oe uf
fécondé jusqu'à son complet développement; utérus'}{}
n'aie que deux sains\wdx{*sein}{m. `espace vide à
l'intérieur d'un corps solide; cavité'}{sain} ou deux
concavités\wdx{concavité}{f.
`espace vide à
l'intérieur d'un corps solide; concavité'}{}
manifestes\wdx{manifest}{adj. `dont l'existence ou la nature
est évidente; manifeste'}{},
toutesvoies chescune en ha trois et une qui est au
milieu. Car, selo\emph{n} Mu\emph{n}din\adx{Mestre Mundin}{}{Mundin},
on y treuve\wdx{*trover}{v.tr. `rencontrer qn ou qch. qu'on cherche;
trouver'}{treuve \emph{3.p.sg. ind.prés.}}
.vij.
receptacles\wdx{receptacle}{m.
`contenant qui
reçoit son contenu de diverses provenances;
réceptacle'}{}. La
dicte matrice\wdx{matrice}{f.
`organe situé dans la
cavité pelvienne destiné à contenir l'\oe uf
fécondé jusqu'à son complet développement; utérus'}{}
ha colligance avec le
cervel et le cuer et le foie et avec l'estomac et est
lié
avec le dos. Item,
entre les mamelles\wdx{*mamele}{f.
`organe glanduleux
qui sécrète le lait (chez les mammifères);
mamelle'}{mamelle}
et la matrice\wdx{matrice}{f.
`organe situé dans la
cavité pelvienne destiné à contenir l'\oe uf
fécondé jusqu'à son complet développement; utérus'}{}
sont lactelles\wdx{lactelle}{f.
`conduit qui transporte le sang aux seins où il
est transformé en lait'}{}
et menstruelles\wdx{menstruelle}{f. `conduit qui
transporte le sang de la menstruation'}{} qui vont
de
\text{l'ung}\fnb{Voranstehend gestrichenes
\emph{lon}.}/
a l'autre. Et pour ce dit Galien\adx{Galien}{}{} ou dit
livre, que Ypocras\adx{Ypocras}{}{} dit, que le
lait\wdx{lait}{m. `liquide blanc, opaque,
nutritif, des femelles des mammifères, servant à
l'alimentation naturelle des jeunes
mammifères (dans l'espece humaine, des
nourrissons); lait'}{}
estoit frere\wdx{frere}{m. `chose considérée comme
apparentée à une autre'}{} au
menstru\wdx{menstru}{m. `sang de la menstruation,
aussi son écoulement'}{} ou que le lait\wdx{lait}{m.
`liquide blanc, opaque,
nutritif, des femelles des mammifères, servant à
l'alimentation naturelle des jeunes
mammifères (dans l'espece humaine, des
nourrissons); lait'}{} est fait du
menstru\wdx{menstru}{m.
`sang de la menstruation,
aussi son écoulement'}{}.
%
[34v\hoch{o}b]
Et pour ce, en ung mesme temps\wdx{*tens}{m. `milieu indéfini où paraissent se dérouler irréversiblement
les existences dans leur changement, les événements et les
phénomènes dans leur succession; temps'}{temps}, la fame\wdx{*feme}{f.
`être humain du sexe
féminin; femme'}{fame},
peult
avoir lait\wdx{lait}{m.
`liquide blanc, opaque,
nutritif, des femelles des mammifères, servant à
l'alimentation naturelle des jeunes
mammifères (dans l'espece humaine, des
nourrissons); lait'}{} et menstrus\wdx{menstru}{m.
`sang de la menstruation,
aussi son écoulement'}{}.
Et en la matrice\wdx{matrice}{f.
`organe situé dans la
cavité pelvienne destiné à contenir l'\oe uf
fécondé jusqu'à son complet développement; utérus'}{}
peullent venir\wdx{venir}{v.tr.indir.
1\hoch{o} `avoir
son origine dans'}{}
pluseurs maladies et la maniere du
mieger\wdx{*megier}{v.tr. empl.abs. `traiter
médicalement'}{mieger \emph{inf.}}
si est par pessaires\wdx{pessaire}{m. `médicament qui
sert à provoquer les menstrues ou à guérir les
maladies de la matrice'}{}; et vous souffice\wdx{*sofire}{v.intr.
`avoir la quantité, la qualité, la force
etc. nécessaire pour (qch.);
suffire'}{\textbf{*sofire a qn} \emph{v.impers.
+ subj.prés.}
vous souffice} ce que dit est de la matrice\wdx{matrice}{f.
`organe situé dans la
cavité pelvienne destiné à contenir l'\oe uf
fécondé jusqu'à son complet développement; utérus'}{}.
\pend
\pstart
Item, dessoubz
les dictes p\emph{ar}ties est le
longaon\wdx{longäon}{m.
terme d'anat.
`portion terminale du gros intestin qui
s'étend du côlon jusqu'à l'anus; rectum'}{longaon}
ou le droit
intestin\wdx{intestin droit}{m. terme d'anat.
`partie terminale du gros intestin entre le côlon et
l'anus; rectum'}{droit intestin}
-- duquel nous
avons laissé la anathomie si devant
-- et c'est
cilz
les sup\emph{er}fluités de la premiere
digestion\wdx{digestion}{f. terme de méd.
`conversion des substances dans le corps
en sucs nécessaires pour l'assimilation et
la désassimilation (des aliments en chyle, le chyle en sang,
etc.)'}{},
et est de
substance panicullaire\wdx{panicullaire}{adj.
terme d'anat.
`qui appartient au \flq pannicle\frq , i.e.
à la couche de tissu musculaire ou cellulaire qui
recouvre une structure organique du corps humain (un
organe, un os, une articulation, un
muscle, etc.)'}{},
ainsi
que les autres intes\-tins; sa longueur\wdx{*longor}{f.
`dimension d'une
chose dans le sens de sa plus grande
étendue; longueur'}{longueur}
\text{est}\fnb{\emph{et} über der Zeile
nachgetragen, l. \emph{est}.}/
ainsi que d'une paulme\wdx{*paume}{f.
`mesure d'environ un travers de main; palme'}{paulme} et va
jusques pres des rongnons en
gisant\wdx{gesir}{v.intr. `être étendu, être
placé, se trouver (en parlant de choses)'}{gisant \emph{p.prés.}}
droictemant\wdx{*droitement}{adv.
`en ligne droite'}{droictemant} sur les os de
la
keue\wdx{os de la queue}{m. `extrémité
inférieure de la colonne vertébrale, articulée avec le
sacrum et formé de trois petit os; coccyx'}{os de la keue}. La
partie de dessoubz\wdx{*desoz}{adv. `à la face
inférieure'}{\textbf{de dessoubz} \emph{loc.adj.
`qui est situé au côté inférieur'}}
du dit longaon\wdx{longäon}{m.
terme d'anat.
`portion terminale du gros intestin qui
s'étend du côlon jusqu'à l'anus; rectum'}{longaon}
est appellee le cul\wdx{cul}{m.
`partie postérieure chez
l'homme; cul'}{}, et y a deux
muscules qui le coevrent et qui le
cloient\wdx{clore}{v.tr. `boucher ce qui est ouvert;
fermer'}{cloient \emph{3.p.pl. ind.prés.}}, et en ce
lieu la \text{se}\fnb{Ms. \emph{se}, Zeilenumbruch und
erneutes \emph{se}.}/
appliquent\wdx{*apliquier}{v.pron.
`se
placer sur (qch.) de manière à y
adhérer'}{appliquent \emph{3.p.pl. ind.prés.}}
.v. rames de vaines q\emph{ue}
sont appellees
emoroïdalles\wdx{*hemorrhoïdal}{adj. terme
d'anat. `qui se trouve
à
la région ano-rectale'}{emoroïdalles \emph{f.pl.}} et
ont grant\wdx{grant}{adj. 3\hoch{o}
dans l'ordre qualitatif, non quantifiable `qui est
d'un
degré supérieur à la moyenne en ce qui concerne la
qualité, l'intensité, l'importance'}{grant
\emph{f.sg.}} colligance
a la vecie et pour ce, quant l'ung est malade,
l'autre ha douleur\wdx{*dolor}{f.
`sensation pénible en un
point ou dans une région du corps;
douleur'}{douleur}. Et puis, se tu
eslieves\wdx{eslever}{v.tr.
`mettre ou porter (qch.)
plus haut'}{eslieves \emph{2.p.sg. ind.prés.}}
le dit
longaon\wdx{longäon}{m.
terme d'anat.
`portion terminale du gros intestin qui
s'étend du côlon jusqu'à l'anus; rectum'}{longaon},
tu porras voir\wdx{*vëoir}{v.tr. `percevoir (qch.)
par le sens de la vue'}{voir \emph{inf.}}
les vaines et les arteres et
les nerfz, co\emph{m}me ilz se
ramefient\wdx{*ramifier}{v.pron.
`se diviser en plusieurs ramifications qui partent
d'un axe ou d'un centre de qch. (en parlant d'une
chose concrète)'}{ramefient
 \emph{3.p.pl. ind.prés.}} et co\emph{m}me
ilz
descendent aux parties basses.
\pend
\pstart
Item, des parties qui
vont dehors
nous en vollons
parler\wdx{parler}{\textbf{\emph{parler de}} v.tr.indir. `s'entretenir
de; parler de'}{}, et premier nous
vollo\emph{n}s dire\wdx{dire}{v.tr.indir. \textbf{\emph{dire de}} `parler de'}{} du
dindime\wdx{*didime}{m.
terme d'anat.
`membrane qui
s'étend du péritoine au scrotum et qui enveloppe le
conduit séminal'}{dindime}
et de l'ossee\wdx{ossee}{m.
`enveloppe des
testicules; scrotum'}{}, et
premier \text{y}\fnb{Nachfolgend expungiertes
\emph{ne}.}/ nous co\emph{n}vient voir\wdx{*vëoir}{v.tr.
`percevoir (qch.) par le sens de la vue'}{voir
\emph{inf.}} deux choses, c'est
assavoir premier les p\emph{ar}ties qui co\emph{n}tiennent\wdx{contenir}{v.tr.
`comprendre en soi,
dans sa capacité, son étendue, sa substance;
contenir'}{\emph{empl.abs.}}
 et
\text{aprés les parties}\fnb{\emph{ties} über der
Zeile nachgetragen, nachfolgend gestrichenes
\emph{pi}.}/ qui so\emph{n}t contenues\wdx{contenir}{v.tr.
`comprendre en soi,
dans sa capacité, son étendue, sa substance;
contenir'}{contenu \emph{p.p.}}. Les choses qui
sont
\text{co\emph{n}tenus}\fnb{Es muß hier heißen \emph{Les
choses qui contiennent}.}/,
%
[35r\hoch{o}a]
se sont celles du ventre qui sont
devant
no\emph{m}mees\wdx{*nomer}{v.tr.
`indiquer (une
personne, une chose) en disant ou en écrivant son
nom'}{nommé \emph{p.p.}}, car de
\text{ces parties
dedans viennent}\fnb{Ms. \emph{{\dots} viennet}.}/
les parties
de dehors\wdx{*defors}{adv. `à
l'extérieur'}{\textbf{de dehors} \emph{loc.adj.
`qui est situé à l'extérieur'}} -- si co\emph{m}me de
mira\emph{n}ce\wdx{mirance}{subst.
terme d'anat.
`membrane séreuse
qui tapisse les parois
intérieures de la cavité abdominale et pelvienne;
péritoine pariétal' (?)}{}
vient mirac\wdx{mirac}{m.
terme d'anat. `membrane séreuse
qui tapisse les parois
intérieures de la cavité abdominale et pelvienne;
péritoine pariétal'}{}
et de siphance\wdx{*sifance}{subst.
terme d'anat.
`membrane séreuse
qui recouvre les organes contenus dans
la cavité abdominale et pelvienne, à l'exception de
l'ovaire; péritoine viscéral' (?)}{siphance}
vient
ciphac --
et pe\emph{n}dent\wdx{*pendiier}{v.intr. `être fixé par le
haut, la partie inférieure restant libre;
pendre'}{pendent \emph{3.p.pl. ind.prés.}
}{} par dehors en
passant par dessus\wdx{*desus}{prép.
qui marque la position en haut par rapport à ce qui
est en bas `sur'}{dessus}
l'os du
pectiné\wdx{pectiné}{m. terme d'anat. `partie antérieure de
l'os iliaque; pubis'}{}. Et le
co\emph{m}mencemant\wdx{*comencement}{m. `première partie de qch.,
celle que d'autres doivent suivre et qu'aucune ne précède
(dans le temps ou dans l'espace)'}{commencemant}, quant il
\text{ist}\fnb{Im Ms. \emph{est} zu \emph{ist}
korrigiert\,?}/, on le appelle le
dindime\wdx{*didime}{m.
terme d'anat.
`membrane qui
s'étend du péritoine au scrotum et qui enveloppe le
conduit séminal'}{dindime},
\text{car}\fnb{\emph{car} über der Zeile ersetzt
expungiertes
\emph{et}.}/ il est double\wdx{*doble}{adj.
`qui est
répété deux fois, qui vaut deux fois (la
chose désignée) ou qui existe deux fois'}{double}. Et la fin ou
\text{l'autre}\fnb{\emph{u} über der Zeile nachgetragen.}/ chief, on
le appelle le ossee\wdx{ossee}{m.
`enveloppe des
testicules; scrotum'}{} ou la bource\wdx{*borse}{f.
1\hoch{o} `structure organique
en forme de poche ou de sac arrondie'}{bource}
des coillons. Les
parties qui sont co\emph{n}tenues\wdx{contenir}{v.tr.
`comprendre en soi,
dans sa capacité, son étendue, sa substance;
contenir'}{contenu \emph{p.p.}}
sont trois: premiers sont
les coillons\wdx{coillon}{m. 1\hoch{o} `gonade mâle suspendue dans le scrotum, qui
produit les spermatozoïdes; testicule'}{}
qui sont les
principaux\wdx{principal}{adj. `qui est le plus
important; principal'}{principaux \emph{m.pl.}}
instrumens
de generacion\wdx{generacïon}{f. `action ou faculté
d'engendrer'}{generacion} humaine\wdx{*umain}{adj.
`qui appartient ou qui
est propre à l'homme; humain'}{humain}, car es
coillons\wdx{coillon}{m. 1\hoch{o} `gonade mâle suspendue dans le scrotum, qui
produit les spermatozoïdes; testicule'}{}
se
p\emph{ar}fait\wdx{parfaire}{v.pron.
`devenir achevé'}{}
l'esparme\wdx{esperme}{m. terme de méd.
`semence, cellule ou groupe de cellules
dont se forme un organisme'}{esparme},
et sont de substance charneuse,
glandellouse\wdx{*glandulos}{adj. `qui contient des
glandes'}{glandellouse \emph{f.sg.}}
et blanche\wdx{blanc}{adj. `qui est de
la couleur de la neige; blanc'}{}. Aprés viennent les vaisseaux
sp\emph{er}matiques\wdx{spermatique}{adj. terme de méd.
`qui a rapport au sperme'}{}
qui viennent des parties devant dictes et sont
doubles\wdx{*doble}{adj.
`qui est
répété deux fois, qui vaut deux fois (la
chose désignée) ou qui existe deux fois'}{double}, c'est assavoir
dellactoires\wdx{*dilatoire}{adj. terme
de méd. `qui a la faculté de s'étendre, d'augmenter
en volume' (dit d'un vaisseau sanguin
du scrotum)}{dellactoire} ou
portatis\wdx{*portatif}{adj. `qui est
capable à porter (qch.)'}{portatis
\emph{pl.}}
et les expulsoires\wdx{expulsoire}{adj. terme de méd.
`qui a la faculté d'expulser (qch.) du corps'}{}. Les
\text{delatoires}\fnb{Im Ms. \emph{dechillores}
korrigiert in
\emph{delatoires}.}/\wdx{*dilatoire}{adj. terme
de méd. `qui a la faculté de s'étendre, d'augmenter
en volume' (dit d'un vaisseau sanguin
du scrotum)}{delatoire},
ce sont
les vaines et les arteres qui naissent de la vaine
kily\wdx{*veine kilis}{f.
terme
d'anat. `vaisseau sanguin qui ramène le sang
nutritif du foie à tout le corps'}{vaine kily} et de
aborchi\wdx{*veine aborthi}{f.
terme d'anat. `vaisseau sanguin qui naisse au
c\oe ur'}{aborchi}. Les
expulsoires\wdx{expulsoire}{adj. terme de méd.
`qui a la faculté d'expulser (qch.) du
corps'}{}, ce sont ceulx qui
montent pres du col de la vecie et expellent
l'esparme\wdx{esperme}{m. terme de méd.
`semence, cellule ou groupe de cellules
dont se forme un organisme'}{esparme}
ou trou de la verge\wdx{verge}{f.
`organe de la copulation
chez l'homme; pénis'}{}. Et avec ce,
y a ung nerf suspensoire\wdx{suspensoire}{adj.
terme d'anat. `qui sert à tenir suspendu un
organe'}{} et
sencitifs\wdx{*sensitif}{adj.
2\hoch{o} `qui a rapport aux sens'}{sencitifs
\emph{m.pl.}}
qui
descend aux coillons\wdx{coillon}{m. 1\hoch{o} `gonade mâle suspendue dans le scrotum, qui
produit les spermatozoïdes; testicule'}{}.
Donc, depuis\wdx{depuis}{prép. `à partir de
(en parlant de l'espace)'}{}
le
dindime\wdx{*didime}{m.
terme d'anat.
`membrane qui
s'étend du péritoine au scrotum et qui enveloppe le
conduit séminal'}{dindime} jusques
a lla bource\wdx{*borse}{f. 2\hoch{o} `enveloppe des
testicules; scrotum'}{bource}, par dedans il sont
quatre
corps distingués\wdx{distinguer}{v.tr. `permettre
de reconnaître (une personne ou une chose d'une
autre), en parlant d'une différence constitutive,
d'un trait caractéristique; distinguer'}{distingué
\emph{p.p.}}. Et par
ce il
appert que vers le inguine\wdx{inguine}{m. `partie latérale
et inférieure du bas-ventre; aine'}{}, dedans le
%
[35r\hoch{o}b]
mirach\wdx{mirac}{m.
terme d'anat. `membrane séreuse
qui tapisse les parois
intérieures de la cavité abdominale et pelvienne;
péritoine pariétal'}{mirach} et
siphac\wdx{sifac}{m. terme d'anat. `membrane séreuse qui
recouvre les organes contenus dans
la cavité abdominale et pelvienne, à l'exception de
l'ovaire; péritoine viscéral'}{siphac}, est
-- et \text{doit}\fnb{Nachfolgend gestrichener
Buchstabenansatz.}/ estre
-- ung
trou par lequel descendent les trois corps devant dis,
c'est assavoir la vaine, l'artere, la
delatoire\wdx{*dilatoire}{adj. terme
de méd. `qui a la faculté de s'étendre, d'augmenter
en volume' (dit d'un vaisseau sanguin
du scrotum)}{delatoire}, et
le nerf. Et par dehors, pres du col de la vecie, en la
racine\wdx{racine}{f. 1\hoch{o} terme d'anat.
`portion d'un organe
servant à son implantation dans un autre organe'}{}
de la verge\wdx{verge}{f.
`organe de la copulation
chez l'homme; pénis'}{}, est l'autre qui
est
le quart, par quoy l'esp\emph{er}me\wdx{esperme}{m. terme de méd.
`semence, cellule ou groupe de cellules
dont se forme un organisme'}{}
mo\emph{n}te et est expellé au
chanel\wdx{chanel}{m.
terme d'anat. `conduit naturel dans le
corps par lequel s'écoule un liquide ou
une matière organique'}{}
de la verge\wdx{verge}{f.
`organe de la copulation
chez l'homme; pénis'}{}. Et pour ce appart aussi
que, quant le trou qui \text{est}\fnb{Nachfolgend
gestrichenes \emph{q}.}/ vers
les inguines\wdx{inguine}{m. `partie latérale et inférieure
du bas-ventre; aine'}{}
\text{se dilate aultre nature}\fnb{L. \emph{se dilate
en aultre nature}\,?}/\wdx{dilater}{v.pron. `augmenter de volume; s'étendre'}{se
dilater aultre nature}, les parties
de dessus, si co\emph{m}me le zirbus\wdx{zirbus}{m.
terme d'anat.
`repli du péritoine qui relie entre eux
les organes abdominaux; épiploon'}{}
et les
intestins,
peullent issir et descendre dedans le
dindime\wdx{*didime}{m.
terme d'anat.
`membrane qui
s'étend du péritoine au scrotum et qui enveloppe le
conduit séminal'}{dindime} et
dedans l'ossee\wdx{ossee}{m.
`enveloppe des
testicules; scrotum'}{}, et peullent fere
roupture\wdx{*rupture}{f. `action de se
rompre, le résultat de cette action;
rupture'}{roupture} ou crepature\wdx{crepature}{f.
`action de crever,
le résultat de cette action'}{}
et, se aultre matere\wdx{matiere}{f. 1\hoch{o} `substance
qui constitue les corps, qui est objet
d'intuition dans l'espace et qui possède une
masse mécanique'}{matere}
y descend, ilz peullent fere
\text{hernie}\fnb{\emph{r} über der Zeile nachgetragen.
Nachfolgend gestrichenes
\emph{le}.}/\wdx{hernie}{f. terme de méd. `tumeur
molle formée par un organe totalement ou partiellement
sorti (par un orifice naturel ou accidentel) de la
cavité qui le contient à l'état normal'}{}, desqueulx la maniere du
saner\wdx{saner}{v.tr.
 `délivrer d'un mal physique;
guérir'}{}
s\emph{er}a dicte
ci aprés.
\pend
\pstart
Cy aprés il nous co\emph{n}vient dire
de la verge\wdx{verge}{f.
`organe de la copulation
chez l'homme; pénis'}{},
car c'est le ahe\emph{n}nier\wdx{*ahanier}{m. `celui qui
laboure' (comme métaphore)}{ahennier}
et cultivour\wdx{*cultiveur}{m. `celui qui
cultive la terre' (comme métaphore)}{cultivour}
ou ortollain\wdx{*hortolain}{m. `celui qui
cultive les jardins; jardinier' (comme métaphore)}{ortollain} de nature
humaine\wdx{nature}{f. 1\hoch{o} `ensemble des
caractères, des propriétés qui définissent un être,
une chose concrète ou
abstraite'}{\textbf{nature humaine}
\emph{`ensemble des
caractères, des propriétés qui définissent l'espèce
humaine'}},
et aussi l'urine\wdx{orine}{f. `liquide organique
clair et ambré qui se forme dans le rein,
qui séjourne dans la vessie et qui est evacué par
l'urètre'}{urine}
passe parmy; et sa substance est composee de cuir, de
muscules, de \text{tenans}\fnb{\emph{s} über der
Zeile ersetzt gestrichenen Buchstaben.}/, de vaines,
de arteres et de nerfz
et de gros\wdx{gros}{adj.
1\hoch{o}
dans l'ordre physique, quantifiable `qui, dans
son genre, dépasse le volume ordinaire; gros (du corps
humain et de ses parties)'}{}
liguemans. Elle est
assise\wdx{asseoir}{v.tr. 2\hoch{o} `placer, poser
(qch.)'}{assis \emph{p.p.}}
et pla\emph{n}tee\wdx{planter}{v.tr. `fixer (qch.)'}{} sur
l'os du pectiné\wdx{pectiné}{m. terme d'anat. `partie antérieure
de l'os iliaque; pubis'}{}; les
liguema\emph{n}s viennent de l'os sacré\wdx{os sacré}{m. terme d'anat. `os formé par la réunion des cinq
vertèbres sacrées, a la partie inférieure de la
colonne vertébrale; sacrum'}{}
et les vaines, les arteres, les nerfz, la char, le
cuir viennent des parties de dessus. Ite\emph{m}, en la
verge\wdx{verge}{f.
`organe de la copulation
chez l'homme; pénis'}{}
sont deux
conduis\wdx{conduit}{m. `canal
ou tuyau qui sert à l'ecoulement ou au
transport d'une matière (un liquide, l'air, un
gaz, etc.)'}{conduis \emph{pl.}}
principaulx
ou deux voies\wdx{voie}{f.
2\hoch{o} terme d'anat. `conduit
naturel dans le corps par lequel s'écoule un liquide
ou une matière organique'}{},
\text{l'une}\fnb{Voranstehend gestrichenes
\emph{la}.}/ du sp\emph{er}me\wdx{esperme}{m. terme de méd.
`semence, cellule ou groupe de cellules
dont se forme un organisme'}{sperme},
l'autre de
l'urine\wdx{orine}{f. `liquide organique
clair et ambré qui se forme dans le rein,
qui séjourne dans la vessie et qui est evacué par
l'urètre'}{urine}. La fin de la verge\wdx{verge}{f.
`organe de la copulation
chez l'homme; pénis'}{}, on le
appelle ballanu\emph{m}\wdx{*balanum}{lt.
`extrémité antérieure de la verge;
gland'}{ballanum}, le trou,
%
[35v\hoch{o}a]
on le appelle mitre\wdx{mitre}{f. `sortie de l'urètre' (par analogie de forme)}{}, le
chapel\wdx{chapel}{m.
`ce dont la forme
ressemble à celle d'une coiffure' (par analogie de forme)}{}, on le
appelle le
\text{prepuce}\fnb{Ms. \emph{pripuce} (?);
\emph{r} über der Zeile
nachgetragen.}/\wdx{prepuce}{m.
`repli tégumentaire qui
entoure le gland de la verge; prépuce'}{}. La quantité de la
verge\wdx{verge}{f.
`organe de la copulation
chez l'homme; pénis'}{}
contient .viij. ou .ix. polz\wdx{*pouz}{m. `le plus
gros et
le plus fort des doigts de la main et du
pied'}{polz \emph{pl.}}, et doit estre
grosse modereemant\wdx{*modereement}{adv. `d'une
manière modérée; modérément'}{modereemant} selon la
grandeur\wdx{*grandor}{f. dans l'ordre
quantitatif `dimension, taille, étendue'}{grandeur} de la
matrice\wdx{matrice}{f.
`organe situé dans la
cavité pelvienne destiné à contenir l'\oe uf
fécondé jusqu'à son complet développement; utérus'}{}.
\pend
\pstart
Perineu\emph{m}\wdx{perineum}{lt.
terme d'anat. `partie inférieure, plancher du
petit bassin qui s'étend entre l'anus et les
parties génitales; périnée'}{}
-- ou peritoneu\emph{m}\wdx{peritoneum}{lt.
terme d'anat. `membrane séreuse
qui tapisse les parois
intérieures de la cavité abdominale et pelvienne et
qui recouvre les organes contenus dans
la cavité abdominale et pelvienne, à l'exception de
l'ovaire'; péritoine'}{} en
arabic\wdx{*arabique}{m. `la langue parlée
par les arabes'}{arabic}
--, c'est le
lieu qui est entre le cul\wdx{cul}{m.
`partie postérieure chez
l'homme; cul'}{}
et la verge\wdx{verge}{f.
`organe de la copulation
chez l'homme; pénis'}{}. Et la est une
cousture\wdx{*costure}{f.
terme d'anat. `ligne
apparente constituant la jonction entre deux organes, deux
parties d'un os, etc.'}{cousture} qui va par la
bourse\wdx{*borse}{f. 1\hoch{o}
`structure organique
en forme de poche ou de sac arrondie'}{bourse}
de la coille\wdx{coille}{f. `gonade mâle
suspendue dans le scrotum, qui
produit les spermatozoïdes, aussi avec
ses enveloppes'}{}.
Les inguines\wdx{inguine}{m. `partie latérale et inférieure
du bas-ventre; aine'}{},
ce sont les emu\emph{n}ctoires\wdx{*emomptoire}{m.
terme d'anat.
`organe
qui élimine les substances inutiles formées au
cours des processus de désassimilation (l'anus,
l'uretère, etc.)'}{emunctoire}
du foie, et
so\emph{n}t chars
glandelleuses
qui sont ordo\emph{n}nees\wdx{*ordener}{v.tr. 1\hoch{o}
`disposer, mettre
dans un certain ordre; ordonner'}{ordonné
\emph{p.p.}}
en la
plicature\wdx{plicature}{f. `formation de pli dans le
corps humain'}{} de la cuisse\wdx{cuisse}{f. `partie
de la jambe qui s'articule à la hanche et s'étend
jusqu'au genou; cuisse'}{}.
Les nages\wdx{*nache}{f.
`chacune des deux parties
charnues de la région du bassin, dans l'espèce
humaine et chez certains mammifères; fesse'}{nage},
ce
sont grosses\wdx{gros}{adj. 3\hoch{o} `qui manque de
finesse, qui est rudimentaire mais solide'}{}
chars musculeuses\wdx{musculeux}{adj. terme d'anat. `qui est de la nature
des muscles'}{}
qui sont sur les os de la
cuisse.
Et finalmant\wdx{finalment}{adv. `à la
fin'}{finalmant} des
hanches\wdx{hanche}{f.
`chacune des deux parties
du corps formant saillie au-dessous des flancs,
entre la fesse en arrière et le pli de l'aine
en avant; hanche'}{}
descendent muscules, cordes
et liguemans ligans\wdx{liguer}{v.tr. `entourer plusieurs choses avec un lien pour
qu'elles tiennent ensemble'}{ligans
\emph{p.prés.}} et mouvans
la cuisse\wdx{cuisse}{f. `partie
de la jambe qui s'articule à la hanche et s'étend
jusqu'au genou; cuisse'}{}
et la grant jambe\wdx{grant jambe}{f. terme d'anat. `membre
inférieur en entier de l'homme y compris le pied'}{} avec les ha\emph{n}ches.
\pend
%
% \memorybreak
%
\pstartueber
Veci le .viij. chappitre: De
l'anathomie des jambes\wdx{jambe}{f.
`partie du membre
inférieur de l'homme comprenant le segment entre le
genou et l'articulation du pied et, par extension, le
membre inférieur en entier, pied inclu ou exclu;
jambe'}{}
et
des piés\wdx{pié}{m. `partie inférieure articulée à
l'extrémité de la jambe'}{}.
\pendueber
%
% \memorybreak
%
\pstart
LA JAMBE\wdx{jambe}{f.
`partie du membre
inférieur de l'homme comprenant le segment entre le
genou et l'articulation du pied et, par extension, le
membre inférieur en entier, pied inclu ou exclu;
jambe'}{} ou le
grant pié\wdx{grant pié}{m. terme d'anat. `membre inférieur
en entier
de l'homme y compris le pied'}{} dure\wdx{durer}{v.intr. `s'étendre (dans l'espace)'}{}
depuis\wdx{depuis}{prép. `à partir de
(en parlant de l'espace)'}{}
la joincture de la scie\wdx{scie}{f.
terme d'anat. `concavité dans la
jointure de la hanche dans laquelle s'emboîte
la tête du fémur'}{} jusques aux extremités\wdx{extremité}{f.
`partie extrème qui termine une chose'}{}
des
dois\wdx{doi}{m. 2\hoch{o} `chacun des
cinq prolongements
qui terminent le pied'}{}.
Et pour ce que les parties de telle
jambe\wdx{jambe}{f.
`partie du membre
inférieur de l'homme comprenant le segment entre le
genou et l'articulation du pied et, par extension, le
membre inférieur en entier, pied inclu ou exclu;
jambe'}{} ou de
tel\wdx{tel}{adj. `qui est
semblable, du même genre; tel'}{}
pié\wdx{pié}{m. `partie inférieure articulée à l'extrémité de
la jambe'}{} ont co\emph{n}venance\wdx{*covenance}{f.
`caractère de ce qui est approprié, qui convient à sa
destination'}{convenance} en pluseurs choses avec
les parties de la grant main\wdx{grant main}{f.
terme d'anat. `membre supérieur de l'homme qui
s'attache au tronc, compris
entre l'épaule et les doigts; bras'}{} --
si co\emph{m}me Galien\adx{Galien}{}{} le
demonstre\wdx{demonstrer}{v.tr.
`faire
voir, mettre devant les yeux; montrer'}{}
ou tiers livre de \flq Utilitate\frq\ --,
pour ce est il que ceste
jambe\wdx{jambe}{f.
`partie du membre
inférieur de l'homme comprenant le segment entre le
genou et l'articulation du pied et, par extension, le
membre inférieur en entier, pied inclu ou exclu;
jambe'}{}
ou ce pié grant\wdx{grant pié}{m. terme d'anat. `membre inférieur
en entier
de l'homme y compris le pied'}{pié
grant}
est divisee en trois parties, ainsi que est la gra\emph{n}t
mein\wdx{grant main}{f.
terme d'anat. `membre supérieur de l'homme qui
s'attache au tronc, compris
entre l'épaule et les doigts; bras'}{grant mein}.
L'une partie est appellee cuisse\wdx{cuisse}{f. `partie
de la jambe qui s'articule à la hanche et s'étend
jusqu'au genou; cuisse'}{},
l'autre la petite
jambe\wdx{petite jambe}{f. terme d'anat.
`segment du membre
inférieur de l'homme compris entre le genou et
l'articulation du pied'}{} et l'autre le
%
[35v\hoch{o}b]
pié\wdx{pié}{m. `partie inférieure articulée à
l'extrémité de la jambe'}{}. Mais en lengue\wdx{langue}{f. `organe charnu, musculeux, allongé et mobile, placé dans la bouche;
langue'}{lengue}
arabique\wdx{arabique}{adj. `qui appartient, est
relatif à l'Arabie et ses habitants'}{} elles ont aultres
noms\wdx{nom}{m.
`mot
servant à désigner les êtres, les choses
qui appartiennent à une même catégorie
logique'}{}.
Le grant pié\wdx{grant pié}{m. terme d'anat. `membre inférieur
en entier
de l'homme y compris le pied'}{} en toutes ces parties est
composé de tieulz\wdx{tel}{adj. `qui est
semblable, du même genre; tel'}{tieulz
\emph{f.pl.}} partie que est la grant
main\wdx{grant main}{f.
terme d'anat. `membre supérieur de l'homme qui
s'attache au tronc, compris
entre l'épaule et les doigts; bras'}{}:
de cuir, de char, de vaines, de arteres, de nerfz, de
muscules, de thenans\wdx{*tendant}{m. terme d'anat. `structure conjonctive fibreuse par laquelle
un muscle s'insère sur un os'}{thenans
\emph{pl.}}, de coligances\wdx{colligance}{f.
2\hoch{o}
terme d'anat. `faisceau de tissu blanchâtre, résistant et peu extensible, unissant les éléments
d'une articulation ou maintenant en place un organe ou une partie d'un
organe; ligament'}{coligance}
et d'os,
desqueulx nous voulons dire l'un aprés l'autre. Du
cuir et de la char dit est cy devant. Des vaines et
des arteres tout ensa\emph{m}ble\wdx{ensemble}{adv. `l'un avec l'autre'}{ensamble}
est assavoir que -- puis que
les vaines en elles
ramefient\wdx{*ramifier}{v.pron.
`se diviser en plusieurs ramifications qui partent
d'un axe ou d'un centre de qch. (en parlant d'une
chose concrète)'}{ramefient
\emph{3.p.pl. ind.prés.}},
sont descendues de leur
origine\wdx{origine}{f.
`endroit d'où
quelque chose provient'}{} ou darnier\wdx{*derrenier}{adj.
`qui vient
après tous les autres, après lequel il n'y a pas
d'autre' (temporel ou spatial)}{darnier}
spondille --
elles se devisent en
deux parties: l'une va a la cuisse\wdx{cuisse}{f. `partie
de la jambe qui s'articule à la hanche et s'étend
jusqu'au genou; cuisse'}{}
dextre et l'autre
a la senextre. Et la, elles se
divisent\wdx{deviser}{v.pron. `se séparer en
parties; se diviser'}{divisent \emph{3.p.pl.
ind.prés.}} en deux grans
rames, l'ung va p\emph{ar} dehors et l'autre par dedans. Et
en ramefiant\wdx{*ramifier}{v.pron.
`se diviser en plusieurs ramifications qui partent
d'un axe ou d'un centre de qch. (en parlant d'une
chose concrète)'}{ramefiant \emph{p.prés.}}, elles
\text{vont}\fnb{Voranstehend gestrichenes
\emph{sont}.}/ es
chevilles\wdx{cheville}{f. `saillie
des os de l'articulation du pied, formée en dedans
par le tibia, en dehors par le péroné, aussi la partie située
entre le pied et la jambe; cheville'}{} et aux
piés\wdx{pié}{m. `partie inférieure articulée à
l'extrémité de la jambe'}{}.
\pend
\pstart
Et sont la
quatre vaines qui sont
co\emph{m}munema\emph{n}t\wdx{*comunement}{adv. `en
général'}{communemant}
flebothomees\wdx{*flebotomier}{v.tr.
`faire une saignée à'}{flebothomé
\emph{p.p.}} pour certaines maladies, c'est assavoir
la sophene\wdx{*saphene}{f.
terme d'anat. `vaisseau sanguin de la jambe dont
l'origine est le sacrum et qui
se manifeste dessous la cheville à la face interne du
pied'}{sophene}
qui est dessoubz
la cheville\wdx{cheville}{f. `saillie
des os de l'articulation du pied, formée en dedans
par le tibia, en dehors par le péroné, aussi la partie située
entre le pied et la jambe; cheville'}{} par dedans
\text{le}\fnb{Über der Zeile nachgetragen.}/
pié\wdx{pié}{m. `partie inférieure articulée à
l'extrémité de la jambe'}{}, vers le talon\wdx{talon}{m.
terme d'anat.
`partie
postérieure du pied; talon'}{},
et la
sciatique\wdx{*veine sciatique}{f.
terme d'anat. `vaisseau sanguin de la jambe dont
l'origine est le sacrum et qui
se manifeste dessous la cheville à la face externe du
pied'}{sciatique} qui est dessoubz la
cheville\wdx{cheville}{f. `saillie
des os de l'articulation du pied, formée en dedans
par le tibia, en dehors par le péroné, aussi la partie située
entre le pied et la jambe; cheville'}{} par dehors,
et la poplitique\wdx{*veine poplitique}{f. terme
d'anat.
`vaisseau sanguin de la jambe qui se
manifeste au jarret'}{poplitique} qui est
dessoubz le genoul\wdx{genoil}{m. `articulation de
la cuisse et de
la jambe, aussi la région avoisinante; genou'}{genoul}, et
la vaine renale\wdx{*veine renale}{f.
terme d'anat. `vaisseau sanguin de la jambe dont
l'origine est le sacrum et qui se manifeste entre le
quatrième et le cinquième orteil'}{vaine renale}
q\emph{ue} est
entre le petit doy\wdx{petit doi}{m. 2\hoch{o} terme d'anat. `le plus petit
des orteils'}{petit doy}
du pié\wdx{pié}{m. `partie inférieure articulée à
l'extrémité de la jambe'}{} et l'autre
ensuivant\wdx{*ensivant}{adj. `qui vient
immédiatement après; suivant'}{ensuivant}.
Donc es jambes\wdx{jambe}{f.
`partie du membre
inférieur de l'homme comprenant le segment entre le
genou et l'articulation du pied et, par extension, le
membre inférieur en entier, pied inclu ou exclu;
jambe'}{} sont
quatre
vaines manifestes\wdx{manifest}{adj. `dont l'existence ou la nature
est évidente; manifeste'}{}
et grosses q\emph{ue} peullent fere gra\emph{n}t\wdx{grant}{adj.
3\hoch{o} dans
l'ordre qualitatif, non quantifiable `qui est d'un
degré supérieur à la moyenne en ce qui concerne la
qualité, l'intensité, l'importance'}{}
flux\wdx{*flus}{m. `action de couler (dit
d'un liquide)'}{flux} de sang et
grant\wdx{grant}{adj. 3\hoch{o} dans
l'ordre qualitatif, non quantifiable `qui est d'un
degré supérieur à la moyenne en ce qui concerne la
qualité, l'intensité, l'importance'}{}
peril\wdx{peril}{m.
`état, situation où l'on
court de grands risques; péril'}{}
me\emph{n}tes\wdx{maint}{adj. `plusieurs; maint'}{mentes
\emph{pl.}} fois\wdx{*foiz}{f.
`cas où un fait se produit, moment du temps où un
événement, conçu comme identique à d'autres
événements, se produit; fois'}{fois}, et se
y a pluseurs aultres rames desqueulx le cirurgien ne
doit pas gra\emph{n}demant\wdx{*grantment}{adv. dans
l'ordre qualitatif `au dessus de la moyenne,
considérablement'}{grandemant} curer\wdx{curer}{v.tr.
`soumettre à un traitement médical'}{}.
\pend
\pstart
Item, des nerfz du
pié\wdx{pié}{m. `partie inférieure articulée à l'extrémité de
la jambe'}{},
%
[36r\hoch{o}a]
Avicene\adx{Avicene}{}{} dit q\emph{ue}
ilz
se diverciffient\wdx{*diversefiier}{v.pron.
`être
différent; différer'}{diverciffient
\emph{3.p.pl. ind.prés.}} moult
des nerfz des mains; et toutesfois, \text{des
darnieres
spondilles des reins et de l'os
sacré}\lemma{des darnieres{\dots} l'os
sacré}\fnb{Verb fehlt (cp. GuiChaul\textsc{jl}
48,16 \emph{ipsi oriuntur a spondylis ultimis renum
et ossis sacri}).}/\wdx{*derrenier}{adj.
`qui vient
après tous les autres, après lequel il n'y a pas
d'autre' (temporel ou
spatial)}{darnier}\wdx{rein}{m. 2\hoch{o} au plur.
`la partie inférieure du dos au niveau des
vertèbres lombaires'}{}\wdx{os
sacré}{m. terme d'anat.
`os formé par la réunion des cinq
vertèbres sacrées, a la partie inférieure de la colonne
vertébrale; sacrum'}{}; et la plus grant\wdx{grant}{adj. 1\hoch{o}
dans l'ordre
physique, quantifiable `qui est d'une extension
au-dessus de la moyenne; grand (des choses)'}{grant
\emph{f.sg.}} p\emph{ar}tie passe par le trou de l'os de la
cuisse
et descend au
muscules
du genoil\wdx{genoil}{m. `articulation de la cuisse et de la jambe,
aussi la région
avoisinante; genou'}{}. Et de ces nerfz ci q\emph{ue} se
co\emph{n}joingnent\wdx{conjoindre}{v.pron.
`se mettre ensemble de manière à se
toucher ou tenir ensemble'}{conjoingnent
\emph{3.p.pl. ind.prés.}}
avec les muscules et avec
les cordes qui mouvent la joincture et
\text{descendent}\fnb{Ms. \emph{descendendent}.}/
des hanches et se appliquent\wdx{*apliquier}{v.pron.
`se placer sur (qch.) de manière à y
adhérer'}{appliquent \emph{3.p.pl. ind.prés.}}
a l'os de la
cuisse,
de cela se font les grans muscules q\emph{ue} sont sur
la cuisse\wdx{cuisse}{f. `partie
de la jambe qui s'articule à la hanche et s'étend
jusqu'au genou; cuisse'}{}
et mouve\emph{n}t le genoil\wdx{genoil}{m.
`articulation de la cuisse et de la jambe, aussi la région
avoisinante; genou'}{} et la
jambe\wdx{jambe}{f.
`partie du membre
inférieur de l'homme comprenant le segment entre le
genou et l'articulation du pied et, par extension, le
membre inférieur en entier, pied inclu ou exclu;
jambe'}{}, et aussi
font les musculles\wdx{muscle}{m. terme d'anat. `structure
organique contractile qui assure les mouvements;
muscle'}{musculle} sur la
jambe\wdx{jambe}{f.
`partie du membre
inférieur de l'homme comprenant le segment entre le
genou et l'articulation du pied et, par extension, le
membre inférieur en entier, pied inclu ou exclu;
jambe'}{} qui
mouvent le pié\wdx{pié}{m.
`partie inférieure articulée à l'extrémité de la jambe'}{} en
la cheville\wdx{cheville}{f. `saillie
des os de l'articulation du pied, formée en dedans
par le tibia, en dehors par le péroné, aussi la partie située
entre le pied et la jambe; cheville'}{},
et les muscules des piés\wdx{pié}{m.
`partie inférieure articulée à l'extrémité de la jambe'}{}
qui mouvent les
dois\wdx{doi}{m. 2\hoch{o} `chacun des cinq prolongements
qui terminent le pied'}{} -- tout ainsi co\emph{m}me dit est
des mains --, mais
ilz font une autre op\emph{er}acion\wdx{operacïon}{f.
`action d'un pouvoir, d'une fonction,
d'un organe qui produit un effet selon sa
nature; opération'}{operacion}
q\emph{ue}
ne varie\wdx{*variier}{v.tr. `rendre autre ou
différent (qch., qn); changer'}{varie \emph{3.p.sg. ind.prés.}}
pas moult
le cirurgien. Et dois savoir que les plaies pres des
joinctures sont moult perilleuses\wdx{*perillos}{adj.
`qui constitue un danger, présente du
danger; dangereux'}{perilleuses \emph{f.pl.}}.
\text{Les}\fnb{\emph{Nota} in Foliomitte.}/
colligances\wdx{colligance}{f. 2\hoch{o}
terme d'anat. `faisceau de tissu blanchâtre, résistant et peu extensible, unissant les éléments
d'une articulation ou maintenant en place un organe ou une partie d'un
organe; ligament'}{}
et les liguemans longs et gros descendent par toute
la ja\emph{m}be\wdx{jambe}{f.
`partie du membre
inférieur de l'homme comprenant le segment entre le
genou et l'articulation du pied et, par extension, le
membre inférieur en entier, pied inclu ou exclu;
jambe'}{}, et se
manifestent\wdx{manifester}{v.pron.
`se révéler
clairement dans son existence ou sa nature'}{}
moult dessoubz les
inguines\wdx{inguine}{m. `partie latérale et inférieure du
bas-ventre; aine'}{}
et dessoubz le genoil\wdx{genoil}{m. `articulation de la
cuisse et de la jambe, aussi la région
avoisinante; genou'}{} et
deseur\wdx{*desor}{prép. qui marque la position en
haut `sur'}{} le talon\wdx{talon}{m.
terme d'anat.
`partie
postérieure du pied; talon'}{} et sur la joincture des
dois\wdx{doi}{m. 2\hoch{o} `chacun des cinq prolongements
qui terminent le pied'}{};
et toute la sole\wdx{sole}{f. `face inférieure;
plante'}{}
des piés\wdx{pié}{m. `partie inférieure articulée à
l'extrémité de la jambe'}{} est faite de liguemans.
\pend
\pstart
Finalmant\wdx{finalment}{adv. `à la fin'}{finalmant}
il no\emph{us} co\emph{n}vient dire
des os, selon la division\wdx{*devisïon}{f.
`action de
diviser (qch.) en parties, le résultat de
cette action'}{division}
du pié ou de la
grant jambe\wdx{grant jambe}{f. terme d'anat. `membre inférieur
en entier de l'homme y compris le pied'}{}. En la premiere
partie, q\emph{ue} on appelle la cuisse, ha ung seul\wdx{*sol}{adj. `qui n'est pas
avec d'autres semblables; seul'}{seul}
os qui
est grant\wdx{grant}{adj. 2\hoch{o}
dans l'ordre
physique, quantifiable `qui est d'une extension
au-dessus de la moyenne; grand (du corps humain
et de ses parties)'}{}
et plain\wdx{*plein}{adj.
`qui contient toute
la quantité possible; plein'}{plain}
de moelle\wdx{*mëole}{f.
terme d'anat.
`substance molle et
grasse de l'intérieur des os; moelle'}{moelle} et est rond a chescun
chief. La \text{rondesse}\fnb{Nachfolgend
gestrichenes \emph{q}.}/\wdx{*rëondece}{f. `état de ce qui
est rond; rondeur'}{rondesse}
de
dessus q\emph{ue}
est toute seule\wdx{*sol}{adj. `qui n'est pas
avec d'autres semblables; seul'}{seul}, q\emph{ue} on
apelle
le
vertebre\wdx{vertebre}{m. terme d'anat. `extrémité arrondie
supérieure de l'os de la cuisse; tête du fémur'}{} ou le
tour[36r\hoch{o}b]na\emph{n}t\wdx{tournant}{m. terme d'anat. `extrémité
arrondie d'un os'}{}, q\emph{ue} se
trait\wdx{traire}{v.pron. `se diriger quelque
part'}{}
par dedans, elle est receue en l'os de la ha\emph{n}che et
est ung
petit\wdx{petit}{adj. 3\hoch{o}
dans
l'ordre qualitatif, non quantifiable `qui est d'un
degré inférieur à la moyenne en ce qui concerne la
qualité, l'intensité,
l'importance'}{\textbf{un petit} \emph{adv.
`un peu'}}
bossue\wdx{*boçu}{adj.
`qui
présente une ou plusieurs saillies arrondies; bossu'}{bossu}
p\emph{ar} dehors. Mais en la p\emph{ar}tie de desoubz,
vers le genoil\wdx{genoil}{m. `articulation de la
cuisse et de la jambe, aussi la région avoisinante;
genou'}{}, il ha deux rondesses\wdx{*rëondece}{f. `état de ce
qui est rond; rondeur'}{rondesse} q\emph{ue} sont receues
et tourne\emph{n}t\wdx{tourner}{v.intr.
`se mouvoir circulairement ou décrire une ligne
courbe'}{} en deux
co\emph{n}cavités\wdx{concavité}{f.
`espace vide à
l'intérieur d'un corps solide; concavité'}{}
qui
sont ou fossile\wdx{focile}{m.
terme d'anat.
`chacun des deux os de l'avant-bras ou de la jambe; cubitus, radius,
tibia, péroné'}{fossile}
de la gra\emph{n}t jambe\wdx{grant jambe}{f. terme d'anat. `membre
inférieur en entier de l'homme y compris le pied'}{}; et par dessus est ung os tout
rond q\emph{ue} on appelle le patine\wdx{patine du genoil}{f.
terme d'anat. `os plat triangulaire, légèrement
bombé, qui est situé à la face antérieure du
genou; rotule'}{} ou la
pallete du genoil\wdx{*palette du genoil}{f.
terme d'anat. `os plat triangulaire, légèrement
bombé, qui est situé à la face antérieure du
genou; rotule'}{pallete du genoil}.
Aprés vient la jambe\wdx{jambe}{f.
`partie du membre
inférieur de l'homme comprenant le segment entre le
genou et l'articulation du pied et, par extension, le
membre inférieur en entier, pied inclu ou exclu;
jambe'}{}
et y a deux os q\emph{ue} on
appelle focilles\wdx{focile}{m. terme d'anat. `chacun des deux os
de l'avant-bras ou de la jambe; cubitus, radius, tibia, péroné'}{focille}.
Le pl\emph{us}
grant\wdx{grant}{adj. 2\hoch{o} dans l'ordre
physique, quantifiable `qui est d'une extension
au-dessus de la moyenne; grand (du corps humain
et de ses parties)'}{}
est en la p\emph{ar}tie de
devant, que on appelle
domestiq\emph{ue}\wdx{domestique}{adj. `qui est situé en
dedans'}{}, qui fait
la aguté\wdx{agüeté}{f. `état de ce qui est aigu'}{aguté} de la
jambe\wdx{jambe}{f.
`partie du membre
inférieur de l'homme comprenant le segment entre le
genou et l'articulation du pied et, par extension, le
membre inférieur en entier, pied inclu ou exclu;
jambe'}{} en
descenda\emph{n}t du genoil\wdx{genoil}{m. `articulation de la
cuisse et de la jambe, aussi la région avoisinante;
genou'}{} jusq\emph{ue}s
au pié, et fait la cheville\wdx{cheville}{f. `saillie
des os de l'articulation du pied, formée en dedans
par le tibia, en dehors par le péroné, aussi la partie située
entre le pied et la jambe; cheville'}{}
q\emph{ue} est \text{p\emph{ar} dedens
pié}\fnb{Zu lesen \emph{par dedens le pié}\,?
Fehlender Artikel ebenso in l.\,1326 und
1331.}/. Et le petit\wdx{petit}{adj. 2\hoch{o} dans
l'ordre
physique, quantifiable `qui est d'une extension
au-dessous de la moyenne; petit (du corps humain et
de ses parties)'}{}
os est par dehors, en
la partie que on
appelle sauvage\wdx{sauvage}{adj. `qui est situé en
dehors'}{}, et descend de
pres
le genoil\wdx{genoil}{m. `articulation de la cuisse et de
la jambe, aussi la région avoisinante; genou'}{} bas
jusques au pié, et se
\emph{con}joingt\wdx{conjoindre}{v.pron.
`se mettre ensemble de manière à se
toucher ou tenir ensemble'}{conjoingt
\emph{3.p.sg. ind.prés.}}
avec le
gra\emph{n}t focile\wdx{grant focile}{m. terme d'anat.
`le plus
gros des deux os de l'avant-bras ou de la jambe;
cubitus, tibia'}{} et
fait la cheville\wdx{cheville}{f. `saillie
des os de l'articulation du pied, formée en dedans
par le tibia, en dehors par le péroné, aussi la partie située
entre le pied et la jambe; cheville'}{}
q\emph{ue} est p\emph{ar}
dehors pié\wdx{*defors}{prép. `à
l'extérieur de'}{dehors}.
Mais Guille\emph{m} de Saliceto\adx{Guillem de Saliceto}{}{}
et Lanfrant\adx{Lanfrant}{}{}, qui
l'ensuit\wdx{*ensivre}{v.pron. 2\hoch{o}
`penser ou agir selon (les idées, la
conduite de qn)'}{ensuit \emph{3.p.sg.
ind.prés.}}, dient le
contraire\wdx{contraire}{m.
`ce qui est
opposé logiquement; contraire'}{} et ne die\emph{n}t
pas vray\wdx{dire}{v.tr.
`lire à haute voix;
réciter'}{\textbf{dire vrai}
\emph{`dire la vérité'} dire vray},
et q\emph{ue} le veult veoir, si le vait regarder.
La fourme de ceux deux focilles\wdx{focile}{m. terme d'anat. `chacun des deux os
de l'avant-bras ou de la jambe; cubitus, radius, tibia, péroné'}{focille}
est telle q\emph{ue}
le grant ha deux concavités\wdx{concavité}{f.
`espace vide à
l'intérieur d'un corps solide;
concavité'}{} vers le
genoil\wdx{genoil}{m. `articulation de la cuisse et de la
jambe, aussi la région avoisinante; genou'}{} esquelles
entrent\wdx{entrer}{v.tr.indir.
`s'emboîter (dans qch.) (de choses)'}{} les
rondesses\wdx{*rëondece}{f.
`état de ce qui est rond;
rondeur'}{rondesse} de l'os de la cuisse, car le petit\wdx{petit}{adj.
2\hoch{o} dans l'ordre
physique, quantifiable `qui est d'une extension
au-dessous de la moyenne; petit (du corps humain et
de ses parties)'}{}
os ne vient pas jusques a la
joincture. Mais il est planté si co\emph{m}me dit est et est
joingt\wdx{joindre}{v.tr.
`mettre des choses
ensemble, de façon qu'elles se touchent ou tiennent
ensemble; joindre'}{joingt \emph{p.p. m.sg.}}
a l'autre pres de dessoubz
le
genoil\wdx{genoil}{m. `articulation de la cuisse et de la
jambe, aussi la région avoisinante; genou'}{}, par
dehors\wdx{*defors}{prép. `à
l'extérieur de'}{dehors} jambe,
et pour ce on le appelle la aguille\wdx{aguille}{f.
2\hoch{o} `ce dont la forme ressemble
à celle
d'une fine tige pointue' (dit d'un os) (par analogie de forme)}{} de la
jambe.
Mais vers le pié il se joingt\wdx{joindre}{v.pron.
`se mettre ensemble de
manière à se toucher ou tenir ensemble'}{joingt
\emph{3.p.sg. ind.prés.}} au grant\wdx{grant}{adj.
2\hoch{o} dans l'ordre
physique, quantifiable `qui est d'une extension
au-dessus de la moyenne; grand (du corps humain
et de ses parties)'}{}
os, et entre
eulx deux ilz font une \emph{con}cavité
lunaire\wdx{lunaire}{adj. `dont la forme ressemble à
celle du croissant de la lune'}{} ou
cornue\wdx{cornu}{adj. `dont la forme ressemble à
celle d'une corne'}{} en
la\mbox{quel}le entre\wdx{entrer}{v.tr.indir.
`s'emboîter (dans qch.) (de choses)'}{}
le pre[36v\hoch{o}a]mier os du pié.
\pend
\pstart
Item, au pié sont
trois
assies\wdx{acie}{f. `catégorie de choses considérée d'après sa
structure, son organisation ou sa place dans une série'
(dit des os)}{assie} d'os: en la p\emph{re}miere assie
sont trois os qui
sont assemblés ensemble, a
maniere\wdx{maniere}{f. 2\hoch{o} `forme particulière
que revêt l'accomplissement d'une action, le
déroulement d'un fait, l'être ou
l'existence'}{\textbf{a\,/\,en maniere de}
\emph{loc.prép. `comme'}}
d'une rondiole\wdx{rondiole}{f. `pièce ronde, peu
épaisse et généralement évidée; rondelle'}{}. Le
premier est appellé cahab\wdx{os cahab}{m.
terme d'anat.
`os du pied qui forme, avec le calcanéum, la
rangée postérieure du tarse; astragale'}{} en
arabic\wdx{*arabique}{m. `la langue
parlée par les arabes'}{arabic}. Et en
grec\wdx{grec}{m. `la langue parlée par les
grecs'}{} on le appelle
astragalus\wdx{astragalus}{lt. terme d'anat.
`os du pied qui forme, avec le calcanéum, la
rangée postérieure du tarse; astragale'}{}, et est
ainsi q\emph{ue} une noix de arbalestre\wdx{noix de
arbalestre}{f. `pièce de
l'arme de trait qui consiste d'un arc d'acier
monté sur un fût, et qui est retenue par le
ressort et retient dans une de ses encoches la
corde de l'arc bandé'}{}, rond d'une p\emph{ar}tie et d'autre,
et en la rondesse\wdx{*rëondece}{f. `état de ce qui est rond;
rondeur'}{rondesse} de dessus se
ferme\wdx{fermer}{v.pron. `être attaché (à qch.)'}{}
la co\emph{n}cavité des
focilles\wdx{focile}{m. terme d'anat. `chacun des deux os
de l'avant-bras ou de la jambe; cubitus, radius, tibia, péroné'}{focille}
et la
se meut\wdx{*movoir}{v.pron. `faire un
mouvemant; bouger'}{meut
\emph{3.p.sg. ind.prés.}} le pié.
Et en l'autre rondesse\wdx{*rëondece}{f. `état de ce qui est
rond; rondeur'}{rondesse} se
ferme\wdx{fermer}{v.pron. `être attaché (à qch.)'}{}
la \emph{con}cavité de l'os
naviculaire\wdx{os naviculaire}{m. terme d'anat.
`os de la rangée antérieure du tarse articulé
avec l'astragale; os naviculaire'}{}.
Aprés l'os cahab\wdx{os cahab}{m. terme d'anat.
`os du pied qui forme, avec le calcanéum, la
rangée postérieure du tarse; astragale'}{}, vers le pié,
vient
l'os naviculaire\wdx{os naviculaire}{m.
terme d'anat.
`os de la rangée antérieure du tarse articulé
avec l'astragale; os naviculaire'}{} qui est
ainsi q\emph{ue} une nef\wdx{nef}{f.
`construction
flottante de forme allongée destinée aux transports
sur mer; navire'}{}
\emph{con}cave\wdx{concave}{adj.
`qui présente une
surface en creux; concave'}{} d'une part et d'autre.
En la
premiere concavité entre\wdx{entrer}{v.tr.indir. `s'emboîter
(dans qch.) (de choses)'}{} la
ro\emph{n}desse\wdx{*rëondece}{f. `état de ce
qui est rond; rondeur'}{rondesse} de l'os
cahab\wdx{os cahab}{m.
terme d'anat.
`os du pied qui forme, avec le calcanéum, la
rangée postérieure du tarse; astragale'}{}, si
co\emph{m}me dit est, et en l'autre \emph{con}cavité
\text{entrent}\fnb{Ms. \emph{entrentrent}.}/\wdx{entrer}{v.intr.
`passer du dehors en dedans'}{} les
rondesses\wdx{*rëondece}{f.
`état de ce qui est rond; rondeur'}{rondesse}
des os de la seconde assie\wdx{acie}{f. `catégorie de choses
considérée d'après sa structure, son organisation ou sa place dans
une série' (dit des os)}{assie} du pié.
Et dessoubz ces deux os est le
calcane\wdx{calcane}{m. terme d'anat. `os du tarse
qui forme le talon; calcanéum'}{} qui est fait en
la fourme d'ung talon\wdx{talon}{m. terme d'anat.
`partie
postérieure du pied; talon'}{}, ou se
ferme\wdx{fermer}{v.pron. `être attaché (à qch.)'}{}
tout
le pié, et ist par
derrier\wdx{derrier}{adv. `du
côté opposé au visage, à la face'}{}
pour les
liguema\emph{n}s q\emph{ue} sont plantés dedens lui. Item,
ap\emph{ré}s le
naviculaire\wdx{os naviculaire}{m.
terme d'anat.
`os de la rangée antérieure du tarse articulé
avec l'astragale; os naviculaire'}{naviculaire
\emph{adj.subst.}} vient
la seconde assie\wdx{acie}{f. `catégorie de choses
considérée
d'après sa structure, son organisation ou sa place dans une série'
(dit des os)}{assie} des os des piés; en
laquelle assie\wdx{acie}{f. `catégorie de choses
considérée d'après
sa structure, son organisation ou sa place dans une série'
(dit des os)}{assie}
so\emph{n}t quatre os assis\wdx{assez}{adv.
`en suffisance;
assez'}{assis} certains\wdx{*certain}{adv. `d'une manière
qui ne peut manquer de se produire; certainement'}{certains},
et l'ung d'iceux
est appellé grandineux\wdx{grandineux}{adj. `qui a la
forme d'un grain de grêle' (dit d'un os dans le
pied)}{} et est par dehors vers le
petit doy\wdx{petit doi}{m. 2\hoch{o} terme d'anat. `le plus petit
des orteils'}{petit doy}. Et les dis os sont
rons\wdx{*rëont}{adj. `qui a la forme
circulaire'}{rons \emph{m.pl.}} vers le
naviculaire\wdx{os naviculaire}{m.
terme d'anat.
`os de la rangée antérieure du tarse articulé
avec l'astragale; os naviculaire'}{naviculaire
\emph{adj.subst.}} et si sont
\emph{con}caves\wdx{concave}{adj.
`qui présente une
surface en creux; concave'}{} vers la tierce
assie\wdx{acie}{f. `catégorie de choses considérée
d'après sa
structure, son organisation ou sa place dans une série'
(dit des os)}{assie}. Item,
en la tierce assie\wdx{acie}{f. `catégorie de choses
considérée
d'après sa structure, son organisation ou sa place dans une série'
(dit des os)}{assie} sont .v. os qui
sont assis\wdx{assez}{adv. `en suffisance; assez'}{assis}
\text{longs}\fnb{Über der Zeile nachgetragen.}/ et
coresponde\emph{n}t\wdx{*correspondre}{v.intr.
`être placé de façon symétrique'}{corespondent
\emph{3.p.pl. ind.prés.}} et ressoivent les
.v. dois\wdx{doi}{m. 2\hoch{o}
`chacun des cinq prolongements
qui terminent le pied'}{}. Et ch\emph{acu}m
des dois\wdx{doi}{m. 2\hoch{o}
`chacun des cinq prolongements
qui terminent le pied'}{}
ha trois os, excepté\wdx{excepté}{prép.
`à la réserve
de'}{}
le
poulz\wdx{*pouz}{m. `le plus gros et le plus fort des doigts de la
main et du pied'}{poulz}
qui ne ha que deux. Donc le pié
%
[36v\hoch{o}b]
ha tarsum\wdx{*tarsus}{mlt.
terme d'anat. `partie du squelette du pied
formée de sept os courts disposés
en double rangée qui occupent la moitié
postérieure du pied; tarse'}{tarsum},
methatarsum\wdx{*metatarsus}{mlt.
terme d'anat. `partie du squelette du pied
formée de cinq os courts, comprise entre le tarse et
les phalanges des orteils; métatarse'}{methatarsum}
\text{et}\fnb{Sinngemäß zu lesen \emph{ou}, cf.
{\pfeilr} \kr\textsc{peigne}.}/ le pigne\wdx{*peigne}{m.
terme d'anat.
`ensemble des os (métacarpiens, métatarsiens)
qui constituent la partie du squelette de la main
entre le carpe et les premiers phalanges des doigts,
et celle du squelette du pied entre le tarse et les
premiers phalanges des orteils;
métacarpe,
métatarse'}{pigne},
tout ainsi que la mein. Donc ou petit
pié\wdx{petit pié}{m. terme d'anat. `partie inférieure articulée à
l'extrémité de la jambe; pied'}{} sont
.xxvj.
os et en tout le grant pié\wdx{grant pié}{m. terme d'anat. `membre
inférieur
en entier
de l'homme y compris le pied'}{} ou la jambe en ha
.xxx.
\pend
\pstart
Et
par ce que dit est, le cirurgien peult considerer\wdx{considerer}{v.tr.
`regarder (qch.) attentivement;
considérer'}{}
la
maniere et co\emph{m}me les jambes et les
bras et les autres
membres se peullent
deloguer\wdx{deloguer}{v.pron. `se luxer'}{}
ou briser\wdx{*brisier}{v.tr. empl.abs.
`mettre en morceaux d'une manière soudaine, par coup
ou pression; briser'}{briser \emph{inf.}}
et, par co\emph{n}sequent\wdx{consequent}{adj.}{\textbf{par
consequent}
\emph{loc.adv. `comme suite logique'}}, il peult
considerer\wdx{considerer}{v.tr.
`regarder (qch.) attentivement;
considérer'}{}
la maniere co\emph{m}me
on les pourra ramener\wdx{ramener}{v.tr. `faire
revenir une chose au lieu qu'elle avait quitté'}{} et
remetre\wdx{remetre}{v.tr.
`faire passer (qch.) de nouveau dans son ancien
état, son ancienne place'}{} a point\wdx{point}{m. `endroit fixé et déterminé (où qch. à
lieu)'}{}.
Item, aussi
par ce que dit est, le dit cirurgien peult savoir et
veoir que entre ces joinctures devant dictes, laquelle
est la plus difficille\wdx{*dificile}{adj.
`qui
n'est pas facile; difficile'}{difficille}
a
desloier\wdx{*desliier}{v.tr.
`provoquer la
luxation de (un os, une articulation);
disloquer'}{desloier \emph{inf.}} et a
remectre\wdx{remetre}{v.tr. `faire
passer (qch.) de nouveau dans son ancien
état, son ancienne place'}{remectre \emph{inf.}} a son
droit\wdx{droit}{adj. 2\hoch{o}
`qui suit un
raisonnement correct'}{}
point\wdx{point}{m. `endroit fixé et déterminé (où qch. à
lieu)'}{}, et quelle est plus
legiere\wdx{legier}{adj. 2\hoch{o} `qui se
fait facilement'}{} a
desloiier\wdx{*desliier}{v.tr.
`provoquer la
luxation de (un os, une articulation);
disloquer'}{desloiier \emph{inf.}} et
a ramener\wdx{ramener}{v.tr.
`faire
revenir une chose au lieu qu'elle avait quitté'}{} a
son lieu. Car la
joincture du petit pié\wdx{petit pié}{m. terme d'anat. `partie
inférieure articulée à l'extrémité de la jambe; pied'}{}, c'est la
plus difficile\wdx{*dificile}{adj.
`qui
n'est pas facile; difficile'}{difficille}
a
desloiier\wdx{*desliier}{v.tr.
`provoquer la
luxation de (un os, une articulation);
disloquer'}{desloiier \emph{inf.}} et a
remetre\wdx{remetre}{v.tr. `faire
passer (qch.) de nouveau dans son ancien
état, son ancienne place'}{}
en son lieu. Et la joincture du
genoil\wdx{genoil}{m. `articulation de la cuisse
et de la jambe, aussi la région avoisinante; genou'}{}, c'est la
plus
legiere\wdx{legier}{adj. 2\hoch{o} `qui se fait
facilement'}{} a desloiier\wdx{*desliier}{v.tr.
`provoquer la
luxation de (un os, une articulation);
disloquer'}{desloiier \emph{inf.}} et
la plus aisee\wdx{aisé}{adj. `qui se fait avec aise;
facile'}{} a remectre\wdx{remetre}{v.tr. `faire
passer (qch.) de nouveau dans son ancien
état, son ancienne place'}{remectre \emph{inf.}}
a point\wdx{point}{m. `endroit fixé et déterminé (où qch. à
lieu)'}{}. Mais la joincture de la scie\wdx{scie}{f.
terme d'anat. `concavité dans la
jointure de la hanche dans laquelle s'emboîte
la tête du fémur'}{}, c'est a dire
de la cuisse
ou de la hanche, elle est moyenne a l'ung
et a l'autre. Et vous souffise\wdx{*sofire}{v.intr.
`avoir la quantité, la qualité, la force
etc. nécessaire pour (qch.);
suffire'}{\textbf{*sofire a qn}
\emph{v.impers. + subj.prés.} vous souffise}
ce que dit est de la
anathomie des membres.
\pend
%
% \memorybreak
%
\pstartueber
EXPLICIT PRIMUS LIBER
\pend
\endnumbering
%\clearpage
\end{edition}
\endinput
